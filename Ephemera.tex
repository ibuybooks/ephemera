\documentclass[11pt,oneside,openany,extrafontsizes]{memoir}

\usepackage{fontspec}

\setmainfont{EB Garamond}[
  Ligatures=TeX,
  Numbers=Proportional
]
\setsansfont{Open Sans}[
  Ligatures=TeX,
  Scale=MatchLowercase
]

\usepackage[dvipsnames,svgnames]{xcolor}

\definecolor{primaryMain}{RGB}{33, 33, 33} 
\definecolor{accentMain}{RGB}{33, 33, 33}  
\definecolor{secondaryMain}{RGB}{33, 33, 33}    
\definecolor{primaryLight}{RGB}{255, 255, 255}
\definecolor{primaryDark}{RGB}{33, 33, 33}    

\definecolor{questionBg}{RGB}{255, 255, 255}  
\definecolor{emailBg}{RGB}{255, 255, 255}     
\definecolor{quoteBg}{RGB}{255, 255, 255}    
\definecolor{answerBg}{RGB}{255, 255, 255}      
\definecolor{warningBg}{RGB}{255, 255, 255}   
\definecolor{warningFrame}{RGB}{33, 33, 33}  

\definecolor{textMain}{RGB}{33, 33, 33}   
\definecolor{noteColor}{RGB}{75, 75, 75}     
\definecolor{metadataColor}{RGB}{117, 117, 117}  
\definecolor{externalLinkColor}{RGB}{60, 60, 60}  
\definecolor{borderColor}{RGB}{189, 189, 189}    
\definecolor{secondaryLight}{RGB}{117, 117, 117} 
\definecolor{ruleColor}{RGB}{189, 189, 189} 
\definecolor{linkColor}{RGB}{33, 33, 33}      
\definecolor{accentLight}{RGB}{255, 255, 255}  
\definecolor{accentDark}{RGB}{33, 33, 33}    
\definecolor{secondaryExtraLight}{RGB}{255, 255, 255}

\usepackage{graphicx}
\usepackage{tikz}
\usetikzlibrary{shadows, shadows.blur}
\usepackage[english]{babel}
\usepackage[autostyle]{csquotes}
\usepackage[final]{microtype}     
\usepackage{selnolig} 
\usepackage{fontawesome5}
\usepackage[version=4]{mhchem}
\usepackage{siunitx}
\sisetup{per-mode=symbol}
\DeclareSIUnit\IU{IU}
\DeclareSIUnit\day{day}

\setstocksize{11in}{8.5in}
\settrimmedsize{\stockheight}{\stockwidth}{*}
\settrims{0pt}{0pt}

\settypeblocksize{*}{40pc}{1.618}

\setlrmarginsandblock{1.25in}{1.25in}{*}
\setulmarginsandblock{1.25in}{1.25in}{*}

\setheadfoot{1.5\onelineskip}{2\onelineskip}
\setheaderspaces{*}{2\onelineskip}{*}

\checkandfixthelayout

\setSingleSpace{1.3}
\SingleSpacing

\setlength{\parindent}{1.5em}
\setlength{\parskip}{0pt}

\clubpenalty=10000
\widowpenalty=10000
\raggedbottom

\hyphenpenalty=1000
\exhyphenpenalty=1000
\tolerance=3000

\emergencystretch=1em

\makechapterstyle{raypeat}{%
    \renewcommand{\chapterheadstart}{\vspace*{\beforechapskip}}
    \renewcommand{\chapnamefont}{\normalfont\Large\scshape\raggedright\color{metadataColor}}
    \renewcommand{\chapnumfont}{\normalfont\Large\scshape\color{primaryMain}}
    \renewcommand{\chaptitlefont}{\normalfont\huge\bfseries\raggedright\color{primaryMain}}
    \renewcommand{\printchaptername}{\chapnamefont\chaptername}
    \renewcommand{\chapternamenum}{\ }
    \renewcommand{\printchapternum}{\chapnumfont\thechapter}
    \renewcommand{\afterchapternum}{\par\nobreak\vskip 0.5\onelineskip}
    \renewcommand{\printchaptertitle}[1]{%
        \chaptitlefont ##1\par\nobreak%
        \vskip 0.5\onelineskip%
        {\color{primaryMain}\hrule height 1.5pt}%
    }
    \setlength{\beforechapskip}{3\onelineskip}
    \setlength{\afterchapskip}{2\onelineskip}
}

\chapterstyle{raypeat}

\setsecheadstyle{\Large\bfseries\raggedright\color{primaryMain}}
\setsubsecheadstyle{\large\bfseries\raggedright\color{accentMain}}
\setsubsubsecheadstyle{\normalsize\bfseries\raggedright\color{accentDark}}

\setbeforesecskip{1.5\baselineskip}
\setaftersecskip{0.5\baselineskip}

\setbeforesubsecskip{1.5\baselineskip}
\setaftersubsecskip{0.5\baselineskip}

\setsecnumdepth{subsection}
\settocdepth{subsection}

\renewcommand{\contentsname}{Contents}

\renewcommand{\cftchapterfont}{\sffamily\large\bfseries\color{primaryMain}}
\renewcommand{\cftchapterpagefont}{\sffamily\large\bfseries\color{primaryMain}}
\renewcommand{\cftchapterdotsep}{\cftdotsep}
\setlength{\cftbeforechapterskip}{1.5em}
\setlength{\cftchapterindent}{0pt}
\renewcommand{\cftchapterleader}{\hspace{1em}\color{borderColor}\cftdotfill{\cftchapterdotsep}}

\renewcommand{\cftsectionfont}{\normalfont\color{textMain}}
\renewcommand{\cftsectionpagefont}{\normalfont\color{metadataColor}}
\setlength{\cftbeforesectionskip}{0.3em}
\setlength{\cftsectionindent}{2em}
\renewcommand{\cftsectionleader}{\hspace{0.75em}\color{borderColor}\cftdotfill{\cftsectiondotsep}}

\renewcommand{\cftsubsectionfont}{\small\color{secondaryLight}}
\renewcommand{\cftsubsectionpagefont}{\small\color{metadataColor}}
\setlength{\cftbeforesubsectionskip}{0.2em}
\setlength{\cftsubsectionindent}{3.5em}
\renewcommand{\cftsubsectionleader}{\hspace{0.5em}\color{borderColor}\cftdotfill{\cftsubsectiondotsep}}

\renewcommand{\cftdotsep}{2.5}

\makepagestyle{raypeat}
\makeevenhead{raypeat}{\small\itshape\color{metadataColor}\leftmark}{}{}
\makeoddhead{raypeat}{}{}{\small\itshape\color{metadataColor}\rightmark}
\makeevenfoot{raypeat}{}{\small\bfseries\color{primaryMain}\thepage}{}
\makeoddfoot{raypeat}{}{\small\bfseries\color{primaryMain}\thepage}{}
\makeheadrule{raypeat}{\textwidth}{0.5pt}

\makepagestyle{plain}
\makeevenfoot{plain}{}{\small\bfseries\color{primaryMain}\thepage}{}
\makeoddfoot{plain}{}{\small\bfseries\color{primaryMain}\thepage}{}

\pagestyle{raypeat}

\renewcommand{\makeheadrule}{%
    \color{borderColor}\rule[-.3\baselineskip]{\linewidth}{0.5pt}%
}

\usepackage{tcolorbox}
\tcbuselibrary{skins,breakable}

\newif\iffirstquestion
\firstquestionfalse

\newtcolorbox{qaexchange}[2][]{
    enhanced,
    unbreakable,
    colback=white,         
    colframe=primaryMain,
    sharp corners,
    boxrule=1pt,
    left=6pt,
    right=6pt,
    top=\dimexpr8pt+3mm,
    bottom=8pt,
    fonttitle=\small\bfseries\sffamily\color{white},
    colbacktitle=primaryMain,   
    attach boxed title to top left={yshift=-3mm,xshift=8mm},
    boxed title style={sharp corners,boxrule=0pt,colback=primaryMain},
    title={\faComments~~Q\&A~\textbar~#2},
    before upper={\stepcounter{qacount}\setlength{\parindent}{1.5em}\setlist[itemize]{topsep=3pt,itemsep=2pt}\setlist[enumerate]{topsep=3pt,itemsep=2pt}\global\firstquestiontrue\color{textMain}},
    before skip=0.8\baselineskip,
    after skip=0.8\baselineskip,
    drop fuzzy shadow={shadow xshift=1pt, shadow yshift=-1pt, opacity=0.2},
    #1
}

\newtcolorbox{emailexchange}[2][]{
    enhanced,
    unbreakable,
    colback=white,   
    colframe=primaryMain,
    sharp corners,
    boxrule=1pt,
    left=6pt,
    right=6pt,
    top=\dimexpr8pt+3mm,
    bottom=8pt,
    fonttitle=\small\bfseries\sffamily\color{white},
    colbacktitle=primaryMain,
    attach boxed title to top left={yshift=-3mm,xshift=8mm},
    boxed title style={sharp corners,boxrule=0pt,colback=primaryMain},
    title={\faEnvelope~~Email~\textbar~#2},
    before upper={\stepcounter{emailcount}\setlength{\parindent}{1.5em}\setlist[itemize]{topsep=3pt,itemsep=2pt}\setlist[enumerate]{topsep=3pt,itemsep=2pt}\color{textMain}},
    before skip=0.8\baselineskip,
    after skip=0.8\baselineskip,
    drop fuzzy shadow={shadow xshift=1pt, shadow yshift=-1pt, opacity=0.2},
    #1
}

\newtcolorbox{standalonequote}[2][]{
    enhanced,
    unbreakable,
    colback=white,
    colframe=primaryMain, 
    sharp corners,
    boxrule=1pt,
    left=6pt,
    right=6pt,
    top=\dimexpr8pt+3mm,
    bottom=8pt,
    fonttitle=\small\bfseries\sffamily\color{white},
    colbacktitle=primaryMain, 
    attach boxed title to top left={yshift=-3mm,xshift=8mm},
    boxed title style={sharp corners,boxrule=0pt,colback=primaryMain},
    title={\faQuoteLeft\ \ Quote \textbar\ #2},
    before upper={\stepcounter{quotecount}\setlength{\parindent}{1.5em}\setlist[itemize]{topsep=3pt,itemsep=2pt}\setlist[enumerate]{topsep=3pt,itemsep=2pt}\color{textMain}},
    before skip=0.8\baselineskip,
    after skip=0.8\baselineskip,
    drop fuzzy shadow={shadow xshift=1pt, shadow yshift=-1pt, opacity=0.2},
    #1
}

\newtcolorbox{warningbox}[2][]{
    enhanced,
    unbreakable,
    colback=white,
    colframe=primaryMain,
    sharp corners,
    boxrule=1.5pt,
    left=8pt,
    right=8pt,
    top=8pt,
    bottom=8pt,
    fonttitle=\small\bfseries\sffamily\color{white},
    colbacktitle=primaryMain,
    attach boxed title to top left={yshift=-3mm,xshift=8mm},
    boxed title style={sharp corners,boxrule=0pt,colback=primaryMain},
    title={\faExclamationTriangle\ #2},
    before upper={\setlength{\parindent}{0em}\color{textMain}},
    before skip=1.5\baselineskip,
    after skip=1.5\baselineskip,
    drop fuzzy shadow={shadow xshift=1pt, shadow yshift=-1pt, opacity=0.2},
    #1
}

\ExplSyntaxOn

\NewDocumentEnvironment{question}{}{
    \par
    \iffirstquestion
        \global\firstquestionfalse
    \else
        \vspace{4pt}
    \fi
    \begin{description}[
        leftmargin=2.5em,
        rightmargin=0.8em,
        labelindent=0.8em,
        labelwidth=1.2em,
        labelsep=0.5em,
        itemindent=0pt,
        font=\normalfont\bfseries,
        topsep=0pt,
        partopsep=0pt,
        parsep=0pt,
        align=right
    ]
    \item[Q:]
}{
    \end{description}
    \vspace{3pt}
}

\NewDocumentEnvironment{note}{}{
    \par
    \iffirstquestion
        \global\firstquestionfalse
    \else
        \vspace{4pt}
    \fi
    \noindent
    \textcolor{noteColor}{\raisebox{-0.35em}{\rule{2.5pt}{1.3em}}}\space\space\color{noteColor}\itshape
}{
    \par
    \vspace{3pt}
}

\NewDocumentEnvironment{answer}{}{
    \par
    \vspace{4pt}
    \begin{description}[
        leftmargin=2.5em,
        rightmargin=0.8em,
        labelindent=0.8em,
        labelwidth=1.2em,
        labelsep=0.5em,
        itemindent=0pt,
        font=\normalfont\bfseries,
        topsep=0pt,
        partopsep=0pt,
        parsep=0pt,
        align=right
    ]
    \item[RP:]
}{
    \end{description}
    \vspace{4pt}
}

\NewDocumentCommand{\rp}{}{%
    \par\vspace{4pt}%
    \begin{qalist}%
    \item[RP:]%
}

\NewDocumentCommand{\qq}{}{%
    \par\vspace{4pt}%
    \begin{qalist}%
    \item[Q:]%
}

\NewDocumentCommand{\closeqa}{}{\end{qalist}\par}

\keys_define:nn { raypeat / metadata }
  {
    source     .tl_set:N   = \l__rp_metadata_source_tl,
    date       .tl_set:N   = \l__rp_metadata_date_tl,
    topic      .tl_set:N   = \l__rp_metadata_topic_tl,
    tags       .tl_set:N   = \l__rp_metadata_tags_tl,
    source     .initial:n  = {},
    date       .initial:n  = {},
    topic      .initial:n  = {},
    tags       .initial:n  = {},
  }

\NewDocumentCommand{\metadata}{m}
  {
    \group_begin:
    \keys_set:nn { raypeat / metadata } { #1 }
    \noindent
    {\small\ttfamily\color{metadataColor}
	\tl_if_empty:NF \l__rp_metadata_topic_tl {Topic:~\l__rp_metadata_topic_tl \quad}
    \tl_if_empty:NF \l__rp_metadata_date_tl {Date:~\l__rp_metadata_date_tl \quad}
    \tl_if_empty:NF \l__rp_metadata_source_tl {Source:~\l__rp_metadata_source_tl \quad}
    \tl_if_empty:NF \l__rp_metadata_tags_tl {Tags:~\l__rp_metadata_tags_tl}
    }
    \par
    \vspace{8pt}
    % Automatic indexing
    \tl_if_empty:NF \l__rp_metadata_topic_tl { \index{\l__rp_metadata_topic_tl} }
    \group_end:
  }

\NewDocumentCommand{\entrynum}{m}{
    \marginpar{\small\color{metadataColor}\bfseries\sffamily\#\,#1}
}

\NewDocumentCommand{\idx}{m}{
    #1\index{#1}
}

\NewDocumentCommand{\topic}{m}{
    \index{#1}
}

\NewDocumentCommand{\source}{m}{
    {\small\itshape\color{metadataColor}#1}
}

\NewDocumentCommand{\extlinkicon}{}{%
    \,{\fontsize{4}{0}\selectfont\raisebox{1.5ex}{\color{metadataColor}\faExternalLink*}}%
}

\NewDocumentCommand{\extlink}{m m}{%
    \href{#1}{\color{externalLinkColor}#2\extlinkicon}%
}

\NewDocumentCommand{\sourcelink}{m m}{%
    \par\vspace{4pt}%
    \noindent\hfill{\small\itshape\color{externalLinkColor}Source:~\extlink{#1}{#2}}%
    \par%
}

\NewDocumentCommand{\sourcelinks}{m}{%
    \par\vspace{4pt}%
    \noindent\hfill{\small\itshape\color{externalLinkColor}Sources:~#1}%
    \par%
}

\NewDocumentCommand{\ornament}{O{rule}}{
  \par\vspace{1em}
  \noindent\hfil
  \str_case:nn {#1} {
    {rule}{{\color{borderColor}\rule{3em}{0.5pt}}}
    {dots}{{\color{borderColor}\kern.2em\raisebox{.4ex}{.}\kern.2em\raisebox{.4ex}{.}\kern.2em\raisebox{.4ex}{.}\kern.2em}}
    {star}{{\color{accentMain}\large\faStar}}
  }
  \hfil
  \par\vspace{1em}
}

\NewDocumentCommand{\sidenote}{m}{
    \marginpar{\footnotesize\raggedright\color{metadataColor}#1}
}

\ExplSyntaxOff

\usepackage{totcount}

\newcounter{qacount}
\newcounter{emailcount}
\newcounter{quotecount}
\newcounter{totalcount}

\regtotcounter{qacount}
\regtotcounter{emailcount}
\regtotcounter{quotecount}
\regtotcounter{totalcount}

\usepackage{enumitem}
\setlist[itemize]{leftmargin=*,topsep=6pt,itemsep=3pt}
\setlist[enumerate]{leftmargin=*,topsep=6pt,itemsep=3pt}
\setlist[description]{leftmargin=0pt,labelindent=0pt,itemsep=6pt}

\newlist{qalist}{description}{1}
\setlist[qalist]{
    leftmargin=2.5em,
    rightmargin=0.8em,
    labelindent=0.8em,
    labelwidth=1.2em,
    labelsep=0.5em,
    itemindent=0pt,
    font=\normalfont\bfseries,
    topsep=0pt,
    partopsep=0pt,
    align=right
}

\usepackage[
    pdfencoding=unicode,
    bookmarks=true,
    colorlinks=true,
    linkcolor=linkColor,
    urlcolor=accentMain,
    citecolor=secondaryMain,
    breaklinks=true,
    unicode=true,
    hypertexnames=false,
    linktoc=all
]{hyperref}

\hypersetup{
    pdftitle={Ephemera: Emails \& Exchanges from the Archives (2012-2024)},
    pdfauthor={Raymond Peat, PhD},
    pdfsubject={Q\&A, Emails, and Quotes from the Ray Peat Forum},
    pdfkeywords={Ray Peat; Bioenergetics; Health; Nutrition; Biology},
    pdflang={en-US}
}

\usepackage[noabbrev,nameinlink]{cleveref}

\begin{document}

\frontmatter

\begin{titlingpage}
    \centering

    \begin{tikzpicture}[remember picture,overlay]
        \draw[primaryDark, line width=1.5pt]
            ([shift={(0.5,0.5)}]current page.south west) rectangle
            ([shift={(-0.5,-0.5)}]current page.north east);
    \end{tikzpicture}

    \vspace*{1.2cm}

    \vspace{0.5cm}
    {%
    \fontsize{72}{0}\selectfont
    EPHEMERA\\[0.5cm]\fontsize{18}{0}\selectfont\textit{or}
    }
    \vspace{0.6cm}

    {\fontsize{18}{0}\selectfont\sffamily\color{primaryMain}Emails \& Exchanges from the Archives\\[0.2cm](2012--2024)\par}
    \vspace{0.8cm}

    {\LARGE\itshape\color{secondaryMain}First Edition\par}
    \vspace{1mm}

    \begin{minipage}{0.75\textwidth}
        \centering
        \color{secondaryLight}
        \sffamily\small
        October, 2025
    \end{minipage}
\end{titlingpage}

\clearpage

\newpage
\thispagestyle{empty}
\null
\clearpage

\tableofcontents*

\newpage
\thispagestyle{empty}
\null
\clearpage

\mainmatter

\chapter{Foundations of Bioenergetic Health}

\section{Metabolism \& Energy Production}

\begin{standalonequote}{Metabolism \& Energy Production}
    \metadata{topic={Bag Breathing Duration}, source={Email Wiki}}

    \begin{answer}
        Just until it's uncomfortable, usually a minute or two, depending on the size of the bag. If you do it a few times in a day, you might notice that it makes your skin (e.g., under nails) pinker, by improving circulation.
    \end{answer}
\end{standalonequote}

\begin{qaexchange}{Metabolism \& Energy Production}
    \metadata{topic={Bag Breathing Frequency}, source={Email Wiki}}

    \begin{question}
        At sea level, roughly how often would someone need to do it (2 mins duration) during a day to maintain a significant, noticeable elevation in \ce{CO2} levels?
    \end{question}

    \begin{answer}
        2 or 3 times a day will usually do it, you can check blood pressure to see its cumulative effect, but you should see a lingering increase of the pinkness of your nail beds.
    \end{answer}
\end{qaexchange}

\begin{qaexchange}{Metabolism \& Energy Production}
    \metadata{topic={Carbonated Drinks And \ce{CO2}}, source={Email Wiki}}

    \begin{question}
        Do carbonated drinks have a significant effect on tissue concentrations of \ce{CO2}?
    \end{question}

    \begin{answer}
        In a crisis situation, it (or baking soda in water) can be helpful, but it's more effective to rebreathe in a paper bag.
    \end{answer}
\end{qaexchange}

\begin{qaexchange}{Metabolism \& Energy Production}
    \metadata{topic={\ce{CO2} And Glycolysis}, source={Email Wiki}}

    \begin{question}
        When you say several effects of \ce{CO2} shuts off glycolysis, do you mean anaerobic glycolysis or all glycolysis, if all glycolysis how does glucose enter mitochondria without breaking down to pyruvate?
    \end{question}

    \begin{answer}
        Meaning the entry of lactate into the blood stream inappropriately, which would usually be called aerobic glycolysis, though you can't be sure how much oxygen is getting to the cells when \ce{CO2} is deficient, since its absence causes many problems in oxygen delivery and use.
    \end{answer}
\end{qaexchange}

\begin{qaexchange}{Metabolism \& Energy Production}
    \metadata{topic={Glycolysis In Oxidative Metabolism}, source={Email Wiki}}

    \begin{question}
        So when \ce{CO2} isn't deficient glycolysis, meaning glucose to pyruvate, is fine?
    \end{question}

    \begin{answer}
        Yes, as part of oxidative metabolism, it's better than burning too much fat.
    \end{answer}
\end{qaexchange}

\begin{standalonequote}{Metabolism \& Energy Production}
    \metadata{topic={Randle Cycle Activation}, source={Email Wiki}}

    \begin{answer}
        It's mostly from large fat meals, at first, but then it is increased by stress, and builds up over time.
    \end{answer}
\end{standalonequote}

\begin{standalonequote}{Metabolism \& Energy Production}
    \metadata{topic={Temperature Monitoring Importance}, source={Email Wiki}}

    \begin{answer}
        The temperature rise during the day is the most important thing, since nocturnal stress hormones can give a misleading impression in the morning. Resting pulse rate is another good indicator. Milk and cheese are the best calcium sources.
    \end{answer}
\end{standalonequote}

\begin{standalonequote}{Metabolism \& Energy Production}
    \metadata{topic={High Metabolism From Low Thyroid}, source={Email Wiki}}

    \begin{answer}
        About your high metabolic rate and high temperature: In my teens and twenties, I needed about 8000 calories per day when I was physically active, about 4000 to 5000 when I was sedentary, but after I took thyroid, I needed only about half as many calories. Thyroid is the basic regulator of blood glucose, and it causes it to be fully oxidized for energy, so that it produces ATP efficiently, on relatively few calories. If blood glucose falls, because it's being used very quickly, the body responds with stress hormones, including glucagon, adrenalin, and cortisol. They cause fat and protein to be burned for energy, while in hypothyroidism, glucose can still be used inefficiently for glycolysis, producing lactic acid, displacing bicarbonate and carbon dioxide. This causes mineral imbalances, with effects including cramps and nerve-muscle tension, which produce heat and waste energy. When you first start taking thyroid again, your tissues will need some extra magnesium, during the time when the dose is increasing, and when the mineral balance is restored your temperature and metabolic rate might decrease a little. Orange juice, milk, and coffee are good for the main minerals, while salting your food to taste.
    \end{answer}
\end{standalonequote}

\begin{standalonequote}{Metabolism \& Energy Production}
    \metadata{topic={Thyroid Lowering Metabolism}, source={Email Wiki}}

    \begin{answer}
        Supplementing thyroid can sometimes reduce the rate of metabolism, by allowing cells to retain enough magnesium, which stabilizes ATP.
    \end{answer}
\end{standalonequote}

\begin{qaexchange}{Metabolism \& Energy Production}
    \metadata{topic={When To Lower Metabolism}, source={Email Wiki}}

    \begin{question}
        Is there ever a time that one would want to decrease the metabolic rate?
    \end{question}

    \begin{answer}
        When marooned without food, waiting for rescue.
    \end{answer}
\end{qaexchange}

\begin{standalonequote}{Metabolism \& Energy Production}
    \metadata{topic={Temperature Monitoring Protocol}, source={Email Wiki}}

    \begin{answer}
        Checking your temperature when you wake up, then about an hour after breakfast, can give you an idea of your thyroid status, it should get up to about 98.5 by mid-morning. With restful sleep, the waking temperature is somewhat low; poor sleep, with high stress hormones, can cause the waking temperature to be high.
    \end{answer}
\end{standalonequote}

\begin{standalonequote}{Metabolism \& Energy Production}
    \metadata{topic={Ketones And Metabolic Stress}, source={Email Wiki}}

    \begin{answer}
        Ketones are very protective as a fuel, but the problem is that they are produced as a result of metabolic stress. If the liver is extremely good, it can store enough glycogen for a day, but chronic, frequent, stress usually damages the liver's ability to store glycogen.
    \end{answer}
\end{standalonequote}

\begin{standalonequote}{Metabolism \& Energy Production}
    \metadata{topic={Nighttime Blood Sugar Management}, source={Email Wiki}}

    \begin{answer}
        Blood sugar falls during the night, causing inflammatory mediators and adrenalin, etc., to increase. Sleeping with a wool cap and stockings can help. Having plenty of sugar just before bedtime, or if you wake during the night, will usually alleviate the night-stress problems.
    \end{answer}
\end{standalonequote}

\begin{standalonequote}{Metabolism \& Energy Production}
    \metadata{topic={Optimal Pulse Rate}, source={Email Wiki}}

    \begin{answer}
        I think 85/minute resting is a good average. For the last 35 years I have tried to keep it averaging a little over 90. When people are using thyroid to recover from tumors or cataracts or other chronic problem, they sometimes hold their resting pulse rate at 100 or more for a few months, without any harmful effects. Sometimes I think you'll be able to figure it out by yourself.
    \end{answer}
\end{standalonequote}

\begin{qaexchange}{Metabolism \& Energy Production}
    \metadata{topic={Pulse Rate From Metabolism}, source={Email Wiki}}

    \begin{question}
        Pulse rate driven by adrenaline?
    \end{question}

    \begin{answer}
        If it's from your basic metabolism, it will stay close to that all day while sitting.
    \end{answer}
\end{qaexchange}

\begin{qaexchange}{Metabolism \& Energy Production}
    \metadata{topic={Pyruvate Context-Dependent}, source={Email Wiki}}

    \begin{question}
        In mitochondria and mortality you mention pyruvate having similar effects of lactate. Isn't pyruvate necessary for oxidative metabolism? Is pyruvate only bad when mitochrondia can't fully metabolize it?
    \end{question}

    \begin{answer}
         Yes.
    \end{answer}
\end{qaexchange}

\begin{standalonequote}{Metabolism \& Energy Production}
    \metadata{topic={Temperature Curve Monitoring}, source={Email Wiki}}

    \begin{answer}
        The temperature rise during the day is the most important thing, since nocturnal stress hormones can give a misleading impression in the morning. Resting pulse rate is another good indicator. Milk and cheese are the best calcium sources.
    \end{answer}
\end{standalonequote}

\begin{standalonequote}{Metabolism \& Energy Production}
    \metadata{topic={Normal Temperature Variation}, source={Email Wiki}}

    \begin{answer}
        It depends on when you wake up, but anything from 0.6 to 1 degree Fahrenheit can be normal.
    \end{answer}
\end{standalonequote}

\begin{standalonequote}{Metabolism \& Energy Production}
    \metadata{topic={Water Retention From Caffeine}, source={Email Wiki}}
    \begin{note}
        Increased Water Retention After Use of T\textsubscript{3}, hHigh Dose Caffeine, or Similar
    \end{note}

    \begin{answer}
        Have you checked your weight before and after those events, and noticed the quanties of urine afterward? It could be that the changes are produced by shifts in circulation and muscle tone. Too much caffeine can cause a surge of adrenaline, which can cause shifts of fluids and tone.
    \end{answer}
\end{standalonequote}

\begin{qaexchange}{Metabolism \& Energy Production}
    \metadata{topic={Uncoupling, Vitamin E}, source={Ray Peat Forum}}

    \begin{question}
        I think when you increase uncoupling of oxidation and phosphorylation you increase fatty acid oxidation in the liver. Do you think there's an increased need for vitamin E in such a scenario? 
    \end{question}

    \begin{answer}
      Uncoupling itself tends to reduce lipid peroxidation.
    \end{answer}
\end{qaexchange}

\begin{qaexchange}{Metabolism \& Energy Production}
    \metadata{topic={\ce{CO2} Retention Duration}, source={Ray Peat Forum}}

    \begin{question}
        If I do certain things to increase my \ce{CO2} [bag breathing, slow breathing, holding my breath] while living at low altitude, how long will that \ce{CO2} be retained in my blood and tissues before I need to do the exercises again?
    \end{question}

    \begin{answer}
      Even if higher \ce{CO2} is continuous, it keeps changing things for months; studies after months in a submarine with high \ce{CO2} showed that their bones were still assimilating it.
    \end{answer}
\end{qaexchange}

\begin{standalonequote}{Metabolism \& Energy Production}
    \metadata{topic={Metabolic Energy, Balance}, source={Ray Peat Forum}}

    \begin{answer}
      I think the single idea I would emphasize most would be that just being balanced requires a fairly high level of metabolic energy, and that good thyroid function and good nutrition are essential for that.
    \end{answer}
\end{standalonequote}

\begin{standalonequote}{Metabolism \& Energy Production}
    \metadata{topic={Temperature Cycle, Stress Hormones}, source={Ray Peat Forum}}

    \begin{answer}
      I think the decrease of temperature after eating breakfast explains more about the stress hormones than saliva tests could. The normal temperature cycle is lowest around 3 or 4 AM, and highest about 12 hours later. When the temperature falls in the morning, it implies that the night was stressful, rather than restorative.
    \end{answer}
\end{standalonequote}

\section{Hormonal Balance}

\begin{standalonequote}{Hormonal Balance}
    \metadata{topic={Cholesterol And Inflammation}, source={Email Wiki}}

    \begin{answer}
        If low cholesterol is combined with slightly low thyroid, the protective steroids aren't produced in normal amounts, and inflammatory processes develop. Connective tissue pain, waist fat, and constipation relate to the stress-inflammation processes, for example endotoxin slows the liver's detoxifying process, estrogen and serotonin signal defensive reactions that lead to cumulative problems. Sweet fruits are anti-inflammatory and help to keep the liver functioning, including keeping cholesterol up and keeping estrogen and cortisol under control. When estrogen is relatively high, tryptophan turns into serotonin and slows the thyroid, lowers the temperature. Glycine is the main anti-inflammatory amino acid, and it can normally be made in adequate amounts, but some proteins, especially muscle meats, don't have enough Glycine in relation to tryptophan. Fruits and milk or cheese will usually provide a good balance of the main nutrients, but sometimes gelatin is very useful to balance the other proteins. The calcium content of milk and cheese is important for lowering inflammation, and helps to prevent excess fat deposition. Sodium and vitamin K are closely involved in calcium metabolism.
    \end{answer}
\end{standalonequote}

\begin{standalonequote}{Hormonal Balance}
    \metadata{topic={Monitoring Stress Hormones}, source={Email Wiki}}

    \begin{answer}
        Temperature and pulse rate are important to watch; when the stress hormones are lower, they will increase after breakfast, and stay up until late afternoon.
    \end{answer}
\end{standalonequote}

\begin{standalonequote}{Hormonal Balance}
    \metadata{topic={Leptin Resistance Concept}, source={Email Wiki}}

    \begin{answer}
        I doubt that there is any biological significance in the idea of leptin resistance. Leptin promotes inflammation and cancer, so it might be good to be resistant to it, but I think the concept is mainly an outgrowth of the pharmaceutical industry's promotion of leptin as a cure for obesity.
    \end{answer}
\end{standalonequote}

\begin{standalonequote}{Hormonal Balance}
    \metadata{topic={Nighttime Hormone Management}, source={Email Wiki}}

    \begin{answer}
        During the night all of the hormones of stress and inflammation rise, and the ice cream decreases them enough for you to stay asleep, but they still rise. Having more very bright light (several hundred watts of incandescent bulbs) in the hours from sundown until bedtime will lower them a little more. Since T\textsubscript{3} is used up very quickly, allowing the proinflammatory TSH to rise during the night, it would help if you used Cynoplus at bedtime, instead of Cynomel. If you were taking 10 mcg of cynomel, then a third of a tablet of cynoplus would provide that, as well as the T\textsubscript{4} that holds the TSH down longer. Having an egg every day, and liver once a week, will help to balance the effects of the thyroid hormone, which increases your need for vitamins, especially vitamin A.
    \end{answer}
\end{standalonequote}

\begin{qaexchange}{Hormonal Balance}
    \metadata{topic={Temperature Cycle And Stress}, source={Email Wiki}}

    \begin{question}
        High stress hormones on thyroid increases sensitivity to them?
    \end{question}

    \begin{answer}
        Not necessarily, but it's something to watch for. The daily temperature cycle is helpful, if stress is low, there will be a strong cycle, lowest at night, early morning.
    \end{answer}
\end{qaexchange}

\begin{standalonequote}{Hormonal Balance}
    \metadata{topic={Nutrient-Hormone Interactions}, source={Ray Peat Forum}}

    \begin{answer}
      Diet and hormones interact; magnesium can't be taken up by nerves and muscles if vitamin D and thyroid T\textsubscript{3} are deficient; T\textsubscript{3} can't be produced if selenium and sugar are inadequate. Pregnenolone and progesterone support the T\textsubscript{3} effect. Cyproheptadine blocks the mediators of stress that result from the hormonal-nutritional deficiencies. Too much phosphate relative to calcium and magnesium creates tension.
    \end{answer}
\end{standalonequote}

\section{Nutrition Principles}

\begin{standalonequote}{Nutrition Principles}
    \metadata{topic={Raw Food Diet Critique}, source={Email Wiki}}

    \begin{answer}
        Basically wrong, but there are some areas that would be worth investigating, such as the different physiological effects of raw onions and cooked onions. It's possible that the enzymes inhibit some toxic effects of the irritating chemicals in onions.
    \end{answer}
\end{standalonequote}

\begin{standalonequote}{Nutrition Principles}
    \metadata{topic={Fruit Diet Protein Needs}, source={Email Wiki}}

    \begin{answer}
        For best resistance to stress, more protein is desirable.
    \end{answer}
\end{standalonequote}

\begin{standalonequote}{Nutrition Principles}
    \metadata{topic={Fruit-Only Diet Nutrients}, source={Ray Peat Forum}}

    \begin{answer}
      Intestinal bacteria are an important source of B\textsubscript{12}, and many plant materials contain some. Carotene can be converted to vitamin A when B\textsubscript{12} is available. I don't think the real causes of a B\textsubscript{12} deficiency are known. A generous dietary source of both A and B\textsubscript{12} is desirable, but usually not essential. I think the main reason for having a significant amount of fat in the diet is for its effect on digestion.
    \end{answer}
\end{standalonequote}

\begin{standalonequote}{Nutrition Principles}
    \metadata{topic={Fruit Preferable to Starch}, source={Ray Peat Forum}}

    \begin{answer}
      When a non-starchy fruit is available I think it's always preferable to starch. Alkali-processed corn is the only kind that I'm willing to eat, and seldom that (e.g., corundas made with wood ashes).
    \end{answer}
\end{standalonequote}

\begin{standalonequote}{Nutrition Principles}
    \metadata{topic={Foraging Cultures, Calcium/Phosphate}, source={Ray Peat Forum}}

    \begin{answer}
      Wild foraging and gathering cultures usually got lots of vegetable calcium relative to phosphate, better than agricultural societies.
    \end{answer}
\end{standalonequote}

\begin{qaexchange}{Nutrition Principles}
    \metadata{topic={Mongolian Diet, Dairy-Based}, source={Ray Peat Forum}}

    \begin{question}
        What do you think of the traditional Mongolian diet? It is heavily based on dairy and often includes animal organs. Vegetables are very limited. 
    \end{question}

    \begin{answer}
      Milk compensates for the lack of vegetables.
    \end{answer}
\end{qaexchange}

\begin{standalonequote}{Nutrition Principles}
    \metadata{topic={Diet Evolution, Brain Development}, source={Ray Peat Forum}}

    \begin{answer}
      Diet choices go with evolution—like the squirrel monkeys' fruit diet, foods that support brain development have to be high in carbohydrate, easy to digest, abundant in the environment, and adequate in overall nutrient content and balance. A less concentrated diet, containing a lot of plant material, causes adaptive enlargement of the intestine.
    \end{answer}
\end{standalonequote}

\chapter{Dietary Guidelines}

\section{Macronutrients}

\begin{standalonequote}{Macronutrients}
    \metadata{topic={Low Carb Protein Waste}, source={Email Wiki}}

    \begin{answer}
        People can do well on high or low fat or carbohydrate, but when the carbohydrate is very low, some of the protein will be wasted as fuel, replacing the missing glucose.
    \end{answer}
\end{standalonequote}

\begin{qaexchange}{Macronutrients}
    \metadata{topic={Sugar vs Saturated Fat as Fuel}, source={Ray Peat Forum}}

    \begin{question}
        Is the main reason you think sugar is better than saturated fat as a fuel source because sugar increases \ce{CO2} more? Or are there other reasons? I was wondering if it would be safe to get a decent amount of calories from hydrogenated coconut oil as apposed to pure sucrose?
    \end{question}

    \begin{answer}
      I use hydrogenated coconut oil and sugar, and I think the combination is good for most people; in situations such as heart failure, more sugar would be preferable.
    \end{answer}
\end{qaexchange}

\subsection{Proteins}

\begin{standalonequote}{Proteins}
    \metadata{topic={Gelatin Purity}, source={Email Wiki}}

    \begin{answer}
        There isn't any MSG in gelatin, but the purity of the product is important. It's best when you extract it yourself, in things like ox-tail soup. Sugar helps the thyroid function, so can improve your blood sugar stability. Hypothyroid people are sensitive to even small amounts of lactic acid, since it tends to deplete the liver's glycogen stores. Squid amino acids are similar to other muscles, but the trace minerals are helpful.
    \end{answer}
\end{standalonequote}

\begin{standalonequote}{Proteins}
    \metadata{topic={Diet For Muscle Building}, source={Email Wiki}}

    \begin{answer}
        A high protein diet is helpful, and avoiding polyunsaturated fats helps to increase testosterone (coconut oil, butter, maybe MCT instead). Excess tryptophan can promote the catabolic cortisol, so supplementing gelatin might be helpful.
    \end{answer}
\end{standalonequote}

\begin{standalonequote}{Proteins}
    \metadata{topic={Gelatin Rich Meats}, source={Email Wiki}}

    \begin{answer}
        Lamb shanks, pigs' feet, various joint bones, and boiled chicken, if the fat is skimmed off.
    \end{answer}
\end{standalonequote}

\begin{standalonequote}{Proteins}
    \metadata{topic={Gelatin Fluoride Content}, source={Email Wiki}}

    \begin{answer}
        The only analysis of gelatin that I have seen showed very little fluoride. Since most of the fluoride in an animal is concentrated in the bones, and gelatin is made from skin, it probably doesn't contain much.
    \end{answer}
\end{standalonequote}

\begin{standalonequote}{Proteins}
    \metadata{topic={Gelatin Preparation}, source={Email Wiki}}

    \begin{answer}
        For most people it's o.k. to eat it undissolved, but it causes gas for some people. I usually cook the ox-tail with just enough water to cover it, for about four hours, until the meat comes off the bone easily. It makes a very concentrated gelatin solution.
    \end{answer}
\end{standalonequote}

\begin{standalonequote}{Proteins}
    \metadata{topic={Gelatin Brands}, source={Email Wiki}}

    \begin{answer}
        I haven't had experience with Knox gelatin, I have mostly used either Great Lakes or Gelatin Innovations brands, which are economical by the pound. When I'm not sure of the origin of the pork, I heat the pork rinds in coconut oil and then drain the oil off, to reduce the PUFA.
    \end{answer}
\end{standalonequote}

\begin{standalonequote}{Proteins}
    \metadata{topic={Beef vs Pork Gelatin}, source={Email Wiki}}

    \begin{answer}
        Beef gelatin is available from Great Lakes; some people say they have trouble digesting the pork gelatin. Because the pork rinds contain a lot of fat, which in pork is highly unsaturated, I re-heat mine in coconut oil and then drain them well, to reduce the amount of polyunsaturated fat. Since the polyunsaturated fats interfere with energy production and promote inflammation, I think it's better to avoid them. (They interfere with thyroid and progesterone, and activate estrogen production.)
    \end{answer}
\end{standalonequote}

\begin{standalonequote}{Proteins}
    \metadata{topic={Gelatin Sources}, source={Email Wiki}}

    \begin{answer}
        Mostly I get gelatin from things like ox-tail soup, but I use a little gelatin from Great Lakes Gelatin.
    \end{answer}
\end{standalonequote}

\begin{standalonequote}{Proteins}
    \metadata{topic={Protein For Athletes}, source={Email Wiki}}

    \begin{answer}
        For intense exercise, it's about a gram per pound of body weight.
    \end{answer}
\end{standalonequote}

\begin{standalonequote}{Proteins}
    \metadata{topic={Protein Whole Animal Eating}, source={Email Wiki}}

    \begin{answer}
        I've always been very sedentary, but I have usually had close to 150 grams daily. The traditional meat eaters didn't waste anything,ate all the skin, ears, tails, snouts, feet,tendons, lungs, intestines, marrow, blood,brains, gonads and other glands, picked the ligaments off the bones, so they had a much better balance of amino acids. (Small town restaurants in Mexico, China, etc., still serve those.) Muscle meats are essentially a refined food.
    \end{answer}
\end{standalonequote}

\begin{standalonequote}{Proteins}
    \metadata{topic={Excess Protein With Hypothyroidism}, source={Email Wiki}}

    \begin{answer}
        That's more than enough, and with low thyroid function the excess of tryptophan, methionine, and cystein can lower your thyroid even more. Until your metabolic rate is higher, 80 to 100 grams would be better. Replacing it with sugar, or very well cooked starch, would support thyroid function.
    \end{answer}
\end{standalonequote}

\begin{standalonequote}{Proteins}
    \metadata{topic={Protein Timing}, source={Email Wiki}}

    \begin{answer}
        It's better to take your protein during the day, sugar and fat in the evening. The powdered protein lacks most of the nutrients, so you probably need some fruit, eggs, and liver, for the other nutrients, including potassium and magnesium.
    \end{answer}
\end{standalonequote}

\begin{standalonequote}{Proteins}
    \metadata{topic={Protein Cortisol Response}, source={Email Wiki}}

    \begin{answer}
        Food proteins stimulate insulin secretion, and to prevent hypoglycemia cortisol is increased. The food proteins (along with tissue proteins) can be used for energy under the influence of cortisol. Meats, other than beef, lamb, venison, and bison, usually contain enough polyunsaturated fat to affect estrogen, testosterone, and energy production. Stress, or increased cortisol, increases the circulating cysteine and tryptophan from muscle (meats), and these together with cortisol tend to increase aromatase. The high ratio of phosphate to calcium in meat activates a variety of stress processes; a high intake of calcium supports energy metabolism. Sugars tend to lower circulating free fatty acids, amino acids, and cortisol, while activating the thyroid hormone.
    \end{answer}
\end{standalonequote}

\begin{standalonequote}{Proteins}
    \metadata{topic={Tryptophan Normalization Time}, source={Email Wiki}}

    \begin{answer}
        I think it just takes a few hours, or a day, to normalize the tryptophan. Vitamin B\textsubscript{6} helps to guide the metabolism of tryptophan away from excessive serotonin.
    \end{answer}
\end{standalonequote}

\begin{qaexchange}{Proteins}
    \metadata{topic={Protein Deficiency Liver Effects}, source={Email Wiki}}

    \begin{question}
        Does a protein deficiency lower liver detoxification due to increased muscle breakdown which inhibits thyroid via amino acids which lowers metabolic efficiency of liver?
    \end{question}

    \begin{answer}
        Yes, I think that's at least part of how it works.
    \end{answer}
\end{qaexchange}

\begin{qaexchange}{Proteins}
    \metadata{topic={Whey Protein Powder Problems}, source={Email Wiki}}

    \begin{question}
        Whey?
    \end{question}

    \begin{answer}
        Powdered foods that contain tryptophan are extremely susceptible to harmful oxidation, and the best things are removed, for example calcium, lactose, and casein, with its anti-stress properties.
    \end{answer}
\end{qaexchange}

\begin{standalonequote}{Proteins}
    \metadata{topic={Gelatin Sources, Ox-Tails, Chicken}, source={Ray Peat Forum}}

    \begin{answer}
      I occasionally use some powdered gelatin for things like making marshmallows, but usually I get my gelatin from soup, such as ox-tails. lamb shanks, or chicken backs and wings.
    \end{answer}
\end{standalonequote}

\begin{qaexchange}{Proteins}
    \metadata{topic={Taurine vs Glycine}, source={Ray Peat Forum}}

    \begin{question}
        If I react badly to glycine and gelatin, but I can tolerate taurine just fine, can the taurine be taken with muscle meat to reduce the muscle meat anti-thyroid effects instead of glycine/gelatin?
    \end{question}

    \begin{answer}
      I don't think it will have a similar effect, but it might provide its own unique benefit.
    \end{answer}
\end{qaexchange}

\begin{standalonequote}{Proteins}
    \metadata{topic={Gelatin Supplementation}, source={Ray Peat Forum}}

    \begin{answer}
      Some people feel better with a tablespoon or two of gelatin with their regular foods.
    \end{answer}
\end{standalonequote}

\begin{qaexchange}{Proteins}
    \metadata{topic={Synthesizing Amino Acids}, source={Ray Peat Forum}}

    \begin{question}
        Do you think the body can still produce enough of the so called \enquote{amino acids} carnosine, carnitine, creatine and taurine, none of which are made by any plant food, when a person does well eating a potato and mushroom based diet that leaves out any muscle meats and other foods that have those 4 preformed amino acids?
    \end{question}

    \begin{answer}
      I think so.
    \end{answer}
\end{qaexchange}

\begin{qaexchange}{Proteins}
    \metadata{topic={Tryptophan Intake Limits}, source={Ray Peat Forum}}

    \begin{question}
        When it comes to total intake from food, what do you think is the upper safe limit on total tryptophan intake for an active male? Or is it more about the balance between it and glycine?
    \end{question}

    \begin{answer}
      I think it's a matter of the balance with other nutrients, keeping all the inflammatory things low, and the metabolic rate high. The calcium of milk is protective.
    \end{answer}
\end{qaexchange}

\begin{qaexchange}{Proteins}
    \metadata{topic={Protein Assimilation, Thyroid}, source={Ray Peat Forum}}

    \begin{question}
        For a woman with difficulty ingesting enough protein, do you think calcium glucarate, or glucaric acid would be helpful?
    \end{question}

    \begin{answer}
      If protein intake is really inadequate, it probably wouldn't be helpful. Increased thyroid can improve protein assimilation.
    \end{answer}
\end{qaexchange}

\begin{standalonequote}{Proteins}
    \metadata{topic={Plant Protein, Glycine}, source={Ray Peat Forum}}

    \begin{answer}
       Plant proteins don't have the dangerous overload of tryptophan and Cysteine that occur in muscle meats, whey, and egg white, so increasing the Glycine isn't so important for vegetarians. Just getting enough protein is the hardest thing, and cooked potato juice is the best one known. The amino acid balance in most fruits seems good. Leaf protein is as good as milk protein, but it's hard to separate it from the antinutrients such as tannins.

      It's very slow to look up the literature on amino acids in the various plant species and tissues, and people haven't been very concerned with the amount of glycine in foods.
    \end{answer}
\end{standalonequote}

\subsection{Carbohydrates}

\begin{standalonequote}{Carbohydrates}
    \metadata{topic={Sugar vs Starch For Hormones}, source={Email Wiki}}

    \begin{answer}
        Since cholesterol is the source of progesterone and testosterone (and pregnenolone, DHEA, etc.), and sugar increases it, having fruit rather than starch might increase the hormones. Those hormones, antagonistic to cortisol, can help to reduce waist fat. Chard, collard, and kale are good greens.
    \end{answer}
\end{standalonequote}

\begin{qaexchange}{Carbohydrates}
    \metadata{topic={Starch Safety Thresholds}, source={Email Wiki}}

    \begin{question}
        How much starchy food is safe?
    \end{question}

    \begin{answer}
        There isn't enough information to judge, but a fair part of the carbohydrate should be in the form of sucrose, fructose, and/or lactose. If it's well cooked, and eaten with butter, it's probably safe for many people.
    \end{answer}
\end{qaexchange}

\begin{standalonequote}{Carbohydrates}
    \metadata{topic={Well-Cooked Starch Safety}, source={Email Wiki}}

    \begin{answer}
        When starch is well cooked, and eaten with some fat and the essential nutrients, it's safe, except that it's more likely than sugar to produce fat, and isn't as effective for mineral balance.
    \end{answer}
\end{standalonequote}

\begin{standalonequote}{Carbohydrates}
    \metadata{topic={Starch Fattening Effects}, source={Email Wiki}}

    \begin{answer}
        Starch is less harmful when eaten with saturated fat, but it's still more fattening than sugars.
    \end{answer}
\end{standalonequote}

\begin{standalonequote}{Carbohydrates}
    \metadata{topic={High Sugar Consumption}, source={Email Wiki}}

    \begin{answer}
        I have often had a gallon of orange juice in a day, with 100 grams of other sugar, and didn't see any problem, even while being sedentary. If your metabolic rate is high, with a pound of sugar you will still have an appetite for quite a bit of fat and protein.
    \end{answer}
\end{standalonequote}

\begin{standalonequote}{Carbohydrates}
    \metadata{topic={Carbohydrate Requirements}, source={Email Wiki}}

    \begin{answer}
        That depends on your size, metabolic rate, and activity, and the other nutrients, but I sometimes have more than that [400 g of carbohydrates], including the sugar in milk and orange juice (and I'm about your size, and very sedentary). The fructose component of ordinary sugar (sucrose) helps to increase the metabolic rate. I think a person of average size should have at least 180 grams per day, maybe an average of about 250 grams.
    \end{answer}
\end{standalonequote}

\begin{standalonequote}{Carbohydrates}
    \metadata{topic={Sugar Appetite Regulation}, source={Email Wiki}}

    \begin{answer}
        Appetite should be the basic guide. When your liver has enough glycogen stored, sweet things aren't appetizing.
    \end{answer}
\end{standalonequote}

\begin{standalonequote}{Carbohydrates}
    \metadata{topic={Sugar Intake Adequacy}, source={Email Wiki}}

    \begin{answer}
        If your temperature increases quickly after eating, that's good. I often eat a kilogram or more of oranges in a day, 150 grams of sugar per day wouldn't be excessive.
    \end{answer}
\end{standalonequote}

\begin{standalonequote}{Carbohydrates}
    \metadata{topic={Fructose Increasing Glycogen}, source={Email Wiki}}

    \begin{answer}
        I think that's one of its basic protective effects, and I think it increases it in the brain, too.
    \end{answer}
\end{standalonequote}

\begin{standalonequote}{Carbohydrates}
    \metadata{topic={Fructose Malabsorption}, source={Email Wiki}}

    \begin{answer}
        The fructose cult generalizes crazily from any apparent evidence they can find.
    \end{answer}
\end{standalonequote}

\begin{qaexchange}{Carbohydrates}
    \metadata{topic={Sugar With High Metabolism}, source={Email Wiki}}

    \begin{question}
        White sugar bad idea when metabolism is high?
    \end{question}

    \begin{answer}
        But sometimes it can lower the stress hormones, so it requires experimenting
    \end{answer}
\end{qaexchange}

\begin{qaexchange}{Carbohydrates}
    \metadata{topic={Molasses Degraded Sugars}, source={Email Wiki}}

    \begin{question}
        Blackstrap molasses?
    \end{question}

    \begin{answer}
        Although it's extremely rich in minerals, I think the intense heat used for concentrating it degrades the sugar into things that are likely to be allergens.
    \end{answer}
\end{qaexchange}

\begin{qaexchange}{Carbohydrates}
    \metadata{topic={Starch Safety Guidelines}, source={Email Wiki}}

    \begin{question}
        How much starch is ok?
    \end{question}

    \begin{answer}
        When starch is well cooked, and eaten with some fat and the essential nutrients, it's safe, except that it's more likely than sugar to produce fat, and isn't as effective for mineral balance.
    \end{answer}
\end{qaexchange}

\begin{qaexchange}{Carbohydrates}
    \metadata{topic={Starchy Food Balance}, source={Email Wiki}}

    \begin{question}
        How much cooked starchy food is safe in the diet?
    \end{question}

    \begin{answer}
        There isn't enough information to judge, but a fair part of the carbohydrate should be in the form of sucrose, fructose, and/or lactose. If it's well cooked, and eaten with butter, it's probably safe for many people.
    \end{answer}
\end{qaexchange}

\begin{standalonequote}{Carbohydrates}
    \metadata{topic={Sugar Energy Stabilization}, source={Email Wiki}}

    \begin{answer}
        If the rest of your diet is good, the energy bursts from sugar should level off, and become a steady increased metabolic rate.
    \end{answer}
\end{standalonequote}

\begin{standalonequote}{Carbohydrates}
    \metadata{topic={Sugar Craving Causes}, source={Email Wiki}}

    \begin{answer}
        There is a great anti-sugar cult, with even moralistic overtones, equating sugar craving with morphine addiction. Sugar craving is usually caused by the need for sugar, generally caused by hypothyroidism.When yeasts have enough sugar, they just happily make ethanol, but when they don't have sugar, they can sink filaments into the intestine wall seeking it, and, if the person is very weak, they can even invade the bloodstream and other organs. Milk, cheese, and fruits provide a very good balance of nutrients. Fruits provide a significant amount of protein. Plain sugar is o.k. when the other nutrients are adequate. Roots, shoots, and tubers are, next to the fruits, a good carbohydrate source; potatoes are a source of good protein. Meat as the main protein can provide too much phosphorus in relation to calcium.
    \end{answer}
\end{standalonequote}

\begin{standalonequote}{Carbohydrates}
    \metadata{topic={Lustig (\textit{Sugar the Bitter Truth}) Critique}, source={Email Wiki}}

    \begin{answer}
        I checked a few of the references that were on the charts in his video, and didn't find any facts that would necessarily support what he was saying. Many people are publishing similar extreme interpretations. His ideas about alcohol, appetite, addiction, insulin, and leptin are stereotyped medical cliches, that aren't supported clearly by good evidence.
    \end{answer}
\end{standalonequote}

\begin{standalonequote}{Carbohydrates}
    \metadata{topic={Sugar Maximum Intake}, source={Email Wiki}}

    \begin{answer}
        I think a total for sugar up to ten ounces can be o.k., depending on your metabolic rate and needs. Budd and Piorry used up to 12 ounces per day therapeutically.
    \end{answer}
\end{standalonequote}

\begin{standalonequote}{Carbohydrates}
    \metadata{topic={Sugar Therapeutic Use}, source={Email Wiki}}

    \begin{answer}
        A daily diet that includes two quarts of milk and a quart of orange juice provides enough fructose and other sugars for general resistance to stress, but larger amounts of fruit juice, honey, or other sugars can protect against increased stress, and can reverse some of the established degenerative conditions. Refined granulated sugar is extremely pure, but it lacks all of the essential nutrients, so it should be considered as a temporary therapeutic material, or as an occasional substitute when good fruit isn't available, or when available honey is allergenic.
    \end{answer}
\end{standalonequote}

\begin{standalonequote}{Carbohydrates}
    \metadata{topic={Plain Sugar Safety}, source={Email Wiki}}

    \begin{answer}
        If your other foods are rich in vitamins and minerals it's safe.
    \end{answer}
\end{standalonequote}

\begin{standalonequote}{Carbohydrates}
    \metadata{topic={Sugar Appetite As Guide}, source={Email Wiki}}

    \begin{answer}
        To prevent stress, or to replenish glycogen stores after stress, your appetite for it is likely to be a good guide.
    \end{answer}
\end{standalonequote}

\begin{standalonequote}{Carbohydrates}
    \metadata{topic={Honey Allergenicity}, source={Email Wiki}}

    \begin{answer}
        I haven't had any experience with manuka. Some honey can be allergenic, so it's good to look for a mild one; white sugar is probably similar, with less allergy risk.
    \end{answer}
\end{standalonequote}

\begin{standalonequote}{Carbohydrates}
    \metadata{topic={Brown Sugar Toxicity}, source={Email Wiki}}

    \begin{answer}
        No, although there are some nutrient minerals in it, the impurities can be slightly toxic and allergenic.
    \end{answer}
\end{standalonequote}

\begin{standalonequote}{Carbohydrates}
    \metadata{topic={Coconut Sugar}, source={Email Wiki}}

    \begin{answer}
        If it's browned from heating, it's more likely to be allergenic, and even without too much heat, some people are likely to be allergic to it. But if it doesn't cause any reactions, then it's very good, with some nutritional value. Honey is in some ways better than white sugar, but depending on the plants it's derived from, it can be allergenic. White sugar has the advantage of being very clean. Fruits have many valuable nutrients, so are the best way to get sugars, when good ones are available.
    \end{answer}
\end{standalonequote}

\begin{standalonequote}{Carbohydrates}
    \metadata{topic={Sugar Less Fattening Than Starch}, source={Email Wiki}}

    \begin{answer}
        Per calorie, sugar is less fattening than starch, partly because it stimulates less insulin, and, when it's used with a good diet, because it increases the activity of thyroid hormone. There are several convenient indicators of the metabolic rate--the daily temperature cycle and pulse rate (the temperature should rise after breakfast), the amount of water lost by evaporation, and the speed of relaxation of muscles (Achilles reflex relaxation).
    \end{answer}
\end{standalonequote}

\begin{standalonequote}{Carbohydrates}
    \metadata{topic={Sugar Increasing Calorie Burn}, source={Email Wiki}}

    \begin{answer}
        People on a standard diet will typically burn 200 or 300 more calories per day when that amount of sugar is added to their diet; but if extra fat is added, too, some of the extra calories are likely to be deposited as fat. It's important to watch the signs of changing heat production as the diet changes.
    \end{answer}
\end{standalonequote}

\begin{qaexchange}{Carbohydrates}
    \metadata{topic={Rice Preparation}, source={Ray Peat Forum}}

    \begin{question}
        If we wanted to use rice to gain weight, how could we prepare rice?
    \end{question}

    \begin{answer}
        If it hasn't been either roasted or irradiated, if it's still alive, you can just soak it for a few hours, it gets softer and less starchy and more nutritious, the enzymes that are breaking down the toxic stuff, are releasing nutrients but also synthetizing new forms of proteins, the storage forms of proteins are almost always irritating, or allergenic, or toxic, but as the enzymes process them, they turn into living enzymes and structural proteins just like the plants.
    \end{answer}
\end{qaexchange}

\begin{qaexchange}{Carbohydrates}
    \metadata{topic={Starch and Endotoxin}, source={Ray Peat Forum}}

    \begin{question}
        Let's say I want to eat starch (rice, potatoes and even wheat), what can I do at the same time or right after to reduce the effect of endotoxin? I think vitamin D and vitamin B\textsubscript{2} help along with coconut oil, what do you think?
    \end{question}

    \begin{answer}
        Keeping a fairly quick transit time usually goes with an abundance of digestive secretions, keeping the small intestine free of bacteria. Fiber, good thyroid function, and antiseptic foods such as cooked mushrooms, bamboo shoots, and raw carrots help.
    \end{answer}
\end{qaexchange}

\begin{standalonequote}{Carbohydrates}
    \metadata{topic={Starch with Butter}, source={Ray Peat Forum}}

    \begin{answer}
        Butter is protective with starch, and it has a mild antiseptic effect; I often use a little penicillin when I think a food has disturbed my intestinal flora. Different types of antibiotic can be combined or alternated, and very small doses can be helpful for digestive problems. The types I have used most often are penicillin VK, tetracycline, and erythromycin.
    \end{answer}
\end{standalonequote}

\begin{qaexchange}{Carbohydrates}
    \metadata{topic={Hypoglycemia and Blood Sugar}, source={Ray Peat Forum}}

    \begin{question}
        In one of Adele Davis' books she comments that hunger does not occur until blood sugar drops to 70 or below and that most Americans have blood sugar levels of 90 or above twelve hours after eating their evening meal; therefore, do not experience true hunger in the mornings. How does this relate to hypoglycemia in the context of trying to \enquote{keep it at a good level} by frequent food intake. Does this mean keeping it high in regards to what ones normal fasting level is (perhaps 90 or above), or high in regards to being above 70 (indicating hunger)? If one never feels hungry could this be because blood sugar is always 15 to 20 points above 70? What is happening when the cold nose of adrenalin surge occurs with a blood sugar level of 90 or above?
    \end{question}

    \begin{answer}
        A really healthy person can go all day on stored glycogen, but most people deplete the glycogen quickly, then produce cortisol to convert tissue protein to glucose, and while doing that, shift to using more fat. The fat blocks the use of glucose, so it can raise blood glucose. In the 1940s and '50s the average american had a different metabolism. With fatty acids interfering with glucose oxidation, creating \enquote{insulin resistance,} adrenaline is secreted even when the blood glucose level seems good.
    \end{answer}
\end{qaexchange}

\begin{emailexchange}{Carbohydrates}
    \metadata{topic={Fructose and Liver Health}, source={Ray Peat Forum}}

    \begin{question}
        There seems to be a startling increase in the number of cases of \enquote{non alcoholic fatty liver syndrome,} and I wonder if a corresponding increase in fructose consumption could at least partially account for it?
    \end{question}

    \begin{answer}
      Sucrose is what I recommend, because starch is pure glucose, and usually doesn't have the appropriate minerals associated with it. Fructose just doesn't occur naturally in practical amounts; well aged jerusalem artichokes are the only food I know of that contains it without balancing glucose, and it's rare to find them properly aged.
    \end{answer}
	
    \begin{question}
        Historically, it looks like fruit was available only seasonally (even in the tropics). I've seen it suggested that it's healthier to binge on fruit at infrequent intervals than to consume it every day. Also, from what I can tell, plenty of the groups studied by Weston A. Price consumed a lot of glucose, but none consumed a lot of fructose.
    \end{question}

    \begin{answer}
      With a good liver, a person can store a lot of glycogen, but I don't know of any advantage in having sugar intermittenly. I suppose there are some places in the tropics without much variety of fruit, but many places have fruits in all seasons. Cactus fruit in desert areas, guavas in slightly moister areas, for example. At my place in Michoacan even the apples and peaches make two crops a year, and various berries, custard apples, citrus fruits are never far.
    \end{answer}
	
    \begin{question}
        If we can't consume more than 35\% of our total calories as protein, then the remaining 65\% needs to be composed of sugar and fat. Should we be replacing starch with fruit, or starch with saturated fat?
    \end{question}

    \begin{answer}
      A little saturated fat, especially with the shorter chains, is helpful. I don't think any amount of starch is beneficial, and without fat undercooked starch can be persorbed.
    \end{answer}
	
    \begin{question}
        Could it be that fructose only damages the liver in the presence of PUFA?
    \end{question}

    \begin{answer}
      PUFA damage the liver.
    \end{answer}
\end{emailexchange}

\begin{standalonequote}{Carbohydrates}
    \metadata{topic={Sugar Bleaching Process}, source={Ray Peat Forum}}

    \begin{answer}
      The bleaching is essentially washing, washing the molasses residue away from brown sugar, and I think charcoal is used to absorb the last traces of impurities. The products vary in the thoroughness of the washing, some have a faint yellow color and weak molasses taste; the whitest is least likely to be allergenic.
    \end{answer}
\end{standalonequote}

\begin{standalonequote}{Carbohydrates}
    \metadata{topic={Deuterium in Beet Sugar}, source={Ray Peat Forum}}

    \begin{answer}
      I think beet sugar from areas east of the Rockies is likely to be low deuterium, since more will be rained out enroute from the ocean. I think a similar process of reducing deuterium is likely to be involved in producing fruit and milk, since the aging organism accumulates deuterium; young tissues are better than old tissues. I think LG Boros talks nonsense.
    \end{answer}
\end{standalonequote}

\begin{standalonequote}{Carbohydrates}
    \metadata{topic={Sugar Intake, Daily Diet}, source={Ray Peat Forum}}

    \begin{answer}
      It usually takes me just a few days of eating extremely large amounts of sugar (milkshakes, marshmallows, ice cream, etc.) to replenish my stores. Sugar stops tasting especially good when I have had enough, and my pulse pressure, the difference between top and bottom blood pressure numbers, falls well below 50 points. Normally, I usually have around 400 grams of carbohydrate. I have about 3 quarts of milk, varying amounts of orange juice (probably over a quart on average), eggs, and about 200 grams of meat or fish, with other things such as coca cola, cheese, ice cream, cheese cakes, some coconut oil and butter, occasional tropical fruits.
    \end{answer}
\end{standalonequote}

\subsection{Fats \& Oils}

\begin{standalonequote}{Fats \& Oils}
    \metadata{topic={Coconut Meat}, source={Email Wiki}}

    \begin{answer}
        It often causes gas and irritation symptoms.
    \end{answer}
\end{standalonequote}

\begin{standalonequote}{Fats \& Oils}
    \metadata{topic={Coconut Fiber Safety}, source={Email Wiki}}

    \begin{answer}
        Do you know how the fiber is manufactured, and from what? Fibrous foods can lower both absorbed cholesterol and estrogen, but some fibers are broken down by bacteria to produce estrogenic materials. The husk fiber, coir, is being sold as a food additive. I don't know whether coir has been tested for the release of lignans, which could be carcinogenic. If it's just dried coconut meat, the problem would probably just be gas from the starches.
    \end{answer}
\end{standalonequote}

\begin{standalonequote}{Fats \& Oils}
    \metadata{topic={Coconut Meat Gas Problem}, source={Email Wiki}}

    \begin{answer}
        I think gas is the only problem from the mature meat.
    \end{answer}
\end{standalonequote}

\begin{standalonequote}{Fats \& Oils}
    \metadata{topic={Coconut Oil Allergens}, source={Email Wiki}}

    \begin{answer}
        If you are using coconut oil regularly, that's a possible source of allergens, if it isn't well refined and deodorized.
    \end{answer}
\end{standalonequote}

\begin{standalonequote}{Fats \& Oils}
    \metadata{topic={Coconut Oil Sources}, source={Email Wiki}}

    \begin{answer}
        Most cities have wholesale grocers that either stock it (in five gallon buckets) or can get it, and they usually charge about \$50 per bucket. GloryBee in Eugene is one place I have bought it, and Tropical Traditions has a good one, called expeller expressed, non-certified, and I think it's shipped from Nevada.
    \end{answer}
\end{standalonequote}

\begin{standalonequote}{Fats \& Oils}
    \metadata{topic={Coconut Oil Refining}, source={Email Wiki}}

    \begin{answer}
        It's just filtered, usually through diatomaceous earth, to remove materials other than the fat; the main problem with the unfiltered oil is that it's allergenic for many people. It also degrades quicker.
    \end{answer}
\end{standalonequote}

\begin{standalonequote}{Fats \& Oils}
    \metadata{topic={MCT Oil Side Effects}, source={Email Wiki}}

    \begin{answer}
        The problem lots of people have is diarrhea or other bowel reaction when they take more than a very small amount at a time. The first times I used it I smelled like a goat for several days, and even a small amount is enough for me to notice on my skin the next day.
    \end{answer}
\end{standalonequote}

\begin{standalonequote}{Fats \& Oils}
    \metadata{topic={Optimal Fat Percentage}, source={Email Wiki}}

    \begin{answer}
        Sugar helps the liver to make cholesterol, switching from starchy vegetables to sweet fruits will usually bring cholesterol levels up to normal. If the fat is mostly saturated, from milk, cheese, butter, beef, lamb or coconut oil, I think it's usually o.k. to get about 50\% of the calories from fat, but since those natural fats typically contain around 2\% polyunsaturated fats, I try to minimize my PUFA intake by having more fruit, and a little less fat, maybe 30 to 35\%.
    \end{answer}
\end{standalonequote}

\begin{standalonequote}{Fats \& Oils}
    \metadata{topic={Dietary Fat Justification}, source={Email Wiki}}

    \begin{answer}
        Although we can make our own fats from sugars, I think it's good to have some fat in our food, because of its effects on the intestine especially. Experiments on an isolated loop of intestine, measuring the nutrients entering the bloodstream, showed that relatively simplified mixtures of nutrients were poorly digested. Fat, protein, sugars, and minerals, in combination, activated the intestine, increasing the digestion of all of them, when they were present at the same time. If the fats are mostly saturated, as in butter, coconut oil, or beef or lamb fat, roughly a third of the calories is good, but the ideal proportion probably depends on the specific foods and the person's level of activity. Increasing either fat or sugar can have some specific therapeutic effects, but when more information becomes available about the composition of particular fruits, I suspect that the ideal balance of nutrients will lean toward the sugars, supported by ketoacids and short-chain saturated fats. The polyunsaturated fatty acids, which break down into toxic fragments and free radicals and prostaglandin-like chemicals, are--along with bacterial toxins produced in the intestine--the source of the main inflammatory and degenerative problems. Sugar and the minerals in fruits are fairly effective in keeping free fatty acids from being released from our tissues, and the fats we synthesize from them are saturated, and aren't likely to be stored as excess fat, because they don't suppress metabolism (as polyunsaturated fats and some amino acids do). The minerals of fruits and milk contribute to metabolic activation, and prevention of free-radical damage.
    \end{answer}
\end{standalonequote}

\begin{standalonequote}{Fats \& Oils}
    \metadata{topic={Minimizing Fat Intake}, source={Email Wiki}}

    \begin{answer}
        The fats in meat and cheese can be minimized by choosing low fat types, and skimmed or 1\% milk can be used.
    \end{answer}
\end{standalonequote}

\begin{qaexchange}{Fats \& Oils}
    \metadata{topic={Minimum Fat Requirements}, source={Email Wiki}}

    \begin{question}
        40 Grams of fat a day enough?
    \end{question}

    \begin{answer}
        I think that's enough, having a little with the other foods is best.
    \end{answer}
\end{qaexchange}

\begin{standalonequote}{Fats \& Oils}
    \metadata{topic={Fungal Toxins In Coconut Oil}, source={Email Wiki}}

    \begin{answer}
        Since animal studies show good health effects of both of them, and bad effects of other foods such as peanuts, wheat, and corn, the contamination is probably low. Hexane extraction seems to eliminate it, and it apparently declines in stored oil with time.
    \end{answer}
\end{standalonequote}

\begin{standalonequote}{Fats \& Oils}
    \metadata{topic={Coconut Oil Weight Loss}, source={Email Wiki}}

    \begin{answer}
        Yes, it's best to lose it slowly. When I tried adding about a tablespoon of coconut oil once a day I lost about two pounds a week, for several weeks, without eating less.
    \end{answer}
\end{standalonequote}

\begin{qaexchange}{Fats}
    \metadata{topic={Lauric Acid and DHT}, source={Ray Peat Forum}}

    \begin{question}
        I've read several studies that show that Lauric Acid inhibits 5 alpha reductase. Do you think men should limit their intake of it so as not to lower DHT?
    \end{question}

    \begin{answer}
        I think the body compensates when something is steadily in the diet.
    \end{answer}
\end{qaexchange}

\begin{qaexchange}{Fats}
    \metadata{topic={Oleic Acid}, source={Ray Peat Forum}}

    \begin{question}
        Is oleic acid harmful? Or whenever you say unsaturated fats are you just specifically talking about the PUFA's?
    \end{question}

    \begin{answer}
        Yes, I mean the PUFA, oleic acid is safe, and we synthesize it ourselves from carbohydrates.
    \end{answer}
\end{qaexchange}

\begin{qaexchange}{Fats}
    \metadata{topic={Oxidized LDL}, source={Ray Peat Forum}}

    \begin{question}
        If oxidized ldl is already oxidized, does it continue to cause lipid peroxidation in our blood vessels and cause inflammation and oxidative stress? Or is oxidized LDL only partially oxidized, and being so, it would continue to have a pathological effect?
    \end{question}

    \begin{answer}
        Yes, the oxidized fragments keep spreading the oxidation, with the smaller products often being the most toxic.
    \end{answer}
\end{qaexchange}

\begin{qaexchange}{Fats}
    \metadata{topic={DHA and Arachidonic Acid}, source={Ray Peat Forum}}

    \begin{question}
        I was wondering what you thought about the fatty acids DHA and Arachidonic acid in regards to them or DHA's precursor being essential for humans?
    \end{question}

    \begin{answer}
        If someone is still talking as though the Burrs demonstrated essentiality of PUFA, I don't know what more to say besides what I've written; for the last several years, the evidence for essentiality has seemed to be that old people's heads are full of DHA. The observation that newborn babies' brains are \enquote{deficient} in those PUFAs would suggest that it's new brains that are impaired, not old brains.
    \end{answer}
\end{qaexchange}

\begin{qaexchange}{Fats}
    \metadata{topic={PUFA Elimination Timeline}, source={Ray Peat Forum}}

    \begin{question}
        If someone is avoiding PUFAs carefully for a few years, is there any way to tell if they've gotten rid of most of their stored PUFA?
    \end{question}

    \begin{answer}
      The resting metabolic rate will be higher, and there will be less tendency to have inflammation after an injury. Although they have to eat more calories, I think they don't experience the intense stress from hunger.
    \end{answer}
\end{qaexchange}

\begin{qaexchange}{Fats}
    \metadata{topic={Metabolic Changes from Saturated Fat}, source={Ray Peat Forum}}

    \begin{question}
        I read that it takes four years for the fat stores in our bodies to change over. In your experience, do most people who switch to mostly saturated dietary fats noticed that their metabolism increases slowly over this four year period?
    \end{question}

    \begin{answer}
      Fat people with low metabolic rate probably take longer; when people use a thyroid supplement they usually decrease it in the first couple of years; it probably depends on age, too.
    \end{answer}
\end{qaexchange}

\begin{standalonequote}{Fats \& Oils}
    \metadata{topic={Coconut Oil, PUFA Displacement}, source={Ray Peat Forum}}

    \begin{answer}
      When the body contains a lot of PUFA, eating coconut oil increases oxidative metabolism, partly because of the shorter fatty acids that are more quickly oxidized, like sugar, and partly because of the antimetabolic effects of the PUFA that they displace. More recently, several investigators have found that a \enquote{deficiency of essential fatty acids} is highly protective against diabetes.
    \end{answer}
\end{standalonequote}

\begin{qaexchange}{Fats \& Oils}
    \metadata{topic={Essential Fatty Acids, Carnivores}, source={Ray Peat Forum}}

    \begin{question}
        If EFAs are absolutely unessential, then what is the explanation for symptoms carnivore animals such as cats get when their diet are devoid of arachidonic acid?
    \end{question}

    \begin{answer}
      I think people have extrapolated ideas from EFAD rats to cats, without recognizing that carnivores have higher metabolic rates and nutritional needs, so that the mistakes of the Burrs are even easier to make.
    \end{answer}
\end{qaexchange}

\begin{emailexchange}{Fats \& Oils}
    \metadata{topic={Palm Oil for Frying}, source={Ray Peat Forum}}

    \begin{question}
        Is it ok to eat potatoes fried in palm oil?
    \end{question}

    \begin{answer}
      If it's unrefined, with a pink color, I would be concerned about the oxidation products of the carotene.
    \end{answer}

    \begin{question}
        What if it says it's refined with yellowish color? Would PUFA be any concerns?
    \end{question}

    \begin{answer}
      I prefer hydrogenated coconut oil because of the PUFA.
    \end{answer}
\end{emailexchange}

\begin{standalonequote}{Fats \& Oils}
    \metadata{topic={Shea Butter PUFA Content}, source={Ray Peat Forum}}

    \begin{answer}
      Shea butter has about 3 times as much linoleic acid as coconut oil.
    \end{answer}
\end{standalonequote}

\begin{standalonequote}{Fats \& Oils}
    \metadata{topic={Hydrogenated Coconut Oil Safety}, source={Ray Peat Forum}}

    \begin{answer}
      In animal tests of oils in chronic feeding, hydrogenated coconut oil was the least carcinogenic. Chronic exposure to nickel is considered to be carcinogenic. I think the main value of coconut oil is that it can displace polyunsatured oils in the diet. We can synthesize our own protective saturated fats from glucose. I notice that the vegetable oil aisle in supermarkets has been shrinking over the last 10 or 15 years, and the soy/canola/corn oil people seem to be stepping up their marketing efforts.
    \end{answer}
\end{standalonequote}

\begin{qaexchange}{Fats \& Oils}
    \metadata{topic={Oleic Acid vs PUFA}, source={Ray Peat Forum}}

    \begin{question}
         In many of your articles and interviews you mention the negatives of the unsaturated fatty acids, and at other times specifically the polyunsaturated fatty acids. My question is, when you are mentioning the unsaturated fats, are you including monounsaturated ones like Oleic acid? Is oleic acid harmful? Or whenever you say unsaturated fats are you just specifically talking about the PUFA's?
    \end{question}

    \begin{answer}
      Yes, I mean the PUFA, oleic acid is safe, and we synthesize it ourselves from carbohydrates.
    \end{answer}
\end{qaexchange}

\begin{qaexchange}{Fats \& Oils}
    \metadata{topic={Saturated Fat Endotoxin Myth}, source={Ray Peat Forum}}

    \begin{question}
        Since saturated fats cause low grade endotoxemia by transporting endotoxin from the digestive system into the blood and other tissues creating an inflammatory response (cytokines, TLR4, etc.) and since we cant completely rid ourselves of bacteria, wouldn't it be better to eat a very low fat high carbohydrate diet so as to avoid chronic, repeated bouts of gut derived endotoxin exposure?
    \end{question}

    \begin{answer}
       I have seen the \enquote{saturated fat increases endotoxin uptake} idea, but they ignored the bacteriostatic action of the saturated fat, and also its acceleration of the degradation process. I consider it to have been a typical misleading propaganda piece.
    \end{answer}
\end{qaexchange}

\begin{qaexchange}{Fats \& Oils}
    \metadata{topic={Low Fat Diet and Skin}, source={Ray Peat Forum}}

    \begin{question}
        Do you think a low fat (about 30 grams per day) diet is bad for skin health and might contribute to its premature aging and wrinkling? Or is sugar and protein sufficient for skin health?
    \end{question}

    \begin{answer}
      We can make fat from protein and carbohydrate; proteins, vitamins, and minerals are the main things for skin health.
    \end{answer}
\end{qaexchange}

\begin{qaexchange}{Fats \& Oils}
    \metadata{topic={Minimum Fat Percentage}, source={Ray Peat Forum}}

    \begin{question}
        Do you think 10\% of total calories coming from fat is enough? (The fat is from milk).
    \end{question}

    \begin{answer}
      For some people it might be, but for good digestion I think 20\% is better.
    \end{answer}
\end{qaexchange}

\begin{qaexchange}{Fats \& Oils}
    \metadata{topic={Nickel in Hydrogenated Oils}, source={Ray Peat Forum}}

    \begin{question}
      Is nickel a concern in hydrogenated fats because of the nickel catalysts used?
    \end{question}

    \begin{answer}
      I think palladium catalysts are more often used for food oils.
    \end{answer}
\end{qaexchange}

\begin{qaexchange}{Fats \& Oils}
    \metadata{topic={Dietary Fat Percentage Recommendation}, source={Ray Peat Forum}}

    \begin{question}
         I wondered roughly what percentage of your daily calories come from fat, do you know? Or what percentage would you recommend most people consume coming from the common saturated fat sources?
    \end{question}

    \begin{answer}
      It depends partly on your metabolic rate and activity level, but you need enough carbohydrate to prevent ketosis. Generally using mainly carbohydrates for energy is better, because a higher respiratory quotient prevents reductive stress, the metabolism that can lead to diabetes, dementia, heart and kidney disease, cancer. I think it would be good to aim for 30\% of calories or less. Milk with 1\% fat is a good staple—the high calcium content helps to keep a higher metabolic rate.
    \end{answer}
\end{qaexchange}

\begin{qaexchange}{Fats \& Oils}
    \metadata{topic={Minimum Dietary Fat Recommendation}, source={Ray Peat Forum}}

    \begin{question}
        What is the minimum amount of fat that you would recommend for a young woman? 
    \end{question}

    \begin{answer}
      The amount in 1\% milk, eggs, some lean meat and cheese, is more than enough. When there's enough milk in the diet, the metabolic rate can tolerate more fat without gaining body fat.
    \end{answer}
\end{qaexchange}

\begin{standalonequote}{Fats \& Oils}
    \metadata{topic={Saturated Fats Protective, Thyroid Function}, source={Ray Peat Forum}}

    \begin{answer}
      Saturated fats are protective; declining thyroid function causes LDL cholesterol to rise, and is an important cause of hypoglycemia, because of failure to store enough glycogen. Eating too much protein can cause declining thyroid function.
    \end{answer}
\end{standalonequote}

\begin{standalonequote}{Fats \& Oils}
    \metadata{topic={PUFA Metabolism, Metabolic Rate}, source={Ray Peat Forum}}

    \begin{answer}
      If the metabolic rate stays high relative to calorie intake, the pufa will be burned quickly, without having an opportunity to shape the physiology very much. Other things become relatively more important as the pufa intake approaches zero---methyl donors, phosphate excess, iron/copper ratio, etc.
    \end{answer}
\end{standalonequote}

\begin{qaexchange}{Fats \& Oils}
    \metadata{topic={Dietary Fat Percentage}, source={Ray Peat Forum}}

    \begin{question}
        Someone shared an email where you advised a maximum of 30\% dietary fat. You seem to praise saturated fat almost in a supplemental context but not as a bulk of the diet. Do you think exceeding that amount above 30\%, even with only saturated fats and no PUFA, causes metabolic problems?
    \end{question}

    \begin{answer}
      Hydrogenated coconut oil doesn't contain any PUFA; I have used larger amounts of that, with no visible effects.
    \end{answer}
\end{qaexchange}

\section{Whole Foods}

\begin{standalonequote}{Whole Foods}
    \metadata{topic={Natural Food Antibiotics}, source={Ray Peat Forum}}

    \begin{answer}
       Raw carrots, cooked bamboo shoots, and cooked mushrooms contain antibiotics that are safe to use everyday. Like tetracycline and the macrolide antibiotics, they (especially mushrooms) are also antiinflammatory. 
    \end{answer}
\end{standalonequote}

\begin{standalonequote}{Whole Foods}
    \metadata{topic={Safe Foods for Tissue Rebuilding}, source={Ray Peat Forum}}

    \begin{answer}
       Safe foods for rebuilding tissues are eggs, scallops, calamari, oysters, orange juice, milk, and cheese, salting things to taste. 
    \end{answer}
\end{standalonequote}

\begin{standalonequote}{Whole Foods}
    \metadata{topic={Bone Broth Fluoride, Gelatin}, source={Ray Peat Forum}}

    \begin{answer}
      The bone fluoride is very insoluble, so I don't think there's a problem. Long bones contain marrow, and prolonged cooking of that produces a lot of fat oxidation products. The tendons and ligaments around joints are the main source of gelatin, rather than the bone itself.
    \end{answer}
\end{standalonequote}

\begin{standalonequote}{Whole Foods}
    \metadata{topic={Stress Hormones, Bone Broth, Fatty Liver}, source={Ray Peat Forum}}

    \begin{answer}
      I think reducing the stress hormones, partly by balancing the nutrients, is more important than the number of calories. With bone broth, it's the cartilage, ligaments and tendons that are helpful; marrow has a lot of iron, and it's good to limit the amount of iron and phosphate. Well cooked greens have a high ratio of calcium and magnesium to phosphate, and these help to activate oxidative metabolism. Endotoxin from bacterial overgrowth is important in both diabetes and fatty liver. Aspirin helps to lower inflammation and blood glucose, while promoting oxidative metabolism. Coffee has some similar effects. Some people have told me that using fructose eliminated the fat from their liver; I think its reduction of phosphate might be involved in its protective effects, while supporting oxidative metabolism.
    \end{answer}
\end{standalonequote}

\begin{standalonequote}{Whole Foods}
    \metadata{topic={Digestible Foods, Inflammation}, source={Ray Peat Forum}}

    \begin{answer}
      Milk and sugar, custard with a minimum of egg (only natural vanilla for flavor), eggs, meat, and sweet ripe orange juice, are the easy things to digest. Even small amounts of plant material can cause inflammation for some people.
    \end{answer}
\end{standalonequote}

\subsection{Grains}

\begin{standalonequote}{Grains}
    \metadata{topic={Cassava Preparation}, source={Email Wiki}}

    \begin{answer}
        Cyanide is a goitrogen, and its quantity varies with the way the cassava is prepared. It's essential for the starch to be very well cooked, and eaten with some fat.
    \end{answer}
\end{standalonequote}

\begin{standalonequote}{Grains}
    \metadata{topic={GMO Corn Allergenicity}, source={Email Wiki}}

    \begin{answer}
        The bacterial genes that are meant to be toxic to insects can be allergenic to people. Barbara McClintock's work with corn showed that a change in the plants' environment causes them to shift their genes around, sort of equivalent to internal hybridization. When something new is added to the genome, it changes the results of the rearrangement, unpredictably. Since seeds always contain toxins, anyway, a new allergen probably isn't too important, and the traditional alkali processing of corn might take care of it.
    \end{answer}
\end{standalonequote}

\begin{standalonequote}{Grains}
    \metadata{topic={Safe Grain Hierarchy}, source={Email Wiki}}

    \begin{answer}
        Masa harina (best), white rice or oats, and brown rice. The phytic acid in the oats block absorption of much of the calcium; cooking the oats much longer than usual might improve its nutritional value.
    \end{answer}
\end{standalonequote}

\begin{standalonequote}{Grains}
    \metadata{topic={Sourdough vs Standard Bread}, source={Email Wiki}}

    \begin{answer}
        Naturally fermented sourdough is less harmful than standard or unleavened wheat products, but any starch tends to stimulate appetite by activating fat synthesis. The same number of calories in fruit would be less fattening, and would keep your blood sugar steadier, improve your sleep and mental energy.
    \end{answer}
\end{standalonequote}

\begin{standalonequote}{Grains}
    \metadata{topic={Sourdough Bread Effects}, source={Email Wiki}}

    \begin{answer}
        Naturally fermented sourdough is less harmful than standard or unleavened wheat products, but any starch tends to stimulate appetite by activating fat synthesis. The same number of calories in fruit would be less fattening, and would keep your blood sugar steadier, improve your sleep and mental energy.
    \end{answer}
\end{standalonequote}

\begin{qaexchange}{Grains}
    \metadata{topic={Oat Preparation}, source={Ray Peat Forum}}

    \begin{question}
        Whole oats or flattened oats? Oat meal or porridge? Which would be a better choice? How would you prepare them?
    \end{question}

    \begin{answer}
       Rolled oats or oat bran either would be be o.k., people vary in their response. The grain should be added to boiling water, and allowed to simmer for several minutes.
    \end{answer}
\end{qaexchange}

\begin{standalonequote}{Grains}
    \metadata{topic={Rice vs Sourdough Bread Safety}, source={Ray Peat Forum}}

    \begin{answer}
       In general, I think white rice is safer. Real sourdough bread, in which most of the gluten has been degraded, is probably about as safe as rice. I think nixtamalized corn is the safest. 
    \end{answer}
\end{standalonequote}

\begin{qaexchange}{Grains}
    \metadata{topic={Masa Harina as Staple}, source={Ray Peat Forum}}

    \begin{question}
        Do you think there are any concerns to using masa harina as the major carb source in one's diet, alongside good fruit (when available) and sugar?
    \end{question}

    \begin{answer}
       I think it's a good staple food.
    \end{answer}
\end{qaexchange}

\begin{qaexchange}{Grains}
    \metadata{topic={Parboiled Rice vs Nixtamalized Corn}, source={Ray Peat Forum}}

    \begin{question}
        Would you consider parboiled rice as a good alternative to nixtamalized corn ? (or white rice is better?)
    \end{question}

    \begin{answer}
       I think thorough cooking of white rice is better. Corn and other grains can be nixtamalized easily with calcium hydroxide, potassium carbonate, or sodium carbonate.
    \end{answer}
\end{qaexchange}

\subsection{Fruits}

\begin{qaexchange}{Fruits}
    \metadata{topic={Potatoes vs Bananas Comparison}, source={Email Wiki}}

    \begin{question}
        Potatoes can feed bacteria in the gut resulting in endotoxin and serotonin production; bananas contain serotonin---yet which of the two is the lesser evil?
    \end{question}

    \begin{answer}
        Potatoes are much better, unless you're allergic to them (it usually goes with allergy to tomato and bell peppers).
    \end{answer}
\end{qaexchange}

\begin{standalonequote}{Fruits}
    \metadata{topic={Allergenic Fruits}, source={Email Wiki}}

    \begin{answer}
        Bananas and jack-fruit are strong allergens, possibly because of their cultivation methods. Mangos, apples, and pears are allergenic to some people. Poorly ripened fruits of all sorts should be avoided.
    \end{answer}
\end{standalonequote}

\begin{standalonequote}{Fruits}
    \metadata{topic={Non-Allergenic Fruit List}, source={Email Wiki}}

    \begin{answer}
        Corossol, lychee, longan, guava, papaya, pawpaw, sapota, guanabana. Some frozen and canned fruits are good; applesauce, corossol, guanabana, longans, and lychees for example.
    \end{answer}
\end{standalonequote}

\begin{standalonequote}{Fruits}
    \metadata{topic={Fruits And Serotonin}, source={Email Wiki}}
    \begin{note}
        Salicylic Acid Intolerance
    \end{note}

    \begin{answer}
        The fruits you mention all seriously increase serotonin. A sore throat is a quick effect, but some people get migraines from them. The pectin in raw apples causes the intestine to release serotonin into the blood, so well cooked apples have much less effect. Fruits contain almost no salicylic acid.
    \end{answer}
\end{standalonequote}

\begin{standalonequote}{Fruits}
    \metadata{topic={Pear Pectin Serotonin}, source={Email Wiki}}

    \begin{answer}
        The fructose content of pears is probably helpful, but you should watch for what effect it might be having on your intestine, from the pectin. Pectin tends to increase serotonin by irritating the intestine. Allergies can increase your blood glucose, so you should watch for effects, usually the next day, sometimes extending for two or three days, from foods that are commonly allergenic, such as tomato sauce and spaghetti; unrefined coconut oil is a possible allergen, too. Do you use any aspirin?
    \end{answer}
\end{standalonequote}

\begin{qaexchange}{Fruits}
    \metadata{topic={Orange Juice and Carotenes}, source={Ray Peat Forum}}

    \begin{question}
        Do you think some people could be sensitive to the carotenes in orange juice? Could the carotenes in orange juice and carrot be keeping people in a hypothyroid state despite doing them everything else right?
    \end{question}

    \begin{answer}
        The content in oranges is so low I don't think even a gallon a day would affect the thyroid, but a glass or two of carrot juice definitely can.
    \end{answer}
\end{qaexchange}

\begin{standalonequote}{Orange Juice}
    \metadata{topic={Orange Juice and Blood Pressure}, source={Ray Peat Forum}}

    \begin{answer}
        Citric acid binds magnesium and calcium, and if the orange juice was sour (commercial juice usually has added citric acid) that might account for the blood pressure change.
    \end{answer}
\end{standalonequote}

\begin{qaexchange}{Fruits}
    \metadata{topic={Cherries}, source={Ray Peat Forum}}

    \begin{question}
        What do you think about sour (tart) cherries?
    \end{question}

    \begin{answer}
        I think any cherries are safe and nutritious, though it's good to investigate whether toxic pesticides have been used on them.
    \end{answer}
\end{qaexchange}

\begin{qaexchange}{Fruits}
    \metadata{topic={Dates for Labor}, source={Ray Peat Forum}}

    \begin{question}
        People swear by the effectiveness of eating dates (saying they have an Oxytocin-like effect) to reduce labor pain. Does that sound possibly helpful?
    \end{question}

    \begin{answer}
        Labor consumes a lot of glucose, and is likely to deplete glycogen stores if it's prolonged, so I suspect that the main effect of dates is to prevent hypoglycemia. They contain some serotonin, which can increase uterine contraction. Hypoglycemia increases adrenaline, and adrenaline can cause uterine inertia, increasing the risk of hemorrhage, so I think any good source of sugar, such as fruit juice, is protective during labor.
    \end{answer}
\end{qaexchange}

\begin{qaexchange}{Fruits}
    \metadata{topic={Malic Acid}, source={Ray Peat Forum}}

    \begin{question}
        Should we worry about the fact that small amounts of malic acids are found in many fruits and vegetables?
    \end{question}

    \begin{answer}
      Malic acid is o.k.; malonic acid is what makes unripe apples toxic.
    \end{answer}
\end{qaexchange}

\begin{qaexchange}{Fruits}
    \metadata{topic={Grapefruit Selection}, source={Ray Peat Forum}}

    \begin{question}
        Would grapefruit be ok?
    \end{question}

    \begin{answer}
      Yes, if they are ripe and sweet.
    \end{answer}
\end{qaexchange}

\begin{qaexchange}{Fruits}
    \metadata{topic={Guava Quality Ranking}, source={Ray Peat Forum}}

    \begin{question}
        I think you have stated that the best fruits, if properly grown, are oranges, watermelons, and grapes. Is guava close to them in quality?
    \end{question}

    \begin{answer}
      Yes, they rank with oranges for their protective qualities.
    \end{answer}
\end{qaexchange}

\begin{qaexchange}{Fruits}
    \metadata{topic={Cooked Apples as Carb Source}, source={Ray Peat Forum}}

    \begin{question}
        What do you think about very well cooked apples being a major source of carbohydrates in the diet?
    \end{question}

    \begin{answer}
       Very good, probably best to peel them. 
    \end{answer}
\end{qaexchange}

\begin{standalonequote}{Fruits}
    \metadata{topic={Pectin in Apples/Pears}, source={Ray Peat Forum}}

    \begin{answer}
      A problem with apples and pears is their pectin content, which can potentially cause bowel inflammtion.
    \end{answer}
\end{standalonequote}

\begin{qaexchange}{Fruits}
    \metadata{topic={Dried Fruits Safety}, source={Ray Peat Forum}}

    \begin{question}
        Whats your thoughts on dried fruits sir like dried organic mango, organic pineapple, organic dates, organic figs?
    \end{question}

    \begin{answer}
      If they agree with you they are fine.
    \end{answer}
\end{qaexchange}

\begin{standalonequote}{Fruits}
    \metadata{topic={Orange Peel Marmalade, Naringin}, source={Ray Peat Forum}}

    \begin{answer}
      When I get sour oranges I make marmalade from the peels, if they are organic. Shred, soak, cook slowly simmering in water for about an hour before adding sugar, and letting that simmer without boiling until it thickens a little. When it's cool it thickens more. The peels are rich in antiinflammatory chemicals [naringin and naringenin], more than the juice, and the marmalade is a good way to get sugar with the cottage cheese or parmesan.
    \end{answer}
\end{standalonequote}

\begin{standalonequote}{Fruits}
    \metadata{topic={Fruit-Only Diet, Amino Acids}, source={Ray Peat Forum}}

    \begin{answer}
      Fruits vary in their protein content and amino acid balance; if we had more knowledge about the amino acids in each fruit, a pure fruit diet might be ideal, but I think it would be risky without that information. Independent researchers have trouble buying the reagents needed for that kind of study, so I haven't done it.
    \end{answer}
\end{standalonequote}

\subsection{Vegetables}

\begin{standalonequote}{Vegetables}
    \metadata{topic={Carrot Salad Preparation}, source={Email Wiki}}

    \begin{answer}
        Just chewing a carrot is best, any saturated fat around the same time is o.k. What doesn't work very well is to grind the carrot very fine.
    \end{answer}
\end{standalonequote}

\begin{standalonequote}{Vegetables}
    \metadata{topic={Carrot Alternatives For Fiber}, source={Email Wiki}}

    \begin{answer}
        If you want to avoid the carotene of carrots, they can be rinsed after shredding; washed and cooked bran or psyllium husk can be effective, too.
    \end{answer}
\end{standalonequote}

\begin{standalonequote}{Vegetables}
    \metadata{topic={Potato Varieties Digestibility}, source={Email Wiki}}

    \begin{answer}
        The carotene in sweet potatoes can make them harder to digest. Well cooked white potatoes, such as russets, are very nutritious, and the (cooked) juice is just something for people with extreme metabolic or digestive problems. The juicer I had was the kind that's commonly used for juicing carrots, and it was inexpensive. I don't think it was anything near 700 watts, that's nearly a horsepower, more than enough for a big cement mixer.
    \end{answer}
\end{standalonequote}

\begin{standalonequote}{Vegetables}
    \metadata{topic={Mushrooms}, source={Email Wiki}}

    \begin{answer}
        Since reading about the chemicals in mushrooms I stopped eating them, but using them occasionally is o.k., probably better than many vegetables.
    \end{answer}
\end{standalonequote}

\begin{standalonequote}{Vegetables}
    \metadata{topic={Carrot Timing Between Meals}, source={Email Wiki}}

    \begin{answer}
        Since the fiber [carrot] will delay digestion and reduce absorption of other foods, I think it's best to eat it between meals, usually in the afternoon.
    \end{answer}
\end{standalonequote}

\begin{standalonequote}{Vegetables}
    \metadata{topic={Plain Carrot Efficacy}, source={Email Wiki}}

    \begin{answer}
        Yes, the plain carrot is good. For people who want more antimicrobial effect, the saturated fats and vinegar are helpful.
    \end{answer}
\end{standalonequote}

\begin{standalonequote}{Vegetables}
    \metadata{topic={Carrot Lowering Estrogen}, source={Email Wiki}}

    \begin{answer}
        The intestine is a potential source of reabsorbed estrogen, and a daily raw carrot (grated or shredded, with a little olive oil, vinegar, salt) helps to lower excess estrogen (and endotoxin produced by bacteria). While lowering estrogen, it is likely to lower cortisol and increase progesterone.
    \end{answer}
\end{standalonequote}

\begin{standalonequote}{Vegetables}
    \metadata{topic={Mushroom Intestinal Reflex}, source={Email Wiki}}

    \begin{answer}
        I haven't noticed anything like that with the mushrooms, but peristalsis of the intestine can often trigger very localized sensations in the face or head.
    \end{answer}
\end{standalonequote}

\begin{standalonequote}{Vegetables}
    \metadata{topic={Potato Starch Considerations}, source={Email Wiki}}

    \begin{answer}
        When a person has limited money for food, potatoes are a better staple than beans or oats. Starches associated with saponins, alkaloids, and other potentially pro-inflammatory things make them a less than ideal food, if you have digestion-related health problems, and if you can afford to choose. New potatoes are tastier, less starchy, and probably less likely to cause digestive irritation.
    \end{answer}
\end{standalonequote}

\begin{standalonequote}{Vegetables}
    \metadata{topic={Green Juice PUFA Concerns}, source={Email Wiki}}

    \begin{answer}
        The minerals and vitamin K are definitely valuable, but the high content of PUFA and tannins is a problem. Boiling the leaves and discarding all but the water can produce a good magnesium supplement.
    \end{answer}
\end{standalonequote}

\begin{qaexchange}{Vegetables}
    \metadata{topic={Greens Not Necessary}, source={Email Wiki}}

    \begin{question}
        I supplement 5g of vit K\textsubscript{2} mk-4 once a week, do you think green veggies are even necessary?
    \end{question}

    \begin{answer}
        If you have other sources of magnesium, the green vegetables aren't needed.
    \end{answer}
\end{qaexchange}

\begin{standalonequote}{Vegetables}
    \metadata{topic={Carrot For Weight Loss}, source={Email Wiki}}

    \begin{answer}
        Yes, I know people who have lost weight just by eating a raw carrot every day, reducing endotoxin stress. The liver treats PUFA as it treats toxins, but when their concentration is too high, they poison the detoxifying system. Oleic acid, which we can make ourselves from carbohydrates, greatly activates the detox enzyme system.
    \end{answer}
\end{standalonequote}

\begin{qaexchange}{Vegetables}
    \metadata{topic={Raw Carrot Salad}, source={Ray Peat Forum}}

    \begin{question}
        I've been slicing carrots in olive oil and umi vinegar and eating at end of lunch am I losing any benefit from not grating them?
    \end{question}

    \begin{answer}
        Chewing well has the same effect.
    \end{answer}
\end{qaexchange}

\begin{emailexchange}{Vegetables}
    \metadata{topic={Carotenoids}, source={Ray Peat Forum}}

    \begin{answer}
        We don't convert lycopene [main tomato pigment], but alpha-carotene and beta-cryptoxanthin are converted to vitamin A.
    \end{answer}

    \begin{question}
        Did you mention alpha-carotene and beta-cryptoxanthin as related to tomatoes, or those are only carotenoids we can convert to vitamin A?
    \end{question}

    \begin{answer}
        One other can be converted to vitamin A--gamma-carotene.
    \end{answer}

    \begin{question}
        And those are the ones that can cause problems as I understand. Are you aware of any other fruits that contain the carotenoids in quantities that can be risky?
    \end{question}

    \begin{answer}
        Cooked carrots or the juice, tomatoes, and orange or red sweet potatoes in large quantities are commonly problems in hypothyroid people, or in vegans who lack vitamin B\textsubscript{12}.
    \end{answer}
\end{emailexchange}

\begin{qaexchange}{Vegetables}
    \metadata{topic={Greens Broth Preparation}, source={Ray Peat Forum}}

    \begin{question}
        I was wondering what amount of the greens you put into a pot to boil, when you want to drink the water for magnesium and minerals, and how much water you drink from it? Also, how often do you think a magnesium deficient person should drink this?
    \end{question}

    \begin{answer}
       I packed a pan with the leaves, and just enough water to cover them as they compacted with heat; an ounce of the liquid was enough to stop cramps. Supplementing thyroid makes the cells retain magnesium, so just one or two doses was enough. 
    \end{answer}
\end{qaexchange}

\begin{qaexchange}{Vegetables}
    \metadata{topic={Organic Greens for Broth}, source={Ray Peat Forum}}

    \begin{question}
        When you prepare the greens broth to drink the water, do you make sure to get organic greens, or do you not think it is important?
    \end{question}

    \begin{answer}
       I think it's worth the cost, in the case of greens. 
    \end{answer}
\end{qaexchange}

\begin{standalonequote}{Vegetables}
    \metadata{topic={Beets Carotene Content}, source={Ray Peat Forum}}

    \begin{answer}
       Cooked beets are pleasant food, but the high carotene content could become a problem if thyroid function isn't too good. 
    \end{answer}
\end{standalonequote}

\begin{standalonequote}{Vegetables}
    \metadata{topic={Celery Allergens}, source={Ray Peat Forum}}

    \begin{answer}
      Celery has some valuable flavonoids, but its allergenic proteins cause problems for some people, so it's good to be cautious.
    \end{answer}
\end{standalonequote}

\begin{qaexchange}{Vegetables}
    \metadata{topic={Bamboo Shoot Goitrogenic Effects}, source={Ray Peat Forum}}

    \begin{question}
        There are studies with mice/rats that show prolonged bamboo shoot consumption induce hypothyroidism as well as damage reproductive function. Have you read any of these studies? Have you experienced any thyroid suppression from prolonged bamboo shoot consumption?
    \end{question}

    \begin{answer}
       Yes, like cabbage, cauliflower, mustard and radish it's possible to eat enough to cause enlargement of the thyroid gland, especially when the traditional diet favors a bitter, acidic variety of bamboo, but the blanched, mild tasting variety that's popular in the US and China doesn't seem to cause problems.
    \end{answer}
\end{qaexchange}

\begin{standalonequote}{Vegetables}
    \metadata{topic={Natural Glutamate Safety}, source={Ray Peat Forum}}

    \begin{answer}
       The amount that occurs naturally in a food is accompanied by balancing factors that prevent excitotoxicity. 
    \end{answer}
\end{standalonequote}

\begin{qaexchange}{Vegetables}
    \metadata{topic={Carrot Salad Immediate Effects}, source={Ray Peat Forum}}

    \begin{question}
        If the action of the carrot isn't necessarily providing nutrients that are missing, is the consistency of post euphoria more of a sign of deficiency in your research/experience?

         Does the binding of excess estrogen in the gut really create that much of a instant difference or is it something else?
    \end{question}

    \begin{answer}
      The reduction of irritation combined with gentle stimulation has an immediate effect, probably involving an immediate reduction of serotonin, histamine, and endotoxin absorption; the estrogen effect is probably slower.

      The immediate effect of reducing the toxins can resolve symptoms and change attitudes.
    \end{answer}
\end{qaexchange}

\subsection{Dairy Products}

\begin{standalonequote}{Dairy Products}
    \metadata{topic={Milk Hormones And Metabolism}, source={Email Wiki}}

    \begin{answer}
        The milk estrogen research isn't good. It also contains thyroid and progesterone and other protective substances. The high calcium content helps to increase the metabolic rate, and probably contributes to maintaining the anabolic balance.
    \end{answer}
\end{standalonequote}

\begin{standalonequote}{Dairy Products}
    \metadata{topic={Milk Tryptophan Metabolism}, source={Email Wiki}}

    \begin{answer}
        Regarding milk and its tryptophan content, The calcium helps to keep the metabolic rate high, and the other nutrients help to steer tryptophan away from the serotonin path.
    \end{answer}
\end{standalonequote}

\begin{standalonequote}{Dairy Products}
    \metadata{topic={Cheese Manufacturing Concerns}, source={Email Wiki}}

    \begin{answer}
        In cheese, When the label says \enquote{enzymes,} it is likely that they are using one of the new products; lots of people are having serious intestinal reactions to commercial cheeses. Real animal rennet is still safe, as far as I know. Industrial grade citric acid is a serious allergen for some people, because it contains contaminants that aren't in natural fruit citric acid, but it's probably safer than the industrial \enquote{enzymes.} The producers of the enzyme products claim they are highly purified, but some people react as though they still contain some antigens from the microorganisms. The traditional cheeses were made with milk that soured with the bacteria that lived in the cows, but now it's common to sterilize the milk, and then add cultures, or enzymes, or citric acid, for standardization---but they often put their faith in a commercial product that seems to work well, but that could have serious allergenic contaminants. The same thing has been happening with aged cheeses, many places are no longer letting the native molds infect the cheese curds. Homogenizing doesn't cause any problems---unless they use solvents/detergents for adding the vitamins A and D that are required in milk with reduced fat. The vitamins aren't normally added to whole milk or cream.
    \end{answer}
\end{standalonequote}

\begin{standalonequote}{Dairy Products}
    \metadata{topic={Cheese Tryptophan Variation}, source={Email Wiki}}

    \begin{answer}
        It varies a little with the method of making cheese, and the calcium content varies even more.
    \end{answer}
\end{standalonequote}

\begin{standalonequote}{Dairy Products}
    \metadata{topic={Ricotta Calcium Content}, source={Email Wiki}}

    \begin{answer}
        The calcium content [of ricotta cheese] can vary greatly, depending on whether the whey is separated by acid or by bacterial proteolysis.
    \end{answer}
\end{standalonequote}

\begin{standalonequote}{Dairy Products}
    \metadata{topic={Parmesan Cheese Calcium}, source={Email Wiki}}

    \begin{answer}
        The ratio of calcium to phosphate is more important than the absolute amount of calcium. 4 ounces [Parmegioanno regiano] would usually be enough.
    \end{answer}
\end{standalonequote}

\begin{standalonequote}{Dairy Products}
    \metadata{topic={Milk Processing Preferences}, source={Email Wiki}}

    \begin{answer}
        I normally use pasteurized (and homogenized) milk, and I know people who do best when they use ultrapasteurized milk, and many people who, especially in certain seasons, don't tolerate raw milk. Cows' bacteria change according to what they are eating, and sometimes even the low level of bacteria in pasteurized milk can upset the person's intestinal balance of bacteria. I advise against eating the solid parts of coconut, as a regular part of the diet, and recommend the deodorized refined oil, because so many people are allergic to the proteins (and starches) of coconut. My November newsletter, below, will explain why people tend to lose weight on milk and sugar.
    \end{answer}
\end{standalonequote}

\begin{standalonequote}{Dairy Products}
    \metadata{topic={Ray Peat Milk Consumption}, source={Email Wiki}}

    \begin{answer}
        Over the years I averaged a gallon a day, and I liked to eat butter, fat meat, ice cream, and thick cream in my coffee, so 1\% milk had enough fat. I didn't like the taste of skimmed milk, and the available 1\% happens to be pasteurized. In Mexico when I get it from the farmer, I don't know how much fat it has, but on average it's probably similar.
    \end{answer}
\end{standalonequote}

\begin{standalonequote}{Dairy Products}
    \metadata{topic={Milk Source Variations}, source={Email Wiki}}

    \begin{answer}
        Have you experimented with milk from different sources? Sometimes the goats or cows eat allergenic things, or have bacteria that disturb the intestine. Have you tried boiled or ultrapasteurized milk? Is the cheese the original Parmigiano Reggiano? If you can list all the foods that you have had in the last day or two, I might see some things that are affecting your hormones. Anything that irritates your intestine or increases bacterial activity in the small intestine can increase the absorption of bacterial endotoxin, and that lowers testosterone and thyroid hormone, and increases cortisol. Reducing endotoxin might be all it takes to correct the hormones. Have you had blood tests for thyroid or other hormones?
    \end{answer}
\end{standalonequote}

\begin{standalonequote}{Dairy Products}
    \metadata{topic={Goat Milk Allergen Sources}, source={Email Wiki}}

    \begin{answer}
        Sometimes goats find allergenic weeds when they graze, so trying different kinds of milk, or commercial ultrapasteurized milk could help.
    \end{answer}
\end{standalonequote}

\begin{standalonequote}{Dairy Products}
    \metadata{topic={Kefir Lactic Acid Content}, source={Email Wiki}}

    \begin{answer}
        The neutral lactate salt is at least as toxic as the acid form, but each culture varies a little in the amount of acid formed. The enzyme that thickens the milk sometimes works with very little acid formed. How sour the kefir is suggests how much lactic acid is in it. There are types of yogurt that have much of the acidic whey drained out, that aren't a problem. A spoonful or two of acidic yogurt isn't harmful, but a cupful of the acidic type can be enough to deplete the liver's energy stores, because lactic acid is converted to glucose in the liver, requiring energy. The \enquote{strained} type that isn't acidic is similar to cottage cheese and is safe.
    \end{answer}
\end{standalonequote}

\begin{standalonequote}{Dairy Products}
    \metadata{topic={Milk Freshness}, source={Email Wiki}}

    \begin{answer}
        I think it's a matter of watching for any effects associated with a particular product; if nothing is obvious, the fresher milk is preferable.
    \end{answer}
\end{standalonequote}

\begin{standalonequote}{Dairy Products}
    \metadata{topic={Raw vs Pasteurized Milk}, source={Email Wiki}}

    \begin{answer}
        The difference isn't enough to worry about.
    \end{answer}
\end{standalonequote}

\begin{standalonequote}{Dairy Products}
    \metadata{topic={Milk Estrogen Content}, source={Email Wiki}}

    \begin{answer}
        High estrogen, relative to progesterone, interferes with lactation, and the enzymes that convert estradiol to the less active estrone and estriol are increased by progesterone. The amount of estradiol in milk is usually much less than one microgram per liter, and it's concentrated in the cream, so low-fat milk has very little estrogen. The cow's diet is probably a more important factor in the estrogen content of milk than pregnancy. The information in that abstract isn't enough to tell whether the study was done properly.
    \end{answer}
\end{standalonequote}

\begin{standalonequote}{Dairy Products}
    \metadata{topic={Ultrapasteurized Milk Tolerance}, source={Email Wiki}}

    \begin{answer}
        I know people who tolerate only the ultrapasteurized milk.
    \end{answer}
\end{standalonequote}

\begin{standalonequote}{Dairy Products}
    \metadata{topic={Milk Intolerance Causes}, source={Email Wiki}}

    \begin{answer}
        Yes, I think bowel irritation is behind milk sensitivity.
    \end{answer}
\end{standalonequote}

\begin{standalonequote}{Dairy Products}
    \metadata{topic={Milk Intolerance Cultural Differences}, source={Email Wiki}}

    \begin{answer}
        I have been interested in the subject of \enquote{milk intolerance} for a long time, and have wondered why doctors in the US and England give it so much attention, while the people who drink the most milk, in the Samburu and Masai cultures, and the cultures of northern India, don't seem to have the problem. I doubt that this is a matter of genetic differences; for example this person: \enquote{I was recently diagnosed with lactose intolerance and so i had to eliminated milk and milk products from my diet. I live in the USA. However, on a recent trip to India, I had milk and all possible milk products there and it did not affect me at all! Has anyone else experienced this? Or does anyone have a possible explanation?}  When a woman or a cow eats an allergen, such as peanuts or soybeans, the allergens appear in the milk. Weeds in the pasture are another potential source of tainted milk. In Africa and India, milk production per cow is much lower than in the US, because they seldom give them anything but grass, or in India, hay, probably some fruit. Although insecticides such as lindane are no longer used in US dairies, most of the milk in commerce has synthetic vitamins (dissolved in corn oil) emulsified into the product, which could account for many of the bad reactions.
    \end{answer}
\end{standalonequote}

\begin{standalonequote}{Dairy Products}
    \metadata{topic={Goat Milk Copper Content}, source={Email Wiki}}

    \begin{answer}
        Goat milk contains more copper than cow milk, and copper is important for energy metabolism and blood formation.
    \end{answer}
\end{standalonequote}

\begin{standalonequote}{Dairy Products}
    \metadata{topic={Powdered Milk}, source={Email Wiki}}

    \begin{answer}
        It's not as good as fresh milk, or cheese, but when they aren't available, 100 grams (or more) would be a good addition to the diet, because of the high ratio of calcium to phosphate, as well as other nutrients.
    \end{answer}
\end{standalonequote}

\begin{standalonequote}{Dairy Products}
    \metadata{topic={Reduced Milk Safety}, source={Email Wiki}}

    \begin{answer}
        Reduced milk is o.k. if the heat wasn't very high.
    \end{answer}
\end{standalonequote}

\begin{standalonequote}{Dairy Products}
    \metadata{topic={Yogurt Lactic Acid Problems}, source={Email Wiki}}

    \begin{answer}
        In quantities of an ounce or so, for flavoring, it's o.k., but the lactic acid content isn't good if you are using yogurt as a major source of your protein and calcium. It triggers the inflammatory reactions, leading to fibrosis eventually, and the immediate effect is to draw down the liver's glycogen stores for energy to convert it into glucose.
    \end{answer}
\end{standalonequote}

\begin{qaexchange}{Milk}
    \metadata{topic={Milk Selection}, source={Ray Peat Forum}}

    \begin{question}
        Was listening to your phone interview with Patrick Timpone where you recommend low fat milk. I purchase Maple Hill Grass Milk since raw milk is prohibited where I live. The whole milk is not fortified, but by law the low fat milk is fortified with Vitamin A Palmitate and Vitamin D3. Which do you recommend in this case--whole or low fat?
    \end{question}

    \begin{answer}
        If you aren't very active physically, the fat content of whole milk provides too many calories in relation to the other nutrients. If it isn't homogenized, it's easily skimmed.
    \end{answer}
\end{qaexchange}

\begin{qaexchange}{Milk}
    \metadata{topic={Milk and IGF-1}, source={Ray Peat Forum}}

    \begin{question}
        I understand that growth hormone (IGF-1) is bad but it seems that drinking milk increases it. Is that something to worry about do you think?
    \end{question}

    \begin{answer}
        No.
    \end{answer}
\end{qaexchange}

\begin{qaexchange}{Milk}
    \metadata{topic={Milk Temperature}, source={Ray Peat Forum}}

    \begin{question}
        What do you think is the optimal temperature for milk to be consumed at? Also, why might lukewarm milk produce less gas in my colon than heated milk, while other people seem to digest heated milk better?
    \end{question}

    \begin{answer}
        The digestive functions all work best at 98 or 99 degrees F; lower temperature slows or stops the digestive functions. Lukewarm is the best temperature for food. Food that we don't digest becomes available to support the growth of bacteria, which can cause gas and toxic effects.
    \end{answer}
\end{qaexchange}

\begin{qaexchange}{Milk}
    \metadata{topic={Prolactin in Milk}, source={Ray Peat Forum}}

    \begin{question}
        How is prolactin in milk not a problem for adults drinking it, especially in large amounts?
    \end{question}

    \begin{answer}
        Digestion usually inactivates protein hormones, and processes such as pasteurization would probably do the same.
    \end{answer}
\end{qaexchange}

\begin{qaexchange}{Dairy}
    \metadata{topic={Yogurt and Probiotics}, source={Ray Peat Forum}}

    \begin{question}
        I think I read were you do not recommend eating yogurt; is that correct? Are you against any kind of probiotics?
    \end{question}

    \begin{answer}
        In quantities of an ounce or so, for flavoring, it's o.k., but the lactic acid content isn't good if you are using yogurt as a major source of your protein and calcium. Cottage cheese, that is, milk curds with salt, is very good, if you can find it without additives, but traditional cottage cheese was almost fat-free, so when they make it with whole milk you should watch for other innovations that might not be beneficial.
    \end{answer}
\end{qaexchange}

\begin{qaexchange}{Dairy}
    \metadata{topic={Casein Intolerance}, source={Ray Peat Forum}}

    \begin{question}
        We're finding more and more people who are casein
intolerant and are unable to tolerate even raw grass fed fermented
milk. Given the PUFA issues you've mentioned with eggs, if a person
is unable to include milk in their diet and is minimizing their egg
intake, will they get enough vitamin A \& D from beef liver?
    \end{question}

    \begin{answer}
        Yes, beef liver has so much of the oily vitamins that it
just takes an occasional meal to meet those requirements generously.
The charts have stopped giving its vitamin E content, and rarely
mention vitamin K, but it's very good for those. Charts still don't
reflect the intracellular (lipid soluble dehydro-) form of vitamin
C, but liver is a good source of that too.
    \end{answer}
\end{qaexchange}

\begin{standalonequote}{Dairy Products}
    \metadata{topic={Safe Cheese Brands}, source={Ray Peat Forum}}

    \begin{answer}
      Tillamook vintage white sharp cheddar (2 or 3 years aged), Parmigiano reggiano, and pecorino romano are currently the safest. The Tillamook cheese process has been changed, so only what's currently aging has the traditional composition.
    \end{answer}
\end{standalonequote}

\begin{standalonequote}{Dairy Products}
    \metadata{topic={CLA, Butyric Acid in Dairy}, source={Ray Peat Forum}}

    \begin{answer}
      They [conjugated linoleic acid (CLA) and butyric acid] aren't necessary, but the CLA in cream and butter are probably responsible for some of their good effects. I use 1\% milk, some butter and hydrogenated coconut oil, to keep polyunsaturated fats to a minimum.
    \end{answer}
\end{standalonequote}

\begin{qaexchange}{Dairy Products}
    \metadata{topic={Safe Cheese Types}, source={Ray Peat Forum}}

    \begin{question}
        Are there any cheeses that you like besides Parmigiano-Reggiano for variety?
    \end{question}

    \begin{answer}
      Anything without harmful additives. Mozzarella and Oaxaca, and some kinds of feta, often don't have the harmful additives.
    \end{answer}
\end{qaexchange}

\begin{standalonequote}{Dairy Products}
    \metadata{topic={Cheese for Stress Management}, source={Ray Peat Forum}}

    \begin{answer}
      I have noticed that, when my thyroid activity has been good and my glycogen stores are high, salty fat cheese helps me to tolerate stress, but with lower thyroid function and poor glycogen stores, sugar is essential for keeping stress under control.
    \end{answer}
\end{standalonequote}

\begin{standalonequote}{Dairy Products}
    \metadata{topic={Milk Calcium Benefits}, source={Ray Peat Forum}}

    \begin{answer}
      The calcium, accompanied by some saturated fat, is a major benefit of milk and cheese.
    \end{answer}
\end{standalonequote}

\begin{emailexchange}{Dairy Products}
    \metadata{topic={Dairy and Cancer Risk}, source={Ray Peat Forum}}

    \begin{question}
        Do you think the articles that are claiming dairy causes breast and prostate cancer have any truth to them?
    \end{question}

    \begin{answer}
      Most of the evidence suggests that it increases prostate cancer but decreases other cancers, especially colon, and is protective in other ways.
    \end{answer}

    \begin{question}
        If milk increases prostate cancer risk, why do you think its safe to consume?
    \end{question}

    \begin{answer}
       If it decreases many other risks, it's healthful.
    \end{answer}
\end{emailexchange}

\begin{qaexchange}{Dairy Products}
    \metadata{topic={Beta-Casomorphin-7 in Milk}, source={Ray Peat Forum}}

    \begin{question}
        The opioid peptide beta-casomorphine-7 from A1 milk worries me, do you have any thoughts on this since you haven't spoken fondly of opioids?
    \end{question}

    \begin{answer}
      The experiments involved injecting it into brains. It's a peptide derived from partial digestion of casein; good digestion should reduce it to amino acids.
    \end{answer}
\end{qaexchange}

\begin{qaexchange}{Dairy Products}
    \metadata{topic={Vegetarian Rennet Cheese}, source={Ray Peat Forum}}

    \begin{question}
        Do you think it is safe to regularly consume cheese made with enzymes/vegetarian rennet if it is well tolerated? Or, will regular consumption cause some damage long term? 
    \end{question}

    \begin{answer}
      I think it increases the risk of future inflammation.
    \end{answer}
\end{qaexchange}

\begin{qaexchange}{Dairy Products}
    \metadata{topic={Milk Consumption, Casein vs Whey}, source={Ray Peat Forum}}

    \begin{question}
        How much milk do you drink per day? do you think casein proteins are superior to whey proteins? or are both proteins equally valuable?
    \end{question}

    \begin{answer}
      I have averaged two quarts a day for a long time. The method of separating whey from casein determines how much calcium each has; neither by itself is as good.
    \end{answer}
\end{qaexchange}

\begin{qaexchange}{Dairy Products}
    \metadata{topic={Freezing Milk}, source={Ray Peat Forum}}

    \begin{question}
      If I buy large amounts of good quality fresh/raw goat's or cow's milk - would the freezing of the milk destroy it's beneficial properties or will it be negligible? 
    \end{question}

    \begin{answer}
      It's negligible
    \end{answer}
\end{qaexchange}

\begin{qaexchange}{Dairy Products}
    \metadata{topic={Milk Temperature, Digestion}, source={Ray Peat Forum}}

    \begin{question}
        Do you drink your milk cold? Does the milk temperature affect the digestion of the milk?
    \end{question}

    \begin{answer}
      I usually prefer it at room temperature; too much cold food at once can sometimes affect the heart rhythm.
    \end{answer}
\end{qaexchange}

\begin{qaexchange}{Dairy Products}
    \metadata{topic={Raw vs Pasteurized Milk}, source={Ray Peat Forum}}

    \begin{question}
        Does raw milk have any benefits over pasteurized, and if one is able to obtain it, is it better even though the bacteria could potentially kill?
    \end{question}

    \begin{answer}
      It's important to get it from a clean dairy. Pasteurization reduces the content of vitamin B\textsubscript{2} and B\textsubscript{12} and folic acid very slightly, but the main differences are that raw milk should taste better, and it generally doesn't have the additives that are in most commercial milk.
    \end{answer}
\end{qaexchange}

\subsection{Meat \& Animal Products}

\begin{standalonequote}{Meat \& Animal Products}
    \metadata{topic={Egg Cooking Methods}, source={Email Wiki}}

    \begin{answer}
        I think soft boiling eggs is probably best. Scrambling them probably does cause some heat damage, but the difference in vitamin content is too small to matter.
    \end{answer}
\end{standalonequote}

\begin{qaexchange}{Meat \& Animal Products}
    \metadata{topic={Raw Egg Yolks Safety}, source={Email Wiki}}

    \begin{question}
        Raw egg yolks OK?
    \end{question}

    \begin{answer}
        Yes, eggnogs for example.
    \end{answer}
\end{qaexchange}

\begin{standalonequote}{Meat \& Animal Products}
    \metadata{topic={Balancing Egg Protein}, source={Email Wiki}}

    \begin{answer}
        To use the protein of 2 eggs efficiently it would be good to have a glass of milk and a large glass of orange juice.
    \end{answer}
\end{standalonequote}

\begin{standalonequote}{Meat \& Animal Products}
    \metadata{topic={Pastured Eggs Quality}, source={Email Wiki}}

    \begin{answer}
        There can be a great difference between eggs from chickens that really have adequate pasture, and the standard ones, but the labels aren't likely to contain enough information. 'Organic-free range' chickens in the US are usually fed soy and corn in a crowded outdoor pen. In the US I seldom eat more than one large egg per day, in Mexico where I know where the chickens live and what they eat, I eat more of them.
    \end{answer}
\end{standalonequote}

\begin{standalonequote}{Meat \& Animal Products}
    \metadata{topic={Chicken Bone Arsenic}, source={Email Wiki}}

    \begin{answer}
        In the US, chickens are fed arsenic to make them grow faster, and it concentrates in the bones; you should find out what the chicken feeding practice is in your area.
    \end{answer}
\end{standalonequote}

\begin{standalonequote}{Meat \& Animal Products}
    \metadata{topic={Meat Phosphate Problems}, source={Email Wiki}}

    \begin{answer}
        Meat contains too much phosphate, which destabilizes everything, and energy depletion has similar effects.
    \end{answer}
\end{standalonequote}

\begin{standalonequote}{Meat \& Animal Products}
    \metadata{topic={Aged Meat Polyamines}, source={Email Wiki}}

    \begin{answer}
        In the US, there is a widespread meat cult, that insists meat should be stored for two weeks before it's sold; it's convenient for the corporations that want everything to have an indefinitely long shelf-life, but it's bad for the public health. 150 years ago, when refrigeration was rare, the \enquote{high} flavor of meat was considered to be good, and people who were used to eating the half rotten stuff shaped the meat culture, and people looked for a \enquote{scientific} rationale for keeping meat in storage until it lost its fresh taste. The rationale is that it becomes tender, as the enzymes cause the meat to digest itself. That process starts after the glucose and glycogen in the muscle have been depleted, and the collagen and other proteins begin to be degraded. Besides losing the amino acid balance of fresh meat, the products include the cancer-promoting polyamines. Liver contains far more of the self-digesting enzymes than muscles do, and its glycogen is depleted in just a few hours. This is why liver in the US tastes so terrible. Since liver and eggs contain many of the same essential nutrients in high concentration, and eggs don't digest themselves, that's why I eat a few eggs in the US, despite their known high content of PUFA. When I can avoid the PUFA, I do; and in Mexico, liver and other meats aren't stored, except maybe in the supermarkets that serve foreigners.
    \end{answer}
\end{standalonequote}

\begin{standalonequote}{Meat \& Animal Products}
    \metadata{topic={Bacon}, source={Email Wiki}}

    \begin{answer}
        The nitrate isn't likely to be a problem if you eat it with orange juice. I fry the bacon to remove some of the fat, and then refry it in coconut oil, to remove most of the PUFA.
    \end{answer}
\end{standalonequote}

\begin{standalonequote}{Meat \& Animal Products}
    \metadata{topic={Cooked Meat}, source={Email Wiki}}

    \begin{answer}
        But the juices from inside quick cooked meat inactivate those toxins; the inside should be pink. Milk, cheese, and eggs are better proteins, anyway, because of the better calcium/phosphate ratio.
    \end{answer}
\end{standalonequote}

\begin{standalonequote}{Meat \& Animal Products}
    \metadata{topic={Liver Eating Frequency}, source={Email Wiki}}

    \begin{answer}
        Yes, two to four times a month.
    \end{answer}
\end{standalonequote}

\begin{standalonequote}{Meat \& Animal Products}
    \metadata{topic={Liver Cooking Method}, source={Email Wiki}}

    \begin{answer}
        I cook it quickly in butter.
    \end{answer}
\end{standalonequote}

\begin{standalonequote}{Meat \& Animal Products}
    \metadata{topic={Liver Substitutes}, source={Email Wiki}}

    \begin{answer}
        A combination of eggs and oysters would cover the main nutrients, but not as well.
    \end{answer}
\end{standalonequote}

\begin{standalonequote}{Meat \& Animal Products}
    \metadata{topic={Liver Extract Preparation}, source={Email Wiki}}

    \begin{answer}
        I made extracts myself, and there was a lot of it, but I didn't measure it exactly, just a few milligrams per kilo.
    \end{answer}
\end{standalonequote}

\begin{qaexchange}{Meat \& Animal Products}
    \metadata{topic={Liver Frequency Convenience}, source={Email Wiki}}

    \begin{question}
        Liver/oysters once a week versus every day - is it just convenience?
    \end{question}

    \begin{answer}
        Convenience, because of the time preparing things.
    \end{answer}
\end{qaexchange}

\begin{qaexchange}{Meat \& Animal Products}
    \metadata{topic={Desiccated Liver Quality}, source={Email Wiki}}

    \begin{question}
        Do you think Argentina Liver Powder is okay for supplementing vit B\textsubscript{6}?
    \end{question}

    \begin{answer}
        Probably, but the dehydration probably damages it nutritionally.
    \end{answer}
\end{qaexchange}

\begin{standalonequote}{Meat \& Animal Products}
    \metadata{topic={Liver Vitamin E Content}, source={Email Wiki}}

    \begin{answer}
        Overcooking destroys many nutrients. The vitamin E content is much lower in grain-fed beef than in grass-fed.
    \end{answer}
\end{standalonequote}

\begin{standalonequote}{Meat \& Animal Products}
    \metadata{topic={Organ Meats Considerations}, source={Email Wiki}}

    \begin{answer}
        Brains do contain beneficial steroids, but the other fats aren't necessarily good, so I don't recommend them especially as an isolated food.
    \end{answer}
\end{standalonequote}

\begin{qaexchange}{Meat \& Animal Products}
    \metadata{topic={Kidney Organ Meat}, source={Email Wiki}}

    \begin{question}
        Do you have any misgivings about eating beef kidney once a week?
    \end{question}

    \begin{answer}
        Yes, I never smelled one that I wanted to eat.
    \end{answer}
\end{qaexchange}

\begin{qaexchange}{Meat}
    \metadata{topic={Sweetbreads}, source={Ray Peat Forum}}

    \begin{question}
        Are sweetbreads (hypothalmus) of grass-fed veal beneficial for thyroid?
    \end{question}

    \begin{answer}
        Pancreas is the traditional sweatbread, thymus is sometimes sold as neck sweatbread. They aren't especially good for the thyroid.
    \end{answer}
\end{qaexchange}

\begin{qaexchange}{Meat}
    \metadata{topic={Gelatin and Taurine}, source={Ray Peat Forum}}

    \begin{question}
        If I react badly to glycine and gelatin, but I can tolerate taurine just fine, can the taurine be taken with muscle meat to reduce the muscle meat anti-thyroid effects instead of glycine/gelatin?
    \end{question}

    \begin{answer}
        I don't think it will have a similar effect, but it might provide its own unique benefit.
    \end{answer}
\end{qaexchange}

\begin{qaexchange}{Gelatin}
    \metadata{topic={Bone Marrow Collagen}, source={Ray Peat Forum}}

    \begin{question}
        I bought this bone marrow collagen, and wondered if it is MORE beneficial than standard gelatin, do you know?
    \end{question}

    \begin{answer}
        Marrow contains little collagen, but lots of iron. Cartilage extract is good, but I don't think it's good to include marrow.
    \end{answer}
\end{qaexchange}

\begin{standalonequote}{Animal Products}
    \metadata{topic={Pork Rinds Preparation}, source={Ray Peat Forum}}

    \begin{answer}
      I soak mine in hot coconut oil and then drain them, to exchange some of the fat.
    \end{answer}
\end{standalonequote}

\begin{standalonequote}{Meat \& Animal Products}
    \metadata{topic={Liver Consumption Frequency}, source={Ray Peat Forum}}

    \begin{answer}
      I think six ounces a week is enough, and safe. Drinking coffee with it reduces iron absorption.
    \end{answer}
\end{standalonequote}

\begin{qaexchange}{Meat \& Animal Products}
    \metadata{topic={Eggs Biological Factors}, source={Ray Peat Forum}}

    \begin{question}
        Is there anything in eggs besides the nutrients that justifies their PUFA?
    \end{question}

    \begin{answer}
      I think eggs contain \enquote{biological} factors of various kinds besides the specific essential nutrients.
    \end{answer}
\end{qaexchange}

\begin{qaexchange}{Meat \& Animal Products}
    \metadata{topic={Eating Animal Brains}, source={Ray Peat Forum}}

    \begin{question}
        What are your thoughts on eating ruminant animal brain?
    \end{question}

    \begin{answer}
       The steroids are probably valuable; although I'm skeptical of the mad cow disease theory, I don't eat ruminant brains. 
    \end{answer}
\end{qaexchange}

\begin{standalonequote}{Meat \& Animal Products}
    \metadata{topic={Egg Timing, Blood Glucose}, source={Ray Peat Forum}}

    \begin{answer}
      I think it's best to start the day with carbohydrates and milk, before having eggs. When glycogen stores are low, the high quality protein of eggs can lower the blood glucose. That's more likely to happen if thyroid and vitamin D are low.
    \end{answer}
\end{standalonequote}

\begin{emailexchange}{Meat \& Animal Products}
    \metadata{topic={Wild Pig Radioactivity}, source={Ray Peat Forum}}

    \begin{question}
        Do you think there's anything special about wild pig meat? Maybe in terms of hormones or fat content, etc.?
    \end{question}

    \begin{answer}
      I have heard that wild pigs in Europe are still highly contaminated with Chernobyl radioactivity.
    \end{answer}

    \begin{question}
        So it seems like the EU allows a maximum of 600 bec/kg. Do you think that's a reasonable number?
    \end{question}

    \begin{answer}
      I think Italy didn't get the radioactive fallout from Chernobyl, so their sausage should be safer.
    \end{answer}
\end{emailexchange}

\subsection{Seafood}

\begin{standalonequote}{Seafood}
    \metadata{topic={Fish Head Soup For Thyroid}, source={Email Wiki}}

    \begin{answer}
        Yes, people used to get very significant amounts from fish heads, chicken necks, various stews and sausages. I knew Norwegians who lived in fishing villages and ate fish head soup every week who said all their relatives were healthy into their 90s.
    \end{answer}
\end{standalonequote}

\begin{standalonequote}{Seafood}
    \metadata{topic={Squid And Goat's Milk}, source={Email Wiki}}

    \begin{answer}
        Yes, squid is very good, with selenium, copper, etc. Some people say that goat's milk is good after they have had trouble with cow's milk. The food the animals eat can contribute allergens to the milk. I use pasteurized milk because the dairies in this region with raw milk happen to use feed that give the milk a bad taste. If you think you might have a real milk allergy, you should start with just a drop or a sip. Adding sugar or honey (if you aren't allergic to honey) will decrease any allergic reaction to the milk.
    \end{answer}
\end{standalonequote}

\begin{qaexchange}{Seafood}
    \metadata{topic={Seafood Cooking Methods}, source={Email Wiki}}

    \begin{question}
        What is the best/safest way to cook seafood? I would assume boiling, steaming or frying in coconut oil.
    \end{question}

    \begin{answer}
        Those, or butter.
    \end{answer}
\end{qaexchange}

\begin{qaexchange}{Seafood}
    \metadata{topic={Fukushima Radiation Concern}, source={Email Wiki}}

    \begin{question}
        Is the quality of seafood, or other foods, a concern after the Fukushima nuclear disaster, for someone living in the West Coast? Could it have become irradiated and pose dangers to health?
    \end{question}

    \begin{answer}
        Yes, sea food from the northern Pacific should be tested periodically, but the US government has stopped the radiation testing that had previously been done, which I think means that the radiation is exceeding their previous safety limits.
    \end{answer}
\end{qaexchange}

\begin{emailexchange}{Seafood}
    \metadata{topic={Oysters}, source={Ray Peat Forum}}

    \begin{question}
        I know you recommend shellfish once a week, maybe more now, are oysters recommended more than once a week, and how many would be healthy until they become too much?
    \end{question}

    \begin{answer}
        More than one serving of oyster per week might provide too much iron.
    \end{answer}

    \begin{question}
        How many regular sized oysters would you consider one serving?
    \end{question}

    \begin{answer}
        I think two or three are enough.
    \end{answer}
\end{emailexchange}

\begin{standalonequote}{Seafood}
    \metadata{topic={Shellfish for Copper}, source={Ray Peat Forum}}

    \begin{answer}
        Having shrimp, oysters, or other shellfish once or twice a week will correct a deficiency.
    \end{answer}
\end{standalonequote}

\begin{emailexchange}{Seafood}
    \metadata{topic={Smoked Oysters in Oil}, source={Ray Peat Forum}}

    \begin{question}
        Do you think it would still be beneficial to eat oysters of somebody's only source was canned, smoked oysters in sunflower oil?
    \end{question}

    \begin{answer}
      After draining them thoroughly, I think once or twice a month is good.
    \end{answer}

    \begin{question}
        What would you say the limiting factor of stopping someone eating them much more often was? The PUFA, the smoking, the canning, or a combination of these or something else?
    \end{question}

    \begin{answer}
      The smoke and oil together would make any carcinogens in the smoke be absorbed.
    \end{answer}
\end{emailexchange}

\begin{standalonequote}{Seafood}
    \metadata{topic={Seafood Frequency for Selenium/Iodine}, source={Ray Peat Forum}}

    \begin{answer}
       I think having seafood once or twice a week is best for selenium and iodine.
    \end{answer}
\end{standalonequote}

\begin{qaexchange}{Seafood}
    \metadata{topic={Scallops vs Imitation Seafood}, source={Ray Peat Forum}}

    \begin{question}
         Is there something unique about scallops as a protein source compared to others?
    \end{question}

    \begin{answer}
      They are similar to other mollusks nutritionally. I stopped mentioning them years ago when there were a lot of imitation scallops on the market. More recently I accidentally got some imitation calamari steaks. It's easier to mention the things that are less likely to be fabricated out of junk and glue.
    \end{answer}
\end{qaexchange}

\begin{qaexchange}{Seafood}
    \metadata{topic={Oyster Extract Toxicity}, source={Ray Peat Forum}}

    \begin{question}
        Do you think oyster extract/powder is an okay replacement, if one can't stomach cooked oysters?
    \end{question}

    \begin{answer}
      No, dehydration produces some toxic materials.
    \end{answer}
\end{qaexchange}

\begin{qaexchange}{Seafood}
    \metadata{topic={Fish Head Soup for Thyroid}, source={Ray Peat Forum}}

    \begin{question}
        I was thinking of using heads of European sprat fish to make a fish head supp for additional thyroid. They are quite small. Do you think that's a good fish to make a fish head soup from?
    \end{question}

    \begin{answer}
      They have a high fat content that could antagonize thyroid function. Supplementing thyroid would be safer and more reliable.
    \end{answer}
\end{qaexchange}

\section{Foods to Minimize or Avoid}

\begin{standalonequote}{Foods to Minimize or Avoid}
    \metadata{topic={Citric Acid}, source={Email Wiki}}

    \begin{answer}
        I'm looking for one without the citric acid. The benzoate isn't necessary with a lot of added sugar, but it's probably used everywhere.
    \end{answer}
\end{standalonequote}

\begin{standalonequote}{Foods to Minimize or Avoid}
    \metadata{topic={Cellulose Derivative Persorption}, source={Email Wiki}}
    \begin{note}
        Hydroxypropyl Methylcellulose (Hypromellose)
    \end{note}

    \begin{answer}
        I think it's good to avoid it when possible; there's an article by Gerhart Volkheimer, on persorption, that explains how particulate matter of all sorts can enter the blood stream from the intestine.
    \end{answer}
\end{standalonequote}

\begin{standalonequote}{Foods to Minimize or Avoid}
    \metadata{topic={Allergenic Foods List}, source={Email Wiki}}

    \begin{answer}
        Buckwheat is considered to be inflammatory. Potatoes are allergenic for quite a few people; honey (depending on the plants the bees used) and mangoes are other things to consider [as allergens].
    \end{answer}
\end{standalonequote}

\begin{standalonequote}{Foods to Minimize or Avoid}
    \metadata{topic={Nicotine Age-Dependent Effects}, source={Email Wiki}}

    \begin{answer}
        In old people, a little nicotine can have a balancing effect, improving alertness, and probably protecting nerves, for example in the negative association with Parkinson's disease. But in younger people, its vasoconstrictive effect tends to promote the development of wrinkles in the skin, and I think it's likely to contribute to periodontal disease.
    \end{answer}
\end{standalonequote}

\begin{standalonequote}{Foods to Minimize or Avoid}
    \metadata{topic={Sorbitol Safety}, source={Email Wiki}}

    \begin{answer}
        A little is o.k.
    \end{answer}
\end{standalonequote}

\begin{standalonequote}{Foods to Minimize or Avoid}
    \metadata{topic={Stevia Arsenic Contamination}, source={Email Wiki}}

    \begin{answer}
        Stevia extract is probably safe. The plant is often highly contaminated with arsenic.
    \end{answer}
\end{standalonequote}

\begin{standalonequote}{Foods to Minimize or Avoid}
    \metadata{topic={Foods to Avoid, Inflammation}, source={Ray Peat Forum}}

    \begin{answer}
      Starch should be avoided. Chocolate is highly allergenic for many people. Some ice creams are incredibly bad. Haagen Dasz seems to be o.k., vanilla and coffee, anyway. Sea food and meats are easy to digest; starches are the worst things for inflammation.
    \end{answer}
\end{standalonequote}

\subsection{Polyunsaturated Fats (PUFAs)}

\begin{standalonequote}{Polyunsaturated Fats (PUFAs)}
    \metadata{topic={Unavoidable PUFA In Diet}, source={Email Wiki}}

    \begin{answer}
        Our foods usually contain enough PUFA, unavoidably, to make fats matter to some extent. After about twenty years of carefully avoiding them, I'm still getting about 2\% of my fat as PUFA (beef, oysters, eggs, etc.). That's why I'm making an effort to increase my sugar intake, to displace some fat.
    \end{answer}
\end{standalonequote}

\begin{qaexchange}{Polyunsaturated Fats (PUFAs)}
    \metadata{topic={Omega-6 For Eczema}, source={Email Wiki}}

    \begin{question}
        Do you recommend someone try omega 6 supplementation from say safflower oil in an extreme case on eczema? Or will the omega 6 appear to heal the eczema because of lowered metabolism?
    \end{question}

    \begin{answer}
        Slowing metabolism and causing inflammation are its two basic functions.
    \end{answer}
\end{qaexchange}

\begin{standalonequote}{Polyunsaturated Fats (PUFAs)}
    \metadata{topic={PUFA Detoxification}, source={Email Wiki}}

    \begin{answer}
        The main way they [PUFA] are detoxified is by attaching glucuronic acid, making them water soluble, so they leave in the urine. Keeping the free fatty acids low in relation to albumin, they will largely be carried bound to the albumin to the liver.
    \end{answer}
\end{standalonequote}

\begin{standalonequote}{Polyunsaturated Fats (PUFAs)}
    \metadata{topic={PUFA Mobilization From Fat Stores}, source={Email Wiki}}

    \begin{answer}
        They [polyunsaturated fats] are more water soluble, so are easier to release [from adipose tissue]. The fat cells themselves preferentially oxidize saturated fatty acids, so the stores tend to become more unsaturated with age.
    \end{answer}
\end{standalonequote}

\begin{standalonequote}{Polyunsaturated Fats (PUFAs)}
    \metadata{topic={Acrylamide From PUFA}, source={Email Wiki}}

    \begin{answer}
        PUFA (omega-3 and -6 oils, also called polyunsaturated fatty acids) break down into several toxic things, including acrolein, which oxidizes to form acrylate, and both of them react with ammonia or amines to form acrylamide. I haven't read the article, but it does seem odd that they would think the starch was the source of the acrylamide.
    \end{answer}
\end{standalonequote}

\begin{standalonequote}{Polyunsaturated Fats (PUFAs)}
    \metadata{topic={PUFA Tissue Clearance Time}, source={Email Wiki}}

    \begin{answer}
        When the polyunsaturated fats in the diet are reduced, the amount of them stored in the tissues decreases for about four years, making it progressively easier to keep the metabolic rate up, and stress hormones down.
    \end{answer}
\end{standalonequote}

\begin{standalonequote}{Polyunsaturated Fats (PUFAs)}
    \metadata{topic={PUFA Slowing Metabolism}, source={Email Wiki}}

    \begin{answer}
        It's the stored PUFA, released by stress or hunger, that slow metabolism. Niacinamide helps to lower free fatty acids, and good nutrition will allow the liver to slowly detoxify the PUFA, if it isn't being flooded with large amounts of them. A small amount of coconut oil with each meal will increase the ability to oxidize fat, by momentarily stopping the antithyroid effect of the PUFA. Aspirin is another thing that reduces the stress-related increase of free fatty acids, stimulating metabolism. Taking a thyroid supplement is reasonable until the ratio of saturated fats to PUFA is about 2 to 1.
    \end{answer}
\end{standalonequote}

\begin{qaexchange}{PUFAs}
    \metadata{topic={PUFA Depletion}, source={Ray Peat Forum}}

    \begin{question}
        Around the internet it has been said that you recommend a number of 0.5 grams of pufa a day as the upper limit for 'pufa depletion'. Is this number correct? And what would be the benefit of this as opposed to, say 2 grams?
    \end{question}

    \begin{answer}
        It's a lot easier to manage a diet that isn't so extreme.
    \end{answer}
\end{qaexchange}

\begin{standalonequote}{PUFAs}
    \metadata{topic={Sugar Cravings}, source={Ray Peat Forum}}

    \begin{note}
        Sugar Cravings Subsided After Stopping PUFA Consumption
    \end{note}

    \begin{answer}
        That happens when your blood sugar is stable, and your liver is storing enough glycogen so that it can continuously make the T\textsubscript{3}, needed for efficient use of the sugar.
    \end{answer}
\end{standalonequote}

\begin{qaexchange}{Polyunsaturated Fats (PUFAs)}
    \metadata{topic={PUFA Accumulation with Age}, source={Ray Peat Forum}}

    \begin{question}
        You have repeatedly mentioned PUFA accumulates with aging. My question is, if someone in their 20's started eating a diet with less than 2 grams of PUFA per day, and maintained that low PUFA intake permanently into old age, would PUFA still accumulate with age?
    \end{question}

    \begin{answer}
      I think some would still accumulate in the fat tissues, unless total fat intake was low, and PUFA intake was half a gram or less.
    \end{answer}
\end{qaexchange}

\begin{qaexchange}{Polyunsaturated Fats (PUFAs)}
    \metadata{topic={Lecithin in Chocolate}, source={Ray Peat Forum}}

    \begin{question}
        Looking through ingredients lists of chocolate, many contain either soy or sunflower lecithin. Do you think either of these is less harmful than the other?
    \end{question}

    \begin{answer}
       It takes just a small amount to emulsify it, so I don't think it would be harmful. When an oil isn't organic, I think the issue is the oil- or water-solubility of the herbicide or insecticide commonly used on the crop, and on that basis I would choose soy over sunflower, since the chemicals used on sunflowers are commonly oil soluble, and would tend to stay in the oil fraction in refining.
    \end{answer}
\end{qaexchange}

\begin{qaexchange}{Polyunsaturated Fats (PUFAs)}
    \metadata{topic={Fermenting Nuts, PUFA Conversion}, source={Ray Peat Forum}}

    \begin{question}
        Will fermenting nuts and seeds biohydrogenate the PUFA into SFA? Or does this require a ruminant animal?
    \end{question}

    \begin{answer}
      The rumen can reach more than 95\% conversion, other systems with lots of vitamin E maybe about 2\%.
    \end{answer}
\end{qaexchange}

\begin{qaexchange}{Polyunsaturated Fats (PUFAs)}
    \metadata{topic={Palm Oil, Sardines Safety}, source={Ray Peat Forum}}

    \begin{question}
        Do you think potato chips in palm oil are safe 2-3x a month? The same question goes for some sardines in olive oil?
    \end{question}

    \begin{answer}
      Sardines in olive oil are safe enough to have about that often; palm oil has a lot of PUFA, probably isn't so safe.
    \end{answer}
\end{qaexchange}

\subsection{Grains \& Gluten}

\begin{qaexchange}{Grains \& Gluten}
    \metadata{topic={Gluten Sensitivity Testing}, source={Email Wiki}}

    \begin{question}
        Test for gluten sensitivity?
    \end{question}

    \begin{answer}
        Endoscopy and biopsy can identify it, blood tests are often not reliable. Home made, slowly leavened bread contains much less gluten than the instantly leavened bread most bakeries sell.
    \end{answer}
\end{qaexchange}

\begin{qaexchange}{Grains \& Gluten}
    \metadata{topic={White Bread Gluten Effects}, source={Email Wiki}}

    \begin{question}
        Can white bread cause intestinal irritation and endotoxins?
    \end{question}

    \begin{answer}
        Yes, white bread is usually artificially leavened, and so contains unmodified starch and gluten.
    \end{answer}
\end{qaexchange}

\begin{qaexchange}{Grains \& Gluten}
    \metadata{topic={Gluten Sensitivity Precautions}, source={Email Wiki}}

    \begin{question}
        Follow up: okay to eat on occasion or is it likely to cause allergic reactions if gluten sensitive?
    \end{question}

    \begin{answer}
        Yes, it's definitely a problem for people who are sensitive to gluten. Home-made bread, that's soaked for about 12 hours during leavening, is much safer than the typical commercial bread.
    \end{answer}
\end{qaexchange}

\begin{standalonequote}{Grains \& Gluten}
    \metadata{topic={Zero Starch For Sensitive Intestines}, source={Email Wiki}}

    \begin{answer}
        For people with really sensitive intestines or bad bacteria, starch should be zero.
    \end{answer}
\end{standalonequote}

\begin{qaexchange}{Grains \& Gluten}
    \metadata{topic={Tapioca Starch Endotoxins}, source={Ray Peat Forum}}

    \begin{question}
        Is tapioca starch, presumably used as a thickening ingredient, something that may cause endotoxins?
    \end{question}

    \begin{answer}
      Yes, any starch can; long cooking, and butter or cream, can reduce that effect.
    \end{answer}
\end{qaexchange}

\begin{qaexchange}{Grains \& Gluten}
    \metadata{topic={Popcorn as Fiber}, source={Ray Peat Forum}}

    \begin{question}
        Is organic popcorn a good fiber?
    \end{question}

    \begin{answer}
      If it's popped in coconut oil with butter it's a fairly safe food, but the high starch content has its drawbacks.
    \end{answer}
\end{qaexchange}

\begin{standalonequote}{Grains \& Gluten}
    \metadata{topic={Coffee Increasing Starch Persorption}, source={Ray Peat Forum}}

    \begin{answer}
       It's why I stopped eating starch. He [Gerhard Volkheimer] thought it was the increased pressure from more vigorous peristalsis. He noticed that eating fat with it reduced persorption. Coffee is important in the diet, and since natural fats contain some PUFA, stopping starch eating most simply solves the problem of persorption. 
    \end{answer}
\end{standalonequote}

\begin{qaexchange}{Grains \& Gluten}
    \metadata{topic={Starch Persorption, Bacterial Growth}, source={Ray Peat Forum}}

    \begin{question}
        Is persorption a risk even with well-cooked starches? If not, what is it about starches (after cooking) that makes them hard on the gut?
    \end{question}

    \begin{answer}
      Natural starches are often embedded in other materials that resist digestion as well as cooking; that can cause them to support bacterial growth.
    \end{answer}
\end{qaexchange}

\subsection{Plant Toxins}

\begin{standalonequote}{Plant Toxins}
    \metadata{topic={Gum Allergenicity}, source={Email Wiki}}
    \begin{note}
        On Acacia Powder/Acacia Gum/Gum Arabic
    \end{note}

    \begin{answer}
        Yes, acacia gum is very allergenic. It has taken me two or three weeks to get over symptoms from it, when I accidentally got it in a food. But other irritating foods might keep the symptoms going, so it's good to watch for changes with different foods.
    \end{answer}
\end{standalonequote}

\begin{standalonequote}{Plant Toxins}
    \metadata{topic={Calcium Glucarate And Cruciferous Vegetables}, source={Email Wiki}}

    \begin{answer}
        I think calcium glucarate can be protective in some circumstances, but manufactured organic compounds (glucaric acid) often contain allergenic impurities. I practically stopped eating all cruciferous vegetables, largely because of that sort of compound---Indoles as a class are very risky. Thyroid and sugars, and saturated fats such as coconut oil, usually help to increase testosterone.
    \end{answer}
\end{standalonequote}

\begin{qaexchange}{Plant Toxins}
    \metadata{topic={Millet Goitrogens}, source={Ray Peat Forum}}

    \begin{question}
        Is there any validity to the claim that millet flour contains substances (i.e. \enquote{goitrogens}) that suppress thyroid function?
    \end{question}

    \begin{answer}
      It's only a diet dominated by goitrogenic foods that's a problem, small amounts don't have noticeable effects.
    \end{answer}
\end{qaexchange}

\begin{qaexchange}{Plant Toxins}
    \metadata{topic={Inflammatory Vegetables}, source={Ray Peat Forum}}

    \begin{question}
        Are you aware of any other vegetables with a reputation for causing inflammation?
    \end{question}

    \begin{answer}
       Green leaves used in salads very commonly support bacterial overgrowth; seeds, nuts, and grains, and some starchy vegetables, especially when they aren't cooked until they soften. 
    \end{answer}
\end{qaexchange}

\begin{emailexchange}{Plant Toxins}
    \metadata{topic={Lutein Sensitivity, Carotenoids}, source={Ray Peat Forum}}

    \begin{question}
        Do you think some people can be sensitive to lutein? I think some people have reported feeling depressed and tired after eating lutein, and I think I am responding similarly.
    \end{question}

    \begin{answer}
       Beta-carotene can have toxic effects when it accumulates, and since lutein isn't convertible to vitamin A it's likely to be more of a problem. Many of the foods with a lot of lutein have lots of other toxic components, so avoiding them might be protective for reasons other than lutein.
    \end{answer}
    
    \begin{question}
        So, could this mean that some people can be sensitive to orange juice, kale, chard, spinach, mustard greens, and eggs? 
    \end{question}

    \begin{answer}
      I think it's the whole array that has the effect, and citrus juice has so many antiinflammatory things I think they usually prevent harm from the small amount of carotenoids. With eggs, the concentrated protein itself can precipitate inflammation by lowering glucose. Leaves have multiple irritants and toxins.
    \end{answer}
\end{emailexchange}

\begin{emailexchange}{Plant Toxins}
    \metadata{topic={Banana Peel Tea Toxicity}, source={Ray Peat Forum}}

    \begin{question}
        Do you have any opinion on boiling banana peels and drinking the juice? It's supposedly a good source of magnesium, potassium, dopamine and some flavonoids. 
    \end{question}

    \begin{answer}
      Probably not a good idea.
    \end{answer}

    \begin{question}
        Because of pesticides?
    \end{question}

    \begin{answer}
      Including the banana's own toxic defensive chemicals.
    \end{answer}
\end{emailexchange}

\section{Food Preparation \& Cooking}

\begin{standalonequote}{Food Preparation \& Cooking}
    \metadata{topic={Cooking Methods And Cholesterol}, source={Email Wiki}}

    \begin{answer}
        Plant enzymes aren't much help after they are eaten. Slow cooking is the worst for oxidizing cholesterol, quick cooking is safer.
    \end{answer}
\end{standalonequote}

\begin{qaexchange}{Food Preparation \& Cooking}
    \metadata{topic={Broth Cooking Time}, source={Email Wiki}}

    \begin{question}
        How long to cook broths?
    \end{question}

    \begin{answer}
        It's mostly for the attached cartilage, ligaments, and tendons, and most of the gelatin is released in 3 or 4 hours. Excess cooking oxidizes nutrients, especially if there's marrow in the bone.
    \end{answer}
\end{qaexchange}

\begin{standalonequote}{Food Preparation \& Cooking}
    \metadata{topic={Ice Cream Recipe}, source={Email Wiki}}

    \begin{answer}
        I blend an egg (warmed to 40 degrees C) with a cup of sugar (also warmed) and a cup of coconut oil until it's smoothly emulsified, and maybe half a cup of powdered milk for extra texture, then add milk to fill the blender (total volume a little over a liter), with strong coffee or orange juice for flavor, or other fruit or vanilla, etc. The high oil content, and powdered milk, make it freeze without crystallizing, so the ice cream machine isn't necessary.
    \end{answer}
\end{standalonequote}

\begin{standalonequote}{Food Preparation \& Cooking}
    \metadata{topic={Milk Powder Pancake Recipe}, source={Ray Peat Forum}}

    \begin{answer}
       I just mix them according to the consistency that I want, a thick batter for puffy pancakes, with just milk powder and egg and a little salt, or a thinner batter, with a little liquid milk added, for crepes; fried in butter. 
    \end{answer}
\end{standalonequote}

\begin{qaexchange}{Food Preparation \& Cooking}
    \metadata{topic={Quick-Cooking Mushrooms}, source={Ray Peat Forum}}

    \begin{question}
        I know you advise people to boil their mushrooms at least one hour to remove toxins from them. What if I am in a hurry? Can I fry them right away in coconut oil or butter and then eat them? Would the heat in frying help remove the toxins too?
    \end{question}

    \begin{answer}
      Yes, when they are well heated their opacity changes to a slight translucence.
    \end{answer}
\end{qaexchange}

\begin{standalonequote}{Food Preparation \& Cooking}
    \metadata{topic={Mushroom Preparation for Safety}, source={Ray Peat Forum}}

    \begin{answer}
       Boiling sliced mushrooms for three hours extracts most of the beneficial things into the water, and I think the balance is favorable.
    \end{answer}
\end{standalonequote}

\begin{standalonequote}{Food Preparation \& Cooking}
    \metadata{topic={Sourdough Fermentation Timing}, source={Ray Peat Forum}}

    \begin{answer}
       It's the time of soaking that lets the enzymes break down the gluten, so I think the longer time is probably better, but I haven't had experience with longer than about 16 hours. The nature of the starter makes a difference, and adding sugar is a way to prevent the development of a bad taste. It's essential to use flour that hasn't been heat treated. 
    \end{answer}
\end{standalonequote}

\begin{emailexchange}{Food Preparation \& Cooking}
    \metadata{topic={Nixtamalizing Oats}, source={Ray Peat Forum}}

    \begin{question}
        Do you think soaking oats in a calcium hydroxide solution for 24h would increase their nutritional value (digestibility, phytate content, calcium) in a similar manner this process improves corn?
    \end{question}

    \begin{answer}
      I know someone who processed it and other grains, starting with boiling, and said they tasted good; I suppose there would be some synthesis of niacin and improved digestibility.
    \end{answer}

    \begin{question}
        Nixtamalization greatly increases the calcium content of corn. Would the oats incorporate similar amounts of calcium during the soaking?
    \end{question}

    \begin{answer}
      I think starting with boiling is important to accelerate the process; calcium will be retained.
    \end{answer}
\end{emailexchange}

\begin{standalonequote}{Food Preparation \& Cooking}
    \metadata{topic={Masa Harina Chips, Safe Starches}, source={Ray Peat Forum}}

    \begin{answer}
      If you make your own chips with masa harina fried in coconut oil, they are pretty safe. Having been cooked in alkali, it's already partly digested, so it's the safest kind of starch
    \end{answer}
\end{standalonequote}

\section{Beverages \& Special Foods}

\begin{standalonequote}{Beverages \& Special Foods}
    \metadata{topic={Alcohol Antioxidant vs Estrogen}, source={Email Wiki}}

    \begin{answer}
        Small amounts of alcohol can have some good antioxidant effects, but beer, wine, and dark whiskey, etc., contain enough estrogen to be harmful.
    \end{answer}
\end{standalonequote}

\begin{standalonequote}{Beverages \& Special Foods}
    \metadata{topic={Alcohol Individual Reactions}, source={Email Wiki}}

    \begin{answer}
        People have very different reactions to it, probably depending on thyroid activity. It can have an antioxidant effect, but it can also cause hypoglycemia with pro-oxidative effects. If a person eats polyunsaturated fats, alcohol is more likely to cause oxidative reactions between iron and the fats.
    \end{answer}
\end{standalonequote}

\begin{standalonequote}{Beverages \& Special Foods}
    \metadata{topic={Heavy Drinking And PUFA}, source={Email Wiki}}

    \begin{answer}
        Heavy drinking inhibits cellular respiration and sets up an inflammatory process, involving iron, which will still be harmful, but less so than in the presence of PUFA. If absolutely none of the dietary PUFA were in the body, no one really knows what that metabolic stress would do, maybe nothing cumulative.
    \end{answer}
\end{standalonequote}

\begin{qaexchange}{Beverages \& Special Foods}
    \metadata{topic={Mitigating Alcohol And Cannabis Effects}, source={Email Wiki}}

    \begin{question}
        Alcohol and cannabis? What is the best immediate measure to mitigate any harmful effects you see in these two drugs?
    \end{question}

    \begin{answer}
        Small amounts of alcohol can have some good antioxidant effects, but beer, wine, and dark whiskey, etc., contain enough estrogen to be harmful. Cannabis is antiandrogenic or estrogenic, but it can be protective in some situations. Protein, thyroid, sugars, and saturated fats are protective against both.
    \end{answer}
\end{qaexchange}

\begin{standalonequote}{Beverages \& Special Foods}
    \metadata{topic={Brewer's Yeast Preparation}, source={Email Wiki}}

    \begin{answer}
        I would use about a fourth of a cup, and let it stand until it finished settling, and then just pour off the clear yellow part.
    \end{answer}
\end{standalonequote}

\begin{standalonequote}{Beverages \& Special Foods}
    \metadata{topic={Brewer's Yeast Phosphate Content}, source={Email Wiki}}

    \begin{answer}
        Because of the phosphate, it depends on your need for it, and the amount of milk and cheese in your diet, etc.
    \end{answer}
\end{standalonequote}

\begin{standalonequote}{Beverages \& Special Foods}
    \metadata{topic={Caffeine With Food}, source={Email Wiki}}

    \begin{answer}
        Caffeine increases your metabolic rate, so it's important to take it with food, including enough sugar. Coffee and cocoa are very good magnesium sources. Cocoa contains both bromocriptine and caffeine, bromocriptine seems to be more stimulating to the heart than to the brain.
    \end{answer}
\end{standalonequote}

\begin{qaexchange}{Beverages \& Special Foods}
    \metadata{topic={High-Dose Caffeine Tablets}, source={Email Wiki}}

    \begin{question}
        Do you think it is safe to use caffeine tablet of 600 mg or more to treat fatty liver?
    \end{question}

    \begin{answer}
        That would be too much caffeine at once, unless it's with a big meal to slow its absorption.
    \end{answer}
\end{qaexchange}

\begin{standalonequote}{Beverages \& Special Foods}
    \metadata{topic={Cannabis Effects}, source={Email Wiki}}

    \begin{answer}
        Cannabis is antiandrogenic or estrogenic, but it can be protective in some situations.
    \end{answer}
\end{standalonequote}

\begin{standalonequote}{Beverages \& Special Foods}
    \metadata{topic={Cocoa Alkalizing Process}, source={Email Wiki}}

    \begin{answer}
        The idea is to remove the fat so that it mixes easily with milk.
    \end{answer}
\end{standalonequote}

\begin{standalonequote}{Beverages \& Special Foods}
    \metadata{topic={Coconut Water Quality}, source={Email Wiki}}

    \begin{answer}
        If it is fresh from the coconut, it's good, also if it has been bottled without additives.
    \end{answer}
\end{standalonequote}

\begin{standalonequote}{Beverages \& Special Foods}
    \metadata{topic={Coffee Storage And Allergens}, source={Email Wiki}}

    \begin{answer}
        A couple of times I have seen coffee that had been stored near herbs that made it slightly allergenic, but that could probably be noticed in the flavor.
    \end{answer}
\end{standalonequote}

\begin{standalonequote}{Beverages \& Special Foods}
    \metadata{topic={Coffee Magnesium Content}, source={Email Wiki}}

    \begin{answer}
        Dry instant coffee is close to 0.5\% magnesium, so a cup of strong coffee has about 40 mg. I make strong drip coffee.
    \end{answer}
\end{standalonequote}

\begin{standalonequote}{Beverages \& Special Foods}
    \metadata{topic={Instant vs Brewed Coffee}, source={Email Wiki}}

    \begin{answer}
        The antioxidants in very fresh coffee might have some special value, but I think instant coffee is on average just as good as brewed coffee. The high temperature of espresso gets the most caffeine, lower temperature processes get the minerals and vitamins (mostly niacin) and aroma, but a little less of the caffeine.
    \end{answer}
\end{standalonequote}

\begin{standalonequote}{Beverages \& Special Foods}
    \metadata{topic={Coffee With Food}, source={Email Wiki}}

    \begin{answer}
        It's important not to drink coffee on an empty stomach, it should always be with food, since it increases the metabolic rate, and can deplete glycogen stores.
    \end{answer}
\end{standalonequote}

\begin{standalonequote}{Beverages \& Special Foods}
    \metadata{topic={Organic vs Conventional Coffee}, source={Email Wiki}}

    \begin{answer}
        Organic coffee is preferable (in the coffee orchards I have seen no pesticides were needed), but the roasting process probably eliminates any added chemicals.
    \end{answer}
\end{standalonequote}

\begin{qaexchange}{Beverages \& Special Foods}
    \metadata{topic={Coconut Oil And Coffee Benefits}, source={Email Wiki}}

    \begin{question}
        How beneficial are coconut oil and coffee to a healthy person with a good diet?
    \end{question}

    \begin{answer}
        If the basic foods were chosen for minimal unsaturated fats, then coconut oil wouldn't add much of value. Coffee is a good source of magnesium and niacin, and has smaller amounts of other essential nutrients, besides the caffeine and antioxidants.
    \end{answer}
\end{qaexchange}

\begin{standalonequote}{Beverages \& Special Foods}
    \metadata{topic={Alternative Fruit Juices}, source={Email Wiki}}

    \begin{answer}
        A few other juices [other than orange juice] are good, for example watermelon. Some fruits contain things that affect the hormones.
    \end{answer}
\end{standalonequote}

\begin{standalonequote}{Beverages \& Special Foods}
    \metadata{topic={Pasteurized Juice Safety}, source={Email Wiki}}

    \begin{answer}
        Usually they are o.k.
    \end{answer}
\end{standalonequote}

\begin{standalonequote}{Beverages \& Special Foods}
    \metadata{topic={Juice Fluid Balance}, source={Email Wiki}}

    \begin{answer}
        Milk and fruit juice are osmotically balanced with minerals and sugar, so they don't cause imbalance of body fluids, the way drinking plain water can in a hypothyroid person. Many doctors have recommended drinking a certain amount of water every day, regardless of thirst, and that often causes problems in people with hormonal problems.
    \end{answer}
\end{standalonequote}

\begin{qaexchange}{Beverages \& Special Foods}
    \metadata{topic={Apple Juice Quality}, source={Email Wiki}}

    \begin{question}
        How to know if it's free of fungal contaminant?
    \end{question}

    \begin{answer}
        There are probably flavor indications, but I don't know whether anyone has studied that.
    \end{answer}
\end{qaexchange}

\begin{qaexchange}{Beverages \& Special Foods}
    \metadata{topic={Apple Juice Fiber Content}, source={Email Wiki}}

    \begin{question}
        Is there any other problem with apple juice besides starch and pectin?
    \end{question}

    \begin{answer}
        It looks like there isn't much fiber in store bought apple juice.
    \end{answer}
\end{qaexchange}

\begin{standalonequote}{Beverages \& Special Foods}
    \metadata{topic={Orange Juice Sweetness}, source={Email Wiki}}

    \begin{answer}
        If oranges aren't sweet, straining it won't prevent irritation.
    \end{answer}
\end{standalonequote}

\begin{standalonequote}{Beverages \& Special Foods}
    \metadata{topic={Orange Juice Enzyme Processing}, source={Email Wiki}}

    \begin{answer}
        Until 2006 I was using mostly frozen pulp-free concentrate, then they introduced the enzyme process (for disposing of waste fiber, making it stay suspended in the juice), affecting even the 'pulp-free' type. So now I use only sweet oranges that I squeeze myself. US people don't realize how ridiculously degraded their standard of living has become. Nutrition is political economical. The governments tell people to eat beans and bread for a reason. I use coca cola as a fill-in when I can't get oranges.
    \end{answer}
\end{standalonequote}

\begin{standalonequote}{Beverages \& Special Foods}
    \metadata{topic={Cooking Orange Juice Effects}, source={Email Wiki}}

    \begin{answer}
        Part of the value of sweet orange juice is its antiinflammatory, antioxidant, antiestrogenic effect, and cooking will change those effects to some extent. What would be the reason for reducing fluid intake?
    \end{answer}
\end{standalonequote}

\begin{qaexchange}{Beverages \& Special Foods}
    \metadata{topic={Canned Orange Juice Safety}, source={Email Wiki}}

    \begin{question}
        Do you think canned not from concentrate juice is ok or should it be avoided?
    \end{question}

    \begin{answer}
        If it's in glass it might be o.k. Even the so-called pulp-free juices now might have been processed with enzymes to liquefy pulp.
    \end{answer}
\end{qaexchange}

\begin{qaexchange}{Beverages \& Special Foods}
    \metadata{topic={Commercial OJ Enzyme Test}, source={Email Wiki}}

    \begin{question}
        What about commercial pulp-free OJ in a plastic container like florida's natural brand?
    \end{question}

    \begin{answer}
        If it separates into an orange-colored sediment and a nearly clear supernatant, it's probably natural, but even the so-called pulp-free juices now often contain the enzyme-solublized pulpy refuse.
    \end{answer}
\end{qaexchange}

\begin{standalonequote}{Beverages \& Special Foods}
    \metadata{topic={Orange Juice Enzyme Technology}, source={Email Wiki}}

    \begin{answer}
        Until a few years ago, I would drink a couple of quarts of orange juice from pulp-free frozen concentrate every day, then I started noticing those allergy symptoms, and investigated their production processes. They had recently introduced an enzyme technology to make pulp more water soluble. For years, it had been used to dispose of massive amounts of otherwise waste pulp by putting it into the \enquote{creamy} or \enquote{home style} pulpy juices, but then suddenly the relatively clear so-called pulp-free juice began leaving a residue on glasses, and resisting passage through filter paper, besides causing allergy symptoms. For several decades I have watched as traditionally safe foods have been altered, and have found that many people have developed allergic problems when their favorite foods were changed by new technologies. Since intestinal bacteria affect the allergenicity of foods that are poorly digested, changing the flora can often relieve the symptoms. Raw carrot contains some antibiotics that can be helpful; oil and vinegar can increase the germicidal effects. It's important to use oil and vinegar that aren't allergenic themselves. Hypothyroidism increases the susceptibility to many foods.
    \end{answer}
\end{standalonequote}

\begin{standalonequote}{Beverages \& Special Foods}
    \metadata{topic={Tea Tannins And Milk}, source={Email Wiki}}

    \begin{answer}
        The tradition of adding either milk or lemon to tea has been known to protect against the tannins, by reducing reactivity in the case of lemon, or by combination in the cup with the milk protein (as defense against the carcinogenic tannins).
    \end{answer}
\end{standalonequote}

\begin{standalonequote}{Beverages \& Special Foods}
    \metadata{topic={Red Bull}, source={Email Wiki}}

    \begin{answer}
        No, I haven't tried it.
    \end{answer}
\end{standalonequote}

\begin{qaexchange}{Beverages \& Special Foods}
    \metadata{topic={Glucuronolactone Safety}, source={Email Wiki}}

    \begin{question}
        Is Glucuronolactone Safe?
    \end{question}

    \begin{answer}
        I think it is, if it's from a reliable source.
    \end{answer}
\end{qaexchange}

\begin{qaexchange}{Beverages}
    \metadata{topic={Coca-Cola and HFCS}, source={Ray Peat Forum}}

    \begin{question}
        Someone told me you drink HFCS coke regularly. Do you think it is not that harmful if someone is healthy?
    \end{question}

    \begin{answer}
        I prefer Mexican coke with real sugar (it tastes very different), but metabolically there isn't much difference.
    \end{answer}
\end{qaexchange}

\begin{emailexchange}{Beverages}
    \metadata{topic={HFCS Reactions}, source={Ray Peat Forum}}

    \begin{question}
        A few years ago, I started having fast heart rate, palpitations, slurred speech, and difficulty to concentrate after eating certain foods (the first event occurred after eating Oreo cookies with whole milk). About a year later I discovered by reading food labels that these reactions were always preceded by consumption of high fructose corn syrup. E.g., if I drank Pepsi with regular sugar, nothing bad happened. If I drank Pepsi with HFCS, the symptoms appeared within minutes, and were very intense to a degree that left me unable to concentrate in my studies for about two hours. The Oreo cookies that preceded my first reaction contained HFCS.
    \end{question}

    \begin{answer}
        I think [a few years ago] some of the HFCS factories were found to be leaving mercury in some of the sweetener, so that might be a possible explanation. But the normally produced HFCS contains a lot of starch-like material, besides the fructose and glucose, and I think that material would be able to cause allergic reactions.
    \end{answer}
\end{emailexchange}

\begin{standalonequote}{Beverages}
    \metadata{topic={Coca-Cola vs Pepsi}, source={Ray Peat Forum}}

    \begin{answer}
        Coca, not Pepsi. From the mineral analyses I have seen, Coke would have more of the botannical extracts.
    \end{answer}
\end{standalonequote}

\begin{qaexchange}{Beverages}
    \metadata{topic={Coca-Cola Benefits}, source={Ray Peat Forum}}

    \begin{question}
        I was wondering if you would mind sharing on your opinion on the phosphoric acid found in Coke?
And/or would it be more beneficial to do bag breathing instead of drinking coke?
Or does Coke have specific special benefits?
    \end{question}

    \begin{answer}
        The coca leaf and cola seed extracts are valuable antiinflammatories.
The amount of phosphate is very small compared to the amount in meat, fish, beans, nuts, and grains.
    \end{answer}
\end{qaexchange}

\begin{standalonequote}{Beverages}
    \metadata{topic={Tonic Water}, source={Ray Peat Forum}}

    \begin{answer}
        I think tonic water should be used only in small amounts, about an ounce at a time, unless it's for treating a viral infection.
    \end{answer}
\end{standalonequote}

\begin{qaexchange}{Beverages}
    \metadata{topic={Coffee}, source={Ray Peat Forum}}

    \begin{question}
        If someone was eating 6 meals per day, and drinking 4-6 cups of coffee per day, for best effects on hormones would drinking 1 cup at a time with each meal be best? Or would it be better to drink 2-3 cups at a time with a meal twice per day?
    \end{question}

    \begin{answer}
        I think spreading it out as much as possible is best, to avoid ups and downs.
    \end{answer}
\end{qaexchange}

\begin{qaexchange}{Beverages}
    \metadata{topic={Coffee Withdrawal}, source={Ray Peat Forum}}

    \begin{question}
        Just wondering why some of us get severe withdrawals (headaches, fatigue, drowsiness, low mood) if we don't use coffee daily. We don't get side effects from drinking coffee, just if we don't drink it for 24 hours (sometimes less). These \enquote{withdrawals} can last as long as 5--6 days. Why would this happen? We are thinking there could be issues with our liver? Adrenals?
    \end{question}

    \begin{answer}
        I suspect that it happens mostly with hypothyroidism, because in the 1970s I averaged dozens of cups a day, and thought about it as soon as I woke up, then suddenly after I took some thyroid, I didn't feel any need for it.
    \end{answer}
\end{qaexchange}

\begin{standalonequote}{Beverages}
    \metadata{topic={Coffee Selection}, source={Ray Peat Forum}}

    \begin{answer}
        Pre-ground coffee is good, but freshly ground beans have a little more flavor. Products with silicon dioxide should be avoided.
    \end{answer}
\end{standalonequote}

\begin{standalonequote}{Beverages \& Special Foods}
    \metadata{topic={Alcohol Safety, Antioxidant Effects}, source={Ray Peat Forum}}

    \begin{answer}
      Pure colorless highly distilled alcohol is the safest. It can have antioxidant effects, but in some people it can interfere with the respiratory enzymes and lower blood sugar. Fructose is protective against some of its toxic effects.
    \end{answer}
\end{standalonequote}

\begin{standalonequote}{Beverages \& Special Foods}
    \metadata{topic={Tea Preparation}, source={Ray Peat Forum}}

    \begin{answer}
      The addition of lemon or milk to tea reduces the reactivity of the tannins. In recent years, the tea industry has very commonly been adulterating the product. Pu erh is one that still seems to be o.k.
    \end{answer}
\end{standalonequote}

\begin{standalonequote}{Beverages \& Special Foods}
    \metadata{topic={Apple Juice Nutrition}, source={Ray Peat Forum}}

    \begin{answer}
      Apple juice is very good nutritionally, apart from the risk of the fungal contaminant.
    \end{answer}
\end{standalonequote}

\begin{standalonequote}{Beverages \& Special Foods}
    \metadata{topic={Dark Roast Coffee Niacin}, source={Ray Peat Forum}}

    \begin{answer}
      Dark roast coffee provides the most niacin.
    \end{answer}
\end{standalonequote}

\begin{qaexchange}{Beverages \& Special Foods}
    \metadata{topic={Coke vs Pepsi}, source={Ray Peat Forum}}

    \begin{question}
        Does Coke have any advantage over Pepsi nutritionally or allergy wise?
    \end{question}

    \begin{answer}
       That has been my impression, but I haven't seen any organized tests of it. 
    \end{answer}
\end{qaexchange}

\begin{standalonequote}{Beverages \& Special Foods}
    \metadata{topic={Maximum Caffeine Dosage}, source={Ray Peat Forum}}

    \begin{answer}
      6000 mg per day is the most that I've known someone to use safely, but many people have problems with much smaller amounts—it's important to adjust it to individual needs, and to use it with food; cream in it reduces the rate of absorption. The effects are systemic, and other hormones are involved.
    \end{answer}
\end{standalonequote}

\begin{qaexchange}{Beverages \& Special Foods}
    \metadata{topic={Caffeine with Low Glycogen}, source={Ray Peat Forum}}

    \begin{question}
        Do you think caffeine is safe for people with a very poor ability to store sugar? Or would thyroid be more appropriate until the ability to store sugar is fixed?
    \end{question}

    \begin{answer}
      Although thyroid is the essential basic thing, I have known people who were able to stabilize their blood sugar quickly by using a small amount of coffee (with cream) with meals.
    \end{answer}
\end{qaexchange}

\begin{qaexchange}{Beverages \& Special Foods}
    \metadata{topic={Optimal Caffeine Intake}, source={Ray Peat Forum}}

    \begin{question}
        How many milligrams of caffeine per day do you think is best to get the optimal benefits from it?
    \end{question}

    \begin{answer}
      Statistically, people who drink 5 or more cups per day are the healthiest, but in many places it's common for coffee to have only about 100 mg per cup.
    \end{answer}
\end{qaexchange}

\begin{qaexchange}{Beverages \& Special Foods}
    \metadata{topic={Coffee Daily Consumption}, source={Ray Peat Forum}}

    \begin{question}
        In regards to coffee's benefits, do you think its best to consume it everyday? Or is it benefical to consume cronically, without taking breaks away from it?
    \end{question}

    \begin{answer}
      I think it's best to use it regularly. The caffeine and other things are still working even when you have adapted so that the caffeine is less stimulating.
    \end{answer}
\end{qaexchange}

\begin{standalonequote}{Beverages \& Special Foods}
    \metadata{topic={Agave Nectar Toxicity}, source={Ray Peat Forum}}

    \begin{answer}
      The high temperature necessary to concentrate the agave fluid (called aguamiel before it's fermented into pulque) into a sort of molasses causes a Maillard reaction, producing toxins, but I don't think that product is sold in the US. The \enquote{agave nectar} that's sold in the US is made from the core of the plant, rather than the juice, so it's a very artificial industrial product, and it's risky in a variety of ways, including allergens. 
    \end{answer}
\end{standalonequote}

\begin{standalonequote}{Beverages \& Special Foods}
    \metadata{topic={Alcohol Antioxidant Dosage}, source={Ray Peat Forum}}

    \begin{answer}
       The antioxidant effect requires only a very small amount, like a fourth of an ounce of vodka. 
    \end{answer}
\end{standalonequote}

\begin{standalonequote}{Beverages \& Special Foods}
    \metadata{topic={Dark Honey Irritating}, source={Ray Peat Forum}}

    \begin{answer}
      Dark honey is more likely to be irritating. The value is mostly the concentrated sugar, but there are small amounts of antioxidant materials.
    \end{answer}
\end{standalonequote}

\begin{qaexchange}{Beverages \& Special Foods}
    \metadata{topic={Orange Juice Preferences}, source={Ray Peat Forum}}

    \begin{question}
        I was wondering if you juice oranges or other fruits yourself and if so which kind of oranges do you prefer? I have read that the oranges from Valencia, Spain are supposed to be the best, do you think that's true? 
    \end{question}

    \begin{answer}
      They are usually good; I use any that are sweet and juicy.
    \end{answer}
\end{qaexchange}

\begin{standalonequote}{Beverages \& Special Foods}
    \metadata{topic={Coffee, Sugar, Nutrient Requirements}, source={Ray Peat Forum}}

    \begin{answer}
      And the coffee, like the glucose, stimulates your metabolic rate and both of those, by increasing your metabolic rate, are going to increase your general nutritional requirements: minerals and all of the vitamins have to be adequate and if you don't substitute the sugar for things like fruit, milk, cheese, shellfish, eggs and so on, then you will very likely become deficient in biotin and Vitamin B\textsubscript{6} and pantothenic acid, selenium and copper are things that are among the first to become deficient if you try to run on too much coffee and sugar and not enough food.
    \end{answer}
\end{standalonequote}

\begin{standalonequote}{Beverages \& Special Foods}
    \metadata{topic={Nicotine Toothpicks Safety}, source={Ray Peat Forum}}

    \begin{answer}
      I think that small amount is probably safe.
    \end{answer}
\end{standalonequote}

\begin{standalonequote}{Beverages \& Special Foods}
    \metadata{topic={White Rum as Antioxidant}, source={Ray Peat Forum}}

    \begin{answer}
      Once in Mexico I accidentally ate something that I'm intensely allergic to, but I had a glass of white rum at the same meal, and there was no reaction at all. It's an effective antioxidant, anti-inflammatory.
    \end{answer}
\end{standalonequote}

\begin{standalonequote}{Beverages \& Special Foods}
    \metadata{topic={Nasal Snuff, Nicotine Absorption}, source={Ray Peat Forum}}

    \begin{answer}
      When I was a kid it was very popular; I think the effect is the same as oral-swallowed, only quicker, with efficient nicotine absorption.
    \end{answer}
\end{standalonequote}

\begin{standalonequote}{Beverages \& Special Foods}
    \metadata{topic={Tobacco Use, Carbon Monoxide}, source={Ray Peat Forum}}

    \begin{answer}
      The carbon monoxide isn't likely to be absorbed in dangerous amounts if the smoke isn't inhaled. I think the safest way to use tobacco is either transdermally or orally; it has a laxative and anti-inflammatory effect.
    \end{answer}
\end{standalonequote}

\begin{standalonequote}{Beverages \& Special Foods}
    \metadata{topic={Coffee Brewing Details}, source={Ray Peat Forum}}

    \begin{answer}
      I've never measured it, but it's probably about half a cup of dry coffee to make two cups of liquid. I start with warm water, and end with boiling water, so the temperature of the product probably doesn't get much above 45 degrees C; it loses the red-orange color if it gets too hot.
    \end{answer}
\end{standalonequote}

\begin{standalonequote}{Beverages \& Special Foods}
    \metadata{topic={Coffee Brewing Method}, source={Ray Peat Forum}}

    \begin{answer}
      I use a fine grind, and moisten the grounds slightly with warm water first, then slowly add water of increasing temperature. The tastier aromatic things dissolve first at low temperature, and as they are removed the hotter water gets the progressively less soluble things. It makes a very intense, dark, almost opaque extract.
    \end{answer}
\end{standalonequote}

\begin{qaexchange}{Beverages \& Special Foods}
    \metadata{topic={Caffeine, Acetylcholinesterase}, source={Ray Peat Forum}}

    \begin{question}
        Do you think there is significance to caffeine inhibiting human acetylcholinesterase? There seems to be a \enquote{moderate} inhibitory effect but there are no in vivo studies as far as I know. Do you know of anyone who's had problems with coffee or caffeine that indicated elevated acetylcholine?
    \end{question}

    \begin{answer}
      If it's integrated with meals it's safe.
    \end{answer}
\end{qaexchange}

\begin{standalonequote}{Beverages \& Special Foods}
    \metadata{topic={Alcohol Antioxidant Effects}, source={Ray Peat Forum}}

    \begin{answer}
      I think about that much, a few milliliters, can have an antioxidant effect. Years ago I was having dinner at a friend's house and ate something that I'm normally extremely allergic to, but I had some tequila with the meal, and had no reaction at all. That was what led me to investigate the antioxidant effect.
    \end{answer}
\end{standalonequote}

\section{Herbs \& Spices}

\begin{standalonequote}{Herbs \& Spices}
    \metadata{topic={Garlic And Onions Raw Benefits}, source={Email Wiki}}

    \begin{answer}
        Raw, they do have some germicidal effects, sometimes improving intestinal function. The effect depends on the nature of an individual's intestinal flora.
    \end{answer}
\end{standalonequote}

\begin{qaexchange}{Herbs \& Spices}
    \metadata{topic={Hibiscus Tea Benefits}, source={Email Wiki}}

    \begin{question}
        Are orange blossom, rose water, saffron and hibiscus tea safe/beneficial?
    \end{question}

    \begin{answer}
        I think they are safe; I have enjoyed all of them at different times. Hibiscus tea is recognized as a treatment for high blood pressure, and saffron has been used successfully for treating many problems.
    \end{answer}
\end{qaexchange}

\begin{standalonequote}{Herbs \& Spices}
    \metadata{topic={Oregano Oil Safety}, source={Email Wiki}}

    \begin{answer}
        It's one of the safest spices (low allergenicity, not mutagenic or carcinogenic), so if it isn't combined with harmful excipients it seems worth trying.
    \end{answer}
\end{standalonequote}

\begin{qaexchange}{Herbs}
    \metadata{topic={Ceylon Cinnamon}, source={Ray Peat Forum}}

    \begin{question}
        Ceylon cinnamon seems to be much lower in harmful substances than the typical store bought type. Do you think ceylon cinnamon is safe and or beneficial for consumption?
    \end{question}

    \begin{answer}
        I think it might be safer to use the metabolite, benzoic acid, because of the reactivity of the cinnamaldehyde, which I suspect might react with larger molecules to produce allergens.
    \end{answer}
\end{qaexchange}

\begin{standalonequote}{Herbs}
    \metadata{topic={Raw Garlic}, source={Ray Peat Forum}}

    \begin{note}
        Raw Garlic as an Antibiotic
    \end{note}

    \begin{answer}
        It's about as hard on the stomach as on the germs.
    \end{answer}
\end{standalonequote}

\begin{standalonequote}{Herbs}
    \metadata{topic={Birch Sap}, source={Ray Peat Forum}}

    \begin{answer}
        I haven't seen any tests of its effects, but it would contain some potassium, magnesium, sugar and amino acids, so would be faintly nutritional. High temperature processing of sap, such as for maple syrup, produces harmful, allergenic, compounds, but careful pasteurization might not do that.
    \end{answer}
\end{standalonequote}

\begin{standalonequote}{Herbs \& Spices}
    \metadata{topic={Nigella Sativa (Black Cumin Seed) Quinones}, source={Ray Peat Forum}}

    \begin{answer}
      The quinones in the oil are very interesting, but I haven't had any experience with them. They could have great biological value, but other things in the seeds might be allergenic.
    \end{answer}
\end{standalonequote}

\begin{qaexchange}{Herbs \& Spices}
    \metadata{topic={Umami Ingredients Safety}, source={Ray Peat Forum}}

    \begin{question}
        Are there any concerns regarding the regular use of umami boosting ingredients such as soya sauce and fish sauce?
    \end{question}

    \begin{answer}
      No.
    \end{answer}
\end{qaexchange}

\begin{qaexchange}{Herbs \& Spices}
    \metadata{topic={Soy Sauce Safety}, source={Ray Peat Forum}}

    \begin{question}
        I was wondering about your opinions on the safety of soy sauce? The only mention I can find of yours is that soy sauce produced outside of Japan does not have the soy estrogens contained in it. Because of this do you consider soy sauce safe?
    \end{question}

    \begin{answer}
       It's safe in moderate amounts. 
    \end{answer}
\end{qaexchange}

\begin{qaexchange}{Herbs \& Spices}
    \metadata{topic={Spice Safety and Allergies}, source={Ray Peat Forum}}

    \begin{question}
         What are your thoughts on spices used in food such as cumin, turmeric, paprika?
    \end{question}

    \begin{answer}
       Certain spices, especially cumin and caraway, are very allergenic for many people; it's good to be watchful, but some spices have protective effects—turmeric, pepper, cinnamon, and clove, for example. 
    \end{answer}
\end{qaexchange}

\begin{qaexchange}{Herbs \& Spices}
    \metadata{topic={Garlic Safety, Dosage}, source={Ray Peat Forum}}

    \begin{question}
         Raw Garlic has very potent antibacterial properties. Do you think its safe to add as a topping on the carrot salad? 
    \end{question}

    \begin{answer}
      It's good for flavor in small amounts, but in a high concentration it's toxic to both human and bacterial cells.
    \end{answer}
\end{qaexchange}

\begin{emailexchange}{Herbs \& Spices}
    \metadata{topic={Peppermint Tea Safety}, source={Ray Peat Forum}}

    \begin{question}
        What are your opinions on peppermint tea?
    \end{question}

    \begin{answer}
       I've had it several times, and it might have helped somewhat for the digestive issue that I took it for, but I didn't have any clear sense of benefit from it. 
    \end{answer}

    \begin{question}
        So you think peppermint tea would be reasonably safe to experiment with?
    \end{question}

    \begin{answer}
       I think the small amount in a mild tea are safe, unless you have an allergy-like sensitivity to it. 
    \end{answer}
\end{emailexchange}

\begin{standalonequote}{Herbs \& Spices}
    \metadata{topic={Nettle Tea Safety}, source={Ray Peat Forum}}

    \begin{answer}
      The tea is safe.
    \end{answer}
\end{standalonequote}

\begin{qaexchange}{Herbs \& Spices}
    \metadata{topic={Black Tea Preference}, source={Ray Peat Forum}}

    \begin{question}
        If you had all the herbs in the world at your disposal at the highest and purest quality - what herbs would you most often use to make a tea? 
    \end{question}

    \begin{answer}
       Black tea.
    \end{answer}
\end{qaexchange}

\section{General Dietary Principles}

\begin{standalonequote}{General Dietary Principles}
    \metadata{topic={Sauna Causing Hypoglycemia}, source={Email Wiki}}

    \begin{answer}
        A sauna or hot bath, by increasing your metabolic rate, can quickly deplete the glycogen in your liver, causing hypoglycemia. If you eat protein without enough sugar or other carbohydrate it can cause hypoglycemia, too, so it's important to have lots of orange juice and milk, for frequent snacks.
    \end{answer}
\end{standalonequote}

\begin{standalonequote}{General Dietary Principles}
    \metadata{topic={Diet Composition Recommendations}, source={Email Wiki}}

    \begin{answer}
        Orange juice and other sweet fruits (with very little starch) would be best. The muscle meats and starches don't provide a good balance of minerals and amino acids (high in phosphate, tryptophan, and cysteine, for example). Shellfish provide trace minerals that are often lacking from other foods. Mercury content is high in the big (old) fish, but not in the small shellfish or small fish such as cod and sole. You are probably deficient in calcium, so gradually adding cheese, eggs, and milk could be helpful.
    \end{answer}
\end{standalonequote}

\begin{standalonequote}{General Dietary Principles}
    \metadata{topic={Calcium To Phosphate Ratio}, source={Email Wiki}}

    \begin{answer}
        Usually the low carbohydrate diets have a high ratio of phosphate to calcium, and I suspect that your present diet does, too. If you powder some eggshells, that's the best way to supplement it, but two quarts of milk per day would be best, providing adequate protein and a safe ratio of P to Ca. Seafood, especially oysters, shrimp, squid, etc., would provide the iodine and selenium you need for good thyroid regulation. Increasing fruits in place of bread would increase blood sugar stability, and would provide vitamin C in a safer form. Taking your temperature before and after breakfast helps to interpret your circadian hormone cycle---hypothyroid people often have very high adrenaline, cortisol, and other stress hormones during the night, causing the temperature to be higher before breakfast than after. A daily raw carrot often helps to balance progesterone, cortisol, and estrogen, by improving intestine-liver functions.
    \end{answer}
\end{standalonequote}

\begin{standalonequote}{General Dietary Principles}
    \metadata{topic={Diet Causing Stress}, source={Email Wiki}}

    \begin{answer}
        It's best to have more calcium than phosphate, and your diet is deficient in calcium, and heavy on phosphate, and that by itself can cause serious stress. Cheese would be a good way to get enough calcium, if you don't use milk. Eating protein by itself can cause a big surge of cortisol. Preceding the protein with some carbohydrate makes the protein go farther, otherwise under the influence of cortisol a lot of protein is used just for energy. Your diet might below in vitamin A, so it would be better to have eggs for breakfast, preceded with a generous amount of orange juice. Bananas can be seriously allergenic, apples are allergenic for some people, but not as intensely as bananas. Well cooked potatoes, with butter or cream,are a very good way to get carbohydrate, if you aren't allergic to them, because they contain a good balance of amino acids, too, as well as minerals and B vitamins.
    \end{answer}
\end{standalonequote}

\begin{standalonequote}{General Dietary Principles}
    \metadata{topic={Short-Term Dietary Priorities}, source={Email Wiki}}

    \begin{answer}
        For a while, the vitamin A is very important, and the PUFA isn't crucial in the short term, so 2 or 3 eggs would be o.k., though in the longer run it's good to eat liver about twice a month, limiting the daily eggs to one or two. The type of cheese doesn't matter much as far as calcium goes. If you don't get much sunlight, and during the winter, a vitamin D supplement is necessary to use the calcium effectively. Plain white rice, well cooked, with butter is o.k. The calcium, vitamin D and vitamin A will greatly improve your immunity,the colostrum wouldn't be necessary.
    \end{answer}
\end{standalonequote}

\begin{standalonequote}{General Dietary Principles}
    \metadata{topic={Meal Frequency And Metabolism}, source={Email Wiki}}

    \begin{answer}
        Small meals help to increase the metabolic rate, single big meals increase fat storage.
    \end{answer}
\end{standalonequote}

\begin{standalonequote}{General Dietary Principles}
    \metadata{topic={Ray Peat Meal Frequency}, source={Email Wiki}}

    \begin{answer}
        There are just occasional intervals when I'm not eating---cafe con leche several times a day, other things in between.
    \end{answer}
\end{standalonequote}

\begin{standalonequote}{General Dietary Principles}
    \metadata{topic={Meal Frequency Adjustments}, source={Email Wiki}}

    \begin{answer}
        Frequent meals are helpful during hypothyroidism, and help to prevent obesity, but when the thyroid and liver are working, 2, 3, or 4 meals are good. For me, 2 meals and some snacks are most convenient. Orange juice is good by itself.
    \end{answer}
\end{standalonequote}

\begin{standalonequote}{General Dietary Principles}
    \metadata{topic={Meal Frequency With Improved Metabolism}, source={Email Wiki}}

    \begin{answer}
        Yes, as the metabolism gets more effective, you don't have to eat as often as when you are starting to change. At first, when glycogen isn't being stored, temperature will rise and fall situationally.
    \end{answer}
\end{standalonequote}

\begin{standalonequote}{General Dietary Principles}
    \metadata{topic={Nausea From High Fat}, source={Email Wiki}}

    \begin{answer}
        Diabetes tends to interfere with the activity of thyroid hormone, and low thyroid function is closely connected to gall bladder problems. It's important to have some carbohydrate with protein foods, to prevent decreased blood sugar symptoms.
    \end{answer}
\end{standalonequote}

\begin{standalonequote}{General Dietary Principles}
    \metadata{topic={Increased Fat Appetite}, source={Email Wiki}}

    \begin{answer}
        Increased metabolic rate could increase fat appetite.
    \end{answer}
\end{standalonequote}

\begin{standalonequote}{General Dietary Principles}
    \metadata{topic={Ray Peat Calorie Intake}, source={Email Wiki}}

    \begin{answer}
        I used to drink at least a gallon of 2\% or 3\% milk daily, and often ate more than 5000 calories, but when I'm completely sedentary for more than ten hours daily, my energy requirement is much lower. The calorie intake should be balanced to your heat production and activity.
    \end{answer}
\end{standalonequote}

\begin{standalonequote}{General Dietary Principles}
    \metadata{topic={Balancing Eggs With Orange Juice}, source={Email Wiki}}

    \begin{answer}
        I find that I need almost a pint of orange juice to balance one egg.
    \end{answer}
\end{standalonequote}

\begin{standalonequote}{General Dietary Principles}
    \metadata{topic={Ray Peat Typical Diet}, source={Email Wiki}}

    \begin{answer}
        Eggs and orange juice, milk and oysters, and a raw carrot. For variety, smoked oysters, crab, cod fried in butter, ox-tail soup, parmigiano reggiano, sapotas, lychees, liver. Completely avoiding unsaturated fats, such as canola and mayonnaise, and minimizing beans, cereals, and vegetables.
    \end{answer}
\end{standalonequote}

\begin{standalonequote}{General Dietary Principles}
    \metadata{topic={Fasting Tissue Destruction}, source={Email Wiki}}

    \begin{answer}
        After the liver's glycogen is depleted, fasting destroys the tissues, starting with the thymus, then the muscles and liver.
    \end{answer}
\end{standalonequote}

\begin{standalonequote}{General Dietary Principles}
    \metadata{topic={Fat Timing For Sleep}, source={Email Wiki}}

    \begin{answer}
        Having a larger proportion of your fat near bedtime often helps to get through the night without inflammation.
    \end{answer}
\end{standalonequote}

\begin{qaexchange}{General Dietary Principles}
    \metadata{topic={Carbs With Protein Rationale}, source={Email Wiki}}

    \begin{question}
        Why do you recommend to have carbohydrates with a protein meal?
    \end{question}

    \begin{answer}
        It mitigates the damage produced by the stress response to hypoglycemia.
    \end{answer}
\end{qaexchange}

\begin{standalonequote}{General Dietary Principles}
    \metadata{topic={Celtic Salt Heavy Metals}, source={Email Wiki}}

    \begin{answer}
        A friend had the celtic salt analyzed, and found it was high in toxic heavy metals. The pure white common salt is best.
    \end{answer}
\end{standalonequote}

\begin{standalonequote}{General Dietary Principles}
    \metadata{topic={Pure Salt Safety}, source={Email Wiki}}

    \begin{answer}
        That standard isn't very strict, but the salt is probably safe, if it's white. I usually use either La Baleine or Morton's canning and pickling salt.
    \end{answer}
\end{standalonequote}

\begin{standalonequote}{General Dietary Principles}
    \metadata{topic={Refined Sea Salt}, source={Email Wiki}}

    \begin{answer}
        Some sea salt is refined, by sequential evaporation, until it's very pure; either kind of pure white salt without additives is good.
    \end{answer}
\end{standalonequote}

\begin{qaexchange}{General Dietary Principles}
    \metadata{topic={Parachuting Salt}, source={Email Wiki}}

    \begin{answer}
        The salt might be causing problems used that way.
    \end{answer}
\end{qaexchange}

\begin{standalonequote}{General Dietary Principles}
    \metadata{topic={Baking Soda As Sodium}, source={Email Wiki}}

    \begin{answer}
        Baking soda in water is helpful for some people.
    \end{answer}
\end{standalonequote}

\begin{qaexchange}{General Dietary Principles}
    \metadata{topic={Sodium Alternatives}, source={Email Wiki}}

    \begin{question}
        Sodium or the chloride part of salt causes gut irritation. Do you recommend any other safe source of sodium?
    \end{question}

    \begin{answer}
        Baking soda in water is helpful for some people.
    \end{answer}
\end{qaexchange}

\begin{qaexchange}{Diet}
    \metadata{topic={Food Sensitivities}, source={Ray Peat Forum}}

    \begin{question}
        Do you think if a specific food causes gas/bloating for an individual that its best avoided? And one should stick to foods that dont give those symptoms?
    \end{question}

    \begin{answer}
        Yes. Everyone has different intestinal bacteria, and their reactions are affected by foods. If there are nutritional deficiencies, more foods can cause problems, because digestion is less efficient. Vitamin D and thyroid are especially important for digestion.
    \end{answer}
\end{qaexchange}

\begin{standalonequote}{Diet}
    \metadata{topic={Vegetarian Diets}, source={Ray Peat Forum}}

    \begin{answer}
        Using milk and cheese as the basic proteins, the other essential nutrients can all be found from other sources. Mushrooms and eggs are very practical, rich sources of nutrients that are harder to find elsewhere, but not essential if other foods are carefully selected.
    \end{answer}
\end{standalonequote}

\begin{standalonequote}{Diet}
    \metadata{topic={Fiber-Free Diet}, source={Ray Peat Forum}}

    \begin{answer}
        I think fiber is always a risk (I avoid them all except for occasional well cooked mushrooms and bamboo shoots, which are germicidal). The foods you list contain all the essential nutrients.
    \end{answer}
\end{standalonequote}

\begin{standalonequote}{General Dietary Principles}
    \metadata{topic={Salty Snacks, Sugar Sources}, source={Ray Peat Forum}}

    \begin{answer}
      There are some very salty cheeses that help to satisfy salt appetite, for example pecorino or feta; sometimes I add salt to cheeses such as gouda, emmental, mozzarella, or cheddar. Sometimes pork rinds, chicharrones, are good for a salty snack; I usually heat them in coconut oil and then drain them, to remove some of the pork fat. Orange juice, guavas, watermelons, cherimoyas, cooked apples, cherries, and ripe papayas are good sources of sugar to have regularly.
    \end{answer}
\end{standalonequote}

\begin{standalonequote}{General Dietary Principles}
    \metadata{topic={Winter Diet with Frozen Juice}, source={Ray Peat Forum}}

    \begin{answer}
       In the winter in the US, I use a lot of frozen orange juice concentrate, because good fruit is scarce. When you use refined sugar it's important to avoid the starchy foods, emphasizing milk, cheese, eggs, fruits, and occasional liver and seafood. Cooked leafy greens and mushrooms should substitute for starchy vegetables. 
    \end{answer}
\end{standalonequote}

\begin{standalonequote}{General Dietary Principles}
    \metadata{topic={Calorie Deficit Protein Use}, source={Ray Peat Forum}}

    \begin{answer}
      Since a calorie deficit causes stress (inhibiting the oxidation of glucose by increasing pyruvate dehydrogenase kinase), it increases the use of protein for energy, and it would probably increase the tendency for feedings with a momentary excess of calories to produce fatty acids.
    \end{answer}
\end{standalonequote}

\begin{qaexchange}{General Dietary Principles}
    \metadata{topic={Caloric Restriction Effects}, source={Ray Peat Forum}}

    \begin{question}
        There seems to be some evidence that caloric restriction is beneficial. I haven't heard you mention anything about it. I was wondering if you think mild calorie restriction is optimal for health? Or beneficial? Or is just eating low PUFA what you consider to be optimal, regardless of calorie intake?
    \end{question}

    \begin{answer}
      I've occasionally mentioned that the typical calorie restricted diet increases the metabolic rate and decreases oxidative damage by reducing PUFA, cysteine, methionine, tryptophan, and iron, possibly some random toxins. In one of the big nurses studies, someone noticed that those who ate the most lived the longest, i.e., had the highest metabolic rate.
    \end{answer}
\end{qaexchange}

\begin{standalonequote}{General Dietary Principles}
    \metadata{topic={Meat and Rice Elimination Diet}, source={Ray Peat Forum}}

    \begin{answer}
      A diet of meat and rice has often been used as a kind of \enquote{elimination} diet, avoiding the thousands of inflammation-promoting substances in vegetables and many grains. Meat and water would avoid even the antigens in rice.
    \end{answer}
\end{standalonequote}

\begin{qaexchange}{General Dietary Principles}
    \metadata{topic={High-Calorie Stress Management}, source={Ray Peat Forum}}

    \begin{question}
        If someone needed a large number of calories to keep stress low (like 5000), what foods would you suggest while they transition to a healthier diet as liquids might pose a problem for those who are hypothyroid? 
    \end{question}

    \begin{answer}
      Magnesium and calcium-rich foods, cooked greens, cheese, mushrooms, help with stress.
    \end{answer}
\end{qaexchange}

\begin{qaexchange}{General Dietary Principles}
    \metadata{topic={High Orange Juice/Milk Consumption}, source={Ray Peat Forum}}

    \begin{question}
        Do you know of any negative health effects of drinking a full gallon of high-quality Orange Juice + a full gallon of low-fat milk per day provided enough salt is taken with it?
    \end{question}

    \begin{answer}
      In hot weather I've used that amount, and have never noticed harmful effects. I usually had some salty cheese or other foods with it.
    \end{answer}
\end{qaexchange}

\begin{qaexchange}{General Dietary Principles}
    \metadata{topic={Food Security, Butter vs Grains}, source={Ray Peat Forum}}

    \begin{question}
        High fat diets or grains as a better source of calories during possible upcoming food shortages, especially fruit?
    \end{question}

    \begin{answer}
      I think butter and cream are safe sources for a large part of the calorie requirement, generally better than grains (unless they are nixtamalized, alkali treated). I have been using frozen concentrate and reconstituted pasteurized juice for a long time, because of the scarcity of ripe oranges in the US. In Mexico, I've never run into an unripe orange, so there I usually juice my own. I think, for important things like oranges, milk, and cheese, the US is going to have more trouble than rural Mexico.
    \end{answer}
\end{qaexchange}

\begin{qaexchange}{General Dietary Principles}
    \metadata{topic={Not Eating at Restaurants}, source={Ray Peat Forum}}

    \begin{question}
        What kind of foods do you like to eat out, which restaurants do you eat out at regularly?
    \end{question}

    \begin{answer}
      I haven't eaten in a restaurant since 1984. Restaurants have always been notoriously unsafe places to eat, but in the age of food chemistry it has become impractical to try to identify trustworthy places.
    \end{answer}
\end{qaexchange}

\begin{qaexchange}{General Dietary Principles}
    \metadata{topic={Nutrition in Prison/Restricted Environment}, source={Ray Peat Forum}}

    \begin{question}
        What do you think would be the best course of action to sustain a good metabolism if one cannot get proper nutrition and avoid PUFA-ladden meals, such when in a prison or in a restricted camp. 
    \end{question}

    \begin{answer}
      Avoiding the worst foods as far as possible (things prepared with vegetable oil, margarine), getting as much sunlight and rest as possible. Potatoes and vegetables would be among the safer foods.
    \end{answer}
\end{qaexchange}

\begin{standalonequote}{General Dietary Principles}
    \metadata{topic={Food Cravings, Nutritional Deficiency}, source={Ray Peat Forum}}

    \begin{answer}
      Any craving is a good starting point, because we have several biological mechanisms for correcting specific nutritional deficiencies. When something is interfering with your ability to use sugar, you crave it because if you don't eat it you will waste protein to make it.
    \end{answer}
\end{standalonequote}

\subsection{Digestion}

\begin{qaexchange}{Digestion}
    \metadata{topic={Baking Soda With Protein}, source={Email Wiki}}

    \begin{question}
        OK with protein meals or does it buffer stomach acid?
    \end{question}

    \begin{answer}
        It slows digestion down for a few minutes.
    \end{answer}
\end{qaexchange}

\begin{standalonequote}{Digestion}
    \metadata{topic={Safe Fiber Types}, source={Email Wiki}}

    \begin{answer}
        Cellulose is the safe fiber, and (boiled) bamboo shoots are another safe fiber. My May newsletter, below, has some information about the effects of other fibers, including pectin. If the fruits don't cause digestive problems, such as gas, then the fiber is good. Apples and pears are often so fibrous (because of incomplete ripening) that the fiber can be harmful.
    \end{answer}
\end{standalonequote}

\begin{standalonequote}{Digestion}
    \metadata{topic={Fiber Not Necessary}, source={Email Wiki}}

    \begin{answer}
        They aren't necessary [fiber], for example milk supports abundant bacterial growth that creates bulk, but when there are digestive and hormonal problems because of bad intestinal flora, the fibers of carrot and bamboo shoots have a disinfecting action. The carrots must be raw for that effect.
    \end{answer}
\end{standalonequote}

\begin{qaexchange}{Digestion}
    \metadata{topic={Fiber-Free Diet}, source={Email Wiki}}

    \begin{question}
        Fiber-free diet possible?
    \end{question}

    \begin{answer}
        I've had a fiber-free diet for many years.
    \end{answer}
\end{qaexchange}

\begin{standalonequote}{Digestion}
    \metadata{topic={Digestive Adaptation Time}, source={Email Wiki}}

    \begin{answer}
        It usually takes several days for the digestive system to adjust, with changes in the intestinal rhythm for example, and during that time things like headache and tooth sensitivity can increase. Increased calcium and fiber (raw carrots or boiled bamboo shoots, for example) can help.
    \end{answer}
\end{standalonequote}

\begin{standalonequote}{Digestion}
    \metadata{topic={Fiber Reducing Endotoxin}, source={Email Wiki}}

    \begin{answer}
        Yes, that's why a resistant (antiseptic) fiber such as bamboo shoots or raw carrot helps with weight loss, it reduces endotoxin and the stress hormones, and lets the liver metabolize more effectively.
    \end{answer}
\end{standalonequote}

\begin{qaexchange}{Digestion}
    \metadata{topic={Carrot Salad Concerns}, source={Ray Peat Forum}}

    \begin{question}
        I'm curious are there any downsides or concerns when utilising the carrot salad on a person with sensitive digestion)?
    \end{question}

    \begin{answer}
      Too much carotene can be a problem if your thyroid function is low.
    \end{answer}
\end{qaexchange}

\begin{qaexchange}{Digestion}
    \metadata{topic={Eating Fibers with Meals}, source={Ray Peat Forum}}

    \begin{question}
        Do you eat fibers (carrot/mushroom/bamboo) alone or with other foods? Does adding other foods negate the beneficial effects of the fibers?
    \end{question}

    \begin{answer}
       I always have them with other foods; the intestine makes appropriate adjustments when the diet is consistent. 
    \end{answer}
\end{qaexchange}

\begin{qaexchange}{Digestion}
    \metadata{topic={Protective Fibers for Bowel Cancer}, source={Ray Peat Forum}}

    \begin{question}
        I know you advise the usual carrot/bamboo/mushrooms for fibers. I saw a comment of yours that commented about potato fiber being protective against bowel cancer, I wondered if there were any other fibers that you don't often talk about, that protect against bowel cancer in a similar way?
    \end{question}

    \begin{answer}
      In general, cellulose fibers do protect against bowel cancer, but a few plant fibers that contain lignin or that are fermentable increase cancer. Wheat bran is the only common cereal fiber that's protective.
    \end{answer}
\end{qaexchange}

\begin{standalonequote}{Digestion}
    \metadata{topic={Indigestible Fiber Benefits}, source={Ray Peat Forum}}

    \begin{answer}
       The fiber should be indigestible; bran can be washed, to remove most of the starch so it doesn't feed bacteria. The indigestible fiber stimulates movement of the intestine, while helping to satisfy hunger feelings. 
    \end{answer}
\end{standalonequote}

\begin{qaexchange}{Digestion}
    \metadata{topic={Fermentable Fiber Hydrogen Production}, source={Ray Peat Forum}}

    \begin{question}
        Do you think its possible that some of reported health benefits of dietary fermentable fibers being because of hydrogen thats produced during their fermentation?
    \end{question}

    \begin{answer}
       Yes, that could have an antiinflammatory effect. 
    \end{answer}
\end{qaexchange}

\begin{qaexchange}{Digestion}
    \metadata{topic={Wheat Bran for Fiber}, source={Ray Peat Forum}}

    \begin{question}
        When looking to increase bowel transit, is the use of pure cellulose (usually derived from trees) or wheat bran good choices of insoluble fiber? 
    \end{question}

    \begin{answer}
      Washed wheat bran is usually safe and effective.
    \end{answer}
\end{qaexchange}

\begin{qaexchange}{Digestion}
    \metadata{topic={Meal Timing, Digestive Rhythm}, source={Ray Peat Forum}}

    \begin{question}
        Do you think it's helpful to leave a few hours between meals, to let food digest?
    \end{question}

    \begin{answer}
      I think the distinct meal times help the intestine to establish a regular rhythm.
    \end{answer}
\end{qaexchange}

\subsection{Sleep Support}

\begin{standalonequote}{Sleep}
    \metadata{topic={Bedtime Nutrition}, source={Ray Peat Forum}}

    \begin{answer}
      Something sugary (milk and honey, for example) right at bedtime might help with sleep.
    \end{answer}
\end{standalonequote}

\begin{qaexchange}{Sleep Support}
    \metadata{topic={Nighttime Eating, Liver Glycogen}, source={Ray Peat Forum}}

    \begin{question}
        Do you normally sleep straight through the night, or do you wake up, have a snack and go back to bed? Does it vary depending on the season, and general stress? It seems that holding onto liver glycogen for a full 8 hours as an adult in this modern world is quite difficult. 
    \end{question}

    \begin{answer}
      I rarely sleep straight through, usually have orange juice or other snack. Long days and good weather help.
    \end{answer}
\end{qaexchange}

\begin{emailexchange}{Sleep Support}
    \metadata{topic={Sleeping Position, Pillows}, source={Ray Peat Forum}}

    \begin{question}
        Is there an optimal sleeping position where we can prevent losing too much \ce{CO2}? On the back, stomach, right side, left side, on the floor, etc. How do you usually sleep? 
    \end{question}

    \begin{answer}
      I don't think position in itself has much effects.
    \end{answer}

    \begin{question}
        How about using pillows and sleeping on comfortable beds? My grandfather claims sleeping on the ground with no pillow makes the body stronger. Is there any truth to that?
    \end{question}

    \begin{answer}
      I think a pillow and a soft bed are good for the health.
    \end{answer}
\end{emailexchange}

\chapter{Vitamins \& Minerals}

\section{Fat-Soluble Vitamins}

\begin{standalonequote}{Fat-Soluble Vitamins}
    \metadata{topic={Vitamin A/D Ratio Variation}, source={Ray Peat Forum}}

    \begin{answer}
      I don't think the ratio matters, the need for them can vary in opposite directions, for example, with lots of sunlight there's no need to supplement D, but the need for A increases. Vitamin E protects against an excess of A.
    \end{answer}
\end{standalonequote}

\begin{standalonequote}{Fat-Soluble Vitamins}
    \metadata{topic={Vitamin A/D Ratio, Calcification}, source={Ray Peat Forum}}

    \begin{answer}
      I don't think there is a functional ratio between them (A \& D), independent of everything else. Calcium, phosphate, protein, vitamins and minerals are involved in preventing improper calcification.
    \end{answer}
\end{standalonequote}

\subsection{Vitamin A}

\begin{standalonequote}{Vitamin A}
    \metadata{topic={Vitamin A Oral Allergenicity}, source={Email Wiki}}

    \begin{answer}
        I think prolactin and TSH would be worth checking. I have had bad headaches when I used vitamin A orally, and even getting a little on my lips was enough to do it. It could be that the Nutrisorb-A was the cause, if you used it orally. I use it only on my legs and feet.
    \end{answer}
\end{standalonequote}

\begin{standalonequote}{Vitamin A}
    \metadata{topic={Soybean Oil in Supplements}, source={Email Wiki}}

    \begin{answer}
        The small amount of oil in a capsule doesn't matter much. Any capsule should have the highest potency in the smallest size, to minimize the junk.
    \end{answer}
\end{standalonequote}

\begin{standalonequote}{Vitamin A}
    \metadata{topic={Vitamin A Thyroid Balance}, source={Email Wiki}}

    \begin{answer}
        Yes, it's definitely hard to get them coordinated when there's an imbalance in one direction or the other. For several years, when I had an extremely high metabolic rate, I needed 100,000 units per day during sunny weather to prevent acne and ingrown whiskers, but when I moved to a cloudy climate, suddenly that much was too much, and suppressed my thyroid. The average person is likely to be hypothyroid, and to need only 5,000 units per day. Avoiding large amounts of carotene, and getting plenty of vitamin B\textsubscript{12} to be able to convert any carotene that's in your food, helps to use vitamin A efficiently.
    \end{answer}
\end{standalonequote}

\begin{standalonequote}{Vitamin A}
    \metadata{topic={Vitamin A Acne Mechanism}, source={Email Wiki}}

    \begin{answer}
        Yes, vitamin A and estrogen are antagonistic, and while estrogen promotes keratinization (shedding of skin cells), vitamin A opposes it. Since vitamin A is highly unsaturated, in excess it suppresses the thyroid, so it has to be balanced with the thyroid; the combination is effective for increasing progesterone and decreasing estrogen, slowing the turnover of skin cells, and making the skin cells function longer before flaking off. Plugged pores, combined with a local shift toward synthesizing inflammatory substances, foster bacterial infection. Bright light stimulates the production of steroids, and consumes vitamin A very quickly, but when the balance is right, the acne clears up in just a day or two. Cream, butter, eggs, and liver are good sources of vitamin A. When people supplement thyroid and eat liver once or twice a week, their acne and dandruff (and many other problems) usually clear up very quickly. It was acne and dandruff that led me into studying the steroids and thyroid, and in the process I found that they were related to constipation and food sensitivity.
    \end{answer}
\end{standalonequote}

\begin{standalonequote}{Vitamin A}
    \metadata{topic={High Vitamin A Requirements}, source={Email Wiki}}

    \begin{answer}
        I found that I had an extremely high vitamin A requirement, increased by stress or bright light, and that it related to thyroid function. Usually, thyroid and vitamin A are the supplements that stop acne.
    \end{answer}
\end{standalonequote}

\begin{standalonequote}{Vitamin A}
    \metadata{topic={Vitamin A Source Preference}, source={Email Wiki}}

    \begin{answer}
        I avoid carotene, because it blocks thyroid and steroid production, and very large, excessive, amounts of vitamin A, retinol, can do the same. I use halibut liver oil-derived vitamin A, or retinyl palmitate.
    \end{answer}
\end{standalonequote}

\begin{standalonequote}{Vitamin A}
    \metadata{topic={Vitamin A Skin Functions}, source={Email Wiki}}

    \begin{answer}
        Estrogen causes the oil glands to atrophy, so the skin doesn't support bacterial growth so well. Topical sulfur's germicidal effect can help, and topical aspirin and caffeine are antiseptic as well as antiinflammatory. One function of vitamin A is to increase progesterone in the skin, and it has to be in balance with thyroid to do that. Another function is to differentiate the skin cells, reducing keratin plugging of the glands.
    \end{answer}
\end{standalonequote}

\begin{standalonequote}{Vitamin A}
    \metadata{topic={Very High Vitamin A Dosing}, source={Email Wiki}}

    \begin{answer}
        For several years, I had a similar need to take 100,000 i.u. daily to prevent acne and ingrown whiskers, so I read a lot about its effects. The toxic effects of extremely big doses, such as 500,000 to a million i.u., seem to be from either oxidative processes (rancidity) that are prevented by adequate vitamin E, or by antithyroid effects. I found that when my need for vitamin A began to decrease I tended to accumulate carotene in my calluses; that happens when the thyroid function is lower, reducing the need for vitamin A. Since you are eating foods with carotene, the calluses on your palms or soles should serve as an indicator of when your tissues are saturated with vitamin A. About 100 i.u. of vitamin E would help to keep the vitamin A from being wasted by oxidation, and possibly could reduce your requirement for it.
    \end{answer}
\end{standalonequote}

\begin{qaexchange}{Vitamin A}
    \metadata{topic={Vitamin A Supplement Allergens}, source={Email Wiki}}

    \begin{question}
        Are vitamin A supplements inherently allergenic or are they allergenic just because of modern production processes?
    \end{question}

    \begin{answer}
        It's something in the manufactured product that's not in the natural.
    \end{answer}
\end{qaexchange}

\begin{standalonequote}{Vitamin A}
    \metadata{topic={Vitamin A Excess Symptoms}, source={Email Wiki}}

    \begin{answer}
        Vitamin A oxidizes easily and an excess can create symptoms of a deficiency, so vitamin E is the most important thing for correcting it; excess vitamin A, like PUFA, interferes with thyroid hormone transport, so it's important to balance the two.
    \end{answer}
\end{standalonequote}

\begin{standalonequote}{Vitamin A}
    \metadata{topic={Oral Supplementation Anxiety}, source={Ray Peat Forum}}

    \begin{answer}
        I use the vitamin A only on my skin, because I get intense symptoms from even a small amount orally.
    \end{answer}
\end{standalonequote}

\begin{standalonequote}{Vitamin A}
    \metadata{topic={Topical Application}, source={Ray Peat Forum}}

    \begin{answer}
        I think it's safest to use the oily vitamins on the skin.
    \end{answer}
\end{standalonequote}

\begin{qaexchange}{Vitamin A}
    \metadata{topic={Deficiency Symptoms}, source={Ray Peat Forum}}

    \begin{question}
        If I'm vitamin A deficient enough to get dandruff and acne, could that cause anxiety too?
    \end{question}

    \begin{answer}
        Since it's needed to make pregnenolone and progesterone, I think it could.
    \end{answer}
\end{qaexchange}

\begin{standalonequote}{Vitamin A}
    \metadata{topic={Needs After Vitamin E}, source={Ray Peat Forum}}

    \begin{answer}
        A higher metabolic rate can increase your need for vitamin A considerably. Vitamin E spares vitamin A by preventing its oxidation. Vitamin D and K interact with it, too, so I think it's important to make sure that you have enough of them. When I moved to Oregon I kept taking large amounts of A for several years, and I think it was mostly using vitamin E that reduced my need for it.
    \end{answer}
\end{standalonequote}

\begin{emailexchange}{Vitamin A}
    \metadata{topic={High Dose Supplementation}, source={Ray Peat Forum}}

    \begin{question}
        When you mention having taken 100,000 units of vitamin A a day in the past, do you mean orally? Would the equivalent dose on the skin be 400,000 to 500,000 units?
    \end{question}

    \begin{answer}
      Yes, it was oral. Emanuel Cheraskin's study involved some people taking 100,000 units orally, too. Vitamin A is light sensitive, so I've used it only on my feet, where sunlight wouldn't reach it under my shoes.
    \end{answer}
	
    \begin{question}
        Wouldn't a lot of it come off into your socks?
    \end{question}

    \begin{answer}
      Yes, it's much more efficient to take it orally, but around 1980 vitamin A started giving me migraines, I think from sulfite contaminating it.
    \end{answer}
\end{emailexchange}

\begin{qaexchange}{Vitamin A}
    \metadata{topic={Pregnenolone Substitution}, source={Ray Peat Forum}}

    \begin{question}
        When you were taking large amounts of pregnenolone daily, did your need for vitamin A decrease? Can you substitute taking pregnenolone instead of vitamin A?
    \end{question}

    \begin{answer}
      Yes, I stopped taking vitamin A then.
    \end{answer}
\end{qaexchange}

\begin{qaexchange}{Vitamin A}
    \metadata{topic={Deficiency and Anxiety}, source={Ray Peat Forum}}

    \begin{question}
        If I'm vitamin A deficient enough to get dandruff and acne, could that cause anxiety too?
    \end{question}

    \begin{answer}
      Since it's needed to make pregnenolone and progesterone, I think it could.
    \end{answer}
\end{qaexchange}

\subsection{Vitamin D}

\begin{standalonequote}{Vitamin D}
    \metadata{topic={Vitamin D Recommendations}, source={Email Wiki}}

    \begin{answer}
        I use Carlson's, and I think most of the informed people are recommending about 2,000 units per day. John Cannell's site, 'the vitamin D council,' has a newsletter, and is a good way to keep up with the vitamin D research.
    \end{answer}
\end{standalonequote}

\begin{standalonequote}{Vitamin D}
    \metadata{topic={Vitamin D And Tanning}, source={Email Wiki}}

    \begin{answer}
        I think getting enough vitamin D increases the ability to tan.
    \end{answer}
\end{standalonequote}

\begin{standalonequote}{Vitamin D}
    \metadata{topic={High-Dose Vitamin D Winter}, source={Email Wiki}}

    \begin{answer}
        During the winter for a couple of months 10,000 units of D should be safe, but it's better to increase calcium and vitamin K, keeping the vitamin D a little lower unless you have the blood level checked occasionally.
    \end{answer}
\end{standalonequote}

\begin{standalonequote}{Vitamin D}
    \metadata{topic={Vitamin D Maintenance Dose}, source={Email Wiki}}

    \begin{answer}
        Usually 2000 i.u. during the winter will make up for no sunlight. Some people need 5000 iu according to their blood tests, to keep it in the middle of the range.
    \end{answer}
\end{standalonequote}

\begin{standalonequote}{Vitamin D}
    \metadata{topic={Vitamin D Deficiency Causes}, source={Email Wiki}}

    \begin{answer}
        I've seen some quick improvements from serious symptoms with a supplement of it. I think low thyroid could increase the need for it. It takes lots of summer sun direct exposure of a lot of skin to make enough vitamin D.
    \end{answer}
\end{standalonequote}

\begin{qaexchange}{Vitamin D}
    \metadata{topic={Chronic Deficiency}, source={Ray Peat Forum}}

    \begin{question}
        Do some people have a chronic vitamin D deficiency problem?
    \end{question}

    \begin{answer}
        Yes, some people seem to metabolize it faster than normal. It's involved in energy metabolism.
    \end{answer}
\end{qaexchange}

\begin{qaexchange}{Vitamin D}
    \metadata{topic={Viral Prevention}, source={Ray Peat Forum}}

    \begin{question}
        What is your view on what's happening in China with the CoV? Should we be concerned and would nebulizing or taking colloidal silver be a good preventative? MB, niacinamide, iron avoidance also come to mind.
    \end{question}

    \begin{answer}
        I think keeping serum vitamin D around 50 ng/ml, and getting adequate vitamin A, calcium, trace minerals, and other nutrients, avoiding polyunsaturated fats, greatly reduce the risk of viral infections.
    \end{answer}
\end{qaexchange}

\begin{standalonequote}{Vitamin D}
    \metadata{topic={Optimal Level}, source={Ray Peat Forum}}

    \begin{answer}
        I think 50 ng/ml is a good goal. The point at which it lowers parathyroid hormone would be the right amount.
    \end{answer}
\end{standalonequote}

\begin{emailexchange}{Vitamin D}
    \metadata{topic={Winter Vitamin D Protocol}, source={Ray Peat Forum}}

    \begin{question}
        What do you do for vitamin D levels in winter?
    \end{question}

    \begin{answer}
       In addition to a few quarts of milk with added vitamin D, and eating liver, I rub several drops of the vitamin D oil onto my skin, and occasionally use an ultraviolet light. 
    \end{answer}

    \begin{question}
        I noticed you mention eating liver, could I ask the relationship between liver and vitamin D?
    \end{question}

    \begin{answer}
       The liver catalyzes the hydroxylation of cholecalciferol to 25-OH-colecalciferol, and many nutritional deficiencies can effect the liver's functions. I've known several liver-avoiders whose serum 25-OH-Vitamin D stays low despite big doses, and I suspect their general nutrition is responsible. It's reasonable to feed their liver optimally. 
    \end{answer}
\end{emailexchange}

\begin{qaexchange}{Vitamin D}
    \metadata{topic={Topical Vitamin D Dosage}, source={Ray Peat Forum}}

    \begin{question}
        When you use vitamin D on your skin during the winter what dose do you use?
    \end{question}

    \begin{answer}
       About 20,000 IU. 
    \end{answer}
\end{qaexchange}

\begin{qaexchange}{Vitamin D}
    \metadata{topic={Vitamin D Optimal Levels}, source={Ray Peat Forum}}

    \begin{question}
        Is keeping vitamin D at about 50ng/ml optimal? Do you think topical supplementation of vitamin D is best?
    \end{question}

    \begin{answer}
      Individuals vary in the amount of each that they need. During northern winters, most people benefit from a supplement of vitamin D. Because of manufacturing impurities, any supplement can be irritating to the digestive system, but usually taking it with food is o.k. When an oral supplement causes problems, the oily vitamins can be used on the skin, but the amount absorbed is usually much less. I think a blood level of 50 to 55 ng/ml is optimal. I have noticed that I tan easily rather than burning when I have supplemented vitamin D.
    \end{answer}
\end{qaexchange}

\begin{standalonequote}{Vitamin D}
    \metadata{topic={Topical Vitamin D Absorption}, source={Ray Peat Forum}}

    \begin{answer}
      If the superficial layer of skin is reduced by a long bath and rubbing, much more is absorbed; the concentration of the solution, the thoroughness of the rubbing in, and not washing it off, all make a big difference. Some people, after using it transdermally for several weeks showed no change in their blood level. Taking it orally, nearly all of it is absorbed.
    \end{answer}
\end{standalonequote}

\begin{qaexchange}{Vitamin D}
    \metadata{topic={Vitamin D Testing Timing}, source={Ray Peat Forum}}

    \begin{question}
        What do you think is the optimal amount of days before a vitamin D test, to stop supplementation? Or should we stop at all?
    \end{question}

    \begin{answer}
      The day before is o.k.
    \end{answer}
\end{qaexchange}

\begin{standalonequote}{Vitamin D}
    \metadata{topic={Vitamin D Receptor Function}, source={Ray Peat Forum}}

    \begin{answer}
      I think the VDR and 1,25-OH are very analogous to the estrogen system. Rodents are supposedly more sensitive to vitamin D than people, but the standard rat poison concentration in human food would probably eventually kill some people.
    \end{answer}
\end{standalonequote}

\subsection{Vitamin E}

\begin{standalonequote}{Vitamin E}
    \metadata{topic={Vitamin E Electron Delocalization}, source={Email Wiki}}

    \begin{answer}
        My thesis adviser, Arnold Soderwall, did some studies showing that vitamin E extended fertility considerably. I found some of his old Sigma (chemical company) vitamin E still in the freezer, and I was working on the idea that oxidative catalysts in the liver were directly related to estrogen's effects. I would extract lipids from the liver, and use paper chromatography to separate them, and for reference points I used the vitamin E and different quinones (coenzyme Q10, Q6, and benzoquinone). I happened to mix the vitamin E with one of the quinones, and noticed that it turned almost black; all of the quinones had the same effect. Putting the mixture on the paper, the moving solvent separated the original components. Delocalized electrons absorb low energy light, causing a dark color (as in black semiconductors), and Szent-Gyorgyi had expressed wonder about what could cause the dark color of the healthy liver, a color that can't be extracted as a pigment. This experiment convinced me that vitamin E could be one of the participants in delocalizing electrons for activating proteins in the way S-G suggested. However, the technology for manufacturing vitamin E has changed greatly over the years, and I have never found anything sold as vitamin E that produces the same dark colors as that old stuff from the freezer. I don't know whether the powerfully therapeutic (anti-estrogenic, clot-clearing, anti-inflammatory, quinone-reactive) old vitamin E contained \enquote{impurities} that were effective, or whether it's that the newer materials contain impurities that reduce their effects.
    \end{answer}
\end{standalonequote}

\begin{standalonequote}{Vitamin E}
    \metadata{topic={Old Vitamin E Properties}, source={Email Wiki}}

    \begin{answer}
        It was labeled d-alphatocopherol, but it was semi-solid, like crystallized honey.
    \end{answer}
\end{standalonequote}

\begin{standalonequote}{Vitamin E}
    \metadata{topic={Vitamin E Safety}, source={Email Wiki}}

    \begin{answer}
        Pure vitamin E doesn't have any toxic effects, except when it's enough to irritate the intestine, probably because of viscosity.
    \end{answer}
\end{standalonequote}

\begin{qaexchange}{Vitamin E}
    \metadata{topic={Vitamin E Manufacturing Concerns}, source={Email Wiki}}

    \begin{question}
        Do you take vitamin E?
    \end{question}

    \begin{answer}
        No, I stopped taking it, partly because of the new manufacturing methods, that were associated for several years with adding soy oil to the product.
    \end{answer}
\end{qaexchange}

\begin{standalonequote}{Vitamin E}
    \metadata{topic={Vitamin E Product Selection}, source={Email Wiki}}

    \begin{answer}
        I think mixed tocopherols are better than just d-alpha, but with d-alpha it's good to choose one that has a high potency per volume. I have noticed that one of Unique's products seems to be mostly other oil. I think polycosanols account for some of the viscosity, so I prefer the thick ones.
    \end{answer}
\end{standalonequote}

\begin{standalonequote}{Vitamin E}
    \metadata{topic={Vitamin E Potency vs Soy Oil}, source={Email Wiki}}

    \begin{answer}
        If the potency of a vitamin E product is around 1000 i.u. per milliliter, the amount of soy oil isn't a concern, but if it's only about 100 i.u./ml, then there's enough oil to matter.
    \end{answer}
\end{standalonequote}

\begin{standalonequote}{Vitamin E}
    \metadata{topic={Vitamin E Soy Oil Content}, source={Email Wiki}}

    \begin{answer}
        The lighter consistency is because less soy oil is removed from some cheaper products. They don't have to add oil, they just leave more of it in the product. But non-GMO soy is mostly produced using other herbicides and pesticides, and each pesticide has different affinities for oil or water, so oil soluble pesticides would be the main concern, and those are generally much more volatile than vitamin E, and so if they were present in the crude oil, they would end up mostly in the refined oil, rather than in the vitamin E. The thickest, darkest, vitamin E is likely to be the cleanest.
    \end{answer}
\end{standalonequote}

\begin{qaexchange}{Vitamin E}
    \metadata{topic={Vitamin E And Food Iron}, source={Email Wiki}}

    \begin{question}
        Should one avoid taking vitamin E with a food that naturally contains iron (eg., eggs, chocolate, liver)?
    \end{question}

    \begin{answer}
        Iron in those foods won't interfere.
    \end{answer}
\end{qaexchange}

\begin{standalonequote}{Vitamin E}
    \metadata{topic={Vitamin E Product Changes}, source={Email Wiki}}

    \begin{answer}
        It can still be very protective against lipid peroxidation and inflammation, but the products have been changing frequently in the last 15 years, so I think it's good to be cautious and use minimal doses. The vitamin E from Sigma in the 1960s and early '70s behaved completely differently in relation to coenzyme Q10 and other quinones, than the more recent products.
    \end{answer}
\end{standalonequote}

\begin{emailexchange}{Vitamin E}
    \metadata{topic={Sources}, source={Ray Peat Forum}}

    \begin{question}
        The vitamin E in supplements are usually sourced from either soy or from wheat. Are there other sources of vitamin E from food? Once, you had mentioned that coconut oil contains Vitamin E, but I can't find supporting documents or articles saying so.
    \end{question}

    \begin{answer}
        It's removed in the refining. I asked a coconut oil producer in Mexico what he did with the diatomacous earth after it's used in the refining, and he said he sold it to a chemical company.
    \end{answer}
\end{emailexchange}

\begin{qaexchange}{Vitamin E}
    \metadata{topic={Eye Irritation}, source={Ray Peat Forum}}

    \begin{question}
        Do you think tocopherols would be an eye irritant? I am curious to try small doses of fat soluble vitamins through the eye.
    \end{question}

    \begin{answer}
        I've accidentally got some in my eye when I was putting it on for sunburn, and it was very uncomfortable.
    \end{answer}
\end{qaexchange}

\begin{qaexchange}{Vitamin E}
    \metadata{topic={Mixed Tocopherols}, source={Ray Peat Forum}}

    \begin{question}
        Do you have a preference between with high alpha or high gamma mixed tocopherols?
    \end{question}

    \begin{answer}
        In similar milligram amounts, I would prefer gamma.
    \end{answer}
\end{qaexchange}

\begin{qaexchange}{Vitamin E}
    \metadata{topic={Topical Absorption}, source={Ray Peat Forum}}

    \begin{question}
        Can we absorb vitamin E topically?
    \end{question}

    \begin{answer}
      Yes, that's how I use it.
    \end{answer}
\end{qaexchange}

\subsection{Vitamin K}

\begin{standalonequote}{Vitamin K}
    \metadata{topic={Vitamin K\textsubscript{1} vs K\textsubscript{2}}, source={Email Wiki}}

    \begin{answer}
        K\textsubscript{1} is probably a little less active than K\textsubscript{2}.
    \end{answer}
\end{standalonequote}

\begin{standalonequote}{Vitamin K}
    \metadata{topic={MK-4 vs MK-7 Timing}, source={Email Wiki}}

    \begin{answer}
        Taking the vitamin with a meal, it will absorb slowly and steadily, and I don't think will make much difference.
    \end{answer}
\end{standalonequote}

\begin{standalonequote}{Vitamin K}
    \metadata{topic={Vitamin K Aspirin Balance}, source={Email Wiki}}
    \begin{note}
        $\pm$25g of Spinach and 5--10g of Beef Liver
    \end{note}

    \begin{answer}
        I think that amount of liver and spinach is likely to be enough.
    \end{answer}
\end{standalonequote}

\begin{qaexchange}{Vitamin K}
    \metadata{topic={Nosebleeds During Pregnancy}, source={Ray Peat Forum}}

    \begin{question}
        My daughter is 35 weeks pregnant and is experiencing frequent nosebleeds. Is this normal because of increase in blood volume?
    \end{question}

    \begin{answer}
        I think a good vitamin K such as Thorne Research's drops would help. Raw leaves aren't digestible, and raw kale is more harmful to the thyroid. When pregnant it's essential to get at least 100 grams of protein per day, and plenty of calcium, for example at least two quarts of milk.
    \end{answer}
\end{qaexchange}

\begin{qaexchange}{Vitamin K}
    \metadata{topic={With Aspirin}, source={Ray Peat Forum}}

    \begin{question}
        If I take one baby aspirin three times weekly prior to my weight training sessions, how much vit K\textsubscript{2} do you recommend daily? How much if I don't take any aspirin?
    \end{question}

    \begin{answer}
        Vitamin K is good for protecting muscles and bones, so 500 micrograms to one milligram is good anyway, and it would be protective against even more aspirin.
    \end{answer}
\end{qaexchange}

\begin{standalonequote}{Vitamin K}
    \metadata{topic={Vitamin K Multiple Effects}, source={Ray Peat Forum}}

    \begin{answer}
       It has many effects, some of them prevent abnormal clotting, others abnormal bleeding, others favor energy production and brain lipid synthesis, calcium regulation. 
    \end{answer}
\end{standalonequote}

\begin{standalonequote}{Vitamin K}
    \metadata{topic={Topical Vitamin K Usage}, source={Ray Peat Forum}}

    \begin{answer}
      I use any K\textsubscript{1} or K\textsubscript{2} if it's in a safe vehicle such as olive oil. If it has MCT in it, I use it on my skin.
    \end{answer}
\end{standalonequote}

\begin{qaexchange}{Vitamin K}
    \metadata{topic={Vitamin K\textsubscript{2} for Fatty Liver}, source={Ray Peat Forum}}

    \begin{question}
        Do you know if there's anything special about vitamin K\textsubscript{2} in regards to defattening the liver?
    \end{question}

    \begin{answer}
      Vitamin K supports the energetic effect of coenzyme Q10.
    \end{answer}
\end{qaexchange}

\section{Water-Soluble Vitamins}
\subsection{B Vitamins}

\begin{standalonequote}{B Vitamins}
    \metadata{topic={Methyl Donor Supplementation}, source={Email Wiki}}

    \begin{answer}
        I think it's risky to supplement methyl donors in substantial quantity, such as choline, betaine, methionine, and S-adenosylmethionine.
    \end{answer}
\end{standalonequote}

\begin{standalonequote}{B Vitamins}
    \metadata{topic={Methylfolate Allergies}, source={Email Wiki}}

    \begin{answer}
        The amount of methyl in that form of B\textsubscript{12} is so tiny that I think it's more likely an allergic reaction; folic acid is allergenic too, and the slightly different manufacturing processes could account for different reactions to different products.
    \end{answer}
\end{standalonequote}

\begin{qaexchange}{B Vitamins}
    \metadata{topic={Niacinamide Standalone Use}, source={Email Wiki}}

    \begin{question}
        Can Niacinamide be taken alone, or must it be combined with other B vitamins?
    \end{question}

    \begin{answer}
        It can be used alone.
    \end{answer}
\end{qaexchange}

\begin{standalonequote}{B Vitamins}
    \metadata{topic={Niacinamide for Quitting Smoking}, source={Email Wiki}}

    \begin{answer}
        After middle age, nicotine isn't likely to become addictive, and in small amounts it has nerve protective effects. Some of those effects probably overlap with the nerve protective effects of niacinamide. I haven't experimented with nicotine or tobacco, but I think transdermal application is preferable to smoking; carbon monoxide and other serious toxins are produced by burning the tobacco.
    \end{answer}
\end{standalonequote}

\begin{standalonequote}{B Vitamins}
    \metadata{topic={B Vitamin Allergies}, source={Email Wiki}}

    \begin{answer}
        Some of the B vitamins, especially B\textsubscript{2}, can be very allergenic. B\textsubscript{6} doesn't affect the others very much; 10 mg per day is a big dose.
    \end{answer}
\end{standalonequote}

\begin{standalonequote}{B Vitamins}
    \metadata{topic={B Vitamin Supplement Timing}, source={Email Wiki}}

    \begin{answer}
        Its effects are usually visible immediately, or within a few days, if it's going to be helpful. It's best in general to get the B vitamins from regular foods, occasionally with liver, because supplements usually contain contaminants that can cause allergic reactions when they are used for a long time. Other B vitamins that are usually safe for occasional use are B\textsubscript{1}, niacinamide, and pantothenic acid.
    \end{answer}
\end{standalonequote}

\begin{standalonequote}{B Vitamins}
    \metadata{topic={B Vitamin Allergy Testing}, source={Email Wiki}}

    \begin{answer}
        Because of individual sensitivities, each one should be tested carefully. Allergic reactions sometimes show up within a few minutes of contacting your mouth, other times it takes a couple of days to see a bad reaction. The worst one is B\textsubscript{2}, folic acid is next for allergies. B\textsubscript{1}, pantothenic acid, niacinamide and B\textsubscript{6} are pretty safe.
    \end{answer}
\end{standalonequote}

\begin{standalonequote}{Niacinamide}
    \metadata{topic={Weight Loss}, source={Ray Peat Forum}}

    \begin{answer}
        In a moderate amount, such as 150 mg/day, it will reduce the stress and help the liver to excrete some fatty acids.
    \end{answer}
\end{standalonequote}

\begin{qaexchange}{B Vitamins}
    \metadata{topic={Lowering Homocysteine}, source={Ray Peat Forum}}

    \begin{question}
        Is there a safe way to lower the Homocysteine level in the blood?
    \end{question}

    \begin{answer}
       A diet rich in folic acid, B\textsubscript{12}, B\textsubscript{6}, and good thyroid function. Milk, eggs, orange juice are helpful foods.
    \end{answer}
\end{qaexchange}

\begin{standalonequote}{B Vitamins}
    \metadata{topic={B Vitamin Supplement Dangers}, source={Ray Peat Forum}}

    \begin{answer}
       You should be able to tell quickly whether it's benefitting you. I just heard from another person, 85, who has used supplements for many years, with hypertension, swollen legs, and typical degenerative diseases, who recently stopped all supplements, and his blood pressure returned to what it was 30 years ago, the other symptoms completely disappeared. It isn't at all rare for people to have chronic disease cause by supplements. 
    \end{answer}
\end{standalonequote}

\begin{qaexchange}{B Vitamins}
    \metadata{topic={Vitamin B\textsubscript{6} and Anxiety}, source={Ray Peat Forum}}

    \begin{question}
        You've said that a requirement for vitamin B\textsubscript{6} is often implicated in anxiety, along with hyperventilation. Why is this specific nutrient so closely connected to that condition?
    \end{question}

    \begin{answer}
       It's essential for regulating neurotransmitters, and is wasted by estrogen.
    \end{answer}
\end{qaexchange}

\begin{qaexchange}{B Vitamins}
    \metadata{topic={High-Dose Thiamine Therapy}, source={Ray Peat Forum}}

    \begin{question}
        What are your thoughts on high dose thiamine therapy?
    \end{question}

    \begin{answer}
      I think it's appropriate to keep investigating its use in serious conditions such as Parkinson's.
    \end{answer}
\end{qaexchange}

\subsection{Vitamin C}

\begin{emailexchange}{Vitamin C}
    \metadata{topic={Sources}, source={Ray Peat Forum}}

    \begin{question}
        I know that you do not recommend supplimetal vitamin C because of purity issues, would an organic rosehips tea be a better option between the two?
    \end{question}

    \begin{answer}
        Most people get enough vitamin C without supplements.
    \end{answer}

    \begin{question}
        I only have fruit about every other day, so would I be getting enough vitamin C from dairy, seafood, meat and eggs? I thought the rosehips could be an emergency vitamin C source.
    \end{question}

    \begin{answer}
        Those foods (vegetables, too) provide adequate vitamin C.
    \end{answer}

    \begin{question}
        Doesn't \enquote{well cooked vegetables,} like broccoli, kill vitamin C?
    \end{question}

    \begin{answer}
        Broccoli, kale, etc., are very rich in vitamin C; some goes into the water, which should be used.
    \end{answer}
\end{emailexchange}

\begin{qaexchange}{Vitamin C}
    \metadata{topic={Liposomal Formulation}, source={Ray Peat Forum}}

    \begin{question}
        Is the liposomal method of delivery of nutrients good?
    \end{question}

    \begin{answer}
        The liposomal formulations are entirely for marketing. Things like insulin can be dissolved in oil, and the digestive process incorporates the oil into chylomicrons, i.e., liposomes.
    \end{answer}
\end{qaexchange}

\begin{standalonequote}{Vitamin C}
    \metadata{topic={Vitamin C from Milk/Meat}, source={Ray Peat Forum}}

    \begin{answer}
      A gallon provides enough vitamin C. The vitamin C in meat is usually also much higher than the charts say. At a time when my diet was mostly milk, meat, eggs, cheese, oatmeal, and bread, I decided to check the amount of vitamin C in my urine, and over the period that I checked it, it was steadily more than 2000 mg per day.
    \end{answer}
\end{standalonequote}

\subsection{Other Vitamins}

\begin{standalonequote}{Other Vitamins}
    \metadata{topic={Vitamin C Supplement Impurities}, source={Email Wiki}}

    \begin{answer}
        The trace impurities in synthetic ascorbic acid can increase free radical production, and quite a few people have allergic reactions to it. The situation isn't as clear with citric acid, but I think manufacturing impurities could account for some of the effects I see. For many years, I have been seeing more symptoms relieved by stopping all the chemical supplements, than by using them.
    \end{answer}
\end{standalonequote}

\section{Essential Minerals}

\begin{standalonequote}{Essential Minerals}
    \metadata{topic={Calcium To Phosphorus Ratio Importance}, source={Email Wiki}}

    \begin{answer}
        I think it's good to choose foods with a high ratio of calcium to phosphorous. Supplementing calcium (and often vitamin D is needed too) is usually necessary with the typical modern diet.
    \end{answer}
\end{standalonequote}

\subsection{Calcium}

\begin{standalonequote}{Calcium}
    \metadata{topic={Calcium Supplementation Sources}, source={Email Wiki}}

    \begin{answer}
        Unless you like cheese and milk, a calcium supplement would be the only way to balance the phosphate. Powdered eggshells are the best calcium supplement, oyster shells are the next best. Having some fruit, such as orange juice each time you eat meat will make the protein assimilation much more efficient, so less is needed. The natural sugar in fruit is mostly sucrose, equal parts of glucose and fructose, and the fruits have some of the minerals needed to use carbohydrate efficiently.
    \end{answer}
\end{standalonequote}

\begin{standalonequote}{Calcium}
    \metadata{topic={Calcium From Dairy}, source={Email Wiki}}

    \begin{answer}
        Milk and cheese are the best foods for getting enough calcium, and they will help to keep your protein intake up; an active person needs at least 100 grams daily for efficiency. \dots{}80 grams of protein daily is probably enough for a medium sized person who isn't very active. I have known people whose thyroid function improved noticeably when they increased their protein from 20 grams to 40 grams daily. (A quart of milk has 32 grams of protein, an egg about 6 grams.) If you depend on chicken for your major protein, it will contribute to suppressing your thyroid and progesterone. Increased salt helps to increase your metabolic rate. Low thyroid makes you lose salt too easily, and temporarily just eating more salt helps to make up for low thyroid-adrenals-progesterone.
    \end{answer}
\end{standalonequote}

\begin{standalonequote}{Calcium}
    \metadata{topic={Calcium Regulation By \ce{CO2}}, source={Email Wiki}}

    \begin{answer}
        Did they mention the \ce{CO2} or bicarbonate? That's usually low with hypothyroidism, and \ce{CO2} is what regulates calcium. Powdered eggshell (mixed with food) is a safe way to supplement calcium.
    \end{answer}
\end{standalonequote}

\begin{qaexchange}{Calcium}
    \metadata{topic={Calcium To Phosphorus Ratio}, source={Email Wiki}}

    \begin{question}
        Is 2000 mg calcium and 1000 mg phosphorus a safe ratio?
    \end{question}

    \begin{answer}
        Yes, that's safe. Even a 1 to 1 ratio is probably safe, but the ideal hasn't been clearly defined.
    \end{answer}
\end{qaexchange}

\begin{standalonequote}{Calcium}
    \metadata{topic={Calcium Phosphorus Weight Loss}, source={Email Wiki}}

    \begin{answer}
        The ratio of calcium to phosphate is very important; that's why milk and cheese are so valuable for weight loss, or for preventing weight gain. For people who aren't very active, low fat milk and cheese are better, because the extra fat calories aren't needed.
    \end{answer}
\end{standalonequote}

\begin{qaexchange}{Calcium}
    \metadata{topic={Eggshell Calcium}, source={Ray Peat Forum}}

    \begin{question}
        Is there a maximum amount of calcium from eggshells that someone can digest without problems in a single sitting? Also, is it better to space out eggshell calcium intake throughout the day, or does it matter?
    \end{question}

    \begin{answer}
        I've known people who took a tablespoonful with good results, but some people have digestive discomfort with that amount.
    \end{answer}
\end{qaexchange}

\begin{qaexchange}{Calcium}
    \metadata{topic={Intracellular Calcium and \ce{CO2}}, source={Ray Peat Forum}}

    \begin{question}
        What would be the difference between the people whose cells take up too much calcium leading to cell death and excitotoxicity and people who don't experience these calcium issues - where the calcium remains outside of the cell.

		Why would the cell take up too much calcium?

		I know you've said \ce{CO2} would expel excess calcium from the cell but would the lack of \ce{CO2} also cause the cells to retain it? Are there any other mechanisms through which this could happen?
    \end{question}

    \begin{answer}
      Yes, the lack of \ce{CO2} increases lactate; too little vitamin D and/or calcium in the diet increased parathyroid hormone, increasing lactate and intracellular calcium.
    \end{answer}
\end{qaexchange}

\begin{qaexchange}{Calcium}
    \metadata{topic={Calcium Glucarate Effects}, source={Ray Peat Forum}}

    \begin{question}
        Does calcium glucarate increase the elimination of testosterone and progesterone as well as estrogen?
    \end{question}

    \begin{answer}
      I'm not sure that it reliably does anything.
    \end{answer}
\end{qaexchange}

\begin{qaexchange}{Calcium}
    \metadata{topic={Eggshell Calcium with Vinegar}, source={Ray Peat Forum}}

    \begin{question}
        Do you have an opinion on the consumption of eggshells reacted with citric juices or vinegar?
    \end{question}

    \begin{answer}
      Carbonate has many beneficial effects, I think it's good to limit intake of acetate and citrate.
    \end{answer}
\end{qaexchange}

\begin{qaexchange}{Calcium}
    \metadata{topic={D-Calcium Phosphate Safety}, source={Ray Peat Forum}}

    \begin{question}
        Do you think D-Calcium Phosphate is safe to use?
    \end{question}

    \begin{answer}
      Yes.
    \end{answer}
\end{qaexchange}

\begin{qaexchange}{Calcium}
    \metadata{topic={Calcium and Magnesium from Leaves}, source={Ray Peat Forum}}

    \begin{question}
        Boil leaves for calcium, magnesium, and protein instead of drinking milk?
    \end{question}

    \begin{answer}
      The leaf water is good for magnesium and calcium, but has very little protein. About 3 cups of greens would provide enough of the minerals. If you don't eat the leaves, it takes more, since not all the calcium is released into the water.
    \end{answer}
\end{qaexchange}

\subsection{Magnesium}

\begin{standalonequote}{Magnesium}
    \metadata{topic={Magnesium Supplement Forms}, source={Email Wiki}}

    \begin{answer}
        Both carbonate and glycine are beneficial in themselves, but each of the compounds has its own impurities. Supplements of citrate have other effects on metabolism, that could be harmful.
    \end{answer}
\end{standalonequote}

\begin{standalonequote}{Magnesium}
    \metadata{topic={Magnesium Salicylate}, source={Email Wiki}}

    \begin{answer}
        I haven't tried magnesium salicylate, but most magnesium compounds have been seriously irritating to my intestine; I have mixed baking soda with salicylic acid, and it seems similar to aspirin. If the magnesium doesn't cause irritation, it would be a good form of salicylate. Magnesium salicylate is popular for arthritis, and it releases salicylic acid in the intestine and blood.
    \end{answer}
\end{standalonequote}

\begin{standalonequote}{Magnesium}
    \metadata{topic={Magnesium Food Sources}, source={Email Wiki}}

    \begin{answer}
        Cooked green leaves, or the water they were boiled in, is a very good source of magnesium, with other minerals in safe ratio. Coffee is another good magnesium source.Over 72 trace minerals from the Great Salt Lake, with 99\% of the salt removed, would be dirty salt, without the salt.
    \end{answer}
\end{standalonequote}

\begin{standalonequote}{Magnesium}
    \metadata{topic={Magnesium Supplement Allergies}, source={Email Wiki}}

    \begin{answer}
        All of the magnesium supplements that I have tried caused allergy symptoms and bowel inflammation, but some people don't have a problem with magnesium carbonate or magnesium glycinate. I think it's best to use the foods with high magnesium content, and to start the thyroid slowly, allowing the tissues time to absorb magnesium; the other minerals, Ca, Na, and K, have antistress effects that spare magnesium.
    \end{answer}
\end{standalonequote}

\begin{standalonequote}{Magnesium}
    \metadata{topic={Magnesium Carbonate vs Oxide}, source={Email Wiki}}

    \begin{answer}
        I don't recommend the oxide, because it's very poorly absorbed, but the carbonate is well absorbed. I don't recommend chemical supplements of magnesium, though, because they all contain some manufacturing impurities that can cause bowel inflammation, such as hemorrhoids. Well cooked greens are very good sources, coffee and chocolate are, too.
    \end{answer}
\end{standalonequote}

\begin{qaexchange}{Magnesium}
    \metadata{topic={Daily Intake}, source={Ray Peat Forum}}

    \begin{question}
        Do you have any recommendation for daily intake of magnesium? Is RDA of 420 mg for adult sufficient?
    \end{question}

    \begin{answer}
        With the average diet, that amount is enough. Good thyroid function, and plenty of calcium, potassium, and sodium can decrease the amount of magnesium needed.
    \end{answer}
\end{qaexchange}

\begin{qaexchange}{Magnesium}
    \metadata{topic={Deficiency with Thyroid}, source={Ray Peat Forum}}

    \begin{question}
        Is there a particular nutrient that could be deficient if someone gets muscle tension when increasing thyroid supplements?
    \end{question}

    \begin{answer}
      Usually magnesium. Cells contain more when they have thyroid.
    \end{answer}
\end{qaexchange}

\begin{standalonequote}{Magnesium}
    \metadata{topic={Magnesium Malate Safety}, source={Ray Peat Forum}}

    \begin{answer}
      I haven't seen it sold for a long time, but I think it's o.k.; it's good to watch for allergy reactions such as a headache.
    \end{answer}
\end{standalonequote}

\begin{standalonequote}{Magnesium}
    \metadata{topic={Magnesium Supplement Side Effects}, source={Ray Peat Forum}}

    \begin{answer}
      Magnesium supplements often cause bowel inflammation, and intestinal irritation can cause insomnia.
    \end{answer}
\end{standalonequote}

\begin{standalonequote}{Magnesium}
    \metadata{topic={Magnesium Supplement Forms}, source={Ray Peat Forum}}

    \begin{answer}
      Getting enough sodium in the diet helps to retain magnesium, but both of them are lost easily when thyroid function is low; when the thyroid status is good, the requirement for magnesium is easily met by ordinary foods. The things I most often recommend for magnesium are the water from boiling greens such as beet, chard, turnip and kale, and coffee. Magnesium carbonate is a very good supplement, except that it can cause intestinal irritation. People tell me that they don't have bowel irritation from magnesium glycinate. Either Mg chloride or Mg sulfate with baking soda can be absorbed through the skin.
    \end{answer}
\end{standalonequote}

\begin{qaexchange}{Magnesium}
    \metadata{topic={Magnesium with High Prolactin}, source={Ray Peat Forum}}

    \begin{question}
        Any dangers on taking magnesium when prolactin is high?
    \end{question}

    \begin{answer}
       Correcting a magnesium deficiency can help to correct excessive prolactin, but I haven't heard warnings related to prolactin. When an excess of magnesium is taken up very quickly it can cause profound anesthesia, but the kidneys soon excrete enough to correct it. 
    \end{answer}
\end{qaexchange}

\begin{qaexchange}{Magnesium}
    \metadata{topic={Magnesium Intake, Thyroid Retention}, source={Ray Peat Forum}}

    \begin{question}
        Is there an optimal daily magnesium target you recommend? An optimal Ca:Mg balance? In \textit{Nutrition for Women}, you recommend 1200mg.
    \end{question}

    \begin{answer}
      Thyroid function is the crucial thing for magnesium retention, much more important than the ratio and quantity.
    \end{answer}
\end{qaexchange}

\subsection{Sodium \& Salt}

\begin{qaexchange}{Sodium \& Salt}
    \metadata{topic={Salt Additives, Ferrocyanide}, source={Ray Peat Forum}}

    \begin{question}
        All the table salt here is either iodized, has sodium ferrocyanide or has potassium ferrocyanide. One brand, for example, reported a concentration of 0,005g/kg of the latter in their salt. Do you think this would be a problem? Or would sea salt be a better option?
    \end{question}

    \begin{answer}
      Usually sea salt is better, but a little of the ferrocyanide isn't likely to hurt.
    \end{answer}
\end{qaexchange}

\begin{standalonequote}{Sodium \& Salt}
    \metadata{topic={Salt Appetite as Indicator}, source={Ray Peat Forum}}

    \begin{answer}
      Salt appetite is usually a good indicator of need.
    \end{answer}
\end{standalonequote}

\begin{qaexchange}{Sodium \& Salt}
    \metadata{topic={Salt Restriction Harmful}, source={Ray Peat Forum}}

    \begin{question}
        What are your thoughts on the treatment of this condition and what nutritional changes would you recommend? Would salt restriction be helpful?
    \end{question}

    \begin{answer}
      I think salt restriction is usually harmful. It's important to avoid deficiencies of vitamin D, magnesium, calcium, selenium and thyroid hormone.
    \end{answer}
\end{qaexchange}

\section{Trace Minerals}

\begin{qaexchange}{Trace Minerals}
    \metadata{topic={Smell and Taste Loss}, source={Ray Peat Forum}}

    \begin{question}
        Is there any new research about losing sense of smell and taste due to covid? It's a 100\% loss of smell and taste for me and this point.
    \end{question}

    \begin{answer}
      Trace minerals (e.g., having sea food once a week) and coffee have helped to improve the sense of smell in some studies. \extlink{https://pubmed.ncbi.nlm.nih.gov/32841423/}{Source}
    \end{answer}
\end{qaexchange}

\begin{standalonequote}{Trace Minerals}
    \metadata{topic={Shellfish, Trace Minerals}, source={Ray Peat Forum}}

    \begin{answer}     
		Shell fish are good sources of several trace minerals, cooked greens (with the water they cook in) are good sources of molybdenum and calcium.
    \end{answer}
\end{standalonequote}

\subsection{Zinc}

\begin{standalonequote}{Zinc}
    \metadata{topic={Zinc Supplementation Cautions}, source={Email Wiki}}

    \begin{answer}
        Taking zinc orally, 5 or 10 mg, can replenish the body's stores in a few days, but the supplement can oxidize other nutrients in the stomach or intestine, so it isn't good to use it for a long time.
    \end{answer}
\end{standalonequote}

\subsection{Copper}

\begin{standalonequote}{Copper}
    \metadata{topic={Copper Acetate Preparation}, source={Email Wiki}}

    \begin{answer}
        I made it myself, soaking a piece of pure copper with aspirin in water, until a very pale blue color developed. Later, the solution became a deeper blue color, and at a certain concentration I think it's toxic.
    \end{answer}
\end{standalonequote}

\begin{standalonequote}{Copper}
    \metadata{topic={Dietary Sources}, source={Ray Peat Forum}}

    \begin{answer}
        Having shrimp, oysters, or other shellfish once or twice a week will correct a deficiency.
    \end{answer}
\end{standalonequote}

\begin{emailexchange}{Copper}
    \metadata{topic={Copper Bracelet Absorption}, source={Ray Peat Forum}}

    \begin{question}
        I was not specifically asking if they are safe to wear, but rather, if copper bracelets might actually be beneficial in terms of its electric morphogenesis or simply just absorption through the skin. Maybe a Brass (5\% Zinc) bracelet would be of interest, too?
    \end{question}

    \begin{answer}
      A copper bracelet does release enough copper to be absorbed in a nutritionally useful quantity.
    \end{answer}

    \begin{question}
        Does the copper differ from \enquote{organic copper} found in e.g. milk? Does it matter or can the body produce enzymes to deal with that?
    \end{question}

    \begin{answer}
      With very small amounts there's not much difference.
    \end{answer}
\end{emailexchange}

\subsection{Iron}

\begin{standalonequote}{Iron}
    \metadata{topic={High Ferritin And Inflammation}, source={Email Wiki}}

    \begin{answer}
        High ferritin suggests that there's continuing inflammation. Iron and calcium interact, so it might be worth having your parathyroid hormone tested. Despite your good vitamin D, you might not be getting enough calcium in relation to phosphate, and elevated PTH can cause generalized inflammation. Safe antiinflammatory things would be aspirin, calcium carbonate, coffee especially when taken with meat or eggs, salt or baking soda, and sugar. In the US and Canada, I have noticed that the \enquote{normal range} for prolactin has been expanded upward, after a period in the '80s when it was lowered. I think this reflects a change in the population, from estrogen and PUFA, for example, and that the lower range was better for judging health.
    \end{answer}
\end{standalonequote}

\begin{standalonequote}{Iron}
    \metadata{topic={Ferritin And Hypothyroidism}, source={Email Wiki}}

    \begin{answer}
        Uric acid is important as an antioxidant. High ferritin doesn't directly imply high iron stores, it has a defensive effect, and can be increased by inflammation. TSH promotes inflammation. Hypothyroidism usually involves low temperature of the extremities, and the bones of the arms and legs form red cells slowly at low temperature, so it's possible that ferritin is involved in an adaptive mechanism, too.
    \end{answer}
\end{standalonequote}

\begin{standalonequote}{Iron}
    \metadata{topic={Ferritin And Thyroid Function}, source={Email Wiki}}

    \begin{answer}
        Have you had your thyroid checked? Abnormal ferritin can result from thyroid malfunction.
    \end{answer}
\end{standalonequote}

\begin{standalonequote}{Iron}
    \metadata{topic={Coffee Reducing Iron Absorption}, source={Email Wiki}}

    \begin{answer}
        Drinking coffee with meals will greatly reduce iron absorption. Abnormal thyroid status can affect ferritin level, without necessarily affecting your iron load.
    \end{answer}
\end{standalonequote}

\begin{standalonequote}{Iron}
    \metadata{topic={Ferritin And Inflammation}, source={Email Wiki}}

    \begin{answer}
        I assume that conventional medicine has misunderstood its role, I'm not sure that I can think of anything that conventional medicine doesn't misunderstand. Hypothyroidism increases inflammation and decreases kidney function; even protective antioxidants can become problems in themselves under some circumstances. Ferritin binds iron, and while it's bound it is less likely to produce random free radical damage. If there is inflammation in the liver or bone marrow, the inflammation can cause iron to be released, and ferritin apparently acts as a buffer, absorbing the released iron.
    \end{answer}
\end{standalonequote}

\begin{standalonequote}{Iron}
    \metadata{topic={Iron Absorption With Juice}, source={Email Wiki}}

    \begin{answer}
        Although orange juice would tend to increase iron absorption, that combination hasn't been studied. It isn't an issue for most people, only someone with an iron overload issue. The copper in oysters is protective against iron excess.
    \end{answer}
\end{standalonequote}

\begin{standalonequote}{Iron}
    \metadata{topic={Heme vs Non-Heme Iron}, source={Email Wiki}}

    \begin{answer}
        I think that's true, that coffee affects mainly non-heme iron absorption. The heme has toxic effects, forming carbon monoxide, apart from the iron. As I understand it, the amount of non-heme iron that's absorbed increases with the extent of its reduction, with ferrous iron being absorbed much more than the ferric form. The presence of reductants in the food will increase absorption. 
    \end{answer}
\end{standalonequote}

\begin{standalonequote}{Iron}
    \metadata{topic={Iron Supplement Cautions}, source={Email Wiki}}

    \begin{answer}
        As long as your hemoglobin is o.k., I wouldn't use an iron supplement, because so many things can influence the amount of iron in the blood, even when there's enough in the liver and marrow. Have you been getting enough copper and other trace minerals in your diet? Including shellfish (oysters have a lot of iron as well as other trace minerals) and liver in your diet would be the safe way to increase your iron and hemoglobin. Did you have your hormones measured? High cortisol can reduce the amount of iron in the blood while increasing it in the liver.
    \end{answer}
\end{standalonequote}

\begin{standalonequote}{Iron}
    \metadata{topic={Iron Stress Reactions}, source={Email Wiki}}

    \begin{answer}
        Too much iron, especially in the reduced form, activates a variety of harmful stress reactions.
    \end{answer}
\end{standalonequote}

\begin{qaexchange}{Iron}
    \metadata{topic={Methylene Blue with Iron Overload}, source={Ray Peat Forum}}

    \begin{question}
        Would methylene blue be dangerous or beneficial in an iron overload state?
    \end{question}

    \begin{answer}
        Like ascorbic acid, it can reduce ferric iron to the ferrous form, which creates the most toxic free radicals.
    \end{answer}
\end{qaexchange}

\subsection{Selenium}

\begin{qaexchange}{Selenium}
    \metadata{topic={Selenium Supplementation Frequency}, source={Ray Peat Forum}}

    \begin{question}
        For someone taking a supplement, how many times a week should a 200mcg pill be taken to restore T\textsubscript{4} conversion, presuming adequate protein?
    \end{question}

    \begin{answer}
      During the first week, every day would be o.k., then I think once or twice a week is enough.
    \end{answer}
\end{qaexchange}

\subsection{Other Trace Elements}

\begin{qaexchange}{Other Minerals}
    \metadata{topic={Manganese In Diet}, source={Email Wiki}}

    \begin{question}
        Oranges and milk listed as having low manganese. Is this true?
    \end{question}

    \begin{answer}
        Yes, but occasional eggs, liver, oysters, etc., provide enough.
    \end{answer}
\end{qaexchange}

\begin{standalonequote}{Other Minerals}
    \metadata{topic={Strontium Safety}, source={Email Wiki}}

    \begin{answer}
        The normal kind of strontium seems to be harmless, it was the radioactive kind from nuclear bombs and industry that was used as an indicator of fallout contamination, and was highly associated with leukemia. Around 1960 there were warnings about the danger of milk contamination, but vegetables were the greatest source of it.
    \end{answer}
\end{standalonequote}

\begin{qaexchange}{Trace Minerals}
    \metadata{topic={Strontium}, source={Ray Peat Forum}}

    \begin{question}
        A doctor prescribed strontium for bone health for my mom. Is this needed? I associate strontium with radioactivity. Could this be an offshoot of the idea of hormetics pushed on the public?
    \end{question}

    \begin{answer}
        There's nothing left in official medicine besides marketing; when a market fad fails, new ones will appear.
    \end{answer}
\end{qaexchange}

\begin{emailexchange}{Trace Minerals}
    \metadata{topic={Deuterium}, source={Ray Peat Forum}}

    \begin{answer}
        The body can't eliminate deuterium, except by diluting it when pure H2O is taken in.
    \end{answer}

    \begin{question}
        Have you experimented with this? Or is it more of a strategy eating certain foods that are low (i.e: young fruits I believe, etc.)
    \end{question}

    \begin{answer}
        The exchange of deuterium is slow—it slows the metabolic processes that would eliminate it; it's like the process of aging.
    \end{answer}
\end{emailexchange}

\begin{standalonequote}{Other Trace Elements}
    \metadata{topic={Silicon Safety as Nutrient}, source={Ray Peat Forum}}

    \begin{answer}
      I don't think silicon is safe, or has value as a nutrient.
    \end{answer}
\end{standalonequote}

\begin{standalonequote}{Other Trace Elements}
    \metadata{topic={Silicon in Foods}, source={Ray Peat Forum}}

    \begin{answer}
      I recommend many foods containing considerable amounts of silicon, such as meats, liver, shell fish, bamboo shoots, and mushrooms, but there are many products on the market that contain silicon in forms that can be harmful.
    \end{answer}
\end{standalonequote}

\begin{standalonequote}{Other Trace Elements}
    \metadata{topic={Iodine Cult, Guy Abraham}, source={Ray Peat Forum}}

    \begin{answer}
      Used occasionally as a topical antiseptic, tincture of iodine is safe. Historically iodide has been used to treat a breast infection. That's very different from the cult of daily use of large amounts of iodide, started by Guy Abraham.

      The founder of the current iodine cult, Guy Abraham, was promoting iodine along with their radiation devices to protect against electromagnetic pollution. I couldn't decide whether he really believed those things, or just used them to sell his product.
    \end{answer}
\end{standalonequote}

\subsection{Iodine}

\begin{standalonequote}{Iodine}
    \metadata{topic={Iodine Use Cautions}, source={Email Wiki}}

    \begin{answer}
        Lugol's solution is sometimes helpful for an inflammation, but it's risky when there might be a thyroid problem.
    \end{answer}
\end{standalonequote}

\begin{standalonequote}{Iodine}
    \metadata{topic={Iodine Thyroiditis Risk}, source={Email Wiki}}

    \begin{answer}
        Short term use of iodide is safe at a few milligrams per day, but chronic intake of even one mg. per day increases the risk of thyroiditis.
    \end{answer}
\end{standalonequote}

\section{General Supplementation}

\begin{standalonequote}{General Supplementation}
    \metadata{topic={Supplement Allergens Causing Headaches}, source={Email Wiki}}

    \begin{answer}
        Have you tried a large oral dose of progesterone? A very large amount of sugar will usually relieve a migraine; ice cream (about a quart) or milk shakes with some fat and protein make it easier to assimilate the sugar without stomach upset. Caffeine sometimes makes the aspirin and sugar more effective. Did any of the magnesium chloride get on your lips? In my own migraine experience, I found that a very small amount of either vitamin A or magnesium chloride could cause big headaches for two or three days. If I had put vitamin A anywhere on my face or arms, enough would touch my lips to cause the headache. It wasn't the vitamin A or magnesium itself that did it, but some very powerful allergen in the chemically manufactured products. It's possible that some such substance has entered the T\textsubscript{3} during its manufacture, so using a different brand might avoid the effect. What brands of T\textsubscript{3} and desiccated have you used? Is cyproheptadine available where you are? It's probably the safest of the antiserotonin drugs; here are some articles about it.
    \end{answer}
\end{standalonequote}

\begin{standalonequote}{General Supplementation}
    \metadata{topic={Vitamin Needs With High Metabolism}, source={Email Wiki}}

    \begin{answer}
        I think the mineral and vitamin requirements do increase with calorie requirement.
    \end{answer}
\end{standalonequote}

\begin{standalonequote}{General Supplementation}
    \metadata{topic={Supplement Irritants}, source={Email Wiki}}

    \begin{answer}
        Starches, preservatives, and antioxidants are likely to irritate,stearic acid isn't likely to be a problem.
    \end{answer}
\end{standalonequote}

\begin{standalonequote}{General Supplementation}
    \metadata{topic={Supplement Contamination Concerns}, source={Email Wiki}}

    \begin{answer}
        Because of contaminants in supplements I seldom recommend the oral use of any of them, except aspirin, which can be dissolved in warm water to remove most of the additives. In the winter I use vitamin D, but only on my skin in an oil. Using a thyroid supplement temporarily might help to lower your estrogen.
    \end{answer}
\end{standalonequote}

\begin{standalonequote}{General Supplementation}
    \metadata{topic={Topical Vitamin PUFA Concern}, source={Email Wiki}}

    \begin{answer}
        As long as you use the vitamin topically it would not do any harm but be careful to not expose your skin to direct sunlight.
    \end{answer}
\end{standalonequote}

\begin{standalonequote}{General Supplementation}
    \metadata{topic={Long-Term Supplement Safety}, source={Email Wiki}}

    \begin{answer}
        Most supplements contain enough impurities to eventually cause problems. Thyroid and aspirin are among the safest, and the most likely to be valuable indefinitely. It depends on where you live, but vitamin D3, vitamin K, and selenium deficiencies are extremely widespread.
    \end{answer}
\end{standalonequote}

\begin{standalonequote}{Supplements}
    \metadata{topic={Transdermal Absorption}, source={Ray Peat Forum}}

    \begin{answer}
        The oily vitamins are well absorbed, but it depends on the skin, and how often the person washes. 5\% is probably a typical absorption. Water soluble vitamins are poorly absorbed, but covering a large area a significant amount is absorbed.
The legs, inner arms, tops of feet, and throat are good places for transdermal absorption.
    \end{answer}
\end{standalonequote}

\begin{qaexchange}{General Supplementation}
    \metadata{topic={Hair vs Toenail Mineral Testing}, source={Ray Peat Forum}}

    \begin{question}
        Do you have any opinion on hair mineral testing and analysis? Can it be a useful way to acquire information about mineral status and other things going on in the body, do you think?
    \end{question}

    \begin{answer}
       Toenails are much better, because they absorb more from the body, less from environmental air and water exposure. 
    \end{answer}
\end{qaexchange}

\begin{standalonequote}{General Supplementation}
    \metadata{topic={Vitamin Balance, Thyroid Monitoring}, source={Ray Peat Forum}}

    \begin{answer}
       Increased vitamin A increases your need for vitamin E, phosphate should be balanced with calcium, and it's important to watch thyroid function. 
    \end{answer}
\end{standalonequote}

\begin{standalonequote}{General Supplementation}
    \metadata{topic={Chronic Cough from Supplements}, source={Ray Peat Forum}}

    \begin{answer}
       Poorly digested foods, especially things like spices, nuts, green salads, and legumes, can cause a chronic cough. I had a terrible cough for more than a year, which I discovered was caused by the vitamin C supplement I was using. I learned that many people have chronic problems from their supplements. That can be a problem with malate and succinate. 
    \end{answer}
\end{standalonequote}

\begin{standalonequote}{General Supplementation}
    \metadata{topic={Peptide Supplements Critique}, source={Ray Peat Forum}}

    \begin{answer}
      Currently popular peptides have produced amazing results for the people who sell it. I think most of them are either useless or dangerous.
    \end{answer}
\end{standalonequote}

\begin{standalonequote}{General Supplementation}
    \metadata{topic={Amino Acid Supplement Purity}, source={Ray Peat Forum}}

    \begin{answer}
      The purity of individual amino acids on the market has been a real issue, so their theoretical benefits have to be considered in relation to what's available.
    \end{answer}
\end{standalonequote}

\begin{qaexchange}{General Supplementation}
    \metadata{topic={Freeze-Dried Organ Supplements}, source={Ray Peat Forum}}

    \begin{question}
        What are your thoughts on supplementing freeze dried thyroid gland, liver testicle and other beef organs, deer antler velvet extract, freeze dried oyster, krill oil, nigella sativa oil?
    \end{question}

    \begin{answer}
      I prefer, for sanitation, standard desiccated thyroid. Other dehydrated substances often contain lipid peroxides and tryptophan decomposition products.
    \end{answer}
\end{qaexchange}

\begin{standalonequote}{General Supplementation}
    \metadata{topic={Supplement Safety Concerns}, source={Ray Peat Forum}}

    \begin{answer}
      Some chemicals that are very important metabolically can be disruptive when they are taken orally in inappropriate forms or combinations or amounts. For example, free ionic copper taken with food can degrade vitamins and amino acids. PQQ is probably safe, but I think much more research is needed to be sure. Most vitamin E products contain small amounts of potentially harmful impurities. The amount of aspartate in a tablet of magnesium aspartate is harmless, but it's good to be cautious; I knew someone who began having daily seizures when he used large amounts of magnesium and calcium aspartate; his daily seizures continued for four months, and stopped the day after he discontinuted the supplements. I think there's a risk of methyl imbalance when large amounts of betaine are used chronically.
    \end{answer}
\end{standalonequote}

\section{Other Supplements}

\begin{standalonequote}{Other Supplements}
    \metadata{topic={CBD, Nitric Oxide, Prostaglandins}, source={Ray Peat Forum}}

    \begin{answer}
      It can probably be helpful for lowering nitric oxide in some situations, but there's some evidence that it increases prostaglandins, which could be harmful in a person whose tissues have a lot of PUFA. The acidic form, CBDA, seems likely to be more broadly protective.
    \end{answer}
\end{standalonequote}

\begin{standalonequote}{Other Supplements}
    \metadata{topic={Lactobacilli, Kefiran}, source={Ray Peat Forum}}

    \begin{answer}
      Even dead lactobacilli have an antiinflammatory effect in the intestine, so it's probably something in their cell wall coat, and kefiran might have a physical effect of that sort.
    \end{answer}
\end{standalonequote}

\begin{standalonequote}{Other Supplements}
    \metadata{topic={Melatonin Safety Concerns}, source={Ray Peat Forum}}

    \begin{answer}
      It's produced during the stress of darkness, by adrenergic nervous stimulation, and I think one of its functions is to neutralize serotonin, which could be an important antistress function. An amount much smaller than a milligram can bring on sleep, but I doubt the safety of larger amounts. Here are some articles about its involvement in inflammation:
    \end{answer}
\end{standalonequote}

\begin{qaexchange}{Other Supplements}
    \metadata{topic={Flowers of Sulfur for Intestine}, source={Ray Peat Forum}}

    \begin{question}
        Do you think flowers of sulphur is safe for ingestion for the purpose of killing fungus, bacteria and or parasites? If so, do you know of a good amount for those purposes?
    \end{question}

    \begin{answer}
      I found that a pinch, about 200 mg, daily for 3 or 4 days was effective for changing my intestinal flora; the effect lasted a few months. After using it occasionally for many years, I became sensitized to it, and it produced intestinal irritation.
    \end{answer}
\end{qaexchange}

\begin{qaexchange}{Other Supplements}
    \metadata{topic={C. Butryicum Probiotic Safety}, source={Ray Peat Forum}}

    \begin{question}
        Besides the excipients, do you think C. Butryicum is safe? (Miyarisan)?
    \end{question}

    \begin{answer}
      I think so.
    \end{answer}
\end{qaexchange}

\begin{standalonequote}{Other Supplements}
    \metadata{topic={Bacillus Probiotics}, source={Ray Peat Forum}}

    \begin{answer}
       There used to be some products available in health food stores (they are probably still available somewhere) containing B. subtilis and B. licheniformis that produce antibiotics in the intestine (or in milk if you culture them), which are good for people who are afraid of antibiotics in pill form. 
    \end{answer}
\end{standalonequote}

\begin{standalonequote}{Other Supplements}
    \metadata{topic={Urea Dosage for Health}, source={Ray Peat Forum}}

    \begin{answer}
      The big doses were used for major brain injuries, heart failure, and cancer. It's very non-toxic, but for minor problems a teaspoonful in a glass of juice twice a day is likely to be enough.
    \end{answer}
\end{standalonequote}

\begin{standalonequote}{Other Supplements}
    \metadata{topic={CBD Effects, PUFA}, source={Ray Peat Forum}}

    \begin{answer}
      I think the effects of CBD probably depend on how much polyunsaturated fat the person's tissues have; the endogenous cannabinoids can increase inflammation, so the plant analog can be antiinflammatory by reducing that effect.
    \end{answer}
\end{standalonequote}

\begin{standalonequote}{Other Supplements}
    \metadata{topic={Baking Soda Increasing \ce{CO2}}, source={Ray Peat Forum}}

    \begin{answer}
      Drinking a little baking soda in water helps to increase internal \ce{CO2}.
    \end{answer}
\end{standalonequote}

\begin{standalonequote}{Other Supplements}
    \metadata{topic={Activated Charcoal, Endotoxin}, source={Ray Peat Forum}}

    \begin{answer}
      Activated charcoal can absorb many toxins, including bacterial endotoxin, so it is likely to reduce serotonin absorption from the intestine. Since it can also bind or destroy vitamins, it should be used only intermittently. Frolkis, et al. (1989, 1984) found that it extended median and average lifespan of rats, beginning in old age (28 months) by 43\% and 34\%, respectively, when given in large quantities (equivalent to about a cup per day for humans) for ten days of each month.
    \end{answer}
\end{standalonequote}

\subsection{Inositol}

\begin{standalonequote}{Inositol}
    \metadata{topic={Inositol}, source={Ray Peat Forum}}

    \begin{answer}
        I think it's safe in doses of a few grams/day; it seems to have some protective effects, for example against cataracts (which are promoted by estrogen), but I don't know what its exact mechanisms are.
    \end{answer}
\end{standalonequote}

\chapter{Hormones \& Endocrine Health}

\section{Thyroid Function}

\begin{standalonequote}{Thyroid Function}
    \metadata{topic={Thyroid vs Serotonin Drugs}, source={Email Wiki}}

    \begin{answer}
        Thyroid is the best thing for controlling serotonin's effects. The drugs that act on {receptors} act simultaneously on many things; one effect of some of them is a selective \enquote{agonist} effect on the \enquote{receptor} which is involved in negative feedback, turning off the cells that produce serotonin. Wikipedia is a function of consensus; according to them, serotonin is a happy hormone, and there are no conspiracies of government officials and bankers.
    \end{answer}
\end{standalonequote}

\begin{standalonequote}{Thyroid Function}
    \metadata{topic={Optimal TSH Level}, source={Email Wiki}}

    \begin{answer}
        I think it's best to keep the TSH around 0.4
    \end{answer}
\end{standalonequote}

\begin{standalonequote}{Thyroid Function}
    \metadata{topic={TSH And Hair Loss}, source={Email Wiki}}

    \begin{answer}
        I think it's good to have TSH below 0.4, and that probably contributes to loss of hair.
    \end{answer}
\end{standalonequote}

\begin{standalonequote}{Thyroid Function}
    \metadata{topic={TSH Below 0.4 And Cancer}, source={Email Wiki}}

    \begin{answer}
        I think it's good to have lower TSH. It contributes to some of the circulatory and inflammatory problems seen in hypothyroidism. People with TSH below 0.4 were the freest from thyroid cancer. The amount of body fat contributes to both prostate and breast cancer, largely because it's a chronic source of estrogen, by converting the protective androgens. Milk drinkers tend to be the least obese (e.g., the Masai people). One study saw an association of skimmed milk with prostate cancer, but not whole milk, probably because fat people avoid whole milk. Powdered eggshells are a good alternative source of calcium, but milk and cheese are better. When the TSH is lower, the estrogen will probably be lower too.
    \end{answer}
\end{standalonequote}

\begin{standalonequote}{Thyroid Function}
    \metadata{topic={Cramps And Thyroid Function}, source={Email Wiki}}

    \begin{answer}
        It [Progest-E] can help with cramps, but it would probably take a lot; I think it's better to use thyroid (including T\textsubscript{3}) to solve the basic problem, since it will let you regulate the balance between estrogen and progesterone, while allowing your cells to balance the minerals, retaining the magnesium needed to prevent cramping. Increasing your intake of all the main minerals, calcium, sodium, potassium, and magnesium usually helps in the short term, but the balance isn't stable if your thyroid is low. Milk, orange juice, coffee (even decaffeinated coffee is a good source of magnesium), and well salted foods, support thyroid functions. Aspirin helps with thyroid function and mineral balance, even helps to prevent excessive estrogen production.
    \end{answer}
\end{standalonequote}

\begin{standalonequote}{Thyroid Function}
    \metadata{topic={Thyroid Controlling Serotonin}, source={Email Wiki}}

    \begin{answer}
        Thyroid is the best thing for controlling serotonin's effects. The drugs that act on \enquote{receptors} act simultaneously on many things; one effect of some of them is a selective \enquote{agonist} effect on the \enquote{receptor} which is involved in negative feedback, turning off the cells that produce serotonin. Wikipedia is a function of consensus; according to them, serotonin is a happy hormone, and there are no conspiracies of government officials and bankers.
    \end{answer}
\end{standalonequote}

\begin{standalonequote}{Thyroid Function}
    \metadata{topic={T\textsubscript{4} To T\textsubscript{3} Conversion Factors}, source={Email Wiki}}
    \begin{note}
        Improving T\textsubscript{4} \rightarrow{} T\textsubscript{3} Conversion
    \end{note}

    \begin{answer}
        If you were deficient in selenium, the correcting effect would be quick, but if there was a problem with intestinal flora, that would have to be taken care of before conversion was good. Other nutritional deficiencies could be involved. Daily raw carrot, weekly seafood and liver, enough sunlight and vitamin D, a good ratio of calcium to phosphate, are often helpful.
    \end{answer}
\end{standalonequote}

\begin{standalonequote}{Thyroid Function}
    \metadata{topic={Intestinal Flora And T\textsubscript{4} Conversion}, source={Email Wiki}}

    \begin{answer}
        Calcium (two liters of milk), vitamin D and plenty of orange juice sometimes help to regulate things by balancing the minerals. A daily carrot salad should keep the small intestine fairly sterile.
    \end{answer}
\end{standalonequote}

\begin{standalonequote}{Thyroid Function}
    \metadata{topic={Reverse T\textsubscript{3} And Cortisol}, source={Email Wiki}}

    \begin{answer}
        Yes, it's probably induced by stress, with cortisol inducing the type of deiodinase that makes the inactive rT\textsubscript{3}. A low sugar diet can cause chronically high cortisol. If you are eating enough fruit and protein, I think the T\textsubscript{3} of natural thyroid will help to correct the stress/inflammatory metabolism that is connected with the reverse T\textsubscript{3}.
    \end{answer}
\end{standalonequote}

\begin{standalonequote}{Thyroid Function}
    \metadata{topic={T\textsubscript{3} Cellular Uptake}, source={Email Wiki}}
    \begin{note}
        Things Inhibiting T\textsubscript{3} from Entering Cells
    \end{note}

    \begin{answer}
        It isn't a matter of T\textsubscript{3} entering cells, it's assuring that it is either made by conversion from the T\textsubscript{4}, or taken as a supplement.
    \end{answer}
\end{standalonequote}

\begin{standalonequote}{Thyroid Function}
    \metadata{topic={T\textsubscript{2} Dosing Concerns}, source={Email Wiki}}
    \begin{note}
        Question About T\textsubscript{2} Mentioning 150mcg Capsules
    \end{note}

    \begin{answer}
        Mitochondria have the enzyme for converting T\textsubscript{3} to T\textsubscript{2}. The potency seems crazy, the body needs only about 4 mcg per hour.
    \end{answer}
\end{standalonequote}

\begin{standalonequote}{Thyroid Function}
    \metadata{topic={TSH Target Range}, source={Email Wiki}}

    \begin{answer}
        With your TSH so high, you should probably add a thyroid supplement, until you get it down to about 1.0, or less. (The normal range, according to the American Association of Clinical Endocrinologists, is from 0.3 to 3.0.)
    \end{answer}
\end{standalonequote}

\begin{standalonequote}{Thyroid Function}
    \metadata{topic={Diet And Climate Importance}, source={Email Wiki}}
    \begin{note}
        THyroid Not Enough to Lower Stress Hormones
    \end{note}

    \begin{answer}
        Yes, the diet is an essential part of normalizing them. The climate is important, too.
    \end{answer}
\end{standalonequote}

\begin{standalonequote}{Thyroid Function}
    \metadata{topic={T\textsubscript{4} To T\textsubscript{3} Conversion Requirements}, source={Email Wiki}}

    \begin{answer}
        The liver has to convert T\textsubscript{4} to T\textsubscript{3} for it to be effective. It needs glucose and selenium to make the conversion. Adequate protein, at least 80 grams per day, is necessary. Sea food, once a week will provide selenium, two quarts of milk and a quart of orange juice would provide many of the other essential nutrients. Taking T\textsubscript{4} at bedtime sometimes is helpful. Most people feel best on a ratio of T\textsubscript{4}:T\textsubscript{3} of 4:1 or less. Checking the relaxation rate of the Achilles reflex is a quick way to check the effect of the thyroid on your nerves and muscles; the relaxation should be instantaneous, loose and floppy.
    \end{answer}
\end{standalonequote}

\begin{standalonequote}{Thyroid Function}
    \metadata{topic={Thyroid Preventing Obesity}, source={Email Wiki}}

    \begin{answer}
        Yes, most of the research shows that it increases the metabolic rate, tending to prevent obesity.
    \end{answer}
\end{standalonequote}

\begin{qaexchange}{Thyroid Function}
    \metadata{topic={Thyroid Testing Importance}, source={Ray Peat Forum}}

    \begin{question}
        Are thyroid blood tests (full panels) a worthwhile exercise prior to supplementing cynomel and a few months later?
    \end{question}

    \begin{answer}
      They are interesting, but usually free T\textsubscript{3} and TSH are the most important ones. Judging by symptoms, temperature, and pulse rate is really the basic thing.
    \end{answer}
\end{qaexchange}

\begin{standalonequote}{Thyroid Function}
    \metadata{topic={Thyroid, Estrogen Conversion, Teeth}, source={Ray Peat Forum}}

    \begin{answer}
      Thyroid and other antiinflammatory things (even aspirin) help to prevent conversion to estrogen. Since stress can quickly decalcify teeth, a good state without stress should make teeth whiter.
    \end{answer}
\end{standalonequote}

\begin{standalonequote}{Thyroid Function}
    \metadata{topic={T\textsubscript{4}/T\textsubscript{3} Ratio, Metabolic Response}, source={Ray Peat Forum}}

    \begin{answer}
      The serum T\textsubscript{3}/T\textsubscript{4} ratio decreases with age and sickness. I think dose of a supplement should be based entirely on the signs of metabolic response.
    \end{answer}
\end{standalonequote}

\begin{standalonequote}{Thyroid Function}
    \metadata{topic={T\textsubscript{4} Sparing Cholesterol}, source={Ray Peat Forum}}

    \begin{answer}
      T\textsubscript{4} suppresses the pro-inflammatory TSH, without activating the metabolism, so probably spares the cholesterol and other antiinflammatory things. Does she eat enough sugar? Starches and irritating, bacteria-supporting foods increase inflammation and probably interfere with cholesterol synthesis. Custards, sweet fruits, and Haagen Dazs ice cream are safe ways to increase cholesterol.
    \end{answer}
\end{standalonequote}

\begin{standalonequote}{Thyroid Function}
    \metadata{topic={TSH, Mineral Retention}, source={Ray Peat Forum}}

    \begin{answer}
      Keeping the TSH a little lower is the best way to assure balance of the minerals, since it helps with the retention of sodium and magnesium. Drinking milk and salting food to taste will provide a good balance when the stress hormones are low.
    \end{answer}
\end{standalonequote}

\begin{standalonequote}{Thyroid Function}
    \metadata{topic={T\textsubscript{3} Tissue-Specific Metabolism}, source={Ray Peat Forum}}

    \begin{answer}
       Every tissue of the body that has been tested for it has the enzymes needed to metabolize thyroxine; toxins, such as lipid peroxides, can affect some tissues more than others, making a T\textsubscript{3} supplement more valuable for them. Systemic changes, such as in calcium, phosphate, vitamin D, and sugar, can affect them, and a higher-than-normal concentration of hormone on the skin, can sometimes overcome those regional defects. One researcher found that T\textsubscript{3} is normally converted into 3,5-T\textsubscript{2} by functional mitochondria. 
    \end{answer}
\end{standalonequote}

\begin{standalonequote}{Thyroid Function}
    \metadata{topic={T\textsubscript{4} to T\textsubscript{3} Conversion Variability}, source={Ray Peat Forum}}

    \begin{answer}
      Women and older people generally have reduced conversion of T\textsubscript{4} to T\textsubscript{3}, but anyone with liver malfunction is similar.
    \end{answer}
\end{standalonequote}

\begin{qaexchange}{Thyroid Function}
    \metadata{topic={Pulse Rate Targets}, source={Ray Peat Forum}}

    \begin{question}
        You mention that you keep your pulse averaging over 90. Why not 95 or 100, and are there deleterious effects with going that high?
    \end{question}

    \begin{answer}
      With it that high I easily get out of breath and over-heat with moderate work intensity.
    \end{answer}
\end{qaexchange}

\begin{standalonequote}{Thyroid Function}
    \metadata{topic={Monitoring Thyroid Function}, source={Ray Peat Forum}}

    \begin{answer}
      Pulse rate, combined with temperature, helps, and the temperature of the fingers and toes is very useful. A person can learn to interpret feelings of anxiety or depression, but it's good to have something objective. The Achilles reflex relaxation rate is good, if you have someone to thump, and don't follow the advice of internet doctors.
    \end{answer}
\end{standalonequote}

\begin{qaexchange}{Thyroid Function}
    \metadata{topic={Allergies Lowering Thyroid}, source={Ray Peat Forum}}

    \begin{question}
        Do you think its possible that if someone has been chronically eating foods they are allergic or sensitive to that the body might defensively lower thyroid function?
    \end{question}

    \begin{answer}
      The stress hormones produced by allergy cause the liver to convert thyroxine to reverse T\textsubscript{3}, lowering thyroid function, apparently in a kind of defensive semi-hibernation.
    \end{answer}
\end{qaexchange}

\begin{qaexchange}{Thyroid Function}
    \metadata{topic={Goitrogens Without Thyroid Gland}, source={Ray Peat Forum}}

    \begin{question}
        How important is it for people with NO thyroid gland to still avoid goitrogenic foods?
    \end{question}

    \begin{answer}
      The various tissues have their local thyroid activating enzymes producing T\textsubscript{3}, and some antithyroid agents can block those.
    \end{answer}
\end{qaexchange}

\begin{standalonequote}{Thyroid Function}
    \metadata{topic={Serotonin Enlarging Thyroid}, source={Ray Peat Forum}}

    \begin{answer}
      Shock and stress can cause a surge of serotonin in the brain, which stimulates the enlargement of the thyroid gland, and can inhibit secretion of the hormone.
    \end{answer}
\end{standalonequote}

\begin{qaexchange}{Thyroid Function}
    \metadata{topic={Thyroid Effects on Cognition}, source={Ray Peat Forum}}

    \begin{question}
        How did your \enquote{processing} of information change after starting thyroid decades ago? Did you notice a change in memory, learning stuff, creativity and/or anything surprising?
    \end{question}

    \begin{answer}
      It made it easier to pace myself, not trying to do everything at once.
    \end{answer}
\end{qaexchange}

\begin{qaexchange}{Thyroid Function}
    \metadata{topic={Thyroid Normalizing Cholesterol}, source={Ray Peat Forum}}

    \begin{question}
        It's well know that thyroid can lower high cholesterol, but I've also heard of those with low cholesterol taking thyroid and it normalizing their cholesterol levels. Can you describe how this process works? Something along the lines of the cholesterol enzyme being dependent on thyroid?
    \end{question}

    \begin{answer}
      Glucose metabolism provides the (acetate) substance for making cholesterol, and thyroid hormone is a basic regulator of glucose metabolism---in an extremely stressed system, cholesterol synthesis will be limited.
    \end{answer}
\end{qaexchange}

\begin{standalonequote}{Thyroid Function}
    \metadata{topic={Diuresis, Nocturnal Alarm Reaction}, source={Ray Peat Forum}}

    \begin{answer}
      Diuresis is an effect of thyroid, pregnenolone and progesterone, as antagonists to estrogenic edema. The alarm reaction that happens at night has its effects in every organ; for the brain, it's similar to epilepsy; for the intestine and bladder, the effect can be either adrenergic or cholinergic, ie, too little or too much muscle tone; the blood becomes more concentrated. The bowel reaction causes absorption of endotoxin, which is a broad-spectrum poison; carrot salad, laxatives, etc., are especially important during the adjustment time. Cholesterol has a long history as a protectant against many toxins; I think this relates to the fact that people with very low cholesterol have such a high incidence of endotoxin-related symptoms. It might also relate to the therapeutic effects of eggnogs, though it takes a lot of eggs to raise the cholesterol a little.
    \end{answer}
\end{standalonequote}

\begin{standalonequote}{Thyroid Function}
    \metadata{topic={Iodine Sensitivity, Thyroid}, source={Ray Peat Forum}}

    \begin{answer}
      Too much iodine interferes with thyroid function, and that slows down the liver, letting T\textsubscript{4} accumulate. The high TSH shows that your T\textsubscript{4} isn't high now, and selenium and progesterone will help to restore functions. Cynomel is the product with the least iodine, and it would be good to periodically test doses of about one microgram (starting by dissolving a tablet in an ounce or two of water, and testing the water starting with just a drop or two per day).
    \end{answer}
\end{standalonequote}

\subsection{Symptoms}

\begin{qaexchange}{Thyroid}
    \metadata{topic={Moles and Hypothyroidism}, source={Ray Peat Forum}}

    \begin{question}
        I don't understand why I have sudden bursts of black moles on my face. What is the cause of this?
    \end{question}

    \begin{answer}
      Usually, a sudden multiplication of moles is a sign of hypothyroidism.
    \end{answer}
\end{qaexchange}

\begin{qaexchange}{Thyroid}
    \metadata{topic={T\textsubscript{4} Accumulation}, source={Ray Peat Forum}}

    \begin{question}
        How would I tell if T\textsubscript{4} is accumulating?
    \end{question}

    \begin{answer}
      At an extreme, it has an antithyroid effect, slowing metabolism, dulling consciousness.
    \end{answer}
\end{qaexchange}

\begin{qaexchange}{Thyroid}
    \metadata{topic={Adjustment Effects}, source={Ray Peat Forum}}

    \begin{question}
        Do you think people can feel spacey and sleepy when increasing a thyroid supplement dose, if the thyroid is lowering stress hormones and the pulse is decreasing?
    \end{question}

    \begin{answer}
      Yes, it sometimes does that during the first week or two of using it.
    \end{answer}
\end{qaexchange}

\begin{qaexchange}{Thyroid}
    \metadata{topic={Cognitive Effects}, source={Ray Peat Forum}}

    \begin{question}
        Blake was describing some of the ways our culture and mentalities have gotten stuck into rigid dimensions, so that we make a lot of mistakes based on our faulty perception of the real world. By analyzing biological research, you have worked out the biological mechanisms for our mistaken outlook.

		I was thinking about my own developing views, and how they've changed dramatically in the last few years. I think that my changing diet and other things were mostly responsible, allowing my substance to change to accommodate new insights and see what I didn't see before. Do other people tell you similar things?
    \end{question}

    \begin{answer}
      I occasionally hear that when people start using thyroid or progesterone.
    \end{answer}
\end{qaexchange}

\subsection{Hypothyroidism}

\begin{standalonequote}{Hypothyroidism}
    \metadata{topic={Alcohol Craving And Hypothyroidism}, source={Email Wiki}}

    \begin{answer}
        Have you ever had a thyroid test or tried a thyroid supplement? High serotonin activity is often present in hypothyroidism, and alcohol can probably provide temporary compensation for that.
    \end{answer}
\end{standalonequote}

\begin{qaexchange}{Hypothyroidism}
    \metadata{topic={Hypothyroidism Variable Symptoms}, source={Email Wiki}}

    \begin{question}
        Why do people have such different problems due to hypothyroidism?
    \end{question}

    \begin{answer}
        I think early life imprinting and habitual diet can cause such very different reactions to a thyroid deficiency.
    \end{answer}
\end{qaexchange}

\begin{qaexchange}{Thyroid}
    \metadata{topic={Glandular Supplementation}, source={Ray Peat Forum}}

    \begin{question}
        I've been trying a thyroid glandular, and it seems to warm my nose, feet, and fingers up really well, but I have no energy, and I feel like laying down all day because I'm so tired. Do you think it would be better to just supplement with T\textsubscript{3} by itself?
    \end{question}

    \begin{answer}
        Since the whole body has to adjust, it can take several weeks of gradually increasing doses to get the best results. Thyroid improves the quality of sleep and ability to relax. Hypothyroid people often have extremely high adrenaline, and during the first couple of weeks the decreased stress hormones are noticeable.
    \end{answer}
\end{qaexchange}

\begin{qaexchange}{Thyroid}
    \metadata{topic={Taking with Food}, source={Ray Peat Forum}}

    \begin{question}
        Should you take thyroid glandular with food? If so is it ok to take it with sugared milk-coffee?
    \end{question}

    \begin{answer}
        It's important to record your temperature and pulse rate regularly, watching for effects of the thyroid. It doesn't matter what you eat with it.
    \end{answer}
\end{qaexchange}

\begin{emailexchange}{Thyroid}
    \metadata{topic={Black Thyroid from Tetracycline}, source={Ray Peat Forum}}

    \begin{question}
        What do you think of the so called \enquote{black thyroid} that can be caused by tetracycline use? They identified one of these pigments as neuromelanin. Will accumulation of neuromelanin in the thyroid gland impair its function? Some reports link Tetracycline use with persistent hypothyroidism but in these cases there was also treatment with anti-thyroid drugs like propranolol and methimazole.
    \end{question}

    \begin{answer}
        I don't know of any harmful effects of melanin, unlike other pigments, including lipofuscin. Medical use of antibiotics is usually irrational; maybe this will lead to a product to cure grey hair or vitiligo.
    \end{answer}
\end{emailexchange}

\begin{standalonequote}{Thyroid}
    \metadata{topic={High RHR with Cynomel}, source={Ray Peat Forum}}

    \begin{answer}
        Do you have seafood regularly? A deficiency of selenium can cause odd responses to thyroxin. Have you had your vitamin D and cholesterol measured? Low cholesterol can limit the production of progesterone and DHEA in response to thyroid. Have you tried a pregnenolone supplement? Those steroids tend to support a bigger, slower heartbeat.
    \end{answer}
\end{standalonequote}

\begin{standalonequote}{Thyroid}
    \metadata{topic={T\textsubscript{3} Dose for Dementia}, source={Ray Peat Forum}}

    \begin{answer}
        If someone is in a precarious condition, even smaller amounts at a time might be better. For example, a man in the hospital right after a heart attack started taking one mcg per hour; the doctors had said that at the rate his enzymes were rising they would be expected to keep rising for another day, but they started decreasing exactly when he started the small doses, and they had decreased the next day when he left the hospital, without symptoms. T\textsubscript{3}, sugar, and aspirin are the most heart-protective things.
    \end{answer}
\end{standalonequote}

\begin{qaexchange}{Thyroid}
    \metadata{topic={Antibodies}, source={Ray Peat Forum}}

    \begin{question}
        Do high level of Thyroglobulin Antibodies and Thyroid Peroxidase (TPO) Antibodies show autoimmune thyroditis?
    \end{question}

    \begin{answer}
        When TSH is too high for a long time, it causes inflammation in the gland, and the antibodies are in reaction to that.
    \end{answer}
\end{qaexchange}

\begin{standalonequote}{Hypothyroidism}
    \metadata{topic={Heavy Menstruation, Hypothyroidism}, source={Ray Peat Forum}}

    \begin{answer}
      Hypothyroidism is the basic cause of heavy menstruation. Things strong enough to kill an embryo can't be harmless to a person.
    \end{answer}
\end{standalonequote}

\begin{standalonequote}{Hypothyroidism}
    \metadata{topic={Hypothyroidism Slow Digestion}, source={Ray Peat Forum}}

    \begin{answer}
       Slow digestion is very characteristic of hypothyroidism, and it can result in either constipation or diarrhea. 
    \end{answer}
\end{standalonequote}

\begin{qaexchange}{Hypothyroidism}
    \metadata{topic={POTS Caused by Hypothyroidism}, source={Ray Peat Forum}}

    \begin{question}
        Do you know what causes POTS?
    \end{question}

    \begin{answer}
      Hypothyroidism is a common cause.
    \end{answer}
\end{qaexchange}

\begin{qaexchange}{Hypothyroidism}
    \metadata{topic={Goiter Shrinking Strategy}, source={Ray Peat Forum}}

    \begin{question}
        Do you know of a good strategy to shrink a goiter?
    \end{question}

    \begin{answer}
      Both T\textsubscript{4} and T\textsubscript{3} can lower TSH, helping to shrink a goiter, but doing it with just T\textsubscript{3} would tend to give you a fast pulse and a tendency to over-heat. Cynoplus has a generally good balance of T\textsubscript{4} and T\textsubscript{3}, a 4 to 1 ratio; some people use a little T\textsubscript{3} with it, according to how they feel and function. It's available from \extlink{farmaciadelnino.mx}{farmaciadelnino.mx}. One of the effect of T\textsubscript{3} is to lower your estrogen (aspirin helps too) while increasing your progesterone. Estrogen tends to enlarge a goiter, by blocking the enzyme that converts thyroglobulin (\enquote{colloid}) to the active secreted hormones, and progesterone activates those enzymes. If the goiter is visibly enlarged, a certain amount of progesterone can activate the conversion of thyroglobulin to the extent of producing hyperthyroidism during the process (e.g., with a resting pule rate of about 125 beats per minute). It's better to do it slowly with thyroid supplements suppressing TSH. Pregnenolone, being converted to progesterone, might have similar accelerating effects.
    \end{answer}
\end{qaexchange}

\begin{emailexchange}{Hypothyroidism}
    \metadata{topic={TSH, Thyroglobulin Antibodies}, source={Ray Peat Forum}}

    \begin{question}
        TSH is higher than I'd like, vit D is lower than I'd like, and I'm not sure what to make of the high thyroglobulin antibodies?
    \end{question}

    \begin{answer}
      Prolonged stimulation by TSH promotes inflammation, and that leads to formation of antibodies. When TSH is normalized, the antibodies tend to subside over a period of several months. The low vitamin D contributes to an inflammatory state; keeping calcium intake high relative to phosphate supports vitamin D and thyroid functions.
    \end{answer}

    \begin{question}
        Do you think calcium intake relative to phosphate needs to be high on a meal-by-meal basis, or over the course of a day or week?
    \end{question}

    \begin{answer}
      The mineral absorption is fairly slow, so I think the day's average is what matters.
    \end{answer}
\end{emailexchange}

\begin{standalonequote}{Hypothyroidism}
    \metadata{topic={Hypothyroidism, Hunger, Blood Sugar}, source={Ray Peat Forum}}

    \begin{answer}
      Hypothyroid people waste glucose by failing to oxidize it completely, and that typically causes fluctuations of blood glucose and increased appetite. Keeping glucose stable with the right amount of thyroid prevents inappropriate hunger. A supplement of pantothenic acid sometimes helps, with oxidation and prevention of hyperinsulinemia.
    \end{answer}
\end{standalonequote}

\subsection{Hyperthyroidism}

\begin{standalonequote}{Hyperthyroidism}
    \metadata{topic={Hyperthyroidism Magnesium Treatment}, source={Email Wiki}}

    \begin{answer}
        I have known people with extremely high metabolic rates who benefited temporarily from magnesium, but when magnesium was combined with a thyroid supplement, returned stably to a normal (lower) metabolic rate.
    \end{answer}
\end{standalonequote}

\begin{standalonequote}{Hyperthyroidism}
    \metadata{topic={Thyroid Storm Management}, source={Ray Peat Forum}}

    \begin{answer}
      My impression is that some of the articles describing thyroid storm were written by hysterical people who didn't understand thyroid metabolism. After I had been exposed to a pesticide, I experienced a few weeks of hyperthyroidism, probably a normalizing process after the antithyroid toxin was gone. Besides washing my hair two or three times a day and eating a lot, I didn't do anything. A few people I've known wanted to stop the symptoms without a drug; some of them drank a glass or two of cabbage juice for a couple of days, others ate liver once or twice a day.
    \end{answer}
\end{standalonequote}

\subsection{Thyroid Supplementation}

\begin{standalonequote}{Thyroid Supplementation}
    \metadata{topic={T\textsubscript{4} Adaptation Period}, source={Email Wiki}}
    \begin{note}
        On T\textsubscript{4} for a week
    \end{note}

    \begin{answer}
        Since the half-life of T\textsubscript{4} is about two weeks, your adaptations to it have just begun, but your TSH would already be significantly lowered by it. After you have been on a certain dose for at least two weeks, the blood tests would be easier to interpret.
    \end{answer}
\end{standalonequote}

\begin{standalonequote}{Thyroid Supplementation}
    \metadata{topic={NDT T\textsubscript{4} To T\textsubscript{3} Ratio}, source={Email Wiki}}
    \begin{note}
        Ratio of T\textsubscript{4} to T\textsubscript{3}, Which is Sometimes Suggested to be Wrong in NDT
    \end{note}

    \begin{answer}
        Pigs' and cows' thyroids are very similar to people's, with a ratio usually between 3:1 and 4:1. The blood serum of hypothyroid people can have a ratio of 50:1 or 100:1, when the liver is failing to convert thyroxin. Maybe the authors of the book are physicians, educated by pharmaceutical advertisements.
    \end{answer}
\end{standalonequote}

\begin{qaexchange}{Thyroid Supplementation}
    \metadata{topic={Synthetic vs NDT Comparison}, source={Email Wiki}}

    \begin{question}
        Lack of hormones like calcitonin, T\textsubscript{2} etc., in synthetic T\textsubscript{4}/T\textsubscript{3} combo a problem?
    \end{question}

    \begin{answer}
        The old Armour thyroid, made from beef and pork glands before 1990, did contain other components that were probably valuable, but when T\textsubscript{3} is absorbed by mitochondria it's immediately changed into T\textsubscript{2}, so the synthetic T\textsubscript{3}'s effects can't be distinguished from those of a mixture of T\textsubscript{3} and T\textsubscript{2}. The company that now makes Armour thyroid started removing the calcitonin in the 1990s, to sell as a separate product.
    \end{answer}
\end{qaexchange}

\begin{standalonequote}{Thyroid Supplementation}
    \metadata{topic={T\textsubscript{3} Product Quality}, source={Email Wiki}}

    \begin{answer}
        There isn't any natural T\textsubscript{3} product, in the sense of biologically created, but the activity of T\textsubscript{3} is so great that the effective dose, of a few micrograms, couldn't introduce a significant amount of industrial junk; the excipients are the main concern, and whether the people making the tablets understand what they are doing. Cytomel and Cynomel, so far, have been very well made, and there isn't any other T\textsubscript{3} product that I trust.
    \end{answer}
\end{standalonequote}

\begin{standalonequote}{Thyroid Supplementation}
    \metadata{topic={Thyroid Supplementation Duration}, source={Email Wiki}}

    \begin{answer}
        Sometimes it takes many months to get the metabolic rate stable at a higher level, and it's often necessary to use a thyroid supplement.
    \end{answer}
\end{standalonequote}

\begin{standalonequote}{Thyroid Supplementation}
    \metadata{topic={Starting Thyroid With High Stress}, source={Email Wiki}}

    \begin{answer}
        Thyroid is the only thing that safely lowers cholesterol, but when your stress hormones are very high, you shouldn't take more than about one microgram of Cytomel at a time, and should accompany it with things like milk and orange juice.
    \end{answer}
\end{standalonequote}

\begin{standalonequote}{Thyroid Supplementation}
    \metadata{topic={Initial Thyroid Dosing}, source={Email Wiki}}

    \begin{answer}
        A starting dose of about 1 mcg can produce a noticeable effect, and can be repeated at intervals according to the effect. 5 mcg with a meal is another way to start it. Thyroid tends to lower cholesterol by converting it into pregnenolone and other steroids, and yours is high enough to easily improve your steroid hormone balance.
    \end{answer}
\end{standalonequote}

\begin{standalonequote}{Thyroid Supplementation}
    \metadata{topic={Cynoplus Starting Dose}, source={Email Wiki}}

    \begin{answer}
        Cynoplus (www.farmaciadelnino.com has a good price) is cheaper than Armour, and an eighth to a fourth of a tablet would be a reasonable amount to start with; thyroxine's half-life in the body is two weeks, so the effect is cumulative, and if you get the desired effects in less than two weeks the dose should probably be reduced.
    \end{answer}
\end{standalonequote}

\begin{standalonequote}{Thyroid Supplementation}
    \metadata{topic={Thyroid Dosing Frequency}, source={Email Wiki}}

    \begin{answer}
        At the beginning, once a day, but if your temperature and pulse and symptoms aren't just right after two weeks you could add another dose at a different time of day. Change of seasons affects the amount of thyroid you need, sometimes it isn't needed after using it for a while, but it's always good to watch for signs of change.
    \end{answer}
\end{standalonequote}

\begin{standalonequote}{Thyroid Supplementation}
    \metadata{topic={T\textsubscript{3} Revealing Low Metabolism}, source={Email Wiki}}

    \begin{answer}
        T\textsubscript{3}, by lowering stress, sometimes reveals a low basal metabolic rate, that was hidden by high stress hormones. The body produces about 4 mcg of T\textsubscript{3} per hour, so taking more than that can interfere with regulatory processes. It's helpful to use the resting pulse rate, and the 24 hour temperature curve, along with other signs, such as mood, appearance of veins on the hands, etc. The peak temperature should be in the afternoon.
    \end{answer}
\end{standalonequote}

\begin{standalonequote}{Thyroid Supplementation}
    \metadata{topic={T\textsubscript{3} Lowering Pulse Paradox}, source={Email Wiki}}

    \begin{answer}
        I occasionally see that happen [T\textsubscript{3} will cause low temp/pulse]; sometimes people have had their pulse rate decrease 40 or 50 beats per minute. The temperature of your fingers, toes, and nose helps to interpret the balance between stress and thyroid; your fingers should be less cold as your metabolic rate comes up. In extreme hypothyroidism, the hands and feet can be very cold while the oral temperature looks o.k.; then as the metabolic rate increases, the difference between fingers and mouth decreases.
    \end{answer}
\end{standalonequote}

\begin{standalonequote}{Thyroid Supplementation}
    \metadata{topic={Cytomel Heart Palpitations}, source={Email Wiki}}

    \begin{answer}
        When I used only Cytomel, any little stress would make me suddenly hypothyroid, and my heart would stop several times in a minute; when I started using some thyroid, USP, that contained both T\textsubscript{4} and T\textsubscript{3} it stopped happening.
    \end{answer}
\end{standalonequote}

\begin{standalonequote}{Thyroid Supplementation}
    \metadata{topic={Thyroid Brain Fog Causes}, source={Email Wiki}}

    \begin{answer}
        The body makes up to about 4 mcg of T\textsubscript{3} in an hour, so each dose should be small, with food to delay absorption. Are you having orange juice and milk in your diet? Sometimes a B vitamin deficiency, especially B\textsubscript{1}, can cause the fog. A supplement of 10 mg. is often enough to improve focus and prevent fatigue.
    \end{answer}
\end{standalonequote}

\begin{standalonequote}{Thyroid Supplementation}
    \metadata{topic={T\textsubscript{3} Absorption With Food}, source={Email Wiki}}

    \begin{answer}
        When you take T\textsubscript{3} without food, it enters the blood stream very suddenly, and the liver is likely to detect an excessive amount, causing it to produce enzymes to eliminate it. The result can be a decrease in T\textsubscript{3} for the rest of the day, especially at night if you took it in the morning. Have you tried rebreathing into a paper bag, to see if the increased \ce{CO2} affects the fog?
    \end{answer}
\end{standalonequote}

\begin{standalonequote}{Thyroid Supplementation}
    \metadata{topic={Managing T\textsubscript{3} Heart Rate Effects}, source={Email Wiki}}

    \begin{answer}
        I think regular use of the pregnenolone might help. Are you getting enough milk, and salting your food to taste? Do you have some sea food regularly? (For trace minerals.) Have you tried taking the small amounts of T\textsubscript{3} at different intervals, sooner until the symptoms are gone, then longer intervals until they return? TSH is likely to be high early in the morning, and as it subsides during the day the amount of T\textsubscript{3} needed might decrease.
    \end{answer}
\end{standalonequote}

\begin{standalonequote}{Thyroid Supplementation}
    \metadata{topic={Thyroid Adrenaline Surge}, source={Email Wiki}}
    \begin{note}
       Thyroid Acting Like Caffeine
    \end{note}

    \begin{answer}
        Not like caffeine, but if too much is taken suddenly, a person who has been deficient in thyroid is likely to experience an excess of adrenaline. Since the body normally produces about 4 mcg of T\textsubscript{3} in an hour, taking 10 or 20 mcg at once is unphysiological.
    \end{answer}
\end{standalonequote}

\begin{standalonequote}{Thyroid Supplementation}
    \metadata{topic={Thyroid Gland Recovery Time}, source={Email Wiki}}
    \begin{note}
        How to Stop Thyroid
    \end{note}

    \begin{answer}
        If a person's thyroid gland has been inhibited by very high doses of a supplement, it takes only 2 or 3 days for the gland to resume full activity, and because it takes time for the hormone to be excreted, suddenly stopping a supplement shouldn't be noticeable, when the gland isn't being inhibited or malfunctioning.
    \end{answer}
\end{standalonequote}

\begin{standalonequote}{Thyroid Supplementation}
    \metadata{topic={Thyroid Gland Adaptability}, source={Email Wiki}}
    \begin{note}
        Long-Term Effects of Taking Thyroid
    \end{note}

    \begin{answer}
        Experimenters using isotopes gave large doses of thyroid until the subjects' glands were completely shut off, and when they stopped giving the doses, everyone's gland returned to normal activity in just 2 or 3 days. The gland is extremely quick to adjust its activity, both up and down, except when it's inhibited by stress, or PUFA, or estrogen, etc.
    \end{answer}
\end{standalonequote}

\begin{standalonequote}{Thyroid Supplementation}
    \metadata{topic={Physiological Thyroid Dosing}, source={Email Wiki}}

    \begin{answer}
        The working thyroid gland produces about the equivalent of 4 grains of desiccated thyroid per day, and that is about 70\% thyroxine, T\textsubscript{4}, which allows the liver to make as much of the active T\textsubscript{3} hormone as needed (if it is well nourished, and not blocked by PUFA or estrogen or other inhibitor). So taking that amount makes up for what your gland would be producing; by suppressing TSH, which stimulates the growth and activity of the thyroid, it also protects against the recurrence of cancer if it wasn't all removed (some types of cancer were treated just by supplementing thyroid, without surgery). Since the desiccated thyroid is made available by being digested, it's best to divide the day's dose, with some at each meal and at bedtime, so that the amount of active hormone entering the blood isn't too high at any time.
    \end{answer}
\end{standalonequote}

\begin{standalonequote}{Thyroid Supplementation}
    \metadata{topic={Thyroid Cumulative Effects}, source={Email Wiki}}

    \begin{answer}
        It's important to remember that it's cumulative, and the effect of any daily dose increases with time, and is affected by many things, so it's important to keep a chart [of temperature and pulse], watching for changes during a period of about two weeks.
    \end{answer}
\end{standalonequote}

\begin{standalonequote}{Thyroid Supplementation}
    \metadata{topic={Cynoplus vs Cynomel Dosing}, source={Email Wiki}}

    \begin{answer}
        An eighth of a tablet of either [Cynoplus/Cynomel] is a good starting dose. The difference is that T\textsubscript{3} has a short half-life, and so can be repeated more often, while watching the pulse rate, so it's possible to get a quicker response.
    \end{answer}
\end{standalonequote}

\begin{standalonequote}{Thyroid Supplementation}
    \metadata{topic={Thyroid Individual Requirements}, source={Email Wiki}}

    \begin{answer}
        Sensitivities and requirements vary widely. I've known people who temporarily needed 500 mg of Armour even in the summer, but usually the summer requirement is a fourth of the winter requirement. For some people, 15 mg of Armour was enough, and for some 1 mcg of Cytomel was an effective dose.
    \end{answer}
\end{standalonequote}

\begin{standalonequote}{Thyroid Supplementation}
    \metadata{topic={Winter Thyroid Requirements}, source={Email Wiki}}

    \begin{answer}
        The end of winter is the worst time, because of the cumulative stress injury. Small amounts of T\textsubscript{3}, just 2 to 4 mcg at a time, along with good nutrition, including plenty of calcium (e.g., two quarts of milk), helps to recover from winter.
    \end{answer}
\end{standalonequote}

\begin{standalonequote}{Thyroid Supplementation}
    \metadata{topic={T\textsubscript{3} Immediate Feedback}, source={Email Wiki}}

    \begin{answer}
        T\textsubscript{3} has a short half-life in the body, and by adding small amounts of it you could feel quickly whether it was having the right effects. I don't know how reliable the Erfa is in composition. Mood is a good indicator, and the temperature of the toes and fingers usually changes quickly with thyroid changes.
    \end{answer}
\end{standalonequote}

\begin{standalonequote}{Thyroid Supplementation}
    \metadata{topic={Starting Thyroid In Spain}, source={Email Wiki}}

    \begin{answer}
        Are there any combination products, such as Thyrolar or Cynoplus, that you can get in Spain? It's good to start with a small amount, such as 5 mcg of T\textsubscript{3} twice a day, while watching for changes in your pulse rate, temperature, and ability to sleep. Half a grain of Armour, or about 30 mcg of T\textsubscript{4} and 7.5 mcg of T\textsubscript{3}, is traditionally a common starting dose; it should be taken with a meal, so that it absorbs slowly. Taking a very small amount at bedtime usually helps with insomnia.
    \end{answer}
\end{standalonequote}

\begin{standalonequote}{Thyroid Supplementation}
    \metadata{topic={Gradual Thyroid Increase Protocol}, source={Email Wiki}}

    \begin{answer}
        Try a sixth of a 25 mcg cynomel tablet at first, and watch for the effects in the first two hours. According to what you notice, you could continue that once a day, or twice a day, for about 10 days, then you could try some with each meal, for another week. \#2 and \#3: when you find out how the T\textsubscript{3} affects you, you could change to the combination (Armour or Thyrolar or Cynoplus); the amounts I mentioned would be similar to 12 mcg of T\textsubscript{3} per day.
    \end{answer}
\end{standalonequote}

\begin{standalonequote}{Thyroid Supplementation}
    \metadata{topic={Bedtime Thyroid Dosing}, source={Email Wiki}}

    \begin{answer}
        It depends on what you notice from taking a small amount with meals. If it makes you feel pleasant, calm, confident, then trying it at bedtime would be right.
    \end{answer}
\end{standalonequote}

\begin{standalonequote}{Thyroid Supplementation}
    \metadata{topic={T\textsubscript{3} To Thyroid Grain Equivalence}, source={Email Wiki}}

    \begin{answer}
        25 mcg of T\textsubscript{3} has approximately the activity of a grain (65 mg) of thyroid gland; is ERFA the only one available? A synthetic thyroxine could be combined with the Cynomel. Since the European products aren't necessarily the same as those made elsewhere, and a person's requirements are variable, it's essential to start with small amounts, watching for the effects, including pulse rate and temperature. T\textsubscript{4} builds up slowly in the tissues, over about 14 days, but the T\textsubscript{3} acts immediately. With any product, a single dose of T\textsubscript{3} of about 4 mcg is close to the physiological range; sometimes a smaller amount is enough.
    \end{answer}
\end{standalonequote}

\begin{standalonequote}{Thyroid Supplementation}
    \metadata{topic={Total Daily Thyroid Dosing}, source={Email Wiki}}

    \begin{answer}
        As long as it's divided so that you don't get a big dose of T\textsubscript{3} all at once it should be o.k. to take a total of 25 mcg T\textsubscript{3} and 100 of T\textsubscript{4}.That would be similar to the traditional 2 grain dose of Armour thyroid. A healthy person should produce the equivalent of about four grains per day, so with 2 grains of supplement, or the equivalent, there isn't a risk of over-dosing.
    \end{answer}
\end{standalonequote}

\begin{standalonequote}{Thyroid Supplementation}
    \metadata{topic={Cutting Thyroid Tablets}, source={Email Wiki}}

    \begin{answer}
        I use Cynomel and Cynoplus mostly, but they come in only one size, so I cut the tablets into about ten parts.
    \end{answer}
\end{standalonequote}

\begin{standalonequote}{Thyroid Supplementation}
    \metadata{topic={Cytomel Monitoring Protocol}, source={Email Wiki}}

    \begin{answer}
        Twice a day should be o.k., [cytomel] but every day you should make a note of your pulse rate and temperature, and in a week or ten days you should be able to see a progression.
    \end{answer}
\end{standalonequote}

\begin{standalonequote}{Thyroid Supplementation}
    \metadata{topic={Cytomel Sedating Effects}, source={Email Wiki}}

    \begin{answer}
        It [cytomel] improves the retention of magnesium, and cellular relaxation, and some people want to have a nap in the afternoon when their thyroid is good.
    \end{answer}
\end{standalonequote}

\begin{standalonequote}{Thyroid Supplementation}
    \metadata{topic={NDT vs T\textsubscript{3} Product Quality}, source={Email Wiki}}

    \begin{answer}
        If you use some T\textsubscript{3} (such as Cytomel or Cynomel) it's important to keep each dose small, while watching for changes in your pulse and temperature. Usually 4 or 5 mcg at a time is o.k. (the body makes about 4 mcg per hour). I don't think there's likely to be any problem using desiccated thyroid if the product is good, but because of changing manufacturing methods, that's largely a matter of trial and error. Low ferritin is often a result of hypothyroidism. The need for thyroid increases greatly during the winter in high latitudes, for example when I needed half a grain in the summer, I had to increase it to two grains during the winter. When cholesterol is high, that can make it easier to adapt to a thyroid supplement, since the thyroid will stimulate the conversion of cholesterol into progesterone and the adrenal hormones.
    \end{answer}
\end{standalonequote}

\begin{standalonequote}{Thyroid Supplementation}
    \metadata{topic={Cynoplus Source}, source={Email Wiki}}

    \begin{answer}
        I use Cynoplus (contains T\textsubscript{4} and T\textsubscript{3}) and Cynomel (T\textsubscript{3} only) that I usually get from www.mymexicandrugstore.mx. There is only one size tablet, and a fourth of a tablet is a typical starting dose.
    \end{answer}
\end{standalonequote}

\begin{standalonequote}{Thyroid Supplementation}
    \metadata{topic={T\textsubscript{3} Product Formulation Issues}, source={Email Wiki}}

    \begin{answer}
        Several of the commercially available products aren't well formulated, some are completely inactive. Cytomel's formulation has changed recently, so I'm not sure of its present potency. In areas with fluoridated water, taking a tablet with water can inactivate it. With good Cytomel, once a person has taken a very big dose, the liver produces enzymes to inactivate it quickly, so after 12 hours the blood level will become too low, and another big dose will be needed. Stress hormones are responsible for raising reverse T\textsubscript{3}, and just supplementing T\textsubscript{3} is seldom enough to normalize the stress hormones, so continued use of large doses can maintain improved functioning, but at the risk of developing problems from the continued excess of those hormones.
    \end{answer}
\end{standalonequote}

\begin{standalonequote}{Thyroid Supplementation}
    \metadata{topic={Triyotex T\textsubscript{3} Product}, source={Email Wiki}}

    \begin{answer}
        I think Triyotex is T\textsubscript{3}, 75 mcg, which would be three times as much as Cynomel, but I haven't tried it, and don't know how effective it is. Usually, a 5 mcg dose of T\textsubscript{3} with each meal is effective. Anti-aging Systems in England has a large variety of thyroid products.
    \end{answer}
\end{standalonequote}

\begin{standalonequote}{Thyroid Supplementation}
    \metadata{topic={NDT Product Recommendation}, source={Email Wiki}}

    \begin{answer}
        I haven't seen anything that compares well with the original Armour.
    \end{answer}
\end{standalonequote}

\begin{standalonequote}{Thyroid Supplementation}
    \metadata{topic={Armour Thyroid History}, source={Email Wiki}}

    \begin{answer}
        Armour thyroid, USP, was the standard thyroid used widely for about 80 years. Since ownership of the product name was bought by Revlon and then a series of other companies, I'm not sure anything of the simple original formula remains; maybe magnesium stearate, I haven't looked lately.
    \end{answer}
\end{standalonequote}

\begin{standalonequote}{Thyroid Supplementation}
    \metadata{topic={Armour Calcitonin Extraction}, source={Email Wiki}}

    \begin{answer}
        A few years ago I had some communication from a pharmacist at Forest Pharmaceutical, and he said that over ten years ago they began having thyrocalcitonin extracted from the pig thyroid powder to sell separately as a new drug. I think that left stearic acid as the only ingredient the current product might have in common with traditional Armour thyroid, USP. I don't use any product containing fumed or colloidal silica, or titanium, or various novel polymers, or coloring agents
    \end{answer}
\end{standalonequote}

\begin{standalonequote}{Thyroid Supplementation}
    \metadata{topic={Topical T\textsubscript{3}}, source={Email Wiki}}

    \begin{answer}
        Using it topically doesn't do anything for systemic metabolism, just the skin, at least at the concentrations I'm familiar with.
    \end{answer}
\end{standalonequote}

\begin{qaexchange}{Thyroid Supplementation}
    \metadata{topic={Liquid T\textsubscript{3} Stability}, source={Email Wiki}}

    \begin{question}
        Liquid T\textsubscript{3}, concerns?
    \end{question}

    \begin{answer}
        Trace impurities can inactivate it, but some liquid forms have worked.
    \end{answer}
\end{qaexchange}

\begin{standalonequote}{Thyroid Supplementation}
    \metadata{topic={Thyroid, Adrenaline, Relaxation}, source={Ray Peat Forum}}

    \begin{answer}
      During the first week or two of supplementing thyroid, there is usually an intensification of the effect of adrenaline. It's necessary to watch a variety of signs, especially the temperature of hands and feet and the amount of water evaporated, to judge the actual effect of thyroid. The effect of thyroid after the level of adrenaline has normalized is to increase the depth of relaxation.
    \end{answer}
\end{standalonequote}

\begin{standalonequote}{Thyroid Supplementation}
    \metadata{topic={Thyroid Supplement Brands}, source={Ray Peat Forum}}

    \begin{answer}
      Brands that I have used are Armour, Cynoplus, Novotiral, and Proloid-S. People have told me they have good results from WP-thyroid and Thyroid-S.
    \end{answer}
\end{standalonequote}

\begin{standalonequote}{Thyroid Supplementation}
    \metadata{topic={T\textsubscript{3} Dosing, Adrenalin Control}, source={Ray Peat Forum}}

    \begin{answer}
      When T\textsubscript{3} is used in small doses, such as 3 or 4 mcg at a time, it can be very effective for lowering adrenalin by letting glucose be more fully oxidized. It's helpful to keep a chart of your waking and midday temperature and pulse rate to watch the cumulative effects of the T\textsubscript{3}, so you can adjust the dose. A dose at bedtime typically makes it possible to go to sleep quickly; it should be supported by things like orange juice, cheese, and milk. A natural desiccated thyroid product, in the long run, is a convenient way to keep your metabolic rate where it should be. Cytomel is the only T\textsubscript{3} product that I'm confident of, at present.
    \end{answer}
\end{standalonequote}

\begin{qaexchange}{Thyroid Supplementation}
    \metadata{topic={Monitoring Thyroid Supplementation}, source={Ray Peat Forum}}

    \begin{question}
         I'm considering experimenting with thyroid (mix of T\textsubscript{4} and T\textsubscript{3}), are you aware of any way to make sure endogenous production of thyroid is not shut down when supplementing ? 
    \end{question}

    \begin{answer}
       Recording pulse and temperature at a certain time of day is helpful. When the thyroid gland is suppressed by taking too much of a supplement it recovers as soon as the dose is reduced. 
    \end{answer}
\end{qaexchange}

\begin{standalonequote}{Thyroid Supplementation}
    \metadata{topic={Natural Thyroid Supplement Preferences}, source={Ray Peat Forum}}

    \begin{answer}
      I think a combination or natural thyroid such as Thyrolar, Cynoplus, or Armour is usually best. A basic antistress action of thyroid is to convert cholesterol into pregnenolone, progesterone, and DHEA, but that requires adequate cholesterol, and a good mixed diet helps to maintain that.
    \end{answer}
\end{standalonequote}

\begin{standalonequote}{Thyroid Supplementation}
    \metadata{topic={Taking Thyroid at Night}, source={Ray Peat Forum}}

    \begin{answer}
      I always take my cynoplus at night, to go to sleep faster; it has an antiinflammatory effect.
    \end{answer}
\end{standalonequote}

\begin{qaexchange}{Thyroid Supplementation}
    \metadata{topic={Thyroid Adjustment Period}, source={Ray Peat Forum}}

    \begin{question}
        You mention a period of 2-3 weeks for adrenaline to come down from thyroid. During this time, should the person suspend the thyroid dose or continue taking it?
    \end{question}

    \begin{answer}
      If you start with a small dose, you can usually increase it slightly at intervals of about two weeks while keeping the pulse fairly steady. Other things that improve glucose metabolism help to lower the stress hormones.
    \end{answer}
\end{qaexchange}

\begin{standalonequote}{Thyroid Supplementation}
    \metadata{topic={Topical T\textsubscript{3} Solution}, source={Ray Peat Forum}}

    \begin{answer}
      I know people who crushed a cytomel tablet in about 3 ounces of water, and used it to treat styes or rashes, with good results.

      I haven't tried the liquid thyroid products, but it's easy to test the effectiveness of a T\textsubscript{3} solution, since it's fast acting. 5 or 10 mcg will speed and strengthen the pulse, starting in a few minutes, and persisting for a few hours. I think propolis is safe.
    \end{answer}
\end{standalonequote}

\begin{qaexchange}{Thyroid Supplementation}
    \metadata{topic={Coffee vs Thyroid Supplement}, source={Ray Peat Forum}}

    \begin{question}
        I cannot find a good available thyroid supplement and was wondering if consuming coffee (caffeine) would equally replace a thyroid supplement?
    \end{question}

    \begin{answer}
      Coffee contains some nutrients, including magnesium and niacin, that can help with some of the symptoms resulting from hypothyroidism, but it won't replace the thyroid hormone.
    \end{answer}
\end{qaexchange}

\begin{standalonequote}{Thyroid Supplementation}
    \metadata{topic={T\textsubscript{3} Inactivation in Alcohol}, source={Ray Peat Forum}}

    \begin{answer}
       I suspect that the thyroid activity had mostly disappeared in the alcohol solution, T\textsubscript{3} is extremely easy to inactivate. MCT can produce intestinal inflammation; I have had better results with the cellulose excipient thyroid.
    \end{answer}
\end{standalonequote}

\begin{standalonequote}{Thyroid Supplementation}
    \metadata{topic={Sugar and Coffee Improving T\textsubscript{3}}, source={Ray Peat Forum}}

    \begin{answer}
      Sugar and coffee can sometimes reduce stress enough to improve T\textsubscript{3} level, but it's hard to predict.
    \end{answer}
\end{standalonequote}

\begin{standalonequote}{Thyroid Supplementation}
    \metadata{topic={T\textsubscript{3} Effects, Free Fatty Acids}, source={Ray Peat Forum}}

    \begin{answer}
      The effects of a certain amount of T\textsubscript{3} in the blood vary according to variations in free fatty acids, so neither of them alone is enough to determine the dose needed.
    \end{answer}
\end{standalonequote}

\begin{standalonequote}{Thyroid Supplementation}
    \metadata{topic={Thyroid Increasing Gradually, Adrenaline}, source={Ray Peat Forum}}

    \begin{answer}
      The body tends to compensate for low thyroid by increasing adrenaline; increasing the thyroid supplement gradually over a period of weeks, it's possible to lower the adrenaline. It's necessary to use extra sugar and salt, and foods with calcium and magnesium, during that time.
    \end{answer}
\end{standalonequote}

\begin{standalonequote}{Thyroid Supplementation}
    \metadata{topic={Thyroid Supplement Sources}, source={Ray Peat Forum}}

    \begin{answer}
      I've been getting it from Farmacia del Nino, but Cynoplus is good too. Many people don't need a supplement, especially during the summer, so it would just be a wasteful nuisance to use it when it's not needed.
    \end{answer}
\end{standalonequote}

\begin{qaexchange}{Thyroid Supplementation}
    \metadata{topic={Making Homemade Thyroid Gland Supplement}, source={Ray Peat Forum}}

    \begin{question}
        I've recently moved to a rural area, with lots of sheep and cattle. I'm sure I'd be able to access fresh thyroid glands here. This morning I've been thinking about the idea of making my own NDT as a fun experiment. Do you know much about how it could be done at home?
    \end{question}

    \begin{answer}
      Medical grade acetone is good for the defatting. They stay greasy if the fat isn't all removed, but that's o.k. if they are frozen. Much of the fat can be removed by cooking them, then they can be powdered.
    \end{answer}
\end{qaexchange}

\begin{standalonequote}{Thyroid Supplementation}
    \metadata{topic={T\textsubscript{4} Dosing, T\textsubscript{3} Use}, source={Ray Peat Forum}}

    \begin{answer}   
It takes at least two weeks for a given dose of T\textsubscript{4} to reach a steady concentration in the body, apart from the hormonal-metabolic changes that it will cause, so I think you probably increased the dose too quickly. The half-life of T\textsubscript{3} is much shorter, so by using doses of less than 5 mcg at a time, it's possible to increase your liver's glycogen storage and your other tissues' sensitivity to adrenergic stimulation, without accumulating any excess T\textsubscript{4}. During that time, having orange juice or other sweet fruit frequently, avoiding eating protein by itself, will help to lower the stress hormones.
    \end{answer}
\end{standalonequote}

\begin{standalonequote}{Thyroid Supplementation}
    \metadata{topic={T\textsubscript{4} as Antithyroid}, source={Ray Peat Forum}}

    \begin{answer}
      There's almost no context in which I would speak of \enquote{an appropriate dose of T\textsubscript{4},} since thyroxin is so effective as an antithyroid substance. It's appropriate if you are also taking T\textsubscript{3}, or if you want to shrink your thyroid.
    \end{answer}
\end{standalonequote}

\begin{standalonequote}{Thyroid Supplementation}
    \metadata{topic={Thyroid, Nutrient Deficiency}, source={Ray Peat Forum}}

    \begin{answer}
      Increasing the metabolic rate, thyroid can increase the need for nutrients, so if your diet is already deficient in something, it can get worse. Thyroid improves the assimilation of some nutrients, so it can help to correct deficiencies, but only if you start very slowly.
    \end{answer}
\end{standalonequote}

\begin{standalonequote}{Thyroid Supplementation}
    \metadata{topic={T\textsubscript{3} Intolerance, Thyroid Supplementation}, source={Ray Peat Forum}}

    \begin{answer}
      I think it would help with your symptoms and hormone regulation if you raise the vitamin D to 50 or 60 ng/ml. If you take the glandular thyroid in one dose, it's best around bedtime. Have you kept a record of its effects on your pulse rate and temperature, especially before getting up in the morning? By powdering the Cynomel, and dividing each tablet into about 20 pieces, each dose will be just enough to be able to sense an effect, starting within seconds, and then fading away in the next few hours. By checking your pulse rate, you'll be able to correlate the dose with changes in the fatigue and waakness. If you space the doses by 3 or more hours, after several days you should be able to see some slight but consistent changes in your pulse rate, and when you do, whether it's slower or faster, wait for it to be steady at that rate for a few days before deciding to increase the dose. T\textsubscript{3} and vitamin D work closely together, and help to keep calcium out of soft tissues; that lets them store up more energy, while reducing inflammation and anxiety.
    \end{answer}
\end{standalonequote}

\begin{standalonequote}{Thyroid Supplementation}
    \metadata{topic={Thyroid Supplementation Timing}, source={Ray Peat Forum}}

    \begin{answer}
      Yes, I think it's better to become hypothyroid during the high temperature.
    \end{answer}
\end{standalonequote}

\begin{qaexchange}{Thyroid Supplementation}
    \metadata{topic={Beta Blockers with Thyroid}, source={Ray Peat Forum}}

    \begin{question}
        Using beta blockers to tolerate small amounts of thyroid?
    \end{question}

    \begin{answer}
      I think it's best for anxious people to use the anxiety to motivate a thorough study of the issues. Hypoglycemia can be a factor in anxiety, and when low thyroid and high estrogen are involved, a beta blocker has the possibility of lowering blood sugar.
    \end{answer}
\end{qaexchange}

\section{Sex Hormones}

\begin{standalonequote}{Sex Hormones}
    \metadata{topic={Optimal Hormone Ranges For Men}, source={Email Wiki}}
    \begin{note}
        For 35-Year-Old Male
    \end{note}

    \begin{answer}
        It's best to have cortisol no higher than the middle of the range, estrogen below the middle, testosterone above the middle. A vitamin D supplement often helps to improve the balance, good thyroid function is essential, adequate protein, good digestion.
    \end{answer}
\end{standalonequote}

\begin{standalonequote}{Sex Hormones}
    \metadata{topic={SHBG Importance In Men}, source={Email Wiki}}

    \begin{answer}
        I think the SHBG might be less important for men than for women.
    \end{answer}
\end{standalonequote}

\begin{standalonequote}{Sex Hormones}
    \metadata{topic={Hormone Testing Timing}, source={Email Wiki}}

    \begin{answer}
        It's best to check around the middle of the luteal phase.
    \end{answer}
\end{standalonequote}

\begin{qaexchange}{Sex Hormones}
    \metadata{topic={Excessive Androgens In Women}, source={Email Wiki}}

    \begin{question}
        15mg DHEA and 5mg testosterone every day for a woman?
    \end{question}

    \begin{answer}
        DHEA and testosterone at those doses are likely to grow whiskers. 5 mg. of testosterone is about ten times what a woman produces in a day, and is about what a muscular young man produces.
    \end{answer}
\end{qaexchange}

\begin{qaexchange}{Sex Hormones}
    \metadata{topic={Pregnenolone Protecting DHEA}, source={Email Wiki}}

    \begin{question}
        Would a ratio of 2:1 (pregnenolone  : DHEA) help mitigate DHEA conversion to estrogen, given no more than 15 mg of DHEA is ingested in a day?
    \end{question}

    \begin{answer}
        Yes, I think pregnenolone can protect against stress and conversion of DHEA to estrogen; larger quantities would be o.k.
    \end{answer}
\end{qaexchange}

\begin{standalonequote}{Sex Hormones}
    \metadata{topic={Virilization Causes}, source={Email Wiki}}

    \begin{answer}
        Low thyroid and high estrogen, resulting from various things such as high PUFA, low nutrient diet, interfere with progesterone synthesis, and the adrenals compensate, producing androgens instead. Pregnenolone helps to lower adrenal androgens, progesterone can be used topically on some hairy areas.
    \end{answer}
\end{standalonequote}

\begin{standalonequote}{Sex Hormones}
    \metadata{topic={Masturbation, Hormone Effects}, source={Ray Peat Forum}}

    \begin{answer}
      I don't think masturbation affects the hormones more than regular sex does.
    \end{answer}
\end{standalonequote}

\begin{qaexchange}{Sex Hormones}
    \metadata{topic={Steroid Storage}, source={Ray Peat Forum}}

    \begin{question}
        In the freezer, how long would steroids keep (progesterone, DHT, \&c.)?
    \end{question}

    \begin{answer}
      Hundreds of years.
    \end{answer}
\end{qaexchange}

\begin{qaexchange}{Sex Hormones}
    \metadata{topic={DHEA vs Pregnenolone}, source={Ray Peat Forum}}

    \begin{question}
        If you have good thyroid function, do you think DHEA is better than pregnenolone because it's far fewer milligrams per dose (less allergenic), like testosterone?
    \end{question}

    \begin{answer}
      Pregnenolone has its own effects distinct from DHEA's; I think it's important to find a well refined pregnenolone.
    \end{answer}
\end{qaexchange}

\subsection{Progesterone}

\begin{standalonequote}{Progesterone}
    \metadata{topic={5-Alpha Dihydroprogesterone (5a-DHP)}, source={Email Wiki}}

    \begin{answer}
        Pregnenolone, DHEA, and progesterone are basic stabilizing molecules, and I think it's dangerous to use novel substances that are likely to interact with those, in the absence of large amounts of preliminary experimentation with animals and in vitro. 
    \end{answer}
\end{standalonequote}

\begin{standalonequote}{Progesterone}
    \metadata{topic={5-Alpha Dihydroprogesterone (5a-DHP)}, source={Email Wiki}}

    \begin{answer}
        The farther a substance is from its precursor material, the easier it is to cause unwanted effects when supplementing it. 
    \end{answer}
\end{standalonequote}

\begin{qaexchange}{Progesterone}
    \metadata{topic={Progesterone Timing Irregular Cycles}, source={Email Wiki}}

    \begin{question}
        Irregular cycle - When to take progesterone?
    \end{question}

    \begin{answer}
        If you know the date of the last menstruation, you could go by the calendar, so 6 or 10 or 14 days later might coincide with the ovulation cycle; if you have a sensation of ovulation, that would be a signal to start it, or if you see a sudden rise in morning temperature that could indicate ovulation. But if there's no cycle you can detect, just starting the progesterone could renew the rhythm.
    \end{answer}
\end{qaexchange}

\begin{standalonequote}{Progesterone}
    \metadata{topic={Progesterone Thyroid Interaction}, source={Email Wiki}}

    \begin{answer}
        Sometimes progesterone can cause an underactive enlarged thyroid gland to begin secreting, temporarily producing mild hypothyroidism while the gland returns to a normal size. Supplemental progesterone can reduce excessive cortisol production.
    \end{answer}
\end{standalonequote}

\begin{standalonequote}{Progesterone}
    \metadata{topic={Progesterone Lowering Cortisol}, source={Email Wiki}}

    \begin{answer}
        Since progesterone helps the thyroid to secrete, and helps the liver to regulate glucose and convert T\textsubscript{4} to T\textsubscript{3}, women who are low in progesterone usually have hypothyroid symptoms (because of insufficient T\textsubscript{3}), including high cortisol, which promotes the synthesis of estrogen (in several ways, but never from progesterone). Cortisol is made from progesterone, but increasing the supply of progesterone reliably lowers cortisol synthesis, acting on the brain, pituitary, and adrenal glands. Progesterone, by many mechanisms, including its antagonism to cortisol, lowers the amount of estrogen in cells (causing the estrogen-binding proteins to be degraded, inhibiting the enzymes that release estrogen from the sulfates and glucuronides, and activating the enzymes that detoxify estrogen). So I think the symptoms of increased estrogen and cortisol are the result of either extraneous ingredients in the creams, or from using it at the wrong time, for example, too early, triggering premature ovulation. Supplementing a small amount of T\textsubscript{3}, Cytomel or Cynomel, usually stops symptoms such as breast pain, irritability, and restless energy, in less than an hour.
    \end{answer}
\end{standalonequote}

\begin{standalonequote}{Progesterone}
    \metadata{topic={Progesterone Seasonal Dosing}, source={Email Wiki}}

    \begin{answer}
        I think continuing the progesterone would help to normalize thyroid responses. If you adjust the thyroid dose every two weeks according to how you feel, and according to your temperatures and pulse rate, there should be a point where your cycle is right, without needing progesterone. During the winter the need for thyroid is higher, because of the short days, so it's important to watch for decreasing need when the days are longer in the spring.
    \end{answer}
\end{standalonequote}

\begin{qaexchange}{Progesterone}
    \metadata{topic={Progesterone Not Converting To Estrogen}, source={Email Wiki}}

    \begin{question}
        Could progesterone turn into estrogen?
    \end{question}

    \begin{answer}
        Progesterone won't turn into estrogen, but along with thyroid and aspirin it will tend to reduce the amount of estrogen in the body. If you have symptoms, you could adjust the dose according to the effect; I have seen some people start recovering immediately with just 10 mg of progesterone, but it depends on the balance of other hormones.
    \end{answer}
\end{qaexchange}

\begin{qaexchange}{Progesterone}
    \metadata{topic={Administration Method}, source={Ray Peat Forum}}

    \begin{question}
        Do you think Progest E is better taken via rubbing on my gums or swallowing?
    \end{question}

    \begin{answer}
        Some will enter your blood stream very quickly from the mouth membranes, but taking it with food the effect will be more gradual and prolonged.
    \end{answer}
\end{qaexchange}

\begin{qaexchange}{Progesterone}
    \metadata{topic={Cycling After Menopause}, source={Ray Peat Forum}}

    \begin{question}
        Since I went through surgical menopause 22 years ago, should I rotate off the Progest E every month for several days or stay on it continuously?
    \end{question}

    \begin{answer}
        I think it's most effective when you take it cyclically; imitating the menstrual cycle, with two week on and two off, would be good, unless you are using it to control some symptom.
    \end{answer}
\end{qaexchange}

\begin{qaexchange}{Progesterone}
    \metadata{topic={Continuous Use}, source={Ray Peat Forum}}

    \begin{question}
        I recently started taking progest-E and it's helped my cyclical moods. I have been taking it for days 14 and 21 of my cycle. I hate to stop taking it because I have PMS, moodiness, would there be harm in taking it for a while continuously even if I miss a period or two?
    \end{question}

    \begin{answer}
        I have known women who took it every day and kept cycling without any problems, but what they should be aware of is if you take a little extra just before the expected time of ovulation, it will trigger early ovulation, then if you stop taking it or take less, it will bring on an early menstruation. So if you take it every day then you need to take the same amount.
    \end{answer}
\end{qaexchange}

\begin{qaexchange}{Progesterone}
    \metadata{topic={As Contraceptive}, source={Ray Peat Forum}}

    \begin{question}
        How can progesterone (i.e: progest-e) be used as a contraceptive? Would a certain amount have to be applied inter-vaginally prior to intercourse?
    \end{question}

    \begin{answer}
        I know women who used it successfully, applying it to both sides of a plastic diaphragm, and leaving it in after intercourse.
    \end{answer}
\end{qaexchange}

\begin{qaexchange}{Progesterone}
    \metadata{topic={During Childbirth}, source={Ray Peat Forum}}

    \begin{question}
        I heard that using progesterone liberally during childbirth can be a natural pain relief- by rubbing it on the lower back and abdomen. Do you think this might be effective? Do you think there could be a negative consequence to using so much during labor?
    \end{question}

    \begin{answer}
        A friend, whose doctor gave her progesterone during her second pregnancy, said the baby \enquote{just popped out.} Katharina Dalton wrote about her experience using progesterone during pregnancy. When it's used along with good nutrition and good thyroid function, I think it prevents childbirth pain. I've heard it described as a \enquote{chemical midwife.} Labor consumes a lot of glucose, and is likely to deplete glycogen stores if it's prolonged, so I suspect that the main effect of dates is to prevent hypoglycemia. They contain some serotonin, which can increase uterine contraction. Hypoglycemia increases adrenaline, and adrenaline can cause uterine inertia, increasing the risk of hemorrhage, so I think any good source of sugar, such as fruit juice, is protective during labor.
    \end{answer}
\end{qaexchange}

\begin{qaexchange}{Progesterone}
    \metadata{topic={Acute Head Injury Dosing}, source={Ray Peat Forum}}

    \begin{question}
        I read about using progesterone after head injuries. If I carried a bottle of Progest-E with me, how much would I want to use if I injured my head? For instance, if I got into a car accident, would I want to immediately eat a whole bottle?
    \end{question}

    \begin{answer}
      No, a fourth to half a teaspoonful (100 to 200 mg) would be a moderately anesthetic and protective dose.
    \end{answer}
\end{qaexchange}

\begin{standalonequote}{Progesterone}
    \metadata{topic={Progesterone Dosing, Thyroid Interaction}, source={Ray Peat Forum}}

    \begin{answer}
      It's usual to start with just a couple of drops of progesterone; depending on your symptoms, that amount could be repeated, as needed. Thyroid tends to lower cholesterol, converting some of it to progesterone. Both cholesterol and progesterone are involved in the response to thyroid.
    \end{answer}
\end{standalonequote}

\begin{standalonequote}{Progesterone}
    \metadata{topic={Progesterone Contraceptive Protocol}, source={Ray Peat Forum}}

    \begin{answer}
      The people I knew used a 10\% solution of progesterone in vitamin E (Progest-E Complex) on a plastic diaphragm at the time of intercourse, leaving it in for a day or two. The diaphragm was coated on both sides, so there was probably about 400 or 500 mg of progesterone.
    \end{answer}
\end{standalonequote}

\begin{standalonequote}{Progesterone}
    \metadata{topic={Progesterone Thyroid Interaction}, source={Ray Peat Forum}}

    \begin{answer}
      If your thyroid gland is enlarged, progesterone's normalization of function can lead to a few weeks of increased thyroid activity while the gland unloads excess colloid. Some antiserotonin drugs can be hard on the heart.
    \end{answer}
\end{standalonequote}

\begin{standalonequote}{Progesterone}
    \metadata{topic={Progesterone Dosing Schedule}, source={Ray Peat Forum}}

    \begin{answer}
       Progesterone has a positive feedback effect on the ovaries, helping them to produce it, and it has a favorable (anti-stress, antiaging) effect on the thyroid, pituitary, adrenals, and pancreas. I've been using a little daily for several years. If a woman is using a large amount, I think it's best to imitate a natural menstrual cycle, because after several days, the liver begins to excrete it, and it takes a few days for the liver enzymes to return to the previous level. When it's taken every day, the effect of a dose doesn't last as long, making it less economical. The size of the dose that's effective depends on how much estrogen is present, and fat tissue is a major source of it after menopause. 
    \end{answer}
\end{standalonequote}

\subsection{Pregnenolone}

\begin{qaexchange}{Pregnenolone}
    \metadata{topic={Pregnenolone For Adrenal Insufficiency}, source={Email Wiki}}

    \begin{question}
        Would pregnenolone correct this?
    \end{question}

    \begin{answer}
        Pregnenolone should usually do it, but progesterone is more certain if the adrenals are really destroyed.
    \end{answer}
\end{qaexchange}

\begin{qaexchange}{Pregnenolone}
    \metadata{topic={Pregnenolone Dissolution}, source={Email Wiki}}

    \begin{question}
        Does pregnenolone have to be micronized for it to dissolve in vitamin e?
    \end{question}

    \begin{answer}
        It doesn't dissolve very well either way, it just takes some stirring and a little warmth. Vitamin E breaks down quickly when it's hot, so stirring at room temperature is best; not much dissolves. It's much more economical to use it orally, as powder.
    \end{answer}
\end{qaexchange}

\begin{standalonequote}{Pregnenolone}
    \metadata{topic={Pregnenolone Synthesis Support}, source={Email Wiki}}

    \begin{answer}
        Ordinarily, you can make enough from converting sugar to cholesterol, with thyroid and vitamin A converting cholesterol to the other hormones. But when you have been poisoned with not enough of the needed foods, or too much of the unsaturated oils, heavy metals, causing free radical reactions and so on, then it helps to use all of the supports possible, thyroid supplements, pregnenolone supplements, possibly dhea and progesterone, saturated fats, sugar, everything that works in the same direction.
    \end{answer}
\end{standalonequote}

\begin{standalonequote}{Pregnenolone}
    \metadata{topic={Pregnenolone Supplement Form}, source={Email Wiki}}

    \begin{answer}
        Pregnenolone is a lipid, only pharmaceutical salesmen talk about the need for a lipid matrix. Most people don't have allergic reactions to the rice and magnesium stearate.
    \end{answer}
\end{standalonequote}

\begin{qaexchange}{Pregnenolone}
    \metadata{topic={Pregnenolone For PMS}, source={Email Wiki}}

    \begin{question}
        Can women take pregnenolone oil just like Progest-E and get the same benefits in regards to PMS, cramps etc?
    \end{question}

    \begin{answer}
        Pregnenolone doesn't have the direct hormonal effects, but it's the precursor, and by stopping exaggerated stress reactions it is likely to help.
    \end{answer}
\end{qaexchange}

\begin{standalonequote}{Pregnenolone}
    \metadata{topic={Pregnenolone Impurities}, source={Email Wiki}}
    \begin{note}
        Bad Response to Pregnenolone
    \end{note}

    \begin{answer}
        I think that would be from impurities in the pregnenolone. In animal studies, a dose equivalent to about a pound in a person, caused no change, unless the animal was stressed, and in that case it stopped the stress.
    \end{answer}
\end{standalonequote}

\begin{standalonequote}{Pregnenolone}
    \metadata{topic={Pregnenolone Steal Myth}, source={Email Wiki}}

    \begin{answer}
        Regarding the pregnenolone steal theory, It would be interesting to know who started that, it's a mechanical way of thinking about physiology that ignores the things that really matter. Thyroid hormone, vitamin A, and cholesterol support the formation of pregnenolone, and the well nourished body is able to make large adjustments in these, to minimize the need for cortisol. In health, enough pregnenolone and progesterone are produced to inhibit the stress systems, for example by inhibiting the release of ACTH. When something prevents the formation of pregnenolone and progesterone, rising ACTH will increase its production as conditions permit, but if something, such as thyroid hormone, is lacking, the ACTH will increase cortisol, often with DHEA and the androgens increasing too, if resources permit; sometimes the stressed system is able to sustain only cortisol and aldosterone production, and that leads to degenerative problems.
    \end{answer}
\end{standalonequote}

\begin{standalonequote}{Pregnenolone}
    \metadata{topic={Pregnenolone Purity}, source={Email Wiki}}

    \begin{answer}
        Someone recently tested pregnenolone for Beyond a Century, and said it looks pure. Sometimes at first a few hundred milligrams are needed to lower cortisol.
    \end{answer}
\end{standalonequote}

\begin{standalonequote}{Pregnenolone}
    \metadata{topic={Pregnenolone Side Effects}, source={Email Wiki}}

    \begin{answer}
        Excipients or impurities in capsules can cause symptoms, by irritating the intestine. In animal studies (and in myself), extremely large doses didn't have any more effects than minimal doses. It's possible to eliminate some of the impurities by mixing it with warm vitamin E, and after stirring it, allowing it to settle, and using only what dissolved in the vitamin E.
    \end{answer}
\end{standalonequote}

\begin{standalonequote}{Pregnenolone}
    \metadata{topic={High-Dose Pregnenolone Safety}, source={Email Wiki}}

    \begin{answer}
        When I was buying pregnenolone from the Syntex factory in Mexico, 1984--5, to test its safety I ate a kilogram of it during a year, 3000 to 4000 milligrams per day. I didn't detect any side effects at all, except that my skin, that had been sagging over my eyes and on my neck, firmed up. I know a man in his sixties who is taking a teaspoonful every day, without any bad effects.
    \end{answer}
\end{standalonequote}

\begin{qaexchange}{Pregnenolone}
    \metadata{topic={Dosing}, source={Ray Peat Forum}}

    \begin{question}
        I have been taking three drops of StressNon topically, I am assuming you are familiar with it, and I don't take it everyday just when I feel I anticipate myself in a stressed state. Is that an appropriate way to use pregnenolone, or should it be cycled?
    \end{question}

    \begin{answer}
        I don't know what Stressnon contains. If pregnenolone is pure, it's effective in doses as small as 10 mg; I have taken thousands of milligrams daily, for a year, and found it to have no side effects.
    \end{answer}
\end{qaexchange}

\begin{qaexchange}{Pregnenolone}
    \metadata{topic={vs DHEA}, source={Ray Peat Forum}}

    \begin{question}
        If you have good thyroid function, do you think DHEA is better than pregnenolone because it's far fewer milligrams per dose (less allergenic), like testosterone?
    \end{question}

    \begin{answer}
        Pregnenolone has its own effects distinct from DHEA's; I think it's important to find a well refined pregnenolone.
    \end{answer}
\end{qaexchange}

\begin{standalonequote}{Pregnenolone}
    \metadata{topic={Product Quality}, source={Ray Peat Forum}}

    \begin{answer}
      Nasal congestion, headache, sore throat, and/or hemorrhoids are common reactions to impurities.
    \end{answer}
\end{standalonequote}

\begin{qaexchange}{Pregnenolone}
    \metadata{topic={Historical Effectiveness}, source={Ray Peat Forum}}

    \begin{question}
        Over the years, have you seen a change in the the types of health problems people have? I get the sense from your books and older articles that pregnenolone was able to cure a lot of people's problems. Do people today have very different metabolisms than people did just a few decades ago?
    \end{question}

    \begin{answer}
      Besides the average change in diet, people now have a more complete set of expectations about their health, and the substances used in supplements have changed. Many people expect pregnenolone to act like a hormone, which it doesn't; I've only seen a couple of people younger than their late 40s who felt anything at all from it. And a single dose is often all it takes, and continued, or bigger, doses don't do anything more.
    \end{answer}
\end{qaexchange}

\begin{qaexchange}{Pregnenolone}
    \metadata{topic={High-Dose Pregnenolone Effects}, source={Ray Peat Forum}}

    \begin{question}
        If a large amount of pregnenolone above one gram daily is needed to lower the pulse and make one feel good and relaxed, does this indicate a thyroid deficiency or should one continue to take the larger amounts of pregnenolone?
    \end{question}

    \begin{answer}
      I think large amounts like that (I averaged about 3 grams/day for a year) are acting to inhibit inflammation and shift water balances, as well as inhibiting the stress hormones via the GABA system. Reducing sources of inflammation (for me it was grains and starches mostly) as well as regulating thyroid can take the place of the pregnenolone.
    \end{answer}
\end{qaexchange}

\begin{qaexchange}{Pregnenolone}
    \metadata{topic={Pregnenolone in Oil}, source={Ray Peat Forum}}

    \begin{question}
        Would it be clever to dissolve pregnenolone in oil or vitamin E to increase its tissues availability by bypassing the liver?
    \end{question}

    \begin{answer}
      I haven't noticed much difference, I think because it dissolves quickly in the bile with fats in foods.
    \end{answer}
\end{qaexchange}

\begin{qaexchange}{Pregnenolone}
    \metadata{topic={Pregnenolone Quality, Manufacturing}, source={Ray Peat Forum}}

    \begin{question}
        Have you had any progress on finding a decent quality pregnenolone? Do you think if someone had a bunch of money, they could approach a manufacturer somewhere and they could figure out how to produce a very clean product?
    \end{question}

    \begin{answer}
      I think the surest thing would be to arrange contract production by one of the giant steroid producers—preferably the one that took over Syntec's product. It would probably be more expensive than getting it from the places that produce it as the end product. They might still be using the same technology from the crude saponin to pregnenolone. New, more efficient technologies that are more profitable sometimes produce radically different products, for example ascorbic acid after 1955 and vitamin E after the 1990s.
    \end{answer}
\end{qaexchange}

\begin{standalonequote}{Pregnenolone}
    \metadata{topic={Pregnenolone Dosage}, source={Ray Peat Forum}}

    \begin{answer}
      The first couple of years that I took pregnenolone, I suppose I ate more than a kilogram of it, but when I realized how much it was costing, I found the minimum that's effective, which is very similar on a weight basis for rats and humans; about 30 mg per day is the adequate normal maintenance dose, but it can generally be taken spaced as much as ten days apart, 200 mg per 7 days, 300 mg per ten days. A rat given thousands of milligrams in a single dose shows no side effects, except a loss of appetite while its stomach is full of the powder.
    \end{answer}
\end{standalonequote}

\begin{standalonequote}{Pregnenolone}
    \metadata{topic={Thyroid, Pregnenolone Production}, source={Ray Peat Forum}}

    \begin{answer}
      Thyroid will dependably correct your pregnenolone production, if you have enough cholesterol, vitamin A, and protein. The cholesterol will be consumed to make pregnenolone and progesterone and bile acids. If cholesterol is below 160, fruit sugar helps to raise it. The protein is needed to detoxify estrogen, unsaturated oils, etc, and to maintain the T\textsubscript{3}. Protein deficiency gives antithyroid signals, and T\textsubscript{4} will be used to make reverse T\textsubscript{3} to inhibit T\textsubscript{3}'s effects. About 3 mcg of T\textsubscript{3} especially if it's taken with milk or gelatine-rich salty soup is effective for stopping the nocturnal alarm reaction.
    \end{answer}
\end{standalonequote}

\begin{standalonequote}{Pregnenolone}
    \metadata{topic={Cholesterol, Thyroid, Steroid Synthesis}, source={Ray Peat Forum}}

    \begin{answer}
      If your cholesterol is above 200, and the thyroid supplements didn't warm you up, it's possible that something is interfering with your steroid synthesis, which might be a deficiency of something like vitamin A, or interference from something like iron or carotene. Have you tried a supplement of pregnenolone or DHEA? Were any other hormones, such as prolactin, measured? If you are taking the aspirin regularly, you should make sure to get vitamin K, from kale, liver, or a supplement. Anemia, like cold feet, is a common sign of low thyroid function.
    \end{answer}
\end{standalonequote}

\begin{standalonequote}{Pregnenolone}
    \metadata{topic={Protein Intake, Steroid Synthesis}, source={Ray Peat Forum}}

    \begin{answer}
      If you are eating enough protein, about 100 grams, and salt and thyroid, then I would consider the steroids--something might be interfering with your production of pregnenolone and DHEA. Things that could do that would be very low cholesterol, or a deficiency of vitamin A (retinol), or possibly other deficiencies.
    \end{answer}
\end{standalonequote}

\subsection{DHEA}

\begin{standalonequote}{DHEA}
    \metadata{topic={DHEA Deficiency Timing}, source={Email Wiki}}

    \begin{answer}
        It's very common for people in their forties to become deficient in both pregnenolone and DHEA, but occasionally it happens in younger people, usually because of an imbalance of thyroid and estrogen. In women, too much DHEA can have a masculinizing effect, so it's best to work on the diet, or to use pregnenolone, which doesn't lead to an imbalance between progesterone and DHEA, since it turns into either, according to need.
    \end{answer}
\end{standalonequote}

\begin{standalonequote}{DHEA}
    \metadata{topic={DHEA Side Effects In Men}, source={Email Wiki}}

    \begin{answer}
        Ten milligrams of DHEA is pretty safe for men, the most common side effects are pimples, oily skin, and sex dreams.
    \end{answer}
\end{standalonequote}

\begin{standalonequote}{DHEA}
    \metadata{topic={DHEA With Low Thyroid}, source={Email Wiki}}

    \begin{answer}
        If your thyroid is very low, you should be cautious with the DHEA, because stress hormones can cause it to turn to estrogen. 5 mg of DHEA taken with a little olive oil or butter can have a noticeable effect on your mood and muscle tone in a few hours.
    \end{answer}
\end{standalonequote}

\begin{qaexchange}{DHEA}
    \metadata{topic={DHEA Oral Administration}, source={Email Wiki}}

    \begin{question}
        Oral or topical use?
    \end{question}

    \begin{answer}
        Orally.
    \end{answer}
\end{qaexchange}

\begin{standalonequote}{DHEA}
    \metadata{topic={DHEA In Diabetes}, source={Email Wiki}}

    \begin{answer}
        If thyroid function is good, and inflammation is low, about 5 mg of DHEA is probably safe, but I think pregnenolone and cynomel would probably be as effective.
    \end{answer}
\end{standalonequote}

\begin{standalonequote}{DHEA}
    \metadata{topic={Low Levels}, source={Ray Peat Forum}}

    \begin{answer}
      It's important to get the nutrients in balance--both zinc and vitamin A are involved in the conversion, and will tend to be depleted when the metabolic rate is higher.
    \end{answer}
\end{standalonequote}

\begin{qaexchange}{DHEA}
    \metadata{topic={DHEA Oral Dosing}, source={Ray Peat Forum}}

    \begin{question}
        What is the optimal dose of DHEA if someone opts to supplement it orally?
    \end{question}

    \begin{answer}
      Its production decreases fairly steadily with age, from a daily maximum of 12 to 15 mg in the teens, to nearly zero at 90, so supplements of 5 to 10 milligrams are usually safe for middle aged people.
    \end{answer}
\end{qaexchange}

\begin{qaexchange}{DHEA}
    \metadata{topic={DHEA Application Location}, source={Ray Peat Forum}}

    \begin{question}
        Where is the best place to apply topical DHEA to get maximum benefits for brain function and fat metabolism?
    \end{question}

    \begin{answer}
       I've normally used it orally, but sometimes use it on arms or legs or elsewhere if there's an injury. 
    \end{answer}
\end{qaexchange}

\begin{qaexchange}{DHEA}
    \metadata{topic={DHEA on Testicles}, source={Ray Peat Forum}}

    \begin{question}
        Theoretically, if the solvent is pretty safe, would applying DHEA on the testicles (less than 5 mg) be a good idea given that they are the primary site of estrogen conversion?
    \end{question}

    \begin{answer}
       As a regulator of brain function and connective tissue strength and fat metabolism, it seems odd that it would occur to someone to apply DHEA to the testicles. (The same applies to solvents of any kind other than water.) 
    \end{answer}
\end{qaexchange}

\begin{qaexchange}{DHEA}
    \metadata{topic={Raising DHEA Levels Safely}, source={Ray Peat Forum}}

    \begin{question}
        What do you think is the safest approach to raising DHEA levels in a male that has below reference range DHEA according to testing?
    \end{question}

    \begin{answer}
      Sometimes supplementing pregnenolone can do it, but 5 mg of DHEA is safe, if your thyroid, vitamin D, and other tests are normal.
    \end{answer}
\end{qaexchange}

\begin{standalonequote}{DHEA}
    \metadata{topic={DHEA Increasing Height/Growth}, source={Ray Peat Forum}}

    \begin{answer}
      A friend and her two kids were living with us, and everyone marked their height on the door frame; I was about 40 at the time. About a year before we moved out of the house, I was experimenting with DHEA; I had previously been handling progesterone occasionally for several years. I probably took 10 or 15 mg per day, and after a few days there was a sudden change in a mole on my belly, and my horizontal, semi-impacted lower wisdom teeth suddenly began erupting; over a period of a few weeks, they moved into a fully erupted position. I was about 45 at that time. I noticed that my belt was looser, and I had a visible waste, so I checked my weight, and found that I hadn't lost weight, though considerable fat had disappeared. We were moving out of the house, and before we left we checked our height on the door frame. My younger girl friend was about the same as before, I was considerably taller.
    \end{answer}
\end{standalonequote}

\begin{standalonequote}{DHEA}
    \metadata{topic={DHEA Wisdom Teeth Eruption}, source={Ray Peat Forum}}

    \begin{answer}
      When I first tried using DHEA (about 5 mg/day) my lower wisdom teeth quickly erupted and moved into position. I think chronically slightly low thyroid function with low vitamin D is usually responsible for delayed wisdom tooth eruption.
    \end{answer}
\end{standalonequote}

\begin{standalonequote}{DHEA}
    \metadata{topic={DHEA Erupting Wisdom Teeth}, source={Ray Peat Forum}}

    \begin{answer}
      My impacted lower wisdom teeth rotated into position in about three weeks when I took a little DHEA. Several people have told me of similar experiences.
    \end{answer}
\end{standalonequote}

\begin{standalonequote}{DHEA}
    \metadata{topic={DHEA Dosage, Aromatase}, source={Ray Peat Forum}}

    \begin{answer}
      I don't think that's a risk with 5 mg or less. Things that protect against aromatase include aspirin, thyroid, vitamin D, pregnenolone, and sugar.
    \end{answer}
\end{standalonequote}

\subsection{Testosterone}

\begin{qaexchange}{Testosterone}
    \metadata{topic={Androsterone And Joint Pain}, source={Email Wiki}}

    \begin{question}
        Joint pain from excessively lowering estrogens?
    \end{question}

    \begin{answer}
        I doubt the low estrogen theory.
    \end{answer}
\end{qaexchange}

\begin{standalonequote}{Testosterone}
    \metadata{topic={Testosterone Test Interpretation}, source={Email Wiki}}

    \begin{answer}
        It has to be interpreted in relation to cortisol, estrogen, and sex hormone binding globulin.
    \end{answer}
\end{standalonequote}

\begin{standalonequote}{Testosterone}
    \metadata{topic={DHT vs Testosterone Aromatization}, source={Email Wiki}}

    \begin{answer}
        The special difference between testosterone and DHT is that testosterone is easily aromatized into estrogen, and DHT isn't. There are several ways that the body can dispose of estrogen, but I haven't heard of that way of inactivating it; I don't think it happens in the body.'
    \end{answer}
\end{standalonequote}

\begin{standalonequote}{Testosterone}
    \metadata{topic={DHT Vilification}, source={Email Wiki}}

    \begin{answer}
        I think it was the drug industry, thinking of villains to justify their otherwise crazy treatments.
    \end{answer}
\end{standalonequote}

\begin{standalonequote}{Testosterone}
    \metadata{topic={DHT Dosing Safety}, source={Email Wiki}}

    \begin{answer}
        I haven't heard of any bad effects from DHT, but that might be because it's so rarely used. The liver problems I've heard about have always involved slightly modified molecules. I think the tendency to take too much might be a problem with androgens generally--4 milligrams of testosterone and 15 mg of DHEA is a normal daily production for young men, and half of that amount is effective for middle aged men, unless the problem is something else.
    \end{answer}
\end{standalonequote}

\begin{standalonequote}{Testosterone}
    \metadata{topic={DHT vs Mesterolone}, source={Email Wiki}}

    \begin{answer}
        A little DHT should be safe, but I don't think mesterolone is safe in any quantity.
    \end{answer}
\end{standalonequote}

\begin{standalonequote}{Testosterone}
    \metadata{topic={Diet For Hormone Balance}, source={Email Wiki}}
    \begin{note}
        Low Testosterone, High Cholesterol
    \end{note}

    \begin{answer}
        The problem with chicken is that the fat is highly unsaturated, and the meat provides very little calcium. Milk and cheese have a much better ratio of calcium to phosphate. Having the carrots raw (shredded, with a little olive oil, vinegar, and salt) would help with the hormone balance, and protect the intestine against inflammation. Supplementing pregnenolone wouldn't have the risk of the DHEA being converted to estrogen, which tends to happen when thyroid function is low. A small supplement of Armour thyroid or the equivalent could quickly lower the cholesterol, and since cholesterol is converted by thyroid into pregnenolone and DHEA, that would probably help the testosterone. Some shellfish (oysters, shrimp, squid, etc.) or low fat fish would provide trace minerals that might be lacking in your diet. Several eggs per week, or liver once a week, can help with other nutrients that are probably deficient in your present foods. Well cooked potatoes, with butter or cream, fruit, and well cooked greens are other foods have vitamins and minerals that are helpful.
    \end{answer}
\end{standalonequote}

\begin{qaexchange}{Testosterone}
    \metadata{topic={Testosterone Supplementation}, source={Email Wiki}}

    \begin{question}
        Low Testosterone: Would the addition of topical testosterone supplement work in the same direction as the diet and thyroid supplement?
    \end{question}

    \begin{answer}
        Yes, I think a small supplement of testosterone will work in the same direction.
    \end{answer}
\end{qaexchange}

\begin{standalonequote}{Testosterone}
    \metadata{topic={Testicular Recovery Protocol}, source={Email Wiki}}
    \begin{note}
        Shrunken Testes from Use of High Doses of Exogenous Testosterone/DHT
    \end{note}

    \begin{answer}
        Testosterone and DHT aren't toxic, so the testes probably haven't been damaged, and would resume functioning with good nutritional support. 100 mg of pregnenolone and 5 mg of DHEA would probably help their recovery, but I think it would be good to have your LH and estrogen checked. If the LH is very high, using a little DHT for a while might be protective until it's more normal. Coffee, aspirin, vitamin D, milk and cheese would also protect against high LH.
    \end{answer}
\end{standalonequote}

\begin{qaexchange}{Testosterone}
    \metadata{topic={Women}, source={Ray Peat Forum}}

    \begin{question}
        When is it a viable option for a woman to supplement with testosterone (i.e: testosterone base dissolved in vitamin e?) I've heard of the pellots that are being pushed at anti-aging clinics for women, but those seems just as dangerous as injecting testosterone.
    \end{question}

    \begin{answer}
        Women are even more likely to turn it into estrogen, if they are in a low thyroid, inflammatory state.
    \end{answer}
\end{qaexchange}

\begin{standalonequote}{Testosterone}
    \metadata{topic={Supplemental}, source={Ray Peat Forum}}

    \begin{answer}
        Yes, pure testosterone on the skin is safe if the diet and thyroid function are good, but it's better to try supplements of pregnenolone first, and then DHEA, to normalize the testosterone production.
    \end{answer}
\end{standalonequote}

\begin{standalonequote}{Testosterone}
    \metadata{topic={Testosterone Administration Method}, source={Ray Peat Forum}}

    \begin{answer}
      I think an oil or cream with real testosterone (and DHEA) is much better than the injections, which are usually an ester of testosterone in a toxic solvent.
    \end{answer}
\end{standalonequote}

\begin{qaexchange}{Testosterone}
    \metadata{topic={Raising Androgens Safely}, source={Ray Peat Forum}}

    \begin{question}
        If the diet is good, are there any supplements to take to raise serum androgens safely in a man?
    \end{question}

    \begin{answer}
      Pregenolone has multiple favorable effects. Sometimes thyroid can improve the ratio of androgens to cortisol and estrogen. Aspirin, by lowering cortisol, can improve the ratio. DHEA, in doses of 5 to 10 mg, can have a positive effect if thyroid function is good.
    \end{answer}
\end{qaexchange}

\begin{qaexchange}{Testosterone}
    \metadata{topic={Androsterone Dosage}, source={Ray Peat Forum}}

    \begin{question}
        Would taking a maximum of 2 mg of androsterone a day be safe for a man who is trying to raise his DHT levels?
    \end{question}

    \begin{answer}
       It probably is, but I haven't had any personal experience with it. DHEA usually helps. 
    \end{answer}
\end{qaexchange}

\begin{standalonequote}{Testosterone}
    \metadata{topic={Testosterone Dosing, HCG Danger}, source={Ray Peat Forum}}

    \begin{answer}
      A healthy young man produces about 4 mg of testosterone and 12 to 15 mg of DHEA per day, so 80 mg per week [of testosterone] seems hugely excessive. An excess of either is likely to be turned into estrogen. Pure pregnenolone would be better, but I don't know of a current product that I would trust to be free of estrogens. HCG isn't safe.
    \end{answer}
\end{standalonequote}

\begin{qaexchange}{Testosterone}
    \metadata{topic={Methenolone Safety}, source={Ray Peat Forum}}

    \begin{question}
        Do you have any thoughts on the general safety profile of the synthetic androgen methenolone?
    \end{question}

    \begin{answer}
      No; generally safety studies of new androgens have been extremely inadequate.
    \end{answer}
\end{qaexchange}

\begin{qaexchange}{Testosterone}
    \metadata{topic={Masturbation, Testosterone}, source={Ray Peat Forum}}

    \begin{question}
        Do you think masturbating to pornography would increase estrogen/cortisol and have harmful physical side effects that you do not get from sex with a live partner? Or is masturbation and watching naked women a non-factor when it comes to physical health?
    \end{question}

    \begin{answer}
      Just thinking about, anticipating, sex increases testosterone, makes the whiskers grow faster; general good health keeps the increased testosterone from increasing estrogen and cortisol.
    \end{answer}
\end{qaexchange}

\begin{standalonequote}{Testosterone}
    \metadata{topic={Oral Testosterone Administration}, source={Ray Peat Forum}}

    \begin{answer}
      A small amount of testosterone either in olive oil or as powder seems fully effective when swallowed; DHEA, similarly effective, but with about 3 times the dose.
    \end{answer}
\end{standalonequote}

\subsection{Estrogen Management}

\begin{standalonequote}{Estrogen}
    \metadata{topic={Estrogen Effects On Tissues}, source={Email Wiki}}

    \begin{answer}
        Estrogen can cause ovarian cysts to develop, and can contribute to the development of skin tags and moles. Its effects on the urethra might help with incontinence, but it can cause problems with the bladder muscle, and cystitis.
    \end{answer}
\end{standalonequote}

\begin{standalonequote}{Estrogen}
    \metadata{topic={Estrogen And Libido}, source={Email Wiki}}

    \begin{answer}
        High estrogen does sometimes cause insatiable sexual interest, partly because it increases adrenal androgens, and partly by inhibiting satisfying orgasms. Too much progesterone can suppress or neutralize the androgens. Thyroid is the best way to regulate the system, keeping libido up, making orgasms satisfying.
    \end{answer}
\end{standalonequote}

\begin{qaexchange}{Estrogen}
    \metadata{topic={Effects on Immune System}, source={Ray Peat Forum}}

    \begin{question}
        Do you know what effects estrogen dominance has on immune function and neurotransmitter status?
    \end{question}

    \begin{answer}
        S. Ansar Ahmed and N. Talal have a book and several papers that discuss estrogen excess as the basis of autoimmune disorders, osteoarthritis, lupus, etc. G.C. Desjardin is a good representative of the people studying estrogen's effects on neurotransmitters, but my article(s) on Alzheimer's disease (the editor of Townsend Letter for Doctors said they were planning to print it, but I don't get the magazine so I don't know whether they did) has a lot of references. My newsletter on the physiology of hot flushes coming out in a couple of months and this month's epilepsy article will have some others. Estrogen activates the excitotoxic pathway in a great variety of ways, promoting acetylcholine and excitatory amino acids, but it also has an extremely general association with histamine. It also acts on the prostaglandin and free fatty acid signalling systems, adenosine, endorphin, endozepine, all of the amine transmitters, and interacts with everything else of nervous importance, such as the \enquote{anandamides.} Estrogen's effects on calcium, potassium, nitric oxide, lactic acid, carbon dioxide, and carbon monoxide relate to both the question of autoimmunity and nerve actions.

        People have been mentioning a variety of problems associated with myalgia, such as pulmonary hypertension and diabetes. In past newsletters I have already talked about the meaning of hypothyroidism/high estrogen/free fatty acids in many diseases, but I will soon be trying to put all of the degenerative diseases into context by focussing on the role of cardiolipin in mitochondrial function, and how that relates to diet and hormones. The loss of copper and overloading with iron, seen in inflammatory states and hyperestrogenism/hypothyroidism, is one of the factors that tend to make the respiratory defect get worse.
    \end{answer}
\end{qaexchange}

\begin{qaexchange}{Estrogen}
    \metadata{topic={Blood Test Results}, source={Ray Peat Forum}}

    \begin{question}
        Why did my blood test come back that I had low estrogen? And also how can the body start making it's own progesterone?
    \end{question}

    \begin{answer}
        There are lots of articles on my website about estrogen. Carla Rothenberg and Barbara Seaman have written some very good things about the fraud involved in the origin of \enquote{HRT,} or \enquote{estrogen replacement.} The article on tissue bound estrogen on my site explains that, when progesterone is deficient, estrogen may be produced in many tissues other than the ovaries, without being released into the blood stream, making its measurement in the serum meaningless. Signs that are commonly said to be from estrogen deficiency are often from estrogen excess.
    \end{answer}
\end{qaexchange}

\begin{standalonequote}{Estrogen}
    \metadata{topic={Hair Loss}, source={Ray Peat Forum}}

    \begin{answer}
        Veterinarians have noticed that dogs often lose their fur when they are around a woman who is using a topical estrogen cream.
    \end{answer}
\end{standalonequote}

\begin{standalonequote}{Estrogen Management}
    \metadata{topic={Low Estradiol in Men}, source={Ray Peat Forum}}

    \begin{answer}
       When testosterone is normal, it will locally be converted to estrogen as needed, so I don't think the low serum amount matters. 
    \end{answer}
\end{standalonequote}

\begin{emailexchange}{Estrogen Management}
    \metadata{topic={17alpha-Estradiol Lifespan Extension}, source={Ray Peat Forum}}

    \begin{question}
        I wanted to know your thoughts on 17alpha-estradiol, the \enquote{non-feminizing} weak estrogen that is showing reproducible life span extension in male mice studies. Some interpretations suggest this to be a potential reason for why females tend to live longer than males. Could this legitimately be a \enquote{good} estrogen, or is this another ploy from estrogen pharmaceutical companies to further promote their therapies?
    \end{question}

    \begin{answer}
       I don't know about old male mice, but in humans old men often have much more estrogen than women of the same age, so a relatively neutral competitor of estrogen should be helpful. I think progesterone is likely to be better.
    \end{answer}

    \begin{question}
        Do you think that these milder neutral forms of estrogen compete for ER binding sites to limit the more damaging effects of stronger estrogens, or just dilute their concentration in tissues? Would there be any therapeutic applications to combine 17alpha-estradiol and progesterone to tackle estrogen, perhaps via two different mechanisms?
    \end{question}

    \begin{answer}
      It weakens the effects of estradiol on the receptors. Progesterone acts in a variety of ways, reducing estrogen production and activation, and by degrading the estrogen receptor.
    \end{answer}
\end{emailexchange}

\begin{qaexchange}{Estrogen Management}
    \metadata{topic={Hormone Testing Timing}, source={Ray Peat Forum}}

    \begin{question}
        When is the best part of the cycle to test estrogen and progesterone for menstruating women?
    \end{question}

    \begin{answer}
      For a single test, the luteal phase, a few days after ovulation.
    \end{answer}
\end{qaexchange}

\section{Stress Hormones}

\begin{standalonequote}{Stress Hormones}
    \metadata{topic={Breaking Stress Hormone Patterns}, source={Ray Peat Forum}}

    \begin{answer}
      An intense stress can cause the hormones to get stuck in an inefficient state, often with suppressed thyroid function and decreased steroid production. Sometimes foods that are essential nutrient rich -- eggs, orange juice, oysters, liver -- or just pleasurable, milk shakes, ice cream etc. (especially at bedtime, to improve sleep), can break the pattern, and sometimes supplementing thyroid or pregnenolone can do it.

      Changing your view of your job is a possibility, but a healthy attitude in an unhealthy environment usually causes problems. I think real mindfulness is incompatible with nearly all the contemporary work environments. 
    \end{answer}
\end{standalonequote}

\subsection{Cortisol}

\begin{qaexchange}{Cortisol}
    \metadata{topic={Cortisol Testing Utility}, source={Email Wiki}}

    \begin{question}
        What could a cortisol test help with?
    \end{question}

    \begin{answer}
        It could show whether it's chronically high. It can help to judge the doses of the things that lower it---pregnenolone, progesterone, aspirin, sugar, thyroid, calcium, etc.
    \end{answer}
\end{qaexchange}

\begin{standalonequote}{Cortisol}
    \metadata{topic={Cortisol Supplementation Critique}, source={Email Wiki}}
    \begin{note}
        Jefferies and Safe Uses of Cortisol
    \end{note}

    \begin{answer}
        I don't think his arguments are correct. The amounts he sometimes prescribed weren't always safe.
    \end{answer}
\end{standalonequote}

\begin{standalonequote}{Cortisol}
    \metadata{topic={Cortef And Cushing's Symptoms}, source={Email Wiki}}
    \begin{note}
        Cortef
    \end{note}

    \begin{answer}
        I think William Jefferies' book created a lot of interest in that. Since ACTH can interfere with ovarian function, cortisol can sometimes help the ovaries to make progesterone, by suppressing ACTH. But I knew people who followed his prescription and got Cushing's symptoms. Pregnenolone is something that can always be used with thyroid, to guarantee an easy adrenal response.
    \end{answer}
\end{standalonequote}

\begin{standalonequote}{Cortisol}
    \metadata{topic={Cortisol Drug Comparisons}, source={Email Wiki}}
    \begin{note}
        Cortisol - Cortisone - Cortef
    \end{note}

    \begin{answer}
        Cortisol works in the body although the body can convert cortisol to cortisone. Synthetic cortisol-like drugs, such as prednisone are more like cortisol. Also, hydrocortisone is a drug that acts like cortisol. The body makes 20 mg of cortisol daily. Taking 10 mg of prednisone is equivalent to about 50 mg of cortisol or 2.5 times the daily amount made in-vitro. Cortef is Hydrocortisone which acts like cortisol.
    \end{answer}
\end{standalonequote}

\begin{standalonequote}{Cortisol}
    \metadata{topic={Cortison and Weak Adrenals}, source={Email Wiki}}

    \begin{answer}
        Cortisol is a little more water soluble than progesterone, and a diurnal cycle can be seen in the saliva, but the absolute amounts aren't as meaningful as in the serum. Thyroid is needed for the adrenals to function well, and adequate cholesterol, as raw material. It's popular to talk about \enquote{weak adrenals,} but the adrenal cortex regenerates very well. Animal experimenters can make animals that lack the adrenal medulla by scooping out everything inside the adrenal capsule, and the remaining cells quickly regenerate the steroid producing tissues, the cortex. So I think the \enquote{low adrenal} people are simply low thyroid, or deficient in cholesterol or nutrients.
    \end{answer}
\end{standalonequote}

\begin{standalonequote}{Cortisol}
    \metadata{topic={Cortisol Deficiency}, source={Email Wiki}}

    \begin{answer}
        Addison's disease, with adrenal cortex degeneration, can cause cortisol deficiency, in which case progesterone would compensate, but doctors often tell people they \enquote{don't have enough cortisol} without proper confirmation.
    \end{answer}
\end{standalonequote}

\begin{qaexchange}{Cortisol}
    \metadata{topic={Cortisol Capillary Protection}, source={Ray Peat Forum}}

    \begin{question}
        How does cortisol protect against shock and stress partly by maintaining the resistance and integrity of the capillaries?
    \end{question}

    \begin{answer}
      Progesterone and cortisol both prevent leakiness of capillaries and are antiinflammatory, though they are antagonists in other situations. They both stabilize mast cells, decreasing histamine and serotonin, and inhibit phospholipase and prostaglandin formation and release of various inflammatory cytokines, and protect the glycocalyx.
    \end{answer}
\end{qaexchange}

\subsection{Adrenaline}

\begin{qaexchange}{Stress Hormones}
    \metadata{topic={Adrenaline Surge at Night}, source={Ray Peat Forum}}

    \begin{question}
        The past couple of days, regardless of day or night, whenever I try to sleep I get jolted awake by what feels like an adrenaline rush. It seems to happen regardless of how much I have eaten as my initial thought was that it was low liver glycogen related. Do you have any ideas how to fix this issue?
    \end{question}

    \begin{answer}
        The most frequent cause is the presence of something in the intestine that's causing stress—too much protein, or something such as starch supporting bacterial metabolism, or allergens.
    \end{answer}
\end{qaexchange}

\section{Metabolic Hormones}
\subsection{Prolactin}

\begin{standalonequote}{Prolactin}
    \metadata{topic={Prolactin Antiserotonin Drugs}, source={Email Wiki}}

    \begin{answer}
        What would your doctors think about letting you try an antiserotonin drug, like lisuride or ondansetron or bromocriptine, now that your prolactin was measured so high? I think the prolactin should be around 9 to 12.
    \end{answer}
\end{standalonequote}

\begin{standalonequote}{Prolactin}
    \metadata{topic={Prolactin Levels}, source={Email Wiki}}

    \begin{answer}
        High prolactin
    \end{answer}
\end{standalonequote}

\begin{standalonequote}{Prolactin}
    \metadata{topic={Prolactin Treatment Options}, source={Email Wiki}}

    \begin{answer}
        Either vitex or bromocriptine would probably stop it, but I think it's probably caused by mild hypothyroidism, and that the best way to handle it would be with a thyroid supplement, and that would probably help your libido too.
    \end{answer}
\end{standalonequote}

\begin{standalonequote}{Prolactin}
    \metadata{topic={Prolactin And TSH Relationship}, source={Email Wiki}}

    \begin{answer}
        Prolactin and TSH tend to increase together, so when you didn't need the prolactin to be high, the TSH--which might have been keeping your thyroid active despite high estrogen--could have decreased, letting the gland be suppressed by estrogen (and maybe PUFA, from the nuts and any non-ruminant meats). Optimally, the TSH should be very low, but the thyroid gland should keep functioning without needing much stimulation.
    \end{answer}
\end{standalonequote}

\begin{standalonequote}{Prolactin}
    \metadata{topic={Prolactin Reduction Methods}, source={Email Wiki}}

    \begin{answer}
        Salt and thyroid usually lower it, but you might want to try a little vitamin B\textsubscript{6}; even a small amount, about 10 mg per day, can lower prolactin.
    \end{answer}
\end{standalonequote}

\begin{standalonequote}{Prolactin}
    \metadata{topic={Vitamin D/Calcium and Prolactin}, source={Ray Peat Forum}}

    \begin{answer}
       Deficiency of vitamin D and calcium (relative to phosphate) tends to increase prolactin.
    \end{answer}
\end{standalonequote}

\begin{qaexchange}{Prolactin}
    \metadata{topic={Estrogen, Libido, Prolactin}, source={Ray Peat Forum}}

    \begin{question}
        Why is it that with higher serum estrogen, with normal prolactin levels, that some men have higher libido?
    \end{question}

    \begin{answer}
      It's a brain excitant, and is produced locally in the brain from testosterone even when the serum estrogen isn't high. I think the absence of higher prolactin means that something is protecting systemically against estrogen, which activates prolactin via serotonin and cortisol. Cortisol increases the formation of serotonin a direct stimulus to prolactin, and the androgens and progesterone oppose cortisol's effects.
    \end{answer}
\end{qaexchange}

\subsection{Parathyroid Hormone}

\begin{standalonequote}{Parathyroid Hormone}
    \metadata{topic={PTH And Calcium Balance}, source={Email Wiki}}

    \begin{answer}
        High parathyroid hormone will increase calcium and lower phosphate. I regularly use at least two quarts of milk per day, in the past I have averaged a gallon a day, high calcium intake helps to compensate for low vitamin D, but both vitamin D and calcium in the diet tend to lower parathyroid hormone, and the serum calcium level. Quite a few people are now recommending from 2000 to 6000 i.u. of vitamin D3 daily during the winter.
    \end{answer}
\end{standalonequote}

\begin{standalonequote}{Parathyroid Hormone}
    \metadata{topic={Parathyroid Gland Enlargement}, source={Email Wiki}}

    \begin{answer}
        If your vitamin D was very low for a long time, I think your parathyroid glands probably enlarged, and might take some time to normalize under the influence of a generous amount of vitamin D and calcium.
    \end{answer}
\end{standalonequote}

\begin{standalonequote}{Parathyroid}
    \metadata{topic={Vitamin D Goal}, source={Ray Peat Forum}}

    \begin{answer}
        I think 50 ng/ml is a good goal. The point at which it lowers parathyroid hormone would be the right amount.
    \end{answer}
\end{standalonequote}

\section{Other Hormones}

\begin{qaexchange}{Hormones}
    \metadata{topic={6-Keto Progesterone}, source={Ray Peat Forum}}

    \begin{question}
        Do you have an opinion on the potential benefits/risks of using 6-keto progesterone? There is not much information on it, but an older study found that it has the same anti-catabolic and sedative effects as progesterone while being devoid of progestin, androgenic, or estrogenic properties.
    \end{question}

    \begin{answer}
        Structurally, it looks as though it should do some of the things progesterone and the androgens do. When a molecule has looked very promising , I have tried very small amounts of it, and I've learned to watch for effects on my sleep during the first few days. The quality of dreams is influenced very sensitively by metabolism. Disruptive substances move dream quality from constructive insight toward incoherence. I think this reflects analogous effects on the organism's vital functions.
    \end{answer}
\end{qaexchange}

\begin{standalonequote}{Hormones}
    \metadata{topic={Lanosterol Absorption}, source={Ray Peat Forum}}

    \begin{answer}
        The high viscosity make it very slow to penetrate; mixing it with coconut oil could help it to absorb.
    \end{answer}
\end{standalonequote}

\subsection{Histamine}

\begin{qaexchange}{Histamine}
    \metadata{topic={Serotonin Relationship}, source={Ray Peat Forum}}

    \begin{question}
        This study found that when mice were injected with LPS brain serotonin levels dropped while staying the same in the control group, while also stating that histamine rose in the brain---to block the effect of serotonin.

		Do you think that histamine acts as a protective mechanism against high levels of serotonin? \extlink{https://www.imperial.ac.uk/news/228353/histamine-could-player-depression-according-study/}{Source}
    \end{question}

    \begin{answer}
      No.
    \end{answer}
\end{qaexchange}

\chapter{Body Systems \& Functions}

\section{Nervous System \& Brain}

\begin{standalonequote}{Nervous System \& Brain}
    \metadata{topic={Alice In Wonderland Syndrome Causes}, source={Email Wiki}}

    \begin{answer}
        I have experienced that; I suspect that it has to do with the depletion of brain energy, and endotoxin and serotonin (and fever) are good candidates for causes.
    \end{answer}
\end{standalonequote}

\begin{standalonequote}{Nervous System \& Brain}
    \metadata{topic={Large Brain Glucose Needs}, source={Email Wiki}}

    \begin{answer}
        It makes the body's glycogen stores more important, so thyroid function can benefit especially from avoidance of PUFA; coffee's protective effects, increasing metabolic efficiency, are probably especially helpful.
    \end{answer}
\end{standalonequote}

\begin{standalonequote}{Nervous System \& Brain}
    \metadata{topic={Large Brain Energy Requirements}, source={Email Wiki}}

    \begin{answer}
        Regarding intelligence and a big head---the brain is energetically a very expensive organ in terms of its energy requirements, and the liver has to be very efficient to meet its needs, so when there is a nutritional or hormonal problem, the problems can be especially intense. Nutritional needs for sugar, protein, vitamins, and minerals can be very high.
    \end{answer}
\end{standalonequote}

\begin{standalonequote}{Nervous System \& Brain}
    \metadata{topic={Electric Sensations Causes}, source={Email Wiki}}

    \begin{answer}
        Several things associated with that include reflexes from intestinal inflammation, hypothyroidism, and a pantothenic acid deficiency.
    \end{answer}
\end{standalonequote}

\begin{standalonequote}{Nervous System \& Brain}
    \metadata{topic={Hallucinations vs Optimal Cognition}, source={Email Wiki}}

    \begin{answer}
        There's a point at which thoughts flow freely and luminously, but meaningfully, that can happen when nutrients and hormones are optimal; coffee and vitamin B\textsubscript{1} support that kind of function. Hallucination suggests that there is distortion in their meaning, probably when energy isn't being produced as fast as it's used.
    \end{answer}
\end{standalonequote}

\begin{qaexchange}{Nervous System \& Brain}
    \metadata{topic={Stuttering, Brain Temperature}, source={Ray Peat Forum}}

    \begin{question}
        Very so often I'll have a day where I occasionally stutter, or mix up the starts of words. I notice when this is happening with my speech it also happens similarly when I write. A strong coffee seems to improve my ability to speak fast without fault. I just wondered if you had any thoughts about the reasons behind this?
    \end{question}

    \begin{answer}
       I think it probably has to do with the temperature of the brain, and the associated motor systems, when the intention runs into reflexes that are operating at a slower speed. 
    \end{answer}
\end{qaexchange}

\begin{standalonequote}{Nervous System \& Brain}
    \metadata{topic={Tremor from Thyroid Deficiency}, source={Ray Peat Forum}}

    \begin{answer}
      Several of my close male relatives developed tremors in middle age, and I was getting shakier in my thirties, but when I started supplementing thyroid, the tremor disappeared and hasn't returned. Nutritional deficiencies and toxins affect different parts of the brain differently, but glucose metabolism is protective everywhere.
    \end{answer}
\end{standalonequote}

\begin{qaexchange}{Nervous System \& Brain}
    \metadata{topic={Non-Dominant Hand Training Benefits}, source={Ray Peat Forum}}

    \begin{question}
         Do you think learning to write with ones non dominant hand (voluntarily, not forced during childhood) is stressful? I worry to go against one's instincts voluntarily might have adverse consequences due to it being a slight stressor.
    \end{question}

    \begin{answer}
      Some studies show that it can remedy neuroses to do more things with the non-dominant hand; the brain becomes exaggertedly one-sided from prolonged stress. The endorphins, produced by stress, help to maintain balance when one side is impaired.
    \end{answer}
\end{qaexchange}

\begin{qaexchange}{Nervous System \& Brain}
    \metadata{topic={Brain Regeneration from Drug Abuse}, source={Ray Peat Forum}}

    \begin{question}
        Is it possible to reverse brain damage for someone who used to abuse drugs like methamphetamine for a long time on a daily basis?
    \end{question}

    \begin{answer}
      Brain cells do regenerate. Progesterone, vitamin D, thyroid hormone, and generally good nutrition support the process.
    \end{answer}
\end{qaexchange}

\begin{qaexchange}{Nervous System \& Brain}
    \metadata{topic={Brain Temperature, Metabolism}, source={Ray Peat Forum}}

    \begin{question}
        Does head insulation and increasing temperature for the brain provide a good boost in metabolism?
    \end{question}

    \begin{answer}
        I wear a hat when I'm in a cold place, but generally my thick hair has been enough. Keeping the brain temperature up is extremely important, especially for restful sleep. It's natural for the forehead to be a little cooler than the rear parts of the brain. In the winter I keep a high-watt bulb over my work area, shining on my head.
    \end{answer}
\end{qaexchange}

\begin{standalonequote}{Nervous System \& Brain}
    \metadata{topic={Non-Dominant Hand Training}, source={Ray Peat Forum}}

    \begin{answer}
      Yes, training the non-dominant hand stabilizes the nervous system, according to work done by Pavlov's followers. Kurt Goldstein used some similar principles with brain damaged patients. N.P. Bekhtereva used internal stimulation for some similar effects.
    \end{answer}
\end{standalonequote}

\begin{standalonequote}{Nervous System \& Brain}
    \metadata{topic={Neurological Repair, Brain Damage}, source={Ray Peat Forum}}

    \begin{answer}
      Both pregnenolone and progesterone are safe in large amounts (except that progesterone can be anesthetic if hundreds of milligrams are taken at once), and help with nerve repair and restoration. Vitamin K (Thorne Research drops, 10 mg/day) and aspirin help to normalize brain metabolism. Symptoms are partly from a poor balance of dopamine and serotonin, and cyproheptadine could help by reducing the serotonin dominance. Bright incandescent bulbs, especially the clear front reflector bulbs that are used for heating, sold at farm and hardware stores, have the red-orange wavelengths that are beneficial; they can shine on the head (all sides) and neck, the red light penetrates very well. They can be used all day if you like, as long as they don't cause over-heating.
    \end{answer}
\end{standalonequote}

\subsection{Cognitive Function \& Memory}

\begin{qaexchange}{Brain}
    \metadata{topic={Intelligence Malleability}, source={Ray Peat Forum}}

    \begin{question}
        How malleable is intelligence?
    \end{question}

    \begin{answer}
      In 1962 Mark Rosenzweig showed that an enriched environment caused rats' brains to grow. A little later, someone found that the DNA content of human brains kept increasing until the age of 90, and about 10 years ago, studies started showing experience-related growth in human brains. Thyroid and thiamine can have great effects on mental ability, and the steroids can either shrink or expand the brain substance. The old Weissmanist-Hayflick doctrine has kept people from thinking about the adaptive nature of adult tissues, but more people are starting to realize that the principles of embryology keep functioning throughout life.
    \end{answer}
\end{qaexchange}

\begin{emailexchange}{Brain}
    \metadata{topic={Size and Consciousness}, source={Ray Peat Forum}}

    \begin{question}
        I read somewhere that you said people with big brains might tend to be more aware of their brain processes. What sort of processes would they be aware of?---like how memories are associated, how creativity works, how thoughts are developed?---or other types of things altogether?
    \end{question}

    \begin{answer}
      The brain has a high rate of glucose oxidation, and I think a person is likely to become aware of the effects of activities on their glucose level when the brain's demands are very high. I think there is likely to be a measurable difference in some of the \enquote{reflex} processes.
    \end{answer}
	
    \begin{question}
        So if brains with different sized parts make different personalities and abilities, can can you see the differences reflected in head size and face shape?
    \end{question}

    \begin{answer}
      About 40 years ago a physical anthropologist did a very good study showing that the relative size of major brain parts can be detected on the outside of the skull. Dogmatic antiphrenologists (pointy headed professors) are hard to convince. Marian Diamond's book shows the effects of estrogen-shrunken cortex vs. progesterone-expanded cortex on the rat face; humans' relatively big cortex probably has a bigger effect on the appearance of the face. A few years ago I asked her about something she said in the book and she didn't have any recollection of it.
    \end{answer}
\end{emailexchange}

\begin{qaexchange}{Brain}
    \metadata{topic={Facial Expression Recognition}, source={Ray Peat Forum}}

    \begin{question}
        What kind of things would cause someone to have trouble reading facial expressions?
    \end{question}

    \begin{answer}
      That can happen after a person has a stroke in a certain area, but I think there's a great natural variation, some of it a matter of character and attitude.
    \end{answer}
\end{qaexchange}

\begin{qaexchange}{Cognitive Function \& Memory}
    \metadata{topic={Brain Storage Capacity}, source={Ray Peat Forum}}

    \begin{question}
        Do you believe that the brain has a limit of knowledge it can store? Or is there an unlimited potential? 
    \end{question}

    \begin{answer}
      I don't think there's a limit. The anatomy of the brain changes constanty with experience and learning.
    \end{answer}
\end{qaexchange}

\begin{standalonequote}{Cognitive Function \& Memory}
    \metadata{topic={Meaningful Activity, Brain Health}, source={Ray Peat Forum}}

    \begin{answer}
      Finding novelty and opportunity in a large context, rather than habitual, ritualized, or trivial activity. Stimulating conversation and spontaneous sex are good examples. Our (industrialized corporate) cultures are designed to exclude meaningful activity as far as possible. Music, theater, literature, research, adapting to a new culture, and political innovation are activities that can be meaningful.
    \end{answer}
\end{standalonequote}

\begin{standalonequote}{Cognitive Function \& Memory}
    \metadata{topic={Enriched Environment, Brain Health}, source={Ray Peat Forum}}

    \begin{answer}
      For rats, the alternative to an enriched environment has been living in a rat box, sort of the equivalent of a Tokyo apartment. Gardening, learning to play a musical instrument, sculpting and drawing, are good; practicing the marital arts, going to new places and taking different routes to old places, listening to an unfamiliar language--anything that involves participation and action or learning.

    \end{answer}
\end{standalonequote}

\subsection{Mental Health}

\begin{qaexchange}{Nervous System}
    \metadata{topic={Breathing Problems}, source={Ray Peat Forum}}

    \begin{question}
        I have a constant feeling of breathing problem, I must concentrate myself to breath in and out. I can't let it happen autonomously, because then I think that I will not breath or not breath enough. It's that a problem of my nervous system?
    \end{question}

    \begin{answer}
        Very high stress hormones create that feeling. Sometimes a little cyproheptadine can lower the hormones and sense of stress.
    \end{answer}
\end{qaexchange}

\begin{qaexchange}{Nervous System}
    \metadata{topic={Dysautonomia}, source={Ray Peat Forum}}

    \begin{question}
        For people with dysutonomia, automic disorders, etc, what is usually the culprit behind this?
    \end{question}

    \begin{answer}
        Toxins, trauma, virus, and hypothyroidism.
    \end{answer}
\end{qaexchange}

\begin{qaexchange}{Mental Health}
    \metadata{topic={Serotonin and Personality}, source={Ray Peat Forum}}

    \begin{question}
        Are there any universal effects serotonin has on personality? 
    \end{question}

    \begin{answer}
      C.R. Cloninger saw serotonin as related to \enquote{harm avoidance,} which has parallels to the induction of hibernation, retreating metabolically to avoid starvation, learned helplessness, and passive acceptance of authoritarianism. In general, I think it involves a lack of the energy needed for creative appropriate actions. Everything affects the energy dimension in some way, so there are many ways to improve the capacity for creative adaptation.
    \end{answer}
\end{qaexchange}

\subsection{Sleep}

\begin{standalonequote}{Sleep}
    \metadata{topic={Bruxism And Intestinal Bacteria}, source={Email Wiki}}

    \begin{answer}
        I think it's caused by irritation and inflammation in the intestine, increasing serotonin. Starches and fibers support bacterial growth and can increase serotonin. Restless leg syndrome is another night-time reaction to bacterial overgrowth..
    \end{answer}
\end{standalonequote}

\begin{standalonequote}{Sleep Disorders}
    \metadata{topic={Sleep Apnea Treatments}, source={Email Wiki}}

    \begin{answer}
        Several things have been very effective, for example the drug Diamox, acetazolamide, stimulates respiration by changing \ce{CO2} and pH; caffeine, thyroid, and progesterone are the more natural things that stimulate respiration. Thyroid is the main regulatory and adaptive substance for respiration. I think it's common to call the apnea \enquote{obstructive} when someone is fat, but it's probably essentially the same condition, filtered through the mechanical medical mind.
    \end{answer}
\end{standalonequote}

\begin{standalonequote}{Sleep Disorders}
    \metadata{topic={Aspirin For Sleep}, source={Email Wiki}}

    \begin{answer}
        It can help with sleep, but you should try it first in the afternoon, because sometimes its first effect can stimulate your metabolism and delay sleep. If you use it regularly, you should have some vitamin K (for example liver once a week).
    \end{answer}
\end{standalonequote}

\begin{standalonequote}{Sleep Disorders}
    \metadata{topic={Extra Sleep Benefits}, source={Email Wiki}}
    \begin{note}
        Sleeping 10+ Hours
    \end{note}

    \begin{answer}
        If you wake up feeling refreshed, I think it's very good to now and then get some extra sleep. Under-sleeping increases nitric oxide, and catch-up sleep lowers it. Niacinamide, coffee, and aspirin are things that lower NO.
    \end{answer}
\end{standalonequote}

\begin{qaexchange}{Sleep}
    \metadata{topic={Deep Sleep Induction}, source={Ray Peat Forum}}

    \begin{question}
        In your book \textit{Generative Energy}, you note how artificial sleep can be useful as restorative therapy for old or sick organisms. Are there any substances or activities that can help induce deep sleep when temporarily unable to do so naturally? 
    \end{question}

    \begin{answer}
      Thyroid hormone, vitamin D, calcium, and magnesium are among the natural things that make normal relaxation and deep restorative sleep possible. Sleep studies on hypothyroid people showed that they weren't able to get beyond superficial sleep during the whole night—no deep restorative sleep at all.
    \end{answer}
\end{qaexchange}

\subsection{Vision}

\begin{qaexchange}{Vision}
    \metadata{topic={Visual Perception Processes}, source={Ray Peat Forum}}

    \begin{question}
        I was very interested in your discussion of the visual systems in \textit{Mind and Tissue}. I've been trying to observe my own sight processes, thinking about the box and triangle diagrams you included, but I'm having trouble. What can I do to observe this process?
    \end{question}

    \begin{answer}
      The motion after-effect is easiest to notice, when you have been traveling and watching scenery come toward you for a while, or if you watch an Archimedes spiral rotating on a turntable, until there is an after-effect (when the real movement stops) of apparent motion in the opposite direction. If you close your eyes during the several seconds when the after-effect would be occurring, you might be able to see what looks like flowing sand.
    \end{answer}
\end{qaexchange}

\section{Cardiovascular System}
\subsection{Heart Health}

\begin{standalonequote}{Cardiovascular}
    \metadata{topic={Atrial Fibrillation}, source={Ray Peat Forum}}

    \begin{answer}
        Vitamins D and K, and calcium are important for stabilizing the heart rhythm. Estrogen tends to cause chemical hyperventilation (loss of carbon dioxide), which increases blood viscosity and the tendency toward atrial fibrillation. Progesterone and those other steroids have opposite effects (progesterone is a natural aldosterone antagonist, too). Thyroid is essential for helping cells to retain magnesium. A quart or two of milk, and a glass or two of orange juice every day helps with the main stabilizing minerals, but it's good to have sea food once a week, especially shell fish, for the trace minerals.
    \end{answer}
\end{standalonequote}

\begin{standalonequote}{Heart Health}
    \metadata{topic={Chest Pain, Magnesium Deficiency}, source={Ray Peat Forum}}

    \begin{answer}
      Intestinal gas is the most common, but when it's in the heart, a magnesium deficiency is often involved.
    \end{answer}
\end{standalonequote}

\begin{standalonequote}{Heart Health}
    \metadata{topic={Takotsubo Cardiomyopathy Causes}, source={Ray Peat Forum}}

    \begin{answer}
      There's evidence that it involves low thyroid and increased estrogen, and low body temperature. Women with the syndrome are likely to have a history of migraines and Raynaud's pnenomenon. It happens 9 times as often in women as in men.
    \end{answer}
\end{standalonequote}

\begin{standalonequote}{Heart Health}
    \metadata{topic={Heart Rate Variability}, source={Ray Peat Forum}}

    \begin{answer}
      I think it's just a fad, deriving from an evidence-free theory about the parasympathetic nervous system. Anxiety and hypertension decrease the effects of breathing on heart pumping.
    \end{answer}
\end{standalonequote}

\begin{qaexchange}{Heart Health}
    \metadata{topic={Oxidized LDL, Lipid Peroxidation}, source={Ray Peat Forum}}

    \begin{question}
        If oxidized ldl is already oxidized, does it continue to cause lipid peroxidation in our blood vessels and cause inflammation and oxidative stress? Or is oxidized LDL only partially oxidized, and being so, it would continue to have a pathological effect?
    \end{question}

    \begin{answer}
      Yes, the oxidized fragments keep spreading the oxidation, with the smaller products often being the most toxic.
    \end{answer}
\end{qaexchange}

\subsection{Blood Pressure}

\begin{standalonequote}{Blood Pressure}
    \metadata{topic={Blood Pressure, TSH, Vitamin D}, source={Ray Peat Forum}}

    \begin{answer}
      Have you had a blood test for TSH and vitamin D? High TSH is often the main factor in high blood pressure, so it should be low. A deficiency of calcium or vitamin D can increase parathyroid hormone, increasing serum calcium and blood pressure. Drinking two quarts of low fat milk per day will help to lose weight and lower blood pressure.
    \end{answer}
\end{standalonequote}

\subsection{Circulation}

\begin{standalonequote}{Circulation}
    \metadata{topic={Blood Clot Dissolution}, source={Ray Peat Forum}}

    \begin{answer}
       Enzymes are always able to degrade clots, though the presence of polyunsaturated fats makes them much tougher and more persistent. Both vitamin E and aspirin accelerate the dissolution, aspirin by direct chemical actions. Vitamin K should be used with aspirin. 
    \end{answer}
\end{standalonequote}

\section{Digestive System}

\begin{standalonequote}{Digestive System}
    \metadata{topic={GERD Causes}, source={Email Wiki}}

    \begin{answer}
        It's usually associated with disturbed muscle action of the whole stomach and intestine and gall bladder. High estrogen, serotonin, prostaglandins, low thyroid, inflammation, and bacterial overgrowth in the small intestine are often involved. Stimulating the intestine with a daily raw carrot often helps.
    \end{answer}
\end{standalonequote}

\begin{standalonequote}{Digestive System}
    \metadata{topic={Hiatal Hernia Causes}, source={Email Wiki}}

    \begin{answer}
        Have you had blood tests for hormones? The whole complex of symptoms including hiatal hernia is usually caused by a general weakness of digestive and hormonal processes, and it's especaily important to check thyroid function carefully, with a blood test and recording waking and midday temperature and pulse rate, and average caloric requirement.
    \end{answer}
\end{standalonequote}

\begin{standalonequote}{Digestive System}
    \metadata{topic={Hiatal Hernia Hormone Causes}, source={Email Wiki}}

    \begin{answer}
        It is most likely to develop as a result of reduced thyroid hormone and increased stress hormones (especially cortisol, in relation to testosterone and DHEA), weakening connective tissues. Some foods that cause intestinal irritation can make it worse; a simplified diet makes it possible to identify any specific foods that make the problem worse. Keeping a record of temperature and pulse rate can help to recognize any hormonal problems.
    \end{answer}
\end{standalonequote}

\begin{qaexchange}{Digestive}
    \metadata{topic={Kidney Damage from Endotoxins}, source={Ray Peat Forum}}

    \begin{question}
        Can you tell me what damages the kidney which leads to renal insufficiency and then to ESKD? Is it the body constantly trying to balance out the blood acidity to maintain the proper pH? Endotoxins?
    \end{question}

    \begin{answer}
       It comes from poisoning from things produced in the intestine by bacteria, resulting from poor digestion. Acid balancing isn't a problem.
    \end{answer}
\end{qaexchange}

\begin{standalonequote}{Digestive System}
    \metadata{topic={Colon Health, Varicose Veins}, source={Ray Peat Forum}}

    \begin{answer}
       It was once a commonly recognized explanation for the leg varicose veins that developed in pregnant women. Thyroid hormone is important for maintaining function of the colon as well as veins, and fibrous foods (raw carrots, cooked mushrooms and bamboo shoots, bran, for example) will sometimes correct the problem. Avoiding starchy foods is important. Well aged cascara sagrada along with the other things has corrected some enlarged colons. 
    \end{answer}
\end{standalonequote}

\begin{standalonequote}{Digestive System}
    \metadata{topic={Hiatal Hernia Treatment}, source={Ray Peat Forum}}

    \begin{answer}
       I know people whose hiatal hernia symptoms stopped when they used thyroid, DHEA, and pregnenolone. 
    \end{answer}
\end{standalonequote}

\subsection{Intestinal Health}

\begin{standalonequote}{Intestinal Health}
    \metadata{topic={Raw Carrot Changing Intestinal Flora}, source={Email Wiki}}
    \begin{note}
        Intestinal bacteria
    \end{note}

    \begin{answer}
        A daily raw carrot (shredded, with olive oil and vinegar, for example) can gradually change the ecology. Sometimes very small amounts of an antibiotic can do it.
    \end{answer}
\end{standalonequote}

\begin{standalonequote}{Intestinal Health}
    \metadata{topic={Intestinal Adaptation Timing}, source={Email Wiki}}

    \begin{answer}
        It takes a few days for the intestine to change its rhythm of peristalsis, and a couple of weeks for the enzymes to adjust to a change of foods. A daily raw carrot helps it to adjust.
    \end{answer}
\end{standalonequote}

\begin{standalonequote}{Intestinal Health}
    \metadata{topic={Persorption Mechanism}, source={Email Wiki}}

    \begin{answer}
        Particles absorbed from the intestine can pass from the blood into the lymph, cerebral spinal fluid, and urine. Having fat and fiber in the food reduces persorption.
    \end{answer}
\end{standalonequote}

\begin{standalonequote}{Intestinal Health}
    \metadata{topic={Yeast And PUFA Sugar}, source={Email Wiki}}

    \begin{answer}
        Yes, yeast loves PUFA, and becomes invasive when deprived of sugar. The mania has been circulating for almost 40 years. I wrote about it in the early '80s.
    \end{answer}
\end{standalonequote}

\begin{qaexchange}{Digestive}
    \metadata{topic={Probiotics}, source={Ray Peat Forum}}

    \begin{question}
        What is your opinion on lactobacillus reuteri as a probiotic to reduce endotoxins? This strain produces the antibiotic reuterin and there's some evidence that it helps with bacterial infections.
    \end{question}

    \begin{answer}
        I think it's safe; I've seen good results from other bacterial cultures, such as B. subtilis and B. licheniformis.
    \end{answer}
\end{qaexchange}

\begin{qaexchange}{Digestive}
    \metadata{topic={Leaky Gut}, source={Ray Peat Forum}}

    \begin{question}
        I know you recommend mushrooms, bamboo shoots and carrot salad, and say coconut oil and olive oil have antibiotic effects also. Besides avoiding many things such as stress, constipation, allergens and other inflammatory things, what would you recommend for healing leaky/damaged intestines besides gelatinous bone broths?
    \end{question}

    \begin{answer}
        Milk, orange juice, and grape juice have important antiinflammatory effects. Thyroid hormone, progesterone, and essential nutrients are important.
    \end{answer}
\end{qaexchange}

\begin{standalonequote}{Digestive}
    \metadata{topic={Silent Reflux}, source={Ray Peat Forum}}

    \begin{answer}
        I think reverse peristalsis is causative of both, moving bacteria up into the stomach and neutralizing the stomach's acid, making a favorable environment for the neutrophilic H. pylori, while moving stomach acid up into the mouth during the night, damaging the teeth. Stress and hypothyroidism are known causes of reverse peristalsis.
    \end{answer}
\end{standalonequote}

\begin{qaexchange}{Digestive}
    \metadata{topic={Cat IBD}, source={Ray Peat Forum}}

    \begin{question}
        Can my foster cat's inflammatory bowel disease be healed with pregnenolone and progesterone treatment? She is on a prescription diet containing oat fibre and psyllium husk. It stops the diarrhea. I've been adding some pregnenolone (50mg) to the food and rubbing progesterone on her ears.
    \end{question}

    \begin{answer}
        Those things seem appropriate. Does she like cottage cheese? A high calcium diet can have an antiinflammatory effect.
    \end{answer}
\end{qaexchange}

\begin{standalonequote}{Intestinal Health}
    \metadata{topic={Chewing Gum, Bowel Reflexes}, source={Ray Peat Forum}}

    \begin{answer}
      I think it's healthiest for the bowel to move after each meal, chewing the gum might be reinforcing a natural reflex.
    \end{answer}
\end{standalonequote}

\begin{standalonequote}{Intestinal Health}
    \metadata{topic={Low Blood Sugar, Intestinal Toxin}, source={Ray Peat Forum}}

    \begin{answer}
      Low blood sugar increases the absorption of toxin from the intestine as well as activating nerve reflexes and inflammation, so keeping the intestine active and clean with mild fiber, and blocking inflammation with aspirin and antihistamine, is usually helpful.
    \end{answer}
\end{standalonequote}

\begin{standalonequote}{Intestinal Health}
    \metadata{topic={Diverticulitis Treatment}, source={Ray Peat Forum}}

    \begin{answer}
       Avoiding indigestible things such as green salads, using antiinflammatory foods such as cooked mushrooms, supplementing vitamin D, and keeping the ratio of calcium to phosphate high.
    \end{answer}
\end{standalonequote}

\begin{standalonequote}{Intestinal Health}
    \metadata{topic={Intestinal Flora and Aging}, source={Ray Peat Forum}}

    \begin{answer}
       The intestinal flora change along with changes in the metabolism, and have to be taken into account. The changes in the environment have to be greater when the degeneration is more advanced. Antibiotics and antiinflammatories and hormonal supplements become more important when a person is seriously sick. Often, even after the age of 50, just going to a high sunny place and eliminating the worst foods is all it takes to restore good health. 
    \end{answer}
\end{standalonequote}

\begin{standalonequote}{Intestinal Health}
    \metadata{topic={Intestinal-Dental Connection}, source={Ray Peat Forum}}

    \begin{answer}
      Irritation in the intestine often makes a tooth infection worse, so avoiding supplements while using the antibiotic is likely to help.
    \end{answer}
\end{standalonequote}

\begin{qaexchange}{Intestinal Health}
    \metadata{topic={Lidocaine Curing Colitis}, source={Ray Peat Forum}}

    \begin{answer}
      I had constantly bleeding colitis for more than a year, and when I took about 20 to 30 mg of lidocaine (in a 2\% solution meant for oral, dental use) the symptoms stopped and haven't returned in more than 30 years.
    \end{answer}
\end{qaexchange}

\begin{standalonequote}{Intestinal Health}
    \metadata{topic={Intestinal Health, Consciousness}, source={Ray Peat Forum}}

    \begin{answer}
      The intestine is powerfully related to consciousness, affecting not only our moods but even the way we feel ourselves in relation to our surrounding space. For example, motion sickness demonstrates the way our sense of movement in space is attached to our stomach and intestine—if we are on a ship, looking at things inside the ship that aren't moving in relation to our body, the real motion sensed by our body conflicts with what our eyes are seeing, and we interpret the inner movement as nausea, but if we just glance at the horizon, the inner sensation of motion suddenly is interpreted accurately as our body moving through space, and the nausea disappears. When there are actual forces being applied to our intestine from the inside, created by bacterial growth, gas, and toxins, our consciousness tries to make sense of it, and the result can be dizziness, a sense of disorientation or falling, so that our sense of location can seem ambiguous or confused.
    \end{answer}
\end{standalonequote}

\begin{standalonequote}{Intestinal Health}
    \metadata{topic={Intestinal Permeability, Aging}, source={Ray Peat Forum}}

    \begin{answer}
      Aging, stress, and heavy consumption of alcohol increase the permeability of the intestine, causing increased absorption of microbial toxins. Laxatives, carrot fiber (not carrotjuice), activated charcoal, and a small amount of sodium thiosulfate decrease the formation and absorption of toxins, increasing the organism's adaptive capacity.
    \end{answer}
\end{standalonequote}

\begin{standalonequote}{Intestinal Health}
    \metadata{topic={Intestinal Bleeding, Thyroid}, source={Ray Peat Forum}}

    \begin{answer}
      The basic problem of intestinal inflammation leads to bleeding in response to the extra irritation of certain foods. I have known people whose bleeding was stopped by thyroid alone; in my own experience, a small amount of 2\% lidocaine solution, swallowed once, ended bleeding that had been occurring every day.
    \end{answer}
\end{standalonequote}

\begin{standalonequote}{Intestinal Health}
    \metadata{topic={Bowel Inflammation Treatment}, source={Ray Peat Forum}}

    \begin{answer}
      I had similar problems, had to eliminate practically all foods, then gradually found ways to increase my tolerance for some of them. Thyroid, progesterone, lidocaine, sugar, DHEA, atropine, pregnenolone, bamboo, cyproheptadine, cascara, flowers of sulfur or sodium thiosulfate, penicillin, aspirin, vitamin D, etc.
    \end{answer}
\end{standalonequote}

\subsection{Liver Function}

\begin{standalonequote}{Liver Function}
    \metadata{topic={Liver Glycogen And Headaches}, source={Email Wiki}}
    \begin{note}
        Headaches and Shakiness
    \end{note}

    \begin{answer}
        An inefficient liver that, among other things, doesn't store enough glycogen to last the whole night, is a common cause of both headaches and shakiness. Lots of light exposure to the whole body helps to increase metabolic efficiency. Liver and oysters about once a week, while keeping stress low, have nutrients that help with liver efficiency. Cyproheptadine, a small amount at bedtime, reduces night stress, might help the headaches and shakiness.
    \end{answer}
\end{standalonequote}

\begin{standalonequote}{Liver Function}
    \metadata{topic={Liver Disease Treatment}, source={Email Wiki}}

    \begin{answer}
        Eliminating all PUFA would be the most important thing, and having lots of orange juice, other sugars including honey, and milk and gelatin. Cytomel, aspirin, acetazolamide, and progesterone all protect the liver and help to slow cancer growth. Some people use extremely large amounts of aspirin, which require supplements of vitamin K, to prevent bleeding. Fibrous foods such as bamboo shoots and laxatives such as cascara help to reduce the absorption of bowel toxins that promote cancer and burden the liver.
    \end{answer}
\end{standalonequote}

\begin{qaexchange}{Liver Function}
    \metadata{topic={Aspirin, Cyproheptadine, Liver}, source={Ray Peat Forum}}

    \begin{question}
        Could taking aspirin and cyproheptadine sometimes worsen liver problems?
    \end{question}

    \begin{answer}
      The excipients often used with them are probably as likely to affect the liver as the chemicals themselves.
    \end{answer}
\end{qaexchange}

\subsection{Gallbladder \& Pancreas}

\begin{standalonequote}{Gallbladder \& Pancreas}
    \metadata{topic={Gallstone Treatment}, source={Ray Peat Forum}}

    \begin{answer}
       I think keeping the hormones in balance, with good thyroid function, is the basic thing. Aspirin is probably helpful.
    \end{answer}
\end{standalonequote}

\subsection{Oral Health}

\begin{standalonequote}{Oral Health}
    \metadata{topic={White Tongue Treatment}, source={Email Wiki}}

    \begin{answer}
        Local bacteria are usually involved in the white tongue, but typically the problem is mainly in the intestine. I have experimented with the old-fashioned 'intestinal disinfectant' camphoric acid (it used to be a common pharmaceutical, 80 to 100 years ago), and when I would swallow about 100 to 200 mg of it in the evening, I would wake up with a perfectly clean tongue, not a bit of the white. Bamboo shoots, raw carrot, and flowers of sulfur are other antiseptics that can reduce the white tongue.
    \end{answer}
\end{standalonequote}

\begin{standalonequote}{Oral Health}
    \metadata{topic={Tartar Prevention}, source={Email Wiki}}

    \begin{answer}
        Rinsing the mouth right after eating.
    \end{answer}
\end{standalonequote}

\begin{standalonequote}{Oral Health}
    \metadata{topic={Bleeding Gums}, source={Email Wiki}}

    \begin{answer}
        It's usually a sign of stress, often from over-growth of bacteria in the upper intestine. A daily raw carrot, shredded with a little vinegar and olive oil, can suppress bacteria.
    \end{answer}
\end{standalonequote}

\begin{standalonequote}{Oral Health}
    \metadata{topic={Gum Health Protocol}, source={Email Wiki}}
    \begin{note}
        Receding Gums and Sensitive Teeth
    \end{note}

    \begin{answer}
        Melting a little coconut oil in the mouth frequently during the day can be effective, because it's antiseptic (and swallowing some at intervals during the day contributes to disinfecting the intestine). Vitamins D and K help some people. The problem usually involves endotoxin absorption, so small daily amounts of minocycline help some people. Putting bamboo shoots through a shredding food processor, so they don't take much chewing, might help to reduce endotoxin. Checking the thyroid is important.
    \end{answer}
\end{standalonequote}

\begin{standalonequote}{Oral Health}
    \metadata{topic={Dental Health}, source={Email Wiki}}

    \begin{answer}
        Stress weakens teeth from the inside, drawing mineral from the dentine; the dentine is the part that can regenerate, not the enamel. Thyroid function is the most important thing for limiting stress.
    \end{answer}
\end{standalonequote}

\begin{standalonequote}{Oral Health}
    \metadata{topic={Tooth Discoloration Mechanism}, source={Email Wiki}}

    \begin{answer}
        Stress typically causes calcium to be removed from the small channels in the dentine, and it tends to be unequal, resulting in spots of discoloration in some teeth, that can develop into cavities. Thyroid is the most important antistress hormone as well as influencing the saliva and immunity.
    \end{answer}
\end{standalonequote}

\begin{standalonequote}{Oral Health}
    \metadata{topic={Tooth Infection Prevention}, source={Email Wiki}}

    \begin{answer}
        Intestinal inflammation is often behind recurrent tooth infections, and a daily raw carrot can make a big difference (along with avoiding legumes, undercooked starches and raw or undercooked vegetables).
    \end{answer}
\end{standalonequote}

\begin{standalonequote}{Oral Health}
    \metadata{topic={Dental Care Routine}, source={Email Wiki}}

    \begin{answer}
        I use baking soda, and I rinse my mouth after having sugar, orange juice, etc. The quality of the saliva, regulated mainly by the thyroid hormone, is the main factor in dental health. My newsletter on osteoporosis mentioned some of the studies on thyroid, estrogen, and tooth decay.
    \end{answer}
\end{standalonequote}

\begin{standalonequote}{Oral Health}
    \metadata{topic={Tongue, Teeth, and Gum Problems}, source={Email Wiki}}

    \begin{answer}
        There are some strong nervous and circulatory interactions between the intestine and the mouth, affecting surfaces and periodontal health, tooth sensitivity, etc. I knew a dentist who stopped doing periodontal surgery when he found that his patients were more easily cured with a laxative. Camphoric acid has been used as an intestinal disinfectant to remedy problems such as coated or sore tongue or bad breath. The tetracyclines have similar effects. Vitamins D and K are important.
    \end{answer}
\end{standalonequote}

\begin{standalonequote}{Oral Health}
    \metadata{topic={Tooth Regeneration Methods}, source={Email Wiki}}
    \begin{note}
        Regenerating Teeth With Light
    \end{note}

    \begin{answer}
        Light can apparently activate part of the process; combining it with pressure and electrical and hormonal stimulation might increase the efficiency of regenerating dentine.
    \end{answer}
\end{standalonequote}

\begin{qaexchange}{Oral Health}
    \metadata{topic={Red Light For Teeth}, source={Email Wiki}}

    \begin{question}
        using a laser?
    \end{question}

    \begin{answer}
        Just bright red light should do it.
    \end{answer}
\end{qaexchange}

\begin{qaexchange}{Oral Health}
    \metadata{topic={Chewing For Tooth Stimulation}, source={Email Wiki}}

    \begin{question}
        Effective stimulation?
    \end{question}

    \begin{answer}
        Chewing is the best stimulation.
    \end{answer}
\end{qaexchange}

\begin{standalonequote}{Oral Health}
    \metadata{topic={Gingivitis Treatment}, source={Email Wiki}}

    \begin{answer}
        Besides keeping phosphates low, getting a lot of vitamin K, and maybe rubbing some onto the gums, might help; it's antiinflammatory. Some people have reverse gingivitis by \enquote{rinsing} with coconut oil twice a day, swishing it around for a couple of minutes.
    \end{answer}
\end{standalonequote}

\begin{standalonequote}{Oral Health}
    \metadata{topic={Wisdom Tooth Extraction}, source={Email Wiki}}

    \begin{answer}
        If it's decayed or inflamed, removing a wisdom tooth might be protective.
    \end{answer}
\end{standalonequote}

\begin{standalonequote}{Oral Health}
    \metadata{topic={Root Canal Safety}, source={Email Wiki}}

    \begin{answer}
        There's normally no need to replace root canals, and x-rays aren't necessary even when having a root canal done if the dentist is very competent. The Japanese are probably more aware than Americans of the damage done by diagnositic x-rays. Systemic toxic effects have been demonstrated from a single set of dental x-rays
    \end{answer}
\end{standalonequote}

\begin{qaexchange}{Oral Health}
    \metadata{topic={Root Canal Mercury}, source={Email Wiki}}

    \begin{question}
        Mercury leaching from root canals continuously and replacing it with a metal / mercury-free alternative?
    \end{question}

    \begin{answer}
        It depends on what the filling material was (my dentist used calcium oxide), but the amount absorbed from amalgam fillings and food is probably much more significant.
    \end{answer}
\end{qaexchange}

\begin{qaexchange}{Oral Health}
    \metadata{topic={Cavity Treatment Options}, source={Email Wiki}}

    \begin{question}
        What is the best course of action for a decayed tooth?
    \end{question}

    \begin{answer}
        I think any obvious cavity should be cleaned out and filled. Extractions are usually done for economic reasons, when a good repair would require a lot of work. When a biting surface isn't involved, a zinc oxide eugenol filling (temporary) is often good for a few years. I think composite fillings are better than amalgam, and the Sorel cements (oxy chloride) are ideal filling materials, though they aren't popular with dentists.
    \end{answer}
\end{qaexchange}

\begin{standalonequote}{Oral Health}
    \metadata{topic={Root Filling Materials}, source={Email Wiki}}

    \begin{answer}
        I think filling a cavity soon is best, and with care the pulp can often be preserved, but a root filling is better than extraction; the material used for a root filling doesn't matter much, if calcium oxide is used at the end. Calcium oxide would be best for the whole thing, but the x-ray mania has discouraged that use.
    \end{answer}
\end{standalonequote}

\begin{standalonequote}{Oral Health}
    \metadata{topic={TMJ And Hypothyroidism}, source={Email Wiki}}

    \begin{answer}
        Have you watched your temperature and pulse rate with various foods? The high magnesium content of coffee, combined with milk and fruit, can help a little with hypothyroidism, but you might need a supplement, to normalize the jaw cartilage.
    \end{answer}
\end{standalonequote}

\begin{standalonequote}{Oral Health}
    \metadata{topic={Oral Yeast Infection Causes}, source={Email Wiki}}

    \begin{answer}
        Poor digestion does affect the membranes of the mouth, but a fungal infection of the mouth usually happens when the immune system is weak, from hormonal imbalance or poor nutrition, for example, or when there isn't enough saliva, or when the membranes are affected by a specific vitamin deficiency, such as vitamin A. Yeasts are attracted to estrogen and glucose, and when the thyroid hormone is deficient the antibodies that normally protect membranes tend to be deficient. It's important to know for sure exactly what the problem is, since leukoplakia is sometimes mistaken for thrush. A rinse with a little powdered sulfur usually eliminates yeast, vitamin A, along with other adequate nutrition, can often correct leukoplakia.
    \end{answer}
\end{standalonequote}

\section{Musculoskeletal System}
\subsection{Bones \& Joints}

\begin{standalonequote}{Musculoskeletal Conditions}
    \metadata{topic={Knee Problems Treatment}, source={Email Wiki}}

    \begin{answer}
        I've seen several grossly malfunctioning knees recover immediately (in from 1 to 12 hours) just with topical progesterone, but the first thing should be to make sure her calcium to phosphorus ratio is good, by having two quarts of low fat milk per day, or the equivalent in low fat cheese, with no grains, legumes, nuts, or muscle meats, and with some well cooked greens regularly. Vitamin K is important for calcium metabolism, too.
    \end{answer}
\end{standalonequote}

\begin{emailexchange}{Musculoskeletal}
    \metadata{topic={Wisdom Teeth}, source={Ray Peat Forum}}

    \begin{answer}
        I think it's good to avoid it if possible. Supplements of DHEA often help with impacted wisdom teeth.
    \end{answer}

    \begin{question}
        Is there any specific reason you have in mind to avoid removal, or is it that avoiding organ removals is best practice?
    \end{question}

    \begin{answer}
        The only specific research I have in mind is that removing rats' molars impaired their learning ability.
    \end{answer}
\end{emailexchange}

\begin{qaexchange}{Musculoskeletal}
    \metadata{topic={Root Canal}, source={Ray Peat Forum}}

    \begin{question}
        Would you recommend against root canal therapy and should I consider having my other root canaled tooth -done a couple years ago- extracted as well?
    \end{question}

    \begin{answer}
        My endodontist in Mexico believes that the immune system is normally able to clear out bacteria at the root tip, but if an infection develops, producing pain, she prescribes an antibiotic. She says she hardly ever has had to pull a tooth.

100 years ago (before antibiotics) it was believed that a \enquote{focal infection} under a tooth caused chronic diseases, and some dentists pulled teeth to treat various non-dental problems. I think the interpretation of causality has often been wrong—for example, I knew a dentist who routinely and successfully treated oral problems by prescribing a laxative.
    \end{answer}
\end{qaexchange}

\begin{qaexchange}{Bones \& Joints}
    \metadata{topic={Titanium Implant Safety}, source={Ray Peat Forum}}

    \begin{question}
        Are titanium implants such as those used for teeth and hips safe?
    \end{question}

    \begin{answer}
      They are the best available, as far as I know. I have known people who were planning to have artificial joints, who changed their plans after using progesterone.
    \end{answer}
\end{qaexchange}

\begin{qaexchange}{Bones \& Joints}
    \metadata{topic={Bone Graft Safety}, source={Ray Peat Forum}}

    \begin{question}
        Are bone grafts safe?
    \end{question}

    \begin{answer}
      I think it's safe if the bone is thoroughly sterilized.
    \end{answer}
\end{qaexchange}

\begin{standalonequote}{Bones \& Joints}
    \metadata{topic={Removing Braces Considerations}, source={Ray Peat Forum}}

    \begin{answer}
       The relation of the teeth to the jaw bone is an active process, and I doubt that it's possible to predict the effects of removing a brace that has been there for a long time. Either way, it's good to be sure that the hormonal-nutritional situation is good, especially vitamin D, parathyroid hormone, prolactin, and cortisol. 
    \end{answer}
\end{standalonequote}

\begin{standalonequote}{Bones \& Joints}
    \metadata{topic={Malocclusion Causes}, source={Ray Peat Forum}}

    \begin{answer}
       I think there's good evidence that the habitual position of the tongue is the decisive thing, exerting a constant outward pressure on the palate and teeth; breathing through the mouth at night interferes with that habit. Nocturnal inflammation tends to cause nasal congestion, forcing mouth breathing. Thyroid and vitamin D deficiencies are major causes of chronic inflammation and sleep breathing problems. 
    \end{answer}
\end{standalonequote}

\begin{standalonequote}{Bones \& Joints}
    \metadata{topic={Joint Cartilage Swelling, Hypothyroidism}, source={Ray Peat Forum}}

    \begin{answer}
      Sometimes slightly low thyroid function can cause the cartilage in joints to swell, and sometimes to accumulate fluid. Slight nutritional deficiencies can contribute to it. Having some seafood once a week often helps, and having enough protein (80 to 100 grams) and calcium (about 1500 mg) is important.
    \end{answer}
\end{standalonequote}

\begin{qaexchange}{Bones \& Joints}
    \metadata{topic={Milk and Arthritis}, source={Ray Peat Forum}}

    \begin{question}
        Is my introduction of milk causing my arthritis?
    \end{question}

    \begin{answer}
       Many years ago someone noticed an antigenic overlap of lactobacilli with joint tissue, and suggested that it could explain the high incidence of rheumatoid arthritis during women's fertile years. Another person noticed an overlap of thyroglobulin with joint tissue. Maybe the milk increased the growth of that kind of bacteria. I think the already increased circulation in your thyroid means that progesterone won't increase its activity. Have you used a little olive oil on your skin to make it easier to spread? Covering your whole leg would increase the effect on your ankle. Is there enough sugar in your diet? Have you tried breathing in a bag occasionally, to increase your \ce{CO2}?
    \end{answer}
\end{qaexchange}

\begin{standalonequote}{Bones \& Joints}
    \metadata{topic={Wisdom Teeth, DHEA}, source={Ray Peat Forum}}

    \begin{answer}
      Mine had been impacted for about 15 years, and within 2 or 3 weeks of taking a small amount of DHEA they erupted properly. Antibiotics and topical antiseptic are commonly used.
    \end{answer}
\end{standalonequote}

\subsection{Muscles}

\begin{qaexchange}{Musculoskeletal}
    \metadata{topic={Aspirin and Muscle Hypertrophy}, source={Ray Peat Forum}}

    \begin{question}
        Is there any negative effects of taking aspirin whilst working out/bodybuilding?

I've read that aspirin can affect the hypertrophy of muscles and the natural inflammatory adaptations that occur post training, as aspirin apparently blunts the inflammatory process, and therefore the beneficial adaptations of exercise do not occur, is this true?
    \end{question}

    \begin{answer}
        It's true that inflammation of the muscles causes them to enlarge, but I don't think that's beneficial for the general health---it involves damage to mitochondria, with some effect on the system's energy economy.
    \end{answer}
\end{qaexchange}

\begin{qaexchange}{Musculoskeletal}
    \metadata{topic={Fat Oxidation and Niacinamide}, source={Ray Peat Forum}}

    \begin{question}
        In some of your interviews you said that fat oxidation is best left to muscles at rest, since this is their preferred fuel. Wouldn't taking substances like niacinamide and aspirin interfere with the muscles' abillity to oxidize fat?
    \end{question}

    \begin{answer}
        Aspirin increases oxygen consumption; although niacinamide can reduce excessive lipolysis, I don't know whether it would lower resting lipolysis.
    \end{answer}
\end{qaexchange}

\begin{standalonequote}{Muscles}
    \metadata{topic={Smooth Muscle Tone Regulation}, source={Ray Peat Forum}}

    \begin{answer}
       Smooth muscle tone is weakened by nitric oxide, histamine, and various stresses, while progesterone, DHEA, and pregnenolone increase the tone. 
    \end{answer}
\end{standalonequote}

\begin{standalonequote}{Muscles}
    \metadata{topic={Muscle Knots Causes}, source={Ray Peat Forum}}

    \begin{answer}
      Hypothyroidism and excess parathyroid hormone are the main causes. Deficiencies of vitamin D, vitamin K, and magnesium are other common causes.
    \end{answer}
\end{standalonequote}

\section{Integumentary System}

\subsection{Eyes}

\begin{standalonequote}{Eyes}
    \metadata{topic={Dilated Pupils Treatment}, source={Email Wiki}}

    \begin{answer}
        I would emphasize milk and orange juice, and some salty things, such as consomme or pork rinds (both with a lot of gelatin). Over a period of a few weeks, it helps the thyroid, pituitary, and liver to adjust. If the thyroid is low, estrogen is slow to be eliminated.
    \end{answer}
\end{standalonequote}

\begin{qaexchange}{Eyes}
    \metadata{topic={Eye Exam Pupil Dilation}, source={Email Wiki}}

    \begin{question}
        Liquid used in eye-exam to dilate the pupils, is that best be avoided?
    \end{question}

    \begin{answer}
        It's usually a synthetic variant of atropine, and I don't think there has been enough research to be sure, but the main thing to worry about would be the antiseptic-preservative with it.
    \end{answer}
\end{qaexchange}

\begin{standalonequote}{Eyes}
    \metadata{topic={Contact Lens Safety}, source={Email Wiki}}

    \begin{answer}
        I think they would become uncomfortable if they were damaging the tissues.
    \end{answer}
\end{standalonequote}

\begin{emailexchange}{Eyes}
    \metadata{topic={T\textsubscript{3} Eye Drops}, source={Ray Peat Forum}}

    \begin{note}
        On an Ophthalmic Solution of Coconut Oil and Liquid T\textsubscript{3}
    \end{note}

    \begin{answer}
        I know a couple of people who have used refined coconut oil in their eyes, as a solvent for other things, and they say it isn't especially uncomfortable, but for some people, any oil in the eye can be painful. I think an emulsifier, even egg yolk, would have an effect similar to soap in the eye.
    \end{answer}

    \begin{question}
        Do you think tocopherols would be an eye irritant?
    \end{question}

    \begin{answer}
      I've accidentally got some in my eye when I was putting it on for sunburn, and it was very uncomfortable.
    \end{answer}
\end{emailexchange}

\begin{qaexchange}{Eyes}
    \metadata{topic={Visual Snow}, source={Ray Peat Forum}}

    \begin{question}
        What are some possible causes and solutions for visual snow?
    \end{question}

    \begin{answer}
        I think it's the individual retinal cells that define our visual acuity; many people just don't notice the way their sense receptors interact with the world.
    \end{answer}
\end{qaexchange}

\begin{standalonequote}{Eyes}
    \metadata{topic={Bloodshot Eyes, B Vitamins}, source={Ray Peat Forum}}
      Enlarged vessels in the eye, especially if there's yellow around them, are suggestive of a B vitamin deficiency, especially B\textsubscript{2}, but there are no really safe supplements of B\textsubscript{2}, so having liver once a week would be reasonable.
    \begin{answer}
      
    \end{answer}
\end{standalonequote}

\begin{standalonequote}{Eyes}
    \metadata{topic={Eye Floaters Prevention}, source={Ray Peat Forum}}

    \begin{answer}
      Avoiding stress, with occasional liver in the diet, regular eggs, to keep the vitamin intake high.
    \end{answer}
\end{standalonequote}

\begin{standalonequote}{Eyes}
    \metadata{topic={Night Blindness Treatment}, source={Ray Peat Forum}}

    \begin{answer}
       Having liver and shellfish once a week to help with the vitamins and trace minerals, and two quarts of milk per day, and plenty of fruit, should help with the sensitivity and night blindness.
    \end{answer}
\end{standalonequote}

\begin{qaexchange}{Eyes}
    \metadata{topic={Orange Lenses for Eye Protection}, source={Ray Peat Forum}}

    \begin{question}
        What are your thoughts on the idea of wearing orange lenses the entire day? Would this be more beneficial than just wearing after sunset?
    \end{question}

    \begin{answer}
       It might make a difference in the risk of cataracts and other eye aging effects.
    \end{answer}
\end{qaexchange}

\begin{qaexchange}{Eyes}
    \metadata{topic={Hearing Improvement at Altitude}, source={Ray Peat Forum}}

    \begin{question}
        Have you ever heard of someone improving their hearing---specifically people who are experiencing hearing loss?
    \end{question}

    \begin{answer}
      Some people experience improved hearing when they move to a high altitude; I think it has to do with improved circulation.

      People going to a high altitude, e.g., 6500 feet or more, sometimes notice improved hearing. Adapting to altitude increases the \ce{CO2} retained in the body, improving capillary circulation. The drug acetazolamide increases \ce{CO2}, and can have similar effects. Thyroid and vitamin D, caffeine and thiamine can help to maintain \ce{CO2} levels.
    \end{answer}
\end{qaexchange}

\begin{standalonequote}{Eyes}
    \metadata{topic={Blue Eyes, Light Damage}, source={Ray Peat Forum}}

    \begin{answer}
      Blue eyes don't filter out as much U.V. and blue light, so people who work outside for years are more susceptible to light damage if their eyes aren't pigmented. Prostaglandins and probably heavy metals can cause pigment to increase with aging; generally, melanin is protective against free radical damage, so the continued light color suggests that you haven't been overexposed to some of the irritants.
    \end{answer}
\end{standalonequote}

\begin{standalonequote}{Eyes}
    \metadata{topic={Dark Eye Circles, Adrenal Function}, source={Ray Peat Forum}}

    \begin{answer}
      The dark eye circles are usually considered to be from an adrenal problem, and thyroid deficiency is one thing that can be responsible, but there are other things that could be responsible, and blood or saliva tests could give you a better idea.
    \end{answer}
\end{standalonequote}

\begin{standalonequote}{Eyes}
    \metadata{topic={Dark Eye Circles, Hypothyroidism}, source={Ray Peat Forum}}

    \begin{answer}
      The dark around the eyes is caused by slightly leaky capillaries, and they are commonly blamed on allergies. Although glucocorticoid treatment sometimes helps, my experience has been that hypothyroidism was responsible.
    \end{answer}
\end{standalonequote}

\subsection{Ears}

\begin{qaexchange}{Ears}
    \metadata{topic={Tinnitus Drug Side Effects}, source={Email Wiki}}

    \begin{question}
        Loratidine or pseydoephedrine for tinnitus?
    \end{question}

    \begin{answer}
        Too much pseudoephedrine increases stress hormones, loratadine isn't good for the liver, and anything that irritates the intestine can cause tinnitus by increasing endotoxin absorption.
    \end{answer}
\end{qaexchange}

\begin{standalonequote}{Ears}
    \metadata{topic={Tinnitus}, source={Email Wiki}}

    \begin{answer}
        Endotoxin can produce those effects. Keeping the digestive system free of inflammation often helps, and a daily raw carrot is sometimes enough, sometimes a supplement of thyroid or progesterone helps. The endogenous opioids can have one-sided effects, and they are increased by endotoxin. Having well cooked mushrooms every day, and avoiding green salads, unsaturated fats, grains and beans are other helpful things. A supplement of niacinamide and other B vitamins sometimes helps.
    \end{answer}
\end{standalonequote}

\subsection{Skin Conditions}

\begin{standalonequote}{Skin Conditions}
    \metadata{topic={Acne Treatment With Sulfur Soap}, source={Email Wiki}}

    \begin{answer}
        Have you tried anything topical, such as sulfur, or an antibiotic such as minocycline? 10\% sulfur soap leaves an antiseptic residue on the skin that can prevent infection.
    \end{answer}
\end{standalonequote}

\begin{standalonequote}{Skin Conditions}
    \metadata{topic={Acne Treatment Sulfur Soap}, source={Email Wiki}}

    \begin{answer}
        I think it would be good to try the sulfur soap first.
    \end{answer}
\end{standalonequote}

\begin{standalonequote}{Skin Conditions}
    \metadata{topic={Sulfur Soap}, source={Email Wiki}}

    \begin{answer}
        The simple ones are what I use; I haven't tried one with salicylic acid.
    \end{answer}
\end{standalonequote}

\begin{standalonequote}{Skin Conditions}
    \metadata{topic={Sulfur Soap Frequency}, source={Email Wiki}}

    \begin{answer}
        The sulfur lingers on the skin for at least a day. It can leave you fairly smelly.
    \end{answer}
\end{standalonequote}

\begin{standalonequote}{Skin Conditions}
    \metadata{topic={Acne, Thyroid, And Vitamin A}, source={Email Wiki}}

    \begin{answer}
        Vitamin A affects the differentiation of skin cells, the production of steroids, and resistance to infection and inflammation. The great increase in formation of the sex steroids at puberty increases the need for vitamin A, and makes its regulatory actions more important. Thyroid's important functions for the skin are the production of steroids and preventing their imbalance, and maintaining the immune function and production of sebum. Other nutritional deficiencies, especially the balance between vitamin E and unsaturated fats, affect the functions of vitamin A and thyroid, so it's important to include foods like liver, eggs, oysters, fruits, and milk in the diet. Sometimes it's easier just to use Benadryl or minocycline to reduce the inflammation and infection.
    \end{answer}
\end{standalonequote}

\begin{standalonequote}{Skin Conditions}
    \metadata{topic={Topical Aspirin For Acne}, source={Email Wiki}}

    \begin{answer}
        A solution of aspirin in water on the skin helps with the inflammation, and is mildly germicidal.
    \end{answer}
\end{standalonequote}

\begin{qaexchange}{Skin Conditions}
    \metadata{topic={Contraceptives And Acne}, source={Email Wiki}}

    \begin{question}
        Why do many women get rid of acne using CONTRACEPTIVES (estrogen)?
    \end{question}

    \begin{answer}
        Estrogen causes the oil glands to atrophy, so the skin doesn't support bacterial growth so well. Topical sulfur's germicidal effect can help, and topical aspirin and caffeine are antiseptic as well as antiinflammatory. One function of vitamin A is to increase progesterone in the skin, and it has to be in balance with thyroid to do that. Another function is to differentiate the skin cells, reducing keratin plugging of the glands.
    \end{answer}
\end{qaexchange}

\begin{standalonequote}{Skin Conditions}
    \metadata{topic={Antibiotics For Acne}, source={Email Wiki}}

    \begin{answer}
        ANTIBIOTICS can be used topically, but the tetracycline type is usually taken internally, for their antiinflammatory effect. Changing the diet while using an antibiotic can make the effect permanent.
    \end{answer}
\end{standalonequote}

\begin{standalonequote}{Skin Conditions}
    \metadata{topic={Benzoyl Peroxide Safety}, source={Email Wiki}}

    \begin{answer}
        I suspect that it will age the skin. Have you tried topical sulfur and oral tetracycline?
    \end{answer}
\end{standalonequote}

\begin{qaexchange}{Skin Conditions}
    \metadata{topic={Sulfur Smell And Fungal Infection}, source={Email Wiki}}

    \begin{question}
        Does the precipitated sulfur smell like a match also?
    \end{question}

    \begin{answer}
        Yes, the smell varies slightly. When there's a skin fungal infection, there's a hydrogen sulfide stink, which is what kills the fungus.
    \end{answer}
\end{qaexchange}

\begin{standalonequote}{Skin Conditions}
    \metadata{topic={Zinc, Vitamin A, And Acne}, source={Email Wiki}}

    \begin{answer}
        ZINC isn't directly an oxidant, but when it's used as a chemical supplement it can cause problems that it wouldn't in the form of foods. Both vitamin A and zinc are essential in the right amount for good skin health, but too much of either can disturb the immune function. Irritation of the intestine is often involved in skin problems, and supplements always contain trace contaminants that can cause reactions. When ADM bought Distillation Products from Eastman several years ago, the composition of their vitatmin E products went through several changes, and competing companies began making similar changes. Since then research results haven't been as consistent as they were 40 to 60 years ago, and I stopped recommending amounts up to a few hundred units per day, waiting to see more results of research. Some of the products sold as vitamin E now contain significant amounts of PUFA, and lack some of the substances such as octacosanol that were in traditional products. The right amount of thyroid is essential for skin immunity and metabolism of steroids in the skin. Topical anti inflammatory things such as tetracycline and aspirin often produce the quickest response.
    \end{answer}
\end{standalonequote}

\begin{standalonequote}{Skin Conditions}
    \metadata{topic={Pigment Cell Mobility}, source={Email Wiki}}
    \begin{note}
        Freckles
    \end{note}

    \begin{answer}
        Yes, the pigment cells are very mobile--they can swim through solid tissue at a surprising speed, more than a centimeter per day, if the skin is warm, and if they are motivated. Sometimes Wikipedia is stupidly dogmatic.
    \end{answer}
\end{standalonequote}

\begin{standalonequote}{Skin Conditions}
    \metadata{topic={Eczema Treatment Approach}, source={Email Wiki}}

    \begin{answer}
        Is he getting enough calcium? Liver and thyroid would be better than trying to use separate vitamins---vitamin A deficiency is the most likely, but some B vitamins could be involved, and a vitamin A supplement can increase the need for thyroid hormone, which is increased anyway during the winter.
    \end{answer}
\end{standalonequote}

\begin{standalonequote}{Skin Conditions}
    \metadata{topic={Vitamin A And Dandruff}, source={Email Wiki}}

    \begin{answer}
        Vitamin A deficiency is a common cause of dandruff.
    \end{answer}
\end{standalonequote}

\begin{standalonequote}{Skin Conditions}
    \metadata{topic={Itching And Hypothyroidism}, source={Email Wiki}}

    \begin{answer}
         Starting with the nosebleeds, thyroid has probably been the basic problem, so checking your temperature and resting pulse rate (at least twice a day, at waking, and about an hour after breakfast) would be a place to start. Having someone check your Achilles tendon reflex relaxation rate is helpful; the relaxation should be instantaneous, so that your foot falls floppily. A daily carrot might take care of the itch, if not, 10\% sulfur soap, or dusting with flowers of sulfur, USP, should do it. Milk with sugar or honey (about an ounce in a glass) at bedtime would help to get to sleep, and reduce the night stress. Blood sugar falls at night, too low if your thyroid is deficient, and is compensated by the stress hormones. Your lowest temperature should be at night, but the stress hormones can cause your waking temperature to be higher than the midday temperature. Pulse rate should follow the same pattern, rising with breakfast, and staying above 80/minute during the day. Aspirin can help with increasing your metabolic rate. Mary Shomon's thyroid website and  have good information about thyroid. I use Cynoplus and Cynomel from , which are generally more economical and more consistent than some of the glandular products.
    \end{answer}
\end{standalonequote}

\begin{standalonequote}{Skin Conditions}
    \metadata{topic={Mole Regression With Hormones}, source={Email Wiki}}

    \begin{answer}
        I've always been very cautious about moles, since I think they have the potential to degenerate into metastatic cancer. Around 1978, I had been watching one on my belly, that had enlarged from an original nearly flat soft light brown mole, to a large, irregular, leathery black thing. A couple of doctors had urged me to have it removed. I happened to be experimenting with steroids, including DHEA, at the time. One night as I went to bed, I saw what looked like a maraschino cherry on my belly, with a black crumb on its top; the black thing brushed off, leaving a spot of blood on the red dome, and I realized that it was my mole. Over the next 3 days the red sphere gradually deflated, and what remained was the soft, flat light brown original mole. Every few years I have had suddenly emerging moles, of various sizes and colors. Each time I would apply some progesterone or DHEA dissolved in vitamin E to the surrounding skin. If I applied it to just one side, there would be an emigration of cells on the other side, like a moving shadow of the mole, and the mole would lose volume and become lighter in color. When I had been in Florida and stopped using thyroid because of the heat and humidity, a 2 centimeter diameter mole (jumbo black olive-like) grew in front of my ear during 2 or 3 weeks. A soon as I returned to Oregon I started using thyroid, and the mole immediately began shrinking and fading. About two weeks later, the pale remnant on a dry stalk fell off, without leaving a scar. Since then I have generally recommended just becoming slightly hyperthyroid, if a person is generally in good health with enough cholesterol for conversion to the hormones), and other people have had similar experiences with very quickly shrinking moles.
    \end{answer}
\end{standalonequote}

\begin{qaexchange}{Skin Conditions}
    \metadata{topic={Skin Infection Treatments}, source={Email Wiki}}

    \begin{question}
        Povidone iodine for recurring skin infection? High blood pressure
    \end{question}

    \begin{answer}
        It isn't good to repeatedly cover a very large skin area with iodine, but it's safe to use on small areas. 10\% sulfur soap is another safe disinfectant that works on various types of infection. There are some nutritional deficiencies that can cause recurring infections. Vitamin D deficiencies are very common, and predispose to all sorts of immunity problems. Hypothyroidism and vitamin A deficiencies sometimes lead to prolonged infections, and increased TSH is very closely involved with high blood pressure.
    \end{answer}
\end{qaexchange}

\begin{standalonequote}{Skin Conditions}
    \metadata{topic={Vitamin Needs With Metabolism}, source={Email Wiki}}

    \begin{answer}
        Vitamins and trace minerals have to increase proportionally as the metabolic rate increases.
    \end{answer}
\end{standalonequote}

\begin{standalonequote}{Skin Conditions}
    \metadata{topic={Vitiligo Topical Treatments}, source={Email Wiki}}

    \begin{answer}
        I used a dilute copper solution to restore pigment to my eyebrows and whiskers, but when the solution was too concentrated it produced, within a few hours, a raised pigmented area, so I stopped using it topically. Vitamin D and progesterone favor the survival of pigment cells. I think a topical solution of aspirin and caffeine might be helpful, and is safe.
    \end{answer}
\end{standalonequote}

\begin{standalonequote}{Skin Conditions}
    \metadata{topic={Wound Healing Topicals}, source={Email Wiki}}

    \begin{answer}
        Topical baking soda, honey, and granulated sugar can be helpful for wounds.
    \end{answer}
\end{standalonequote}

\begin{qaexchange}{Skin Conditions}
    \metadata{topic={Vitamin E For Scars}, source={Email Wiki}}

    \begin{question}
        What is the mechanism behind the effect of silicone-sheets for reducing hypertrophic scars?
    \end{question}

    \begin{answer}
        I had a good experience with reducing a scar by coating it with viscous vitamin E. I think trapping \ce{CO2} is probably the main effect of the silicone.
    \end{answer}
\end{qaexchange}

\begin{qaexchange}{Skin}
    \metadata{topic={Vitamin A for Acne/Dandruff}, source={Ray Peat Forum}}

    \begin{question}
        If I'm vitamin A deficient enough to get dandruff and acne, could that cause anxiety too?
    \end{question}

    \begin{answer}
        Since it's needed to make pregnenolone and progesterone, I think it could.
    \end{answer}
\end{qaexchange}

\begin{standalonequote}{Skin Conditions}
    \metadata{topic={Rash Treatment Options}, source={Ray Peat Forum}}

    \begin{answer}
      Topical antibiotics might be more effective than oral, and there are now products available containing bacteriophage, that kill some bacteria that are resistant to chemical antibiotics. A concentrated solution of baking soda sometimes helps a rash of unknown cause.
    \end{answer}
\end{standalonequote}

\begin{qaexchange}{Skin Conditions}
    \metadata{topic={Stretch Marks Prevention}, source={Ray Peat Forum}}

    \begin{question}
        Do you know of any topical substance, that can get rid of stretch marks, and or keep them from forming?
    \end{question}

    \begin{answer}
      I had a cousin with vivid stretch marks on her pregnant belly, and after just a couple of days of adding eggs and oysters to her diet they were completely gone.
    \end{answer}
\end{qaexchange}

\begin{standalonequote}{Skin Conditions}
    \metadata{topic={Mole Shrinking Methods}, source={Ray Peat Forum}}

    \begin{answer}
      I've never applied anything directly to a mole, because of the danger of causing irritation that could cause cancerization. When I have put progesterone or DHEA (dissolved in vitamin E) on the skin about an inch away from a mole, I have seen changes the next day, with steady shrinking during a week or 10 days. Sometimes I have just increased my dose of T\textsubscript{3}, to keep my daytime temperature at least normal, and it increase my resting pulse rate to about 95 beats per minute. I think I have accelerated the dissappearance of some moles by soaking in a bath containing epsom salts, baking soda, and salt.
    \end{answer}
\end{standalonequote}

\begin{standalonequote}{Skin Conditions}
    \metadata{topic={Dandruff Causes}, source={Ray Peat Forum}}

    \begin{answer}
      Usually it's from a slight nutritional deficiency or imbalance. Vitamin D, the ratio of calcium to phosphate in your diet, and thyroid function are some of the things that affect it.
    \end{answer}
\end{standalonequote}

\begin{standalonequote}{Skin Conditions}
    \metadata{topic={Skin Pigmentation, Thyroid}, source={Ray Peat Forum}}

    \begin{answer}
      It's pigmentation that I've seen go away completely when people used thyroid or pregnenolone or (women) progesterone. It's usually thought to be related to allergy and adrenal deficiency.
    \end{answer}
\end{standalonequote}

\subsection{Hair}

\begin{standalonequote}{Hair}
    \metadata{topic={Hair And Thyroid}, source={Email Wiki}}

    \begin{answer}
        Thyroid makes the hair strong, a high metabolic rate can create a static field that helps it to stand up.
    \end{answer}
\end{standalonequote}

\begin{standalonequote}{Hair}
    \metadata{topic={Hair Loss}, source={Email Wiki}}
    \begin{note}
        Low Libido, Hair Loss in Young Male
    \end{note}

    \begin{answer}
        If you are getting enough of the major nutrients, including protein, calcium, and sugar, it's possible that you have a specific stress-related deficiency, for example of B\textsubscript{6}, niacinamide, or selenium. 10 mg of B\textsubscript{6} can sometimes make a quick difference in prostate and libido, 100 mg of niacinamide can reduce some stress symptoms. Applying caffeine solution to the scalp locally helps to promote hair growth. Water and a little alcohol are convenient for applying it.
    \end{answer}
\end{standalonequote}

\begin{standalonequote}{Hair}
    \metadata{topic={Progesterone Straightening Hair}, source={Email Wiki}}

    \begin{answer}
        Fast, vigorous hair growth tends to make it straighter (a rounder shaft).
    \end{answer}
\end{standalonequote}

\begin{standalonequote}{Hair}
    \metadata{topic={Topical Aspirin For Hair}, source={Email Wiki}}

    \begin{answer}
        Yes, topical aspirin and caffeine stimulate hair growth.
    \end{answer}
\end{standalonequote}

\begin{standalonequote}{Hair}
    \metadata{topic={Copper For Hair Pigment}, source={Email Wiki}}

    \begin{answer}
        Do you know how your thyroid function is? Thyroid regulates copper assimilation, and also the hormones that regulate pigment. I found that applying a weak solution of copper just once would restore color immediately to eyebrows, or to about 10\% of sideburn hairs, apparently because the very long-lived hairs have to be in the right phase of growth, and eyebrows, with a very short life, seem to stay receptive to the stimulation. But I also found that a slightly too strong solution could cause a mole to develop almost instantly, with an invasion of pigment cells. I think a safer alternative would be to supplement, either topically or orally, a little DHEA.
    \end{answer}
\end{standalonequote}

\begin{qaexchange}{Hair}
    \metadata{topic={Topical Aspirin Hair Effects}, source={Email Wiki}}

    \begin{question}
        Is topical aspirin useful for healthier hair? Does it cause water retention?
    \end{question}

    \begin{answer}
        It doesn't cause water retention, and since prostaglandins are involved in atrophy of hair follicles, inhibiting prostaglandins locally could help.
    \end{answer}
\end{qaexchange}

\begin{standalonequote}{Hair}
    \metadata{topic={Topical Thyroid For Baldness}, source={Email Wiki}}
    \begin{note}
        Topical Thyroid (NDT) in a Solution of DMSO and Ethanol for Male Pattern Baldness
    \end{note}

    \begin{answer}
        Desiccated thyroid gland doesn't contain any free hormone; the gland contains thyroglobulin, a protein, which when digested releases the hormones.
    \end{answer}
\end{standalonequote}

\begin{standalonequote}{Hair}
    \metadata{topic={Alopecia Areata Treatment}, source={Ray Peat Forum}}

    \begin{answer}
      Hypothyroidism is often a factor, and I think the \enquote{autoimmune} process is promoted by endotoxin and nitric oxide from bacterial overgrowth in the small intestine; a pregnenolone supplement can improve response to thyroid supplements. Well cooked mushrooms and bamboo shoots have antiinflammatory, antiseptic effects that can reduce inflammation and might reduce the stress-induced fat deposition. If you don't get regular sun exposure, vitamin D supplements might help.
    \end{answer}
\end{standalonequote}

\begin{qaexchange}{Hair}
    \metadata{topic={Diffuse Hair Thinning Causes}, source={Ray Peat Forum}}

    \begin{question}
        Do you have any opinions on ideas as the cause of diffuse thinning hair in a young male?
    \end{question}

    \begin{answer}
       Too much phosphate, too little vitamin D, are common causes. 
    \end{answer}
\end{qaexchange}

\begin{qaexchange}{Hair}
    \metadata{topic={Male Pattern Baldness Causes}, source={Ray Peat Forum}}

    \begin{question}
        I've often wondered if hypothyroidism is the main underlying cause of male pattern baldness, and women tend to be more hypothyroid than men, why is hair loss more prevalent in males? Do you have any thoughts?
    \end{question}

    \begin{answer}
      Progesterone is highly protective for the skin and follicles, preventing the formation of harmful prostaglandins, for example.
    \end{answer}
\end{qaexchange}

\begin{standalonequote}{Hair}
    \metadata{topic={Milk Preventing Baldness}, source={Ray Peat Forum}}

    \begin{answer}
      Skull growth is matched by skin growth and formation of new blood vessels. Increased fibrosis is associated with higher estrogen, lower testosterone, and there is a strong increasing trend in recent decades for both of those, and they coincide with a reduction of milk consumption during that time. The incidence of hyperparathyroidism has increased during this time, and that hormone causes hair loss. Calcium and vitamin D help to lower parathyroid hormone. I think milk helps to prevent baldness, as well as obesity, colon cancer, and dementia.
    \end{answer}
\end{standalonequote}

\begin{standalonequote}{Hair}
    \metadata{topic={Male Pattern Baldness, Metabolism}, source={Ray Peat Forum}}

    \begin{answer}
      My first step would be to thoroughly investigate TSH, temperature, vitamin D, and calcium. No, everything is always changing, reflecting your whole situation. Developing baldness is a warning sign of basic metabolic problems, tending toward general circulatory disease.
    \end{answer}
\end{standalonequote}

\begin{standalonequote}{Hair}
    \metadata{topic={Hair Graying from Radiation}, source={Ray Peat Forum}}

    \begin{answer}
      Anti-inflammatory things, like vitamin D, milk, coffee, orange juice, checking your thyroid function, maybe topically supplementing a little DHEA and progesterone.
    \end{answer}
\end{standalonequote}

\subsection{Nails}

\begin{standalonequote}{Nails}
    \metadata{topic={Fingernail Lunala, Thyroid}, source={Ray Peat Forum}}

    \begin{answer}
      I always had them before I took thyroid, have never had one since then. I don't know what they mean.
    \end{answer}
\end{standalonequote}

\section{Respiratory System}

\begin{standalonequote}{Respiratory System}
    \metadata{topic={Recurring Respiratory Infections}, source={Email Wiki}}

    \begin{answer}
        The liver, intestine, and lungs interact very closely, and supporting the liver with nourishment and adequate thyroid, while avoiding irritating foods such as salads, beans, and allergens, will usually prevent recurring respiratory problems.
    \end{answer}
\end{standalonequote}

\begin{qaexchange}{Respiratory}
    \metadata{topic={Breath Holding}, source={Ray Peat Forum}}

    \begin{question}
        From an exercise standpoint, does holding your breathe after a full exhalation differ in short or long term benefit compared to holding your breathe with a full inhalation?
    \end{question}

    \begin{answer}
        The signal to breathe depends on the blood gas concentration; the urge comes earlier after the exhalation.
    \end{answer}
\end{qaexchange}

\section{Urinary System}

\begin{standalonequote}{Urinary System}
    \metadata{topic={UTI And Hypothyroidism}, source={Email Wiki}}

    \begin{answer}
        Slight hypothyroidism is a very common cause of chronic urinary infections. Both thyroid hormone and progesterone increase the (IgA) antibody production on membranes, improving resistance to infection, and they reduce the resistance to histamine, which tends to increase in the bladder under an excess of estrogen. Checking temperature and pulse rate in the morning and middle of the day is helpful as a first way to check for possible hypothyroidism.
    \end{answer}
\end{standalonequote}

\begin{standalonequote}{Urinary}
    \metadata{topic={Incontinence}, source={Ray Peat Forum}}

    \begin{answer}
        Little kids wet the bed because during the night, in sound sleep, the parasympathetic system tends to become dominant, it is one of the things that causes the tendency of the blood sugar to fall at night. Estrogen increases the tension in the wall of the bladder, and tends to activate the emptying reflex. Katharina Dalton has discussed the actions of estrogen and progesterone on the bladder; progesterone relaxes the bladder so that it comfortably retains more urine, and it quiets the emptying reflex. Progesterone sustains blood sugar during the night, and in several ways prevents imbalances in the parasympathetic and sympathetic nervous system.
    \end{answer}
\end{standalonequote}

\begin{standalonequote}{Urinary}
    \metadata{topic={Cyproheptadine Urine Color}, source={Ray Peat Forum}}

    \begin{answer}
        Large amounts of cyproheptadine might affect the gallbladder or ducts in a way that could increase bilirubin, making the urine brown. I have found that less than one milligram of it can be very helpful; I think it should be taken for only a few days at a time. Thyroid, progesterone, and maybe small amounts of DHEA or testosterone (around 2 to 4 mg/day) should help to reduce inflammation, improve immunity and normalize the ureters.
    \end{answer}
\end{standalonequote}

\begin{qaexchange}{Urinary System}
    \metadata{topic={Kidney Stones Causes}, source={Ray Peat Forum}}

    \begin{question}
        I haven't experienced this but I was wondering if you know what causes kidney stones? A lot of doctors say it's from eating too much animal protein and calcium. Do you think drinking a gallon of milk everyday chronically would give someone kidney stones?
    \end{question}

    \begin{answer}
      High phosphate and low vitamin D are probably the main things. Meat, beans, nuts, and grains have a very high ratio of phosphate to calcium. I think milk is protective, with its good ratio of calcium to phosphate.
    \end{answer}
\end{qaexchange}

\section{Reproductive System}

\begin{standalonequote}{Reproductive System}
    \metadata{topic={Circumcision Effects}, source={Email Wiki}}

    \begin{answer}
        It's effects are almost exclusively negative, except when the foreskin is extremely constricted.
    \end{answer}
\end{standalonequote}

\begin{standalonequote}{Reproductive System}
    \metadata{topic={Mastitis Treatment}, source={Email Wiki}}
    \begin{note}
        Rapid onset of Mastitis in a Middle-Aged, Non-Lactating Woman with a History of Prolactinoma, Substances Used for Relief Were Aspirin, Bromocriptine and Topical Lidocaine Gel.
    \end{note}

    \begin{answer}
        Aspirin, lidocaine, and bromocriptine are all likely to help, but low thyroid is usually behind an excess of prolactin; in middle age, estrogen tends to rise as progesterone falls. A good T\textsubscript{3} supplement is usually the quickest way to correct breast inflammation and pain. Have you checked your temperature and pulse rate? A sluggish intestine interferes with the excretion of estrogen, so raw carrots or a laxative can often, in just a day or two, increase the ratio of progesterone to estrogen. Extra salt in your food, and a little vitamin B\textsubscript{6} could help to lower the prolactin. Low thyroid increases water retention but causes sodium loss, and that combination increases swelling and inflammation; the diuretic effect of tea or coffee might help with the swelling.
    \end{answer}
\end{standalonequote}

\begin{standalonequote}{Reproductive System}
    \metadata{topic={Prostate Treatment Protocol}, source={Email Wiki}}

    \begin{answer}
        Correcting hypothyroidism will usually reduce prostate problems, and often pregnenolone helps with that as well as with increasing testosterone. Checking temperature and pulse rate at waking and in the middle of the day, and checking the Achilles tendon reflex relaxation rate, can help to judge the hormonal situation. Having a carrot salad every day (shredded carrot, with a little olive oil, vinegar, and salt) can help to lower the stress hormones that are usually associated with prostate inflammation.
    \end{answer}
\end{standalonequote}

\subsection{Menstrual Cycle}

\begin{standalonequote}{Menstrual Cycle}
    \metadata{topic={Menstrual Cramps And T\textsubscript{3}}, source={Email Wiki}}

    \begin{answer}
        Have you seen effects on your temperature and pulse rate from the Cynomel? The need for T\textsubscript{3} increases premenstrually, and is probably greater with diabetes.
    \end{answer}
\end{standalonequote}

\begin{standalonequote}{Menstrual Cycle}
    \metadata{topic={PMS And Thyroid}, source={Email Wiki}}

    \begin{answer}
        Premenstrual stress suggests that the thyroid function is low, at least during that time. Do you eat liver and shell fish occasionally? The trace nutrients in those sometimes make a difference.
    \end{answer}
\end{standalonequote}

\begin{standalonequote}{Reproductive}
    \metadata{topic={Gestation Past 42 Weeks}, source={Ray Peat Forum}}

    \begin{answer}
        I've known people who gestated for longer than 42 weeks who were very healthy. I suspect that if the body temperature is a little below normal development takes a little longer, but usually the heavier babies with longer gestation have superior brains. I think it's good to check thyroid, the ratio of progesterone to estrogen, vitamin D, and body temperature.
    \end{answer}
\end{standalonequote}

\begin{standalonequote}{Reproductive System}
    \metadata{topic={Contraceptive Safety, Cervical Cap}, source={Ray Peat Forum}}

    \begin{answer}
      Some condoms are coated with silicone lubricant, and could cause allergic or immune problems. A fitted cervical cap, as described by Barbara Seaman, is probably the ideal. Some women have had success with a plastic diaphragm coated with progesterone.
    \end{answer}
\end{standalonequote}

\begin{qaexchange}{Reproductive System}
    \metadata{topic={Vasectomy Hormone Effects}, source={Ray Peat Forum}}

    \begin{question}
        Do you think there are any negative effects from a male vasectomy?
    \end{question}

    \begin{answer}
      There is evidence that it can cause hormone imbalances and long term harm.
    \end{answer}
\end{qaexchange}

\begin{standalonequote}{Reproductive System}
    \metadata{topic={Prostate Swelling Treatment}, source={Ray Peat Forum}}

    \begin{answer}
      Inflammation in the intestine can raise systemic or regional histamine and serotinin enough to cause prostate swelling. Avoiding starchy vegetables, might help; supplementing vitamin D and thyroid can reduce inflammation. An antihistamine (diphenhydrame or cyproheptadine) and aspirin can reduce prostate swelling, and might help the intestine too.      
    \end{answer}
\end{standalonequote}

\begin{standalonequote}{Reproductive System}
    \metadata{topic={Varicocele Causes}, source={Ray Peat Forum}}

    \begin{answer}
      I think they develop because of an excess of cortisol relative to DHEA and pregnenolone. That imbalance is likely to occur with low protein intake, low thyroid, and low vitamins D and A.
    \end{answer}
\end{standalonequote}

\begin{standalonequote}{Reproductive System}
    \metadata{topic={Vulvodynia Treatment}, source={Ray Peat Forum}}

    \begin{answer}
      Endometriosis is clearly the result of too much estrogen, and there are good reasons for thinking a similar imbalance is involved in vulvodynia. Estrogen increases the presence of mast cells, which are found increased in the painful area. Mast cells secrete a variety of proinflammatory and pain-inducing substances including histamine, serotonin, and renin, which is converted locally to angiotensin. Estrogen supports formation of all those. Low vitamin D, low calcium intake relative to phosphate, and low thyroid function contribute to the excess of estrogen relative to progesterone. Vitamin D, milk, cyproheptadine (antagonist to histamine and serotonin), thyroid, and angiotensin blockers (such as telmisartan or candesartan) should help to correct the condition.
    \end{answer}
\end{standalonequote}

\begin{standalonequote}{Reproductive System}
    \metadata{topic={Ovarian Cyst Treatment}, source={Ray Peat Forum}}

    \begin{answer}
		Everyone that I have known with ovarian cysts got them to break by taking a large dose of progesterone. If you have tried that without effect, than maybe the surgery is necessary. Naloxone is another thing that sometimes corrects ovarian cysts.

		Most often it was a single dose of about 100 mg.

		I've heard from women who didn't have shrinkage with progesterone but did with prolonged high thyroid supplementation. Fluid filled capsules anywhere in the body respond to hypothyroidism by swelling, and they shrink in the presence of increased thyroid hormone.
    \end{answer}
\end{standalonequote}

\begin{standalonequote}{Reproductive System}
    \metadata{topic={Prostate Health, Prostatectomy}, source={Ray Peat Forum}}

    \begin{answer}
		The language \enquote{intermediate probability for malignancy} and equivocal presence of clinically significant cancer, means that they haven't understood the concept of \enquote{watchful waiting} that was responsible for reducing mortality from prostate cancer in the late 1990s.

		What they prescribe tamsulosin for is probably usually just weakness of the bladder contraction muscle, the detrusor, that's common with overweight and low testosterone. His TSH is likely to be high, a supplement of thyroid and vitamin D would probably help.
    \end{answer}
\end{standalonequote}

\chapter{Common Health Conditions}

\section{Metabolic Disorders}
\subsection{Diabetes}

\begin{standalonequote}{Diabetes}
    \metadata{topic={Bruising In Diabetes}, source={Email Wiki}}

    \begin{answer}
        Diabetes usually causes some interference with the formation of the active thyroid hormone, leading to increased cortisol relative to progesterone and DHEA. Those vitamins, and pregnenolone and progesterone help to strengthen capillaries.
    \end{answer}
\end{standalonequote}

\begin{standalonequote}{Diabetes}
    \metadata{topic={Diabetes And PUFA Mechanisms}, source={Email Wiki}}

    \begin{answer}
        Diabetics typically have elevated lactate, which shows that glucose doesn't have a problem getting into their cells, just getting oxidized. Sugars, if they are consumed in quantities beyond the ability to metabolize them (and that easily happens in the presence of PUFA) are converted into saturated fatty acids, which have antistress, antiinflammatory effects. Many propaganda experiments are set up, feeding a grossly excessive amount of polyunsaturated fat, causing sugar to form fat, specifically so they can publish their silly diet recommendations, which supposedly explain the obesity epidemic, but the government figures I cited show that vegetable fat consumption has increased, sugar hasn't. My articles have a lot of information on the mechanisms, such as the so-called 'Randle cycle,' in which fatty acids shut down the ability to oxidize sugar. Polyunsaturated fats do many things that increase blood sugar inappropriately, and my articles review several of the major mechanisms.Several years ago, medical people started talking about the harmful effects of insulin, such as stimulating fat production, so 'insulin resistance' which keeps a high level of insulin from producing obesity would seem to be a good thing, but the medical obesity culture really isn't thinking very straight. One factor in the 'insulin resistance' created by PUFA involves estrogen---chronic accumulation of PUFA in the tissues increases the production of estrogen, and the polyunsaturated free fatty acids intensify the actions of estrogen, which acts in several ways to interfere with glucose oxidation.
    \end{answer}
\end{standalonequote}

\begin{standalonequote}{Diabetes}
    \metadata{topic={T\textsubscript{3} Improving Glucose Oxidation}, source={Email Wiki}}

    \begin{answer}
        The T\textsubscript{3} component of the thyroid hormone makes muscles and other tissues oxidize sugar. Calcium, sodium, and aspirin are other things that increase the ability to use glucose.
    \end{answer}
\end{standalonequote}

\begin{standalonequote}{Diabetes}
    \metadata{topic={Diabetes And Thyroid Supplementation}, source={Email Wiki}}
    \begin{note}
        Adrenaline Surge from Thyroid Supplementation
    \end{note}

    \begin{answer}
        Free fatty acids are usually high in diabetes, and interfere with glucose use. Aspirin and niacinamide help to lower stress increased fatty acids, so allow exercised muscles to use more glucose. 2 or 3 mcg of cynomel with some food should help to avoid the adrenaline. Vitamins D and K, and salt and calcium, are other things that improve glucose use. Since the liver needs glucose and glycogen to convert T\textsubscript{4} into T\textsubscript{3}, diabetes usually interferes with the conversion. A blood test could show if the ratio of T\textsubscript{4} to T\textsubscript{3} is very high.
    \end{answer}
\end{standalonequote}

\begin{standalonequote}{Diabetes}
    \metadata{topic={Type 1 Diabetes Reversibility}, source={Email Wiki}}

    \begin{answer}
        I have known people who believed they had insulin deficiency, who recovered completely. The pancreas beta cells can regenerate quickly, polyunsaturated fats are continually damaging them.
    \end{answer}
\end{standalonequote}

\begin{standalonequote}{Diabetes}
    \metadata{topic={Pancreas Beta Cell Regeneration}, source={Email Wiki}}

    \begin{answer}
        Pregnenolone does convert to either DHEA or progesterone. Sugar and brewer's yeast are other things that help with the regeneration. Keeping free fatty acids low is important, and niacinamide could help with that.
    \end{answer}
\end{standalonequote}

\begin{standalonequote}{Diabetes}
    \metadata{topic={Type 1 Diabetes Dietary Approach}, source={Email Wiki}}

    \begin{answer}
        There are several articles relating to diabetes on my website. Polyunsaturated fats damage the pancreas and increase stress hormones, while glucose stimulates the renewal of insulin-secreting cells. Fruits provide minerals that help to regulate glucose metaboiism, and help to regulate thyroid function.
    \end{answer}
\end{standalonequote}

\begin{standalonequote}{Diabetes}
    \metadata{topic={Type 1 Diabetes Supplements}, source={Email Wiki}}
    \begin{note}
        Blood Sugar Control
    \end{note}

    \begin{answer}
        Vitamin B\textsubscript{1} helps to oxidize glucose, so if you try 50 or 100 mg with a meal you should watch for possible hypoglycemia from the insulin. Pantothenic acid is safe in doses of 100 or 200 mg, and helps to limit hypoglycemia. Brewers' yeast has other nutrients that help with repairing the pancreas, but can cause gas, so it's best to start by pouring hot water over an ounce or two of it, and using just the liquid.
    \end{answer}
\end{standalonequote}

\begin{standalonequote}{Diabetes}
    \metadata{topic={Type 1 Diabetes Management}, source={Email Wiki}}

    \begin{answer}
        I think it's valuable to have a blood test for vitamin D to regulate the dose; TSH can be useful, too, but it's very important to check your temperature and pulse rate regularly to judge the effects of a thyroid supplement, since the need for it varies with season and type of activity. Has your cortisol been checked occasionally? I think it's common for kids to be diagnosed as diabetic when they have high blood sugar following a sickness such as flu; insulin treatment can institutionalize an over-production of the stress hormones. Inflammation of the intestine (which can start with an infection) can be sustained by undigested starches, and the resulting endotoxin/nitric oxide/serotonin can cause insulin resistance, so it's important to keep the small intestine relatively germ-free. Melons and potatoes can feed bacteria if they are present. Adequate calcium is extremely important, because of the interactions of parathyroid hormone and serotonin with stress and glucose metabolism.
    \end{answer}
\end{standalonequote}

\begin{standalonequote}{Diabetes}
    \metadata{topic={Type 1 Diabetes Protein Needs}, source={Email Wiki}}

    \begin{answer}
        Daily protein should be at least 80 grams, and fruit should provide a large part of the calories. A little vitamin B\textsubscript{6} (10 mg) can help with amino acid metabolism. 
    \end{answer}
\end{standalonequote}

\begin{standalonequote}{Diabetes}
    \metadata{topic={Thyroid T\textsubscript{4}/T\textsubscript{3} In Diabetes}, source={Email Wiki}}

    \begin{answer}
        The cynoplus tablets can be divided into pieces so that each dose of T\textsubscript{3} is similar to the amount you were taking in cytomel. In diabetes, when cells aren't getting enough glucose, T\textsubscript{4} can't be converted to the active T\textsubscript{3}, and so it can build up in the body to levels that interfere with metabolism, but the advantage of a combination is that T\textsubscript{4} inhibits TSH, and TSH is responsible for many of the symptoms.
    \end{answer}
\end{standalonequote}

\begin{qaexchange}{Diabetes}
    \metadata{topic={Type I Diabetes Diet Composition}, source={Ray Peat Forum}}

    \begin{question}
         If you had T1 diabetes, if you would follow a higher sugar or higher fat diet?
    \end{question}

    \begin{answer}
      Because of the harmful effect of PUFA, I think it's good to keep all fat intake somewhat low, because even butter and coconut oil contain about 2\% PUFA. Fruits and vegetables have sugar in a good balance with the minerals needed to metabolize it.
    \end{answer}
\end{qaexchange}

\begin{qaexchange}{Diabetes}
    \metadata{topic={Pancreatic Beta Cell Regeneration}, source={Ray Peat Forum}}

    \begin{question}
        Once pancreatic cells of the pancreas have been destroyed or are continually being destroyed (autoimmune diabetes) is there any hope for regeneration?
    \end{question}

    \begin{answer}
      Glucose stimulates regeneration, and pregnenolone, progesterone, and DHEA help to keep them alive. Thyroid and good nutrition gradually help to detoxify the stored fats that are responsible for killing them. Aspirin and niacinamide help in different ways.
    \end{answer}
\end{qaexchange}

\begin{standalonequote}{Diabetes}
    \metadata{topic={Aspirin, Thyroid for Blood Sugar}, source={Ray Peat Forum}}

    \begin{answer}
      Aspirin helps to increase the oxidation of glucose, reinforcing the effects of thyroid hormone. It can antagonize vitamin K, potentially increasing a bleeding tendency, but with a supplement of K it's safe to test the effects of a standard 5 grain tablet with a meal twice a day, to see if it helps to prevent hyperglycemia. The cynoplus should start with a sixth of a tablet per day, watching for effects during the first two weeks. Too much could raise blood sugar rather than regulating it; I hope no one is suggesting beginning with a whole tablet per day. The appropriate amount of progesterone depends on the need and time of month. Vitamin B\textsubscript{1} helps to oxidize glucose, so its use should be adjusted according to need.
    \end{answer}
\end{standalonequote}

\begin{standalonequote}{Diabetes}
    \metadata{topic={PUFA Pancreas Damage, Glucose Benefits}, source={Ray Peat Forum}}

    \begin{answer}
      Polyunsaturated fats damage the pancreas and increase stress hormones, while glucose stimulates the renewal of insulin-secreting cells. Fruits provide minerals that help to regulate glucose metabolism, and help to regulate thyroid function.
    \end{answer}
\end{standalonequote}

\begin{standalonequote}{Diabetes}
    \metadata{topic={Type I Diabetes Management}, source={Ray Peat Forum}}

    \begin{answer}
      Starches and polyunsaturated fats keep stressing beta cells as they regenerate. Endotoxin and nitric oxide cause insulin resistance, besides being toxic to the beta cells, so it's essential to keep the small intestine relatively free of bacteria. A daily raw carrot salad is helpful; well cooked mushrooms every day can help in a veriety of ways.
    \end{answer}
\end{standalonequote}

\subsection{Obesity \& Weight Management}

\begin{standalonequote}{Obesity \& Weight Management}
    \metadata{topic={Cellulite Reduction}, source={Email Wiki}}

    \begin{answer}
        Building muscle with an anabolic diet, and the right kind of activity, causes a hormonal shift.
    \end{answer}
\end{standalonequote}

\begin{standalonequote}{Obesity \& Weight Management}
    \metadata{topic={Weight Gain Difficulty Causes}, source={Email Wiki}}

    \begin{answer}
        I had similar symptoms, I often ate several thousand calories per day without getting fat, and small noises would shock me awake. Taking thyroid reduced my caloric requirement, and immediately allowed me to sleep deeply. Deficiencies of magnesium, vitamin A, and selenium probably contribute to that metabolic pattern.
    \end{answer}
\end{standalonequote}

\begin{standalonequote}{Obesity \& Weight Management}
    \metadata{topic={Hypothyroid Weight Gain}, source={Email Wiki}}
    \begin{note}
        Unintended Weight Gain and Feeling Swollen
    \end{note}

    \begin{answer}
        That sounds like standard hypothyroid symptoms, the body simply adjusts to holding more water, while losing sodium quickly. An increased intake of calcium is the single most important nutritional thing for losing weight. Low fat milk and cheese should be the main foods; a carrot salad helps to reduce stress hormones. Temperature and heart rate are useful indicators for judging the amount of a thyroid supplement to use.
    \end{answer}
\end{standalonequote}

\begin{standalonequote}{Obesity \& Weight Management}
    \metadata{topic={Weight Loss Metabolic Approach}, source={Email Wiki}}

    \begin{answer}
        My recommendation is to eat to increase the metabolic rate (usually temperature and heart rate), rather than any particular foods. Usually the increased metabolic rate, with adequate protein, causes some muscle increase, and when that happens the basic calorie requirement will increase. The increase of muscle mass should continue for several weeks, and during that time the weight might increase a little, but usually the loss of water and fat will compensate for the greater muscle mass. I have heard from several people that they think I recommend drinking whole milk, which I don't, because the amount of fat in whole milk is very likely to be fattening when a person is using it to get the needed protein and calcium. When a person wants to lose excess fat, limiting the diet to low fat milk, eggs, orange juice, and a daily carrot or two, will provide the essential nutrients without excess calories.
    \end{answer}
\end{standalonequote}

\begin{standalonequote}{Obesity \& Weight Management}
    \metadata{topic={Muscle Gain During Weight Loss}, source={Email Wiki}}

    \begin{answer}
        There are different kinds of weight gain. When a person's metabolic rate increases, and stress hormones decrease, for example when adding two quarts of milk to the daily diet, their muscle mass is likely to increase, even while their fat is decreasing. Since muscle burns fat faster than fat does, caloric requirements will gradually increase.
    \end{answer}
\end{standalonequote}

\begin{standalonequote}{Obesity \& Weight Management}
    \metadata{topic={Weight Set Point Theory}, source={Email Wiki}}

    \begin{answer}
        I think habituation to a certain environment and way of living is another way of saying it.
    \end{answer}
\end{standalonequote}

\begin{standalonequote}{Obesity \& Weight Management}
    \metadata{topic={Coconut Oil Satiety, Weight Management}, source={Ray Peat Forum}}

    \begin{answer}
      Coconut oil (I prefer hydrogenated) helps satiety, while tending to increase the metabolic rate. Cheese can be especially satisfying, partly because of the flavor. Sometimes trace nutrient deficiencies lead to over-eating; liver, oysters, and cooked mushrooms can satisfy appetite without high calories. Sometimes hypothyroidism, a tendency to hypoglycemia, is responsible for weight gain.
    \end{answer}
\end{standalonequote}

\begin{standalonequote}{Obesity \& Weight Management}
    \metadata{topic={Body Fat Percentage Without PUFA}, source={Ray Peat Forum}}

    \begin{answer}
       With a diet lacking polyunsaturate fat I think the fat percentage would stabilize between 20 and 30\%, and it wouldn't be a chronic source of estrogen. 
    \end{answer}
\end{standalonequote}

\subsection{Metabolic Syndrome}

\begin{emailexchange}{Metabolic Syndrome}
    \metadata{topic={NAFLD, High Fat Diet}, source={Ray Peat Forum}}

    \begin{question}
         Do you think any kind of overfeeding, especially a high fat diet, would lead to NAFLD and NASH?
    \end{question}

    \begin{answer}
      Yes.
    \end{answer}

    \begin{question}
        What about all of the studies that feed rats an \enquote{atherogenic diet} full of saturated fats to induce NAFLD. Isn't that an inefficient mechanism? As PUFA would lead to a faster fattening of the liver with the same amount of calories fed.
    \end{question}

    \begin{answer}
      Yes, PUFA is probably essential for atherogenesis. In the experimental diets, they include several percent PUFA, so with overfeeding it accumulates quickly.
    \end{answer}

    \begin{question}
        Wouldn't that be a benefit of a calorie restrictive diet, that the PUFA will get burned instead of stored?
    \end{question}

    \begin{answer}
      If it's restricted to the right amount, without stress.
    \end{answer}
\end{emailexchange}

\section{Inflammatory Conditions}

\begin{standalonequote}{Inflammatory Conditions}
    \metadata{topic={Tonsil Stones, Hypothyroidism}, source={Ray Peat Forum}}

    \begin{answer}
      Those are white blood cells, usually responding to allergens. I think slight hypothyroidism, possibly low vitamin D, predisposes to them
    \end{answer}
\end{standalonequote}

\subsection{Chronic Inflammation}

\begin{standalonequote}{Chronic Inflammation}
    \metadata{topic={Interleukin-18 And Inflammation}, source={Email Wiki}}

    \begin{answer}
        Although I think knowing your PTH and free fatty acids will be useful (in judging use of calcium, sugar, aspirin, niacinamide, etc.), another test that could help to clarify the nature of the inflammation would be the serum interleukin-18, since it's associated with liver damage and increased ferritin, and symptoms of inflammation. Since TSH increases IL-18, finding it elevated would be another argument for keeping your TSH very low.
    \end{answer}
\end{standalonequote}

\begin{qaexchange}{Chronic Inflammation}
    \metadata{topic={Inflammation And Aging}, source={Email Wiki}}

    \begin{question}
        Isn't inflammation some kind of homeostatic defensive reaction ? What would be the meaning of trying to inhibit this process?
    \end{question}

    \begin{answer}
        Before birth, injuries heal without inflammation and don't leave a scar. Two important causes for that difference are the high concentration of \ce{CO2}, which limits lactic acid production, and the absence of the n-3 and n-6 polyunsaturated fatty acids. In adulthood, the tissues become progressively more loaded with fats of that type, leading to greater production of inflammatory agents such as prostaglandins, and an increasing tendency to produce lactic acid rather than carbon dioxide. Chronic systemic inflammation is the central factor in the various degenerative diseases.
    \end{answer}
\end{qaexchange}

\begin{qaexchange}{Chronic Inflammation}
    \metadata{topic={Most Anti-inflammatory Substances}, source={Ray Peat Forum}}

    \begin{question}
         If I may ask a quick question. In your opinion, what specifically, are the most anti-inflammatory substances a person could use or take?
    \end{question}

    \begin{answer}
      Some of the most potent are also destructive to all tissues. The safest are sugar, aspirin, pregnenolone, DHEA, progesterone, thyroid hormone, lidocaine, testosterone, and food sources of magnesium and calcium.
    \end{answer}
\end{qaexchange}

\subsection{Autoimmune Diseases}

\begin{standalonequote}{Autoimmune Diseases}
    \metadata{topic={Thyroid Antibodies In Diabetes}, source={Email Wiki}}
    \begin{note}
        Type 1 Diabetic With TPO Antibodies
    \end{note}

    \begin{answer}
        Was your TSH tested? Usually the antibodies just mean that the thyroid gland is inflamed, and increased TSH can be responsible for that. T\textsubscript{4} can suppress TSH protectively, but since intracellular glucose is needed for making T\textsubscript{3}, diabetes can interfere with that. I think some T\textsubscript{3} is always appropriate with diabetes.
    \end{answer}
\end{standalonequote}

\begin{standalonequote}{Autoimmune Diseases}
    \metadata{topic={TPO Antibodies And TSH}, source={Email Wiki}}
    \begin{note}
        TSH Was 3.9
    \end{note}

    \begin{answer}
        I think the high TSH explains the antibodies, and a combination of T\textsubscript{4} and T\textsubscript{3} is usually all that's needed; it usually takes a few months after suppressing TSH for the antibodies to decrease. Cortisol would be important to know, also estrogen and prolactin would be more informative than the common thyroid tests.
    \end{answer}
\end{standalonequote}

\begin{standalonequote}{Autoimmune Diseases}
    \metadata{topic={Leaky Gut And Autoimmunity}, source={Email Wiki}}
    \begin{note}
        Autoimmune Diets
    \end{note}

    \begin{answer}
        Have you seen my website article on milk? It mentions some of the things behind gluten sensitivity. Hypothyroidism is one thing that commonly causes leaky gut, as well as leaky liver, muscles, heart, thyroid gland, etc. The immune system reacts to the leakiness, and although it might be an effect, rather than a cause, the presence of antibodies is sometimes said to show autoimmunity.
    \end{answer}
\end{standalonequote}

\begin{standalonequote}{Autoimmune Diseases}
    \metadata{topic={Food Antibody Testing}, source={Email Wiki}}
    \begin{note}
        Test for Cross-Reactive Foods
    \end{note}

    \begin{answer}
        The presence of specific antibodies means that someone has been exposed to an antigen, but it doesn't indicate that they will react badly to it.
    \end{answer}
\end{standalonequote}

\begin{qaexchange}{Autoimmune Diseases}
    \metadata{topic={Protein Cross-Reactivity, Autoimmunity}, source={Ray Peat Forum}}

    \begin{question}
         How can we know what other substances/organisms/bits of protein chain might also cross react? How do such homologies arise?
    \end{question}

    \begin{answer}
      There are lots of areas of similar amino acid sequences in proteins of microorganisms and animals, that are important for shaping the protein and making it fit into its normal place in the cell. Even slight \enquote{denaturation} of a protein can expose those areas, so stress, changing normal relationships, can make things antigenic. Someone found that an animal's own cartilage, normally not antigenic, became antigenic after the cartilage was twisted. Energy depletion of a tissue makes it very susceptible to having its antigenic regions exposed; hypothyroidism, and high estrogen, make tissues swell; our \enquote{immune system} helps to correct the damage, removing things that are seriously disrupted. Progesterone and pregnenolone have a stabilizing effect, supported by thyroid and \ce{CO2}. When the damaged tissues are restored, the antibodies will gradually disappear. Well cooked mushrooms and bamboo shoots are probably more effective than carrots. Small amounts of antibiotics, flowers of sulfur, and aspirin, and bacteriophage can help by reducing bacteria in the small intestine.
    \end{answer}
\end{qaexchange}

\subsection{Allergies}

\begin{qaexchange}{Allergies}
    \metadata{topic={Carrot And Thyroid For Allergies}, source={Email Wiki}}

    \begin{question}
        Do you think the daily raw carrot is among the most effective things?
    \end{question}

    \begin{answer}
        And thyroid and good complete nutrition.
    \end{answer}
\end{qaexchange}

\begin{standalonequote}{Allergies}
    \metadata{topic={Allergen Persistence Effects}, source={Email Wiki}}

    \begin{answer}
        Even traces of allergens in foods or supplements can do that [congestion/stuffed up nose], and depending on the intestinal transit time, a single dose of an allergen can keep producing congestion for days.
    \end{answer}
\end{standalonequote}

\begin{standalonequote}{Allergies}
    \metadata{topic={Nasal Congestion And Allergies}, source={Email Wiki}}

    \begin{answer}
        Like the dark circles, a chronically plugged nostril is suggestive of an allergy, and it usually varies according to the intensity of the intestinal irritation of undigested food. Keeping notes on what you eat, you might notice increased stuffiness during the night after particular foods were eaten, though with some foods the congestion can take a couple of days to develop. Prolonged endurance exercise will usually slow the pulse because of adaptive inhibition of the thyroid. I have seen some people with the dark circles, fatigue, and other symptoms that stopped as soon as they stopped their daily running.
    \end{answer}
\end{standalonequote}

\begin{standalonequote}{Allergies}
    \metadata{topic={Mold}, source={Ray Peat Forum}}

    \begin{answer}
        Mycotoxins don't stay in the body long. Many kinds of oral supplement can cause allergy symptoms such as wheezing and coughing and headaches, so I use vitamin D and vitamin K on my skin in their oily solvent. The dose has to be about ten times higher, and rubbed in well. Using a good thyroid supplement to lower TSH, and a generous amount of calcium in the diet, can reduce inflammatory effects.
    \end{answer}
\end{standalonequote}

\begin{standalonequote}{Allergies}
    \metadata{topic={Hay Fever Treatment}, source={Ray Peat Forum}}

    \begin{answer}
      Coffee, aspirin, and pregnenolone are usually helpful; Benadryl works for some people, though its excipients can be allergenic.
    \end{answer}
\end{standalonequote}

\begin{standalonequote}{Allergies}
    \metadata{topic={Hives, Hypothyroidism, Blood Clots}, source={Ray Peat Forum}}

    \begin{answer}
      Hives usually result from food sensitivities on a background of hypothyroidism, and hypothyroidism often leads to stress syndromes that affect the personality. There are several related articles on my website. Vitamin E, and including some liver and aged cheese in the diet, and correcting hypothyroidism, help to clear up blood clots. 
    \end{answer}
\end{standalonequote}

\begin{emailexchange}{Allergies}
    \metadata{topic={Stuffy Nose Remedies}, source={Ray Peat Forum}}

    \begin{question}
        What is an effective fast way to relieve a stuffy nose due to allergy, bad food or an infection? If I can't breath from my nose, I am afraid I am losing a lot of \ce{CO2} from my mouth which is not good.
    \end{question}

    \begin{answer}
      Antiinflammatories, including aspirin, antihistamines, fibrous foods. Occasional use of a nebulizer with a 4\% saline solution can help with membranes irritated by allergens.
    \end{answer}

    \begin{question}
        Does using nasal spray do any good? Or does the push further irritate the membranes?
    \end{question}

    \begin{answer}
      Hypertonic saline is antiinflammatory.
    \end{answer}
\end{emailexchange}

\begin{emailexchange}{Allergies}
    \metadata{topic={Stuffy Nose Relief, Allergies}, source={Ray Peat Forum}}

    \begin{question}
        What is an effective fast way to relieve a stuffy nose due to allergy, bad food or an infection?
    \end{question}

    \begin{answer}
      Antiinflammatories, including aspirin, antihistamines, fibrous foods. Occasional use of a nebulizer with a 4\% saline solution can help with membranes irritated by allergens.
    \end{answer}

    \begin{question}
        Does using nasal spray do any good? Or does the push further irritate the membranes?
    \end{question}

    \begin{answer}
      Hypertonic saline is antiinflammatory.
    \end{answer}
\end{emailexchange}

\section{Mental Health Conditions}

\begin{standalonequote}{Mental Health Conditions}
    \metadata{topic={Progesterone For Addiction}, source={Email Wiki}}

    \begin{answer}
        I knew someone who had been addicted to morphine and alcohol for 30 years, who was drinking quarts of beer and wine daily when he didn't have morphine, who had an opportunity for a good job if he could get sober. Starting progesterone at bedtime (and stopping the wine), he said it was the first time he didn't have a hangover in the morning. He used enough progesterone to neuter most people, but said it didn't affect his sex function; he was taking a lot of Cytomel and magnesium, but wasn't drunk again as long as I knew him, and his general health improved.

        The person I described who recovered so completely took about 1000 mg of progesterone during the first night, and more than 1000 mg daily for a few weeks, but that much could make some people comatose; it's a matter of individual hormone status. I think the SSRI drugs continue to do harm, even when they reduce withdrawal symptoms.
    \end{answer}
\end{standalonequote}

\begin{standalonequote}{Mental Health Conditions}
    \metadata{topic={Multiple Condition Protocol}, source={Email Wiki}}
    \begin{note}
        Childhood Cancer, PTSD, OCD, High Blood Pressure, Constipation, ED, Muscle Wasting
    \end{note}

    \begin{answer}
        A daily carrot, for constipation and to lower estrogen and cortisol, thyroid to lower blood pressure, and pregnenolone and DHEA to increase the neurosteroids.
    \end{answer}
\end{standalonequote}

\begin{standalonequote}{Mental Health Conditions}
    \metadata{topic={Cannabis Withdrawal Symptoms}, source={Email Wiki}}
    \begin{note}
        Dilated Pupils and Other Symptoms When Coming Off Marijuana
    \end{note}

    \begin{answer}
        Adrenaline can increase to compensate for low thyroid function, and causes pupils to dilate. Thyroxin by itself works when your liver is in good condition, supplied with enough glucose, and not stressed by adrenaline and other stress hormones. Something that contains both T\textsubscript{4} and T\textsubscript{3} is better for getting out of a stress pattern. Armour thyroid or Novotiral might work better than Eutirox. Too much of the weed interferes with liver function, and while it's recovering it's necessary to be careful to get enough protein and other nutrients every day, for example orange juice, eggs, milk, cheese, cooked mushrooms, occasional shellfish.
    \end{answer}
\end{standalonequote}

\begin{standalonequote}{Mental Health Conditions}
    \metadata{topic={Migraines}, source={Email Wiki}}

    \begin{answer}
        Lots of sugar, without coffee, would be quicker for restoring blood sugar. At least a quart of milk shake or ice cream can provide the needed sugar in a form that can be assimilated quickly. 100 mg of progesterone in oil can usually stop it, by stopping the wastage of glucose.
    \end{answer}
\end{standalonequote}

\begin{qaexchange}{Mental Health Conditions}
    \metadata{topic={Panic Attack Causes}, source={Email Wiki}}

    \begin{question}
        Are panic attacks mainly a manifestation of hormonal imbalance, excessive stress hormones?
    \end{question}

    \begin{answer}
        Yes, usually with hyperventilation caused by high estrogen and serotonin, low vitamin B\textsubscript{6}.
    \end{answer}
\end{qaexchange}

\begin{standalonequote}{Mental Health Conditions}
    \metadata{topic={Lower Serotonin And Holism}, source={Email Wiki}}

    \begin{answer}
        Yes, it allows a positive kind of mental energy, since high serotonin causes conservative, defensive authoritarian avoidance.
    \end{answer}
\end{standalonequote}

\begin{standalonequote}{Mental Health Conditions}
    \metadata{topic={Serotonin And Mental Effort}, source={Email Wiki}}

    \begin{answer}
        Effective mental effort is easier to make when serotonin isn't excessive; attitude and chemistry interact, both directions.
    \end{answer}
\end{standalonequote}

\begin{standalonequote}{Mental Health Conditions}
    \metadata{topic={Emotional Trauma, Neurosteroids}, source={Ray Peat Forum}}

    \begin{answer}
      I think the lingering effects involve changes in the brain's metabolism of neurosteroids. Just saturating the system with the hormones has to be combined with redirection of attention, until new habits of attention and brain metabolism are formed. Ordinary authoritarian relationships, which are the norm, make everyone forget what it felt like to just be a self, before language and assigned identities existed. When relaxing after a very nice meal, pleasant feelings pass through the torso. Getting into the habit of looking for those feelings whenever you have a free moment, you can make them occur more often; they represent part of self-possession. The state of the brain models the state of those perceptions; the crucial neurosteroids that let new patterns develop are pregnenolone, progesterone, vitamin D, and (cautiously) a little DHEA; the energy-supporting high calcium diet supports their metabolism. With practice, this sense of well being and wholeness grows.
    \end{answer}
\end{standalonequote}

\subsection{Depression}

\begin{qaexchange}{Mental Health}
    \metadata{topic={Competitiveness}, source={Ray Peat Forum}}

    \begin{question}
        Do you think being competitive has a place in a healthy individual or would they be fairly indifferent about being better than others?
    \end{question}

    \begin{answer}
        I think it's a form of neurosis.
    \end{answer}
\end{qaexchange}

\begin{qaexchange}{Mental Health}
    \metadata{topic={Chess and Mental Exercise}, source={Ray Peat Forum}}

    \begin{question}
        What do you think of chess as a hobby? Does mental exercise have the same potential to be anti-metabolic as physical exercise does?
    \end{question}

    \begin{answer}
        I think the abstract nature of games makes them potentially harmful, addictive.
    \end{answer}
\end{qaexchange}

\begin{qaexchange}{Mental Health}
    \metadata{topic={Philosophy and Neurosis}, source={Ray Peat Forum}}

    \begin{question}
        Sometimes when i read philosophy, or fictional works, I began to feel a bit neurotic/hyper-excited and I start overly living in my head. Do you think doing physical actions are a good way to ground back to reality?
    \end{question}

    \begin{answer}
        I think the aroused moments are the result of glimpsing a better reality; painting with that attitude can exalt the subject, finding its best meanings.
    \end{answer}
\end{qaexchange}

\begin{qaexchange}{Depression}
    \metadata{topic={Grief Coping}, source={Ray Peat Forum}}

    \begin{question}
         I was wondering if you have run across any information, or learned on your own, things that help people grieve over lost loved ones, and or fear of losing others?
    \end{question}

    \begin{answer}
      It activates the \enquote{helplessness} reactions in the body, stress weakening your own life, and I think it can help to get out of that if you think of your life as a continuation of theirs—the same life, though with fewer bodies.
    \end{answer}
\end{qaexchange}

\begin{qaexchange}{Depression}
    \metadata{topic={Dealing with Anger, Trauma}, source={Ray Peat Forum}}

    \begin{question}
        Do you have any advice for dealing with destructive feelings - pent-up anger, resentment, hate, jealousy, childhood trauma, rejection, feeling unloved, feeling world-weary, feeling alienated in a world full of serotonin-driven automatons - that resurface whenever metabolism isn't optimal or when stress becomes too much?

        Food, progesterone, thyroid, sunlight and other prometabolic substances are helpful, but they don't clear up the energy of the past stuck in the system and influencing the evaluation of the present.

        How to protect oneself from being vulnerable, without getting bitter and coldhearted?

        And how does one know when to run away from a stressful situation and when to work through it and improve oneself in the process?
    \end{question}

    \begin{answer}
      Although I think escaping from stressful situations is good in principle, it can be very hard in practice. For acute situations, having a milkshake, some pregnenolone and progesterone and coffee as needed, can make it possible to direct the anger energy into mental actions. The reality is that societies are populated mostly by those authoritarian automatons, but when I realize that the world still contains lots of sentient intelligent beings—some humans, many mammals, insects, mollusks, plants--I see that alienation from the malicious system is better than joining it.

      It's helpful to understand how particular bad actions fit into the bad system, even though it makes you realize that the problem is immensely bigger than the immediate thing that you're reacting to. Wilhelm Reich and Alice Miller wrote about some of the ways that the evil automatons are created. But beyond that system of automatons, there's the real world of living intelligences.

      When I was investigating how my autonomic nervous system works, I realized that my body takes on specific feelings in the presence of different individuals, with a sense of fibers going out from the solar plexus connecting with people that I feel free with, and a sense of shrinking discomfort in other situations. With practice, I found that I had some control over those reactions, and by directing my attention to them I could maintain a sense of myself while in the presence of the manipulators-robots.

      Besides being self protective or therapeutic, the consciousness of alienation from a bad system puts you into a position where you might be able to reduce some of its destructiveness.
    \end{answer}
\end{qaexchange}

\begin{standalonequote}{Depression}
    \metadata{topic={Grief, Childlessness}, source={Ray Peat Forum}}

    \begin{answer}
      In small tribal villages, everyone functioned as family members, and the \enquote{extended family} function persisted longer in agricultural societies, but has disintegrated in recent times. The idea of wives and children as a man's property blended with the doctrine of \enquote{genes,} so that ownership and inheritance of property/genes became a deep part of our culture's ideology. I think the sense of sharing, participating in, contributing to social life is primary, and the sense of private ownership usually has harmful effects on the people involved. Have you had tests recently for TSH and cholesterol?
    \end{answer}
\end{standalonequote}

\subsection{Anxiety}

\begin{emailexchange}{Anxiety}
    \metadata{topic={Hypochondria vs Health Concern}, source={Ray Peat Forum}}

    \begin{question}
        What are your thoughts on hypercondria versus reasonable concern into health?
    \end{question}

    \begin{answer}
       Having a distinct symptom that is uncomfortable is a reason for looking for the cause; if a symptom is frequent and disabling to any extent, then it's appropriate to invest some time and effort in figuring it out. When anxiety is the problem, people sometimes interpret insignificant sensations as danger signals instead of attending to the causes of the anxiety. 
    \end{answer}

    \begin{question}
        What did you mean by the causes of anxiety?
    \end{question}

    \begin{answer}
       Often it's some simple thing, such as hypothyroidism. A chronic external stress (such as crocodiles or secret police) can lead to an internal change, such as thyroid failure. 
    \end{answer}
\end{emailexchange}

\begin{standalonequote}{Anxiety}
    \metadata{topic={Anxiety, Calcium}, source={Ray Peat Forum}}

    \begin{answer}
      Large amounts of calcium have a sedative, anti-inflammatory effect, and the casein and other nutrients have anti-stress actions. Inflammation increases parathyroid hormone, and a large intake of calcium is a safe way to lower that.
    \end{answer}
\end{standalonequote}

\subsection{Cognitive Disorders}

\begin{standalonequote}{Cognitive Disorders}
    \metadata{topic={Autism Causes}, source={Email Wiki}}

    \begin{answer}
        I suspect that it's an adaptive reaction to prenatal exposure to stress. The imbalances of endorphins, serotonin, catecholamines, and other nerve-regulators that have been seen in autism sometimes can be produced in adults by combined fatigue and poor nutrition, and when the liver's glycogen is depleted, it can be hard to restore the balance. Prenatal influences of different types could damage connectivity, which permitting cells to survive. Normally, a large proportion of brain cells die before birth, because of limited availability of glucose.
    \end{answer}
\end{standalonequote}

\begin{standalonequote}{Cognitive Disorders}
    \metadata{topic={Autism Treatment Approaches}, source={Email Wiki}}

    \begin{answer}
        Since autism typically involves high serotonin, things like thyroid, lisuride, tianeptine, and gelatin could be helpful.
    \end{answer}
\end{standalonequote}

\begin{standalonequote}{Cognitive Disorders}
    \metadata{topic={Dementia Treatment Protocol}, source={Ray Peat Forum}}

    \begin{answer}
      Things that increase the metabolic rate improve memory, reaction time, etc., relax bladder, build muscle. Vitamin D, high calcium intake (two liters/day low fat milk), aspirin, angiotensin receptor blockers, thyroid, progesterone and DHEA (5 mg), adequate protein, bowel regularity, are important for supporting oxidative metabolism.
    \end{answer}
\end{standalonequote}

\section{Degenerative Diseases}
\subsection{Cancer}

\begin{standalonequote}{Cancer}
    \metadata{topic={Chronic Leukemia And Estrogen}, source={Email Wiki}}

    \begin{answer}
        Estrogen is a stress and crisis hormone, so there's a steady upward tendency in its effect with age, which is interrupted during the fertile years by progesterone. Progesterone decline usually starts around age 40 because of things interfering with thyroid function. Declining liver function and increasing pituitary activity influence the way problems develop. Oxygen tension is normally low in bone marrow, and stimulates constant cell proliferation, but when estrogen's oxygen-wasting effect is added, it changes the balance of cell growth; its worst effect is to stimulate fibroblasts and collagen production in the marrow, displacing the red and white cells. Estrogen shifts cells from oxidative production of carbon dioxide to the glycolytic formation of lactic acid, tending to prevent normal differentiation of cell function. The effects of estrogen include the leakage of the lactic acid-forming enzyme LDH into the serum, and the increase of copper in the serum, and these are recognized as signs of CLL, but the genetic-clonal ideology of cancer prevents recognition of the metabolic pattern. Estrogen causes relative hyperventilation, reinforcing the cellular changes. CLL seems to be less common in people adapted to high altitude, where lactic acid formation is inhibited. 
    \end{answer}
\end{standalonequote}

\begin{qaexchange}{Cancer}
    \metadata{topic={Cancer Genetic Dogma Critique}, source={Email Wiki}}

    \begin{question}
        Mainstream science beliefs: What is right and what is wrong ?
    \end{question}

    \begin{answer}
        Some of the basic ideas about the mechanisms of energy production are mistaken, for example that glycolysis is controlled by random diffusion, and the chemiosmotic theory of mitochondrial phosphorylation, but there's still useful information available regarding the bigger picture, looking at the \enquote{energy charge} and the ratios of oxidized and reduced molecules, for example. Cancer involves a shift in the direction of reduction---an idea that has been around for over 60 years, but kept in the background by the genetic dogma. Oxidative metabolism shifts the balance away from cell multiplication, allowing specific functions, and so allows an organism to recover from cancer. The doctrine that cancers are genetically mutated cells justifies the standard treatments---surgery, radiation, cytotoxic chemotherapy---intended to kill 100\% of the mutant cells. The problem is that the damaged region where the tumor had been is left in a reductive state, and signals the organism for repair cells, and those become abnormal when they enter the region of destruction. Cancer mortality figures have been manipulated to convince the public that the standard treatments produce a certain rate of cure, but cancer deaths have increased parallel to the increased number of people treated. For example, the biggest surge in prostate cancer deaths came in the 1990s after the discovery of the prostate specific antigen, PSA, which greatly increased the number of people receiving treatment. In the 1950s, Dan Mazia and Albert Szent-Gyorgyi were showing how the reductive condition of a cell relates to cancer; Frances Knock was another person working in that direction. In just the last few years, there are signs that this approach is coming to the foreground of biological research. But it can't be officially acknowledged, because it would reveal the carcinogenic and irrational nature of the standard treatments.
    \end{answer}
\end{qaexchange}

\begin{standalonequote}{Cancer}
    \metadata{topic={Breast Cancer Prevention}, source={Email Wiki}}

    \begin{answer}
        Orange juice and guavas contain aromatase inhibitors, and aspirin and progesterone are other inhibitors. Aspirin and progesterone also oppose the effects of HER2/neu on aromatase and estrogen. 
    \end{answer}
\end{standalonequote}

\begin{standalonequote}{Cancer}
    \metadata{topic={Eye Melanoma Treatment}, source={Email Wiki}}

    \begin{answer}
        I had some probable melanomas years ago, and I found that progesterone and DHEA and increased thyroid caused them to disappear quickly.
    \end{answer}
\end{standalonequote}

\begin{standalonequote}{Cancer}
    \metadata{topic={Eye Melanoma Protocol}, source={Email Wiki}}

    \begin{answer}
        His other symptoms are very suggestive of a deficiency of pregnenolone, progesterone, and thyroid, as well as the vitamin and minerals. Once when I had four ounces of beer daily for a few weeks several moles grew rapidly, and I realized it was probably the estrogen in the beer that was responsible, and two weeks after I stopped the moles dried up and fell off. Orange juice contains naringenin which is effective against melanoma, and guavas contain apigenin, also effective. A diet consisting of milk, orange juice, guavas, cheese, and some eggs, liver, and oysters, with aspirin would be protective against the spread of the tumor. Thorne's high potency vitamin K drops would help with the blood pressure, and vitamin K also has some anticancer activity, and is necessary when you use a lot of aspirin. At least 15 milligrams per day would probably quickly regulate his blood pressure.
    \end{answer}
\end{standalonequote}

\begin{standalonequote}{Cancer}
    \metadata{topic={Melanoma And Pregnenolone}, source={Email Wiki}}

    \begin{answer}
        Pregnenolone isn't a hormone, but it normalizes the steroid hormones, preventing excess cortisol and helping to normalize aldosterone, so it should be helpful for any stress including surgery. Progesterone has a wide spectrum of anticancer activity, but as far as I know only synthetic progestins have been used medically. Although I used myself it on things that appeared to be very active melanomas, I usually recommend a slightly hyperthyroid state for helping to control it.
    \end{answer}
\end{standalonequote}

\begin{standalonequote}{Cancer}
    \metadata{topic={Heme Oxygenase And Cancer}, source={Email Wiki}}

    \begin{answer}
        An enzyme that activates glycolysis, PFKFB4, is normally increased by oxygen deprivation and the hypoxia inducible factor (HIF), but it is also increased by heme oxygenase (Li, et al., 2012). Gluconeogenesis is normally inhibited by heme, which is removed by heme oxygenase. Lactic acid produced by glycolysis activates an enzyme (thioredoxin) that increases cellular sulfhydryl reduction, and increases HIF and also stimulates the formation of new blood vessels by inducing VEGF, the permeability and growth factor which is essential for the growth of cancer, and which is induced by heme oxygenase. While interfering with the functions of mitochondria, heme oxygenase also stimulates the growth of new mitochondria, along with new blood vessels.
    \end{answer}
\end{standalonequote}

\begin{standalonequote}{Cancer}
    \metadata{topic={Skin Cancer Prevention}, source={Email Wiki}}

    \begin{answer}
        Yes, avoidance of unsaturated fats is the most important thing. Aspirin, caffeine, and orange juice are protective. Keeping the TSH low is important, because it stimulates melanoma growth. 
    \end{answer}
\end{standalonequote}

\begin{standalonequote}{Cancer}
    \metadata{topic={Tongue Cancer Treatment}, source={Email Wiki}}

    \begin{answer}
        Besides large amounts of aspirin (grams per day), and vitamin K\textsubscript{1} or K\textsubscript{2} to prevent abnormal bleeding from the aspirin, I think I would use DCA (dichloroacetate), which is available from Canada on the internet (and forums describe its use), and maybe an enzyme related to vitamin D, called GCMAF, that activates the immune system.
    \end{answer}
\end{standalonequote}

\begin{qaexchange}{Cancer}
    \metadata{topic={RB1 Gene}, source={Ray Peat Forum}}

    \begin{question}
        Would a low RB1 gene expression (which is a tumor suppressor gene apparently)
suggest higher disease progression? Or does the metabolic approach to cancer look at the low RB1 gene as irrelevant? Is the RB1 gene mutation a cause of cancer or is it a consequence of cancer?
    \end{question}

    \begin{answer}
        The gene activation is often a matter of degrees, that can be offset by reducing inflammation.
    \end{answer}
\end{qaexchange}

\begin{standalonequote}{Cancer}
    \metadata{topic={Liver and Bile Duct Cancer}, source={Ray Peat Forum}}

    \begin{answer}
        Besides those, I would include an antibiotic to reduce intestinal bacteria, and captopril, a small amount of cyproheptadine or ketotifen or possibly other antihistamines, and some lidocaine, either orally or transdermally (ointment or patch). I've known a few people (and dogs and cats) that lived a long time with liver cancer, just using progesterone.
    \end{answer}
\end{standalonequote}

\begin{standalonequote}{Cancer}
    \metadata{topic={Kidney Cancer}, source={Ray Peat Forum}}

    \begin{answer}
        Watching thyroid, vitamin D, calcium/phospate in the diet, and the balance of serum steroids (low on cortisol, aldosterone, and estrogen), are the most important things. Keeping the intestine active and free of inflammation (using thyroid and coffee), with a low phosphate diet, are the essence of Gerson's method (raw carrots are helpful for the intestine).
    \end{answer}
\end{standalonequote}

\begin{standalonequote}{Cancer}
    \metadata{topic={Treatment Options}, source={Ray Peat Forum}}

    \begin{answer}
        Anticholinergic drug such as scopolamine, belladonna, or atropine could be helpful; aspirin and cyproheptadine are other safe drugs that inhibit cancer promoting signals.
    \end{answer}
\end{standalonequote}

\begin{standalonequote}{Cancer}
    \metadata{topic={Lung Cancer Treatment Approaches}, source={Ray Peat Forum}}

    \begin{answer}
      Lung cancer cells produce increased amounts of some things that promote abnormal growth, and some of these can be inhibited by common harmless materials. The effects of adenosine and (leaked) ATP are inhibited by caffeine, prostaglandins are inhibited by aspirin and pregnenolone, estrogen by progesterone, aromatase inhibitors, orange juice and cooked mushrooms, histamine and serotonin by antihistamines, cyproheptadine, nitric oxide by tetracycline, progesterone, agmatine, etc. Emodin, in cascara, blocks various things in lung cancer, and its laxative effect helps to lower nitric oxide, histamine, and serotonin.
    \end{answer}
\end{standalonequote}

\begin{standalonequote}{Cancer}
    \metadata{topic={Stage 4 Brain Cancer Protocol}, source={Ray Peat Forum}}

    \begin{answer}
       Tumors have multiple causes, so I think it's best to use things that are known to be protective against known causes. I think cancers are continually produced and maintained by general metabolic conditions, and if those are ignored while trying to kill all of the abnormal cells, those cells release signals to recruit replacement cells, which, encountering the same or worse metabolic conditions, renew the tumor. Known causes involve inflammatory, excitatory processes, and when those are eliminated, tumors tend to disintegrate, undergoing \enquote{apoptosis,} a form of cell death that doesn't create new inflammation. Here are some articles describing the effects of some of these antiinflammatory, antiexcitatory, substances—aspirin, caffeine, tetracylines, antihistamine/antiserotonin/anti-nitric oxide agents, progesterone, pregnenolone, antagonists of excitotoxicity, estrogen antagonists, vitamins D, E, and K, etc. Vitamin E (500 mg mixed tocopherols) and aspirin are things that the medical authorities might not object to.
    \end{answer}
\end{standalonequote}

\begin{qaexchange}{Cancer}
    \metadata{topic={Bowel Cancer Prevention}, source={Ray Peat Forum}}

    \begin{question}
        What would you recommend to prevent against bowel cancer?
    \end{question}

    \begin{answer}
       I think fibrous foods and milk are the most protective things, along with good thyroid function, and avoiding polyunsaturated fats. 
    \end{answer}
\end{qaexchange}

\begin{standalonequote}{Cancer}
    \metadata{topic={Glioblastoma Treatment}, source={Ray Peat Forum}}

    \begin{answer}
      Acetazolamide, angiotensin blockers such as losartan, lidocaine, vitamin D, aspirin, and naloxone all have anticancer effects without serious side-effects, and they are inexpensive and available to any doctor.
    \end{answer}
\end{standalonequote}

\begin{emailexchange}{Cancer}
    \metadata{topic={DHT and Melanoma}, source={Ray Peat Forum}}

    \begin{question}
        A friend of mine started taking DHT for skin cancer. He sent me this study, showing supposed evidence for androgens beeing bad for skin cancer. Do you think there's any truth to that? 
    \end{question}

    \begin{answer}
      For melanoma, that sounds likely. Thyroid and progesterone with opposing effects would be more logical.
    \end{answer}

    \begin{question}
        Wouldn't DHT have an anti-cancer effect through its anti-estrogen effects?
    \end{question}

    \begin{answer}
      It has been used that way in breast cancer, but it's risky to generalize about its anticancer effects.
    \end{answer}
\end{emailexchange}

\begin{standalonequote}{Cancer}
    \metadata{topic={Carcinoid Tumor}, source={Ray Peat Forum}}

    \begin{answer}
      It's commonly associated with deficiency of vitamin D and niacin. Besides watching thyroid function, blood \ce{CO2}, and TSH, using acetazolamide might be helpful if tumors are identified.
    \end{answer}
\end{standalonequote}

\subsection{Neurological Conditions}

\begin{standalonequote}{Neurological Conditions}
    \metadata{topic={Multiple Sclerosis Treatment}, source={Email Wiki}}
    \begin{note}
        Treatments for Multiple Sclerosis, Combined With Paranoia in Menopause
    \end{note}

    \begin{answer}
        For multiple sclerosis, thyroid and progesterone are the most helpful things. Sometimes a very low thyroid function is compensated by extreme nerve excitation, leading to mania or paranoia. Their body temperature might be extremely low, or sometimes the 24 hour cycle is reversed; if the temperature decreases in the morning, that suggests that the stress hormones were very high during the night.
    \end{answer}
\end{standalonequote}

\begin{standalonequote}{Neurological Conditions}
    \metadata{topic={Stroke Prevention Protocol}, source={Email Wiki}}

    \begin{answer}
        Constipation might have been responsible for the stroke. Progesterone is already being used medically, 'experimentally,' for brain protection and repair, pregnenolone has some of the similar effects, probably not as powerfully, but it's safe in any quantity. Carbon dioxide increase can often restore circulaton to areas that have a vascular spasm; a little baking soda in water sometimes helps that. Sugar sometimes helps with constipation. They should be doing something for the constipation; inflammation is often involved, and aspirin and cascara (emodin) are helpful if the main blockage can be overcome with enemas. Inflamed tissues are hyperosmotic, so enemas with the standard 0.9\% saline cause tissue swelling; double or triple osmolar saline is usually helpful.
    \end{answer}
\end{standalonequote}

\begin{standalonequote}{Neurological Conditions}
    \metadata{topic={Stroke Recovery Nutrients}, source={Email Wiki}}
    \begin{note}
        Stroke Brain Recovery
    \end{note}

    \begin{answer}
        Both progesterone and pregnenolone are o.k. at the same time. They protect the brain. Vitamin K (1 or 2) is probably better to use before aspirin, since it helps to prevent both more clotting and also bleeding. A day after the vitamin K the aspirin would be safer, and it helps to protect brain cells. Niacinamide, vitamin B\textsubscript{1} and biotin are other brain protective things, even is small amounts. Hospitals often treat strokes with too much oxygen, to reduce brain pressure, but that reduces circulation to the brain; 5\% carbon dioxide with oxygen helps to reduce brain swelling while maintaining circulation.
    \end{answer}
\end{standalonequote}

\begin{qaexchange}{Neurological Conditions}
    \metadata{topic={MS, Swank Diet}, source={Ray Peat Forum}}

    \begin{question}
        Why do you think the Swank diet works so well for MS sufferers?
    \end{question}

    \begin{answer}
      I think the cod liver oil and low iron (red meat) intake were helpful.
    \end{answer}
\end{qaexchange}

\section{Acute Conditions}

\begin{qaexchange}{Acute Conditions}
    \metadata{topic={Septic Shock Treatment}, source={Ray Peat Forum}}

    \begin{question}
        What would you recommend for someone experiencing septic shock? 
    \end{question}

    \begin{answer}
      Sugar, aspirin, naloxone or naltrexone, antibiotics and laxatives.
    \end{answer}
\end{qaexchange}

\subsection{Infections}

\begin{qaexchange}{Infections}
    \metadata{topic={Vaccine Shedding}, source={Ray Peat Forum}}

    \begin{question}
        Do you still think it's safe to work as a schoolteacher, if about 50\% of the students have been vaccinated? I have repeatedly heard that the vaccine spike proteins commonly shed and act like prions, causing prion disease and toxicity even in those unvaccinated.
    \end{question}

    \begin{answer}
        Prion disease is facilitated by polyunsaturated fats, so it's very protective to reduce or eliminate them from the diet. If you'e in good health the spike proteins from shedding are no more harmful than the virus.
    \end{answer}
\end{qaexchange}

\begin{qaexchange}{Infections}
    \metadata{topic={Ivermectin and HCQ}, source={Ray Peat Forum}}

    \begin{question}
        If the spike proteins are so easily transmissible and so dangerous, might ivermectin and HCQ be reasonable prophylactic measures for someone exposed to vaccinated people daily?
    \end{question}

    \begin{answer}
        Ivermectin has been associated with brain damage in a small percentage of users, so I don't think it's suitable for prolonged preventive use. Vitamin D, aspirin, antihistamines, aspirin, progesterone, etc. are safer.
    \end{answer}
\end{qaexchange}

\begin{standalonequote}{Infections}
    \metadata{topic={Ivermectin vs Hydroxychloroquine}, source={Ray Peat Forum}}

    \begin{answer}
        Hydroxychloroquine is less toxic, but I don't think either is a good idea for prophylaxis, when vitamin D and other very safe things are available.
    \end{answer}
\end{standalonequote}

\begin{emailexchange}{Infections}
    \metadata{topic={Viruses and Exosomes}, source={Ray Peat Forum}}

    \begin{question}
        Some of what you described vis a vis portions of DNA transcribe retroviruses fit in with the recently popularized notion (I learnt from Dr. Tom Cowan) that many of these popular viruses either don't exist, or at least aren't sources of contagion (and perhaps that these viruses and what are known as exosomes, can eb one and the same). Would you go as far as agreeing with this notion?
    \end{question}

    \begin{answer}
        I think viruses originated as exosomes, and better knowledge of what exosome are should help to protect against all kinds of disease. We are constantly exposed to foreign DNA and RNA, and it rarely has a harmful effect, but with fatigue and inflammation, normally harmless things become dangerous. Since the new vaccines are intended to make us produce a protein that blocks ACE2, which is possibly our most important protective antiinflammatory defense, along with a lipid adjuvant to increase inflammation, they seem riskier than the previous kinds that evoked antibodies against a foreign antigen.
    \end{answer}
\end{emailexchange}

\begin{qaexchange}{Infections}
    \metadata{topic={Exosomes vs Viruses}, source={Ray Peat Forum}}

    \begin{question}
        I would like to know the difference between exosomes and retro(tran)posons and viruses\dots{} is there a difference?
    \end{question}

    \begin{answer}
        Exosomes are particles secreted into the blood; the transposons operate within the nucleus. Some viruses have a potential to act on our genes, but are usually considered to be genetically alien.
    \end{answer}
\end{qaexchange}

\begin{standalonequote}{Infections}
    \metadata{topic={Hemorrhagic Fever}, source={Ray Peat Forum}}

    \begin{answer}
        Ebola infection seems to involve a lack of interferon, and the amount of nitric oxide in the blood increases in proportion to the intensity of the symptoms. Reductive stress/inflammation that activates interleukin-1 and arachidonic acid metabolites can inhibit interferon, and at the same time increase the production of nitric oxide. Resistance would be improved by oxidative and antiinflammatory things.
    \end{answer}
\end{standalonequote}

\section{Other Conditions}

\begin{standalonequote}{Other Conditions}
    \metadata{topic={Gout Treatment}, source={Email Wiki}}

    \begin{answer}
        Raw carrot or (boiled) bamboo shoots and aspirin, to lower endotoxin absorption.
    \end{answer}
\end{standalonequote}

\begin{standalonequote}{Other Conditions}
    \metadata{topic={Kidney Disease Treatment}, source={Email Wiki}}

    \begin{answer}
        When there's existing kidney disease, supplementing thyroid and progesterone speeds recovery.
    \end{answer}
\end{standalonequote}

\begin{standalonequote}{Other Conditions}
    \metadata{topic={Chronic Kidney Disease Protocol}, source={Email Wiki}}

    \begin{answer}
        Urea is safe, but progesterone, pregnenolone, and thyroid are often curative for chronic kidney disease. Has his vitamin D been checked? When it's low, parathyroid hormone rises, and it's one of the \enquote{uremic toxins.}
    \end{answer}
\end{standalonequote}

\begin{qaexchange}{Other Conditions}
    \metadata{topic={Stone Formation Theories}, source={Email Wiki}}

    \begin{question}
        Nanobacteria and mycoplasma: what do you think of these as causes of disease such as stones?
    \end{question}

    \begin{answer}
        I think there is a cultishness to blaming everything on them, when no other cause is evident.
    \end{answer}
\end{qaexchange}

\begin{standalonequote}{Other Conditions}
    \metadata{topic={PCOS And Hypothyroidism}, source={Email Wiki}}

    \begin{answer}
        PCOS can be produced in animals by removing the thyroid gland. The inability of ovaries to make progesterone without thyroid causes the adrenals to be overstimulated, and they are the source of increased DHEA and other androgens and estrogen.
    \end{answer}
\end{standalonequote}

\begin{qaexchange}{Other Conditions}
    \metadata{topic={Body Magnetism}, source={Ray Peat Forum}}

    \begin{question}
        If someone has not been vaccinated with the covid vaccine but finds that their tissues (back and upper body) are magnetic how can this be explained? What remedies can reduce the magnetism?
    \end{question}

    \begin{answer}
        I don't think body magnetism is a problem unless you cause a compass needle to deviate from its north-south alignment.
    \end{answer}
\end{qaexchange}

\begin{standalonequote}{Other Conditions}
    \metadata{topic={Live Blood Analysis}, source={Ray Peat Forum}}

    \begin{answer}
        Live blood analysis isn't based on science.
    \end{answer}
\end{standalonequote}

\begin{standalonequote}{Other Conditions}
    \metadata{topic={Bird Longevity}, source={Ray Peat Forum}}

    \begin{note}
        On Birds Having a Long Lifespan Compared to Size, Despite High Membrane Polyunsaturation
    \end{note}

    \begin{answer}
        I think their high temperature and high glucose support the high fat saturation.
    \end{answer}
\end{standalonequote}

\begin{standalonequote}{Other Conditions}
    \metadata{topic={Meniere's Disease Treatment}, source={Ray Peat Forum}}

    \begin{answer}
      Hypothyroidism leads to increased estrogen and nitric oxide, and a decreased serum osmolarity (increased dilution of body fluids) and hyponatremia. Salt, magnesium, calcium, aspirin, antihistamine, progesterone, tetracycline (or minocycline, doxycycline) and thyroid can correct the fluid imbalance.
    \end{answer}
\end{standalonequote}

\begin{standalonequote}{Other Conditions}
    \metadata{topic={Chronic Fatigue Syndrome Causes}, source={Ray Peat Forum}}

    \begin{answer}
       Low thyroid function and intestinal inflammation are usually involved, with increased endotoxin, nitric oxide, endorphins, serotonin, sometimes prolactin and an increased ratio of estrogen to progesterone and the androgens. A deficiency of vitamin D and calcium is sometimes involved. 
    \end{answer}
\end{standalonequote}

\begin{standalonequote}{Other Conditions}
    \metadata{topic={Accutane Side Effects Treatment}, source={Ray Peat Forum}}

    \begin{answer}
       Since the symptoms can be produced by activation of the pro-inflammatory angiotension system, and an overdose of vitamin A can create symptoms of a vitamin A deficiency (vitamin A normally works with vitamin D to inhibit that system), I think it might help to supplement some vitamin D (about 5000 IU unless you get good sun exposure), vitamin A (about 10,000 IU), with some vitamin E (20 to 50 IU), thyroid hormone (according to temperature and pulse rate), and a good ratio of calcium to phosphate in the diet. An angiotensin blocker such as candesartan might help with joint pain and depression.
    \end{answer}
\end{standalonequote}

\begin{standalonequote}{Other Conditions}
    \metadata{topic={Post-Finasteride Syndrome Treatment}, source={Ray Peat Forum}}

    \begin{answer}
      Concentrating at one end on intestinal and liver efficiency and freedom from irritation, then I think the main restorative line to concentrate on is keeping angiotensin-aldosterone-parathyroid hormone low, and optimizing vitamin D, thyroid, pregnenolone (DHEA, progesterone) production. Carbon dioxide can be boosted in a variety of ways.
    \end{answer}
\end{standalonequote}

\section{Infections \& Immunity}

\begin{qaexchange}{Infections}
    \metadata{topic={RSV Cases Increase}, source={Ray Peat Forum}}

    \begin{question}
        What do you think could explain the out-of-season increase in RSV cases this summer (2021)? Especially among children.
    \end{question}

    \begin{answer}
       I think the emphasis on vaccinating babies for influenza starting at 6 months is a factor, weakening their resistance to other respiratory infections.
    \end{answer}
\end{qaexchange}

\begin{standalonequote}{Infections \& Immunity}
    \metadata{topic={H. Pylori Treatment}, source={Ray Peat Forum}}

    \begin{answer}
       I think clarithromycin or erythromycin, and pen-V-K or amoxicillin, and olive oil are effective.
    \end{answer}
\end{standalonequote}

\begin{qaexchange}{Infections \& Immunity}
    \metadata{topic={COVID-19 Neurological Effects}, source={Ray Peat Forum}}

    \begin{question}
        I was just wondering what you thought about the research coming out about the long term neurological damage from people recovered from coronavirus?
    \end{question}

    \begin{answer}
      Influenza has similarly high neurological effects, in some cases that's the only symptom of infection. The MRI changes don't seem informative. Being sick affects the brain, lungs, blood vessels, etc.
    \end{answer}
\end{qaexchange}

\begin{qaexchange}{Infections \& Immunity}
    \metadata{topic={Immune Modulators Validity}, source={Ray Peat Forum}}

    \begin{question}
         Besides vitamin D, is there any validity on use of immune modulators like lactoferrin and beta glucans to help normalize immune response against viruses?
    \end{question}

    \begin{answer}
      I think adequate vitamin D, calcium, and minimizing inflammation is best. An inflamed digestive system, sluggish digestion, is a very important factor in viral susceptibility.
    \end{answer}
\end{qaexchange}

\begin{qaexchange}{Infections \& Immunity}
    \metadata{topic={COVID-19 Treatment Protocol}, source={Ray Peat Forum}}

    \begin{question}
        What would be your go-to medicines/food etc, if you started to develop \enquote{COVID} like symptoms like shortness of breath and fever?
    \end{question}

    \begin{answer}
      Aspirin, antihistamines, and antibiotics (azithromycin has been tested in covid), vitamin D, milk, orange juice, nebulized 4\% saline, lidocaine (nebulized or oral), progesterone.
    \end{answer}
\end{qaexchange}

\begin{standalonequote}{Infections \& Immunity}
    \metadata{topic={Lyme Disease Treatment, Antibiotics}, source={Ray Peat Forum}}

    \begin{answer}
      Yes, the spirochetes can be killed with a couple of weeks of antibiotics. But there are some doctors who specialize in permanent treatments. A few doctors have used the internet to create a chronic Lyme cult. Doxycycline, penicillin G and amoxycycline all usually work well.
    \end{answer}
\end{standalonequote}

\section{Injuries \& Healing}

\begin{qaexchange}{Injuries}
    \metadata{topic={Fracture Healing}, source={Ray Peat Forum}}

    \begin{question}
        What can be done to help ensure proper healing and union of a potential fracture? Specifically fractures that are known for delayed healing and nonunion (such as John's Fracture of the foot). Thinking red light, \ce{CO2}, thyroid, vitamin D \& enough calcium, aspirin would be good recommendations. Anything else?
    \end{question}

    \begin{answer}
      Ultrasound treatments can stimulate healing.
    \end{answer}
\end{qaexchange}

\begin{standalonequote}{Injuries \& Healing}
    \metadata{topic={Concussion Treatment, Neurosteroids}, source={Ray Peat Forum}}

    \begin{answer}
      Pregnenolone and progesterone are the most important things for the after-effects of concussions. Pregnenolone, if it's pure, doesn't have any side effects, so it's safe in large doses. 100 mg per day is usually an effective amount (I took 3000 to 4000 mg per day for a year, with no side effects). Progesterone is usually helpful in moderate, physiological amounts, maybe 20 to 30 mg per day (with food), but more would be o.k. if the effects seem better. The frontal lobes of the brain are involved in focussing attention and planning, and these hormones stabilize the major frontal lobe nerves. Vitamin D and calcium are essential for the protective effects of these hormones, so it would be good to have a vitamin D blood test. Many people, when they don't get much direct sunlight exposure, need about 5000 i.u. of vitamin D3 per day to keep the blood level around the normal 50 ng/ml.
    \end{answer}
\end{standalonequote}

\begin{qaexchange}{Injuries \& Healing}
    \metadata{topic={Tissue Regeneration Potential}, source={Ray Peat Forum}}

    \begin{question}
        So after reading this study done on crayfish saying that they have the ability to regrow new eye after it has been lost, as well as frog embryos, it made me wonder if humans can have the same regrowth potential?
    \end{question}

    \begin{answer}
      Regrowth of a well formed finger tip occurs if the wound is protected from the air, probably allowing high \ce{CO2} accumulation. A friend's dog appeared to have lost an eye after a fight—just a red socket was visible; he put it on a fresh goat milk diet, and what looked like a small eye appeared in the socked after a couple weeks, and after several weeks it had a normal eye, with vision.
    \end{answer}
\end{qaexchange}

\section{Pain \& Inflammation}

\begin{qaexchange}{Pain}
    \metadata{topic={Procaine for Joint Pain}, source={Ray Peat Forum}}

    \begin{question}
        Do you have any experience with Gerovital supplements of procaine hydrocloride?
    \end{question}

    \begin{answer}
      Yes, recently a woman in her sixties with multiple joint, muscle, and nerve pains started taking it in the conventional Ana Aslan form, with additives, and now she's limber, and says her joints are rubbery and painless. I have seen amazing things with just plain procaine, and would talk about it more, except that it can be very allergenic for some people. Lidocaine has enough overlap with it to be valuable.
    \end{answer}
\end{qaexchange}

\begin{standalonequote}{Pain \& Inflammation}
    \metadata{topic={CRPS Treatment}, source={Ray Peat Forum}}

    \begin{answer}
      Sometimes it's relieved by better intestinal function, with fiber, thyroid, etc. Vitamin D, calcium, and magnesium are most likely to help.
    \end{answer}
\end{standalonequote}

\chapter{Therapeutic Interventions}

\section{Pharmaceutical Interventions}

\begin{standalonequote}{Pharmaceutical Interventions}
    \metadata{topic={Gadolinium MRI Contrast Toxicity}, source={Ray Peat Forum}}

    \begin{answer}
       From the animal studies, it damages all the essential organs. Competent workers can make fine MRI images without it. It's a holdover from x-ray times, by people who don't understand the principle of MRI. 
    \end{answer}
\end{standalonequote}

\begin{standalonequote}{Pharmaceutical Interventions}
    \metadata{topic={Aluminum Vaccine Adjuvants}, source={Ray Peat Forum}}

    \begin{answer}
      Anti-inflammatories can protect against everything but once you have injected aluminum in the muscle, you are going to have life-long effects of that. Aluminium in the muscle has unpredictable effects, that often particles are sent up the nerve fiber from the muscle to the brain, transported in a specific way into the brain where they cause continuing amplification of inflammation. So I wouldn't encourage anyone to think that they can protect themselves from aluminium-containing injection.
    \end{answer}
\end{standalonequote}

\begin{qaexchange}{Pharmaceutical Interventions}
    \metadata{topic={Vaccine Study Placebos}, source={Ray Peat Forum}}

    \begin{question}
        Is \enquote{saline placebo} in vaccine studies really a saline placebo and nothing else?
    \end{question}

    \begin{answer}
      Technically, \enquote{saline} could refer to a variety of salts in solution, and many safety tests have deliberately used an inflammation-producing \enquote{placebo.} It's hard to find anything honest in the vaccine business.
    \end{answer}
\end{qaexchange}

\begin{qaexchange}{Pharmaceutical Interventions}
    \metadata{topic={CT Scan Radiation Protection}, source={Ray Peat Forum}}

    \begin{question}
        What steps could one take prior to scans (specifically concerning CT scan on the ear/brain) to help mitigate toxic effects to mitochondria? I have heard methylene blue, aspirin, and caffeine might be helpful.
    \end{question}

    \begin{answer}
      Since the brain is most sensitive to radiation damage, the protective substances should be systemic. Thyroid hormone, magnesium, aspirin, and coffee are other protective substances.
    \end{answer}
\end{qaexchange}

\begin{qaexchange}{Pharmaceutical Interventions}
    \metadata{topic={Vaccine Safety, Adjuvants}, source={Ray Peat Forum}}

    \begin{question}
        Is it possible to make a safe vaccine without an adjuvant?
    \end{question}

    \begin{answer}
      Rubbing an inactivated virus into the skin will activate the immune system, probably much more effectively than injecting it, and the more organized reaction is likely to involve much lower production of antibodies. Since Ehrlich and his magic bullet pharmaceutical advertising slogan, the magic bullet antibody has been treated as the essence of immunity. Metchnikoff's developmental/embryological way of seeing immunity was biologically correct, but it didn't fit well into the \enquote{each disease has its appropriate drug} business plan.
    \end{answer}
\end{qaexchange}

\begin{standalonequote}{Pharmaceutical Interventions}
    \metadata{topic={Novavax Vaccine Adjuvant Concerns}, source={Ray Peat Forum}}

    \begin{answer}
      Any adjuvant can be a problem. In principle, a toxic protein is better than the RNA for a toxic protein, but it takes a long time to know the risks and benefits of any vaccine.
    \end{answer}
\end{standalonequote}

\begin{standalonequote}{Pharmaceutical Interventions}
    \metadata{topic={Anti-inflammatories, Antiviral Effects}, source={Ray Peat Forum}}

    \begin{answer}
      The antiinflammatories and local anesthetics themselves have some antiviral effects--aspirin, Benadryl, procaine, lidocaine, belladonna, cascara, etc. Have you tried topical Lanacaine or TigerBalm?
    \end{answer}
\end{standalonequote}

\begin{standalonequote}{Pharmaceutical Interventions}
    \metadata{topic={MRI Contrast Agents, Gadolinium}, source={Ray Peat Forum}}

    \begin{answer}
      The MRI technology was developed (in the 1970s, by Ray Damadian) specifically because of its ability to distinguish tumor tissue from normal tissue without using a contrast medium, because of the appearance of the cell water even if the shape of the tissue hasn't changed, but since the tumors they are looking for form actual lumps, it doesn't take any specialized MRI technique.
    \end{answer}
\end{standalonequote}

\subsection{Antihistamines}

\begin{standalonequote}{Antihistamines}
    \metadata{topic={Cetirizine Toxicity}, source={Email Wiki}}

    \begin{answer}
        I avoid drugs that contain chlorine or fluorine, because of the risk to the liver. [Clarification] Our enzymes aren't designed for the combination of chlorine with carbon molecules.
    \end{answer}
\end{standalonequote}

\begin{emailexchange}{Antihistamines}
    \metadata{topic={Diphenhydramine and Cyproheptadine}, source={Ray Peat Forum}}

    \begin{question}
        Do you know if diphenhydramine is a safe thing to consume on a daily basis or at least several times a week in order to keep my histamine under control, and what is the mechanism of action there? Why would antihistamines help me with motion sickness?
    \end{question}

    \begin{answer}
      Histamine activates the parasympathetic nervous system, which is involved in nausea. Pure diphenhydramine is better than Benadryl. Cyproheptadine is another antihistamine. Histamine supports wakefulness, and too little lowers alertness.
    \end{answer}
	
    \begin{question}
        Do you think Cyproheptadine would be safer than diphenhydramine? Would the minimal amount of these substances be safe to use long term?
    \end{question}

    \begin{answer}
      My experience has been that cyproheptadine had a reparative effect, so that after a few days using a milligram, a half or fourth worked as well, and then I didn't need it again for a long time.
    \end{answer}
\end{emailexchange}

\subsection{Aspirin}

\begin{standalonequote}{Aspirin}
    \metadata{topic={Aspirin Preparation Method}, source={Email Wiki}}

    \begin{answer}
        I don't know anyone who has a stomach reaction when they dissolve the aspirin in hot water, and then take it with food. There are alternatives, such as magnesium salsalate or just plain salicylic acid (which should be used dissolved and with food). People using it with cancer usually take a daily total of 6 grams or more. Vitamin K protects against bleeding and other effects of prolonged aspirin use.
    \end{answer}
\end{standalonequote}

\begin{standalonequote}{Aspirin}
    \metadata{topic={Aspirin Product Forms}, source={Email Wiki}}

    \begin{answer}
        I mostly use the pure crystals, but in Mexico I use the Bayer tablets.
    \end{answer}
\end{standalonequote}

\begin{standalonequote}{Aspirin}
    \metadata{topic={Aspirin Expiration And Storage}, source={Email Wiki}}

    \begin{answer}
        If the aspirin smells like vinegar it's decomposing, otherwise the expiration date doesn't matter. I have some aspirin, USP, that's at least 10 years old that's still good.
    \end{answer}
\end{standalonequote}

\begin{qaexchange}{Aspirin}
    \metadata{topic={Aspirin Tablet Fillers}, source={Email Wiki}}

    \begin{question}
        Starch in aspirin tablets problematic?
    \end{question}

    \begin{answer}
        Only if it causes symptoms such as hemorrhoids, asthma, or headaches.
    \end{answer}
\end{qaexchange}

\begin{standalonequote}{Aspirin}
    \metadata{topic={Veterinary Aspirin Purity}, source={Email Wiki}}

    \begin{answer}
        They sell aspirin, USP, for animals, which is the same as used for people, except that it doesn't contain the toxic additives of the tablets.
    \end{answer}
\end{standalonequote}

\begin{standalonequote}{Aspirin}
    \metadata{topic={Aspirin In Bathwater}, source={Email Wiki}}

    \begin{answer}
        Some therapists have advocated it.
    \end{answer}
\end{standalonequote}

\begin{standalonequote}{Aspirin}
    \metadata{topic={High-Dose Aspirin Effects}, source={Email Wiki}}

    \begin{answer}
        People sometimes take that much aspirin attempting suicide, but I don't think that's relevant to a conclusion about antinociception.
    \end{answer}
\end{standalonequote}

\begin{standalonequote}{Aspirin}
    \metadata{topic={Aspirin Dosage Context}, source={Email Wiki}}

    \begin{answer}
        It depends on the context; aspirin makes you need more vitamin K, even when you aren't using much. People who use aspirin for arthritis or cancer often take several grams a day.
    \end{answer}
\end{standalonequote}

\begin{standalonequote}{Aspirin}
    \metadata{topic={Very High Aspirin Dose}, source={Email Wiki}}

    \begin{answer}
        7000 mg is a lot, and it's very important to take vitamin K with aspirin.
    \end{answer}
\end{standalonequote}

\begin{standalonequote}{Aspirin}
    \metadata{topic={Aspirin Lowering Stress Hormones}, source={Email Wiki}}

    \begin{answer}
        When aspirin and niacinamide lower the temperature I think it's because they lower the stress hormones.
    \end{answer}
\end{standalonequote}

\begin{standalonequote}{Aspirin}
    \metadata{topic={Aspirin Bulk Purchasing}, source={Email Wiki}}

    \begin{answer}
        USP aspirin from any source is usually good. I buy it by the kilo.
    \end{answer}
\end{standalonequote}

\begin{standalonequote}{Aspirin}
    \metadata{topic={Aspirin Allergy}, source={Email Wiki}}

    \begin{answer}
        I know people who had lifelong \enquote{aspirin allergy} who now use it regularly. I think part of it is the metabolic problems caused by PUFA and low thyroid function.
    \end{answer}
\end{standalonequote}

\begin{qaexchange}{Aspirin}
    \metadata{topic={Aspirin Maximum Dosage}, source={Email Wiki}}

    \begin{question}
        Safe upper limit?
    \end{question}

    \begin{answer}
        It depends on the context; aspirin makes you need more vitamin K, even when you aren't using much. People who use aspirin for arthritis or cancer often take several grams a day.
    \end{answer}
\end{qaexchange}

\begin{qaexchange}{Aspirin}
    \metadata{topic={Aspirin As Anticoagulant}, source={Email Wiki}}

    \begin{question}
        Right dosage of aspirin for use as anticoagulant?
    \end{question}

    \begin{answer}
        Clotting time must be measured when taking any anticoagulant.
    \end{answer}
\end{qaexchange}

\begin{qaexchange}{Aspirin}
    \metadata{topic={Aspirin Before Surgery}, source={Email Wiki}}

    \begin{question}
        Aspirin prior to surgery? (testicular cancer)
    \end{question}

    \begin{answer}
        If vitamin K is used generously, aspirin wouldn't be likely to cause a bleeding problem. Progesterone could be used to lower luteinizing hormone before surgery, if that's high. I think tetracycline (or a derivative) would be helpful to use with the aspirin. It's important to check thyroid function and vitamin D.
    \end{answer}
\end{qaexchange}

\begin{standalonequote}{Aspirin}
    \metadata{topic={Aspirin For Acne Inflammation}, source={Email Wiki}}

    \begin{answer}
        A solution of aspirin in water on the skin helps with the inflammation, and is mildly germicidal.
    \end{answer}
\end{standalonequote}

\begin{qaexchange}{Aspirin}
    \metadata{topic={Daily Dose for PUFA Depletion}, source={Ray Peat Forum}}

    \begin{question}
        How much aspirin do you recommend daily, just to add to ones diet while cleaning out PUFAs while generally healthy and of normal weight?
    \end{question}

    \begin{answer}
        Under those conditions, I don't normally recommend it, when good fruits and other fresh foods are available, but along with a small amount of vitamin E it might be protective to have about 50 to 100 mg of pure aspirin for the occasional surges of PUFA and prostaglandins that can be produced by stress.
    \end{answer}
\end{qaexchange}

\begin{qaexchange}{Aspirin}
    \metadata{topic={Aspirin Degradation Safety}, source={Ray Peat Forum}}

    \begin{question}
        I have just received some animal aspirin I ordered, but it smells like vinegar. I understand this could mean it is decomposing. Does this make it unsafe to take?
    \end{question}

    \begin{answer}
      It breaks down into salicylic acid and acetic acid, and both of those are safe.
    \end{answer}
\end{qaexchange}

\begin{standalonequote}{Aspirin}
    \metadata{topic={Aspirin, Oxygen Consumption, Niacinamide}, source={Ray Peat Forum}}

    \begin{answer}
      Aspirin increases oxygen consumption; although niacinamide can reduce excessive lipolysis, I don't know whether it would lower resting lipolysis.
    \end{answer}
\end{standalonequote}

\begin{qaexchange}{Aspirin}
    \metadata{topic={Aspirin Form, Stability}, source={Ray Peat Forum}}

    \begin{question}
        Is there any difference between the crystal and the powder form of acetylsalicylic acid (ASA), particulary with regard to stability and Absorption? Do you think its an good idea to take some ASA every day, in generell e.g. to prevent cancer?
    \end{question}

    \begin{answer}
      I think a little aspirin, regularly if not daily, is good prevention, if you are sure to get enough vitamin K, to prevent excess bleeding. The amount depends on how you react to it, and can change as your metabolism adjusts. Taking some at bedtime can be very helpful for sleeping; sometimes I take about 500 mg at night, but other times just a little. I think the crystals are more stable, but I keep the big container (a multi-year supply) in the freezer, and keep out enough for a couple of months. The powdered forms developed an acetic acid smell with time, the crystals don't.
    \end{answer}
\end{qaexchange}

\begin{standalonequote}{Aspirin}
    \metadata{topic={Maximum Aspirin Dosage}, source={Ray Peat Forum}}

    \begin{answer}
       I have known a few people who took 4 to 6 grams per day for several years for arthritis or cancer, but I usually think of 1000 to 1500 mg per day as a maximum safe dose, but it depends on things like age and thyroid function; with low kidney function, even that much could accumulate to a toxic level.
    \end{answer}
\end{standalonequote}

\begin{qaexchange}{Aspirin}
    \metadata{topic={Salicylic Acid vs Aspirin}, source={Ray Peat Forum}}

    \begin{question}
        Is salicylic acid as good as aspirin?
    \end{question}

    \begin{answer}
      Yes, for most things.
    \end{answer}
\end{qaexchange}

\begin{qaexchange}{Aspirin}
    \metadata{topic={Aspirin Brands}, source={Ray Peat Forum}}

    \begin{question}
        What brand of aspirin do you recommend/use? Is Bayer 100 mg good?
    \end{question}

    \begin{answer}
      I use plain crystals of aspirin, USP. I think Bayer is o.k., though I don't know what excipients they use.
    \end{answer}
\end{qaexchange}

\begin{standalonequote}{Aspirin}
    \metadata{topic={Aspirin Multiple Benefits}, source={Ray Peat Forum}}

    \begin{answer}
      If a newly discovered substance had aspirin's antiinfective, anticancer, antistress, antioxidant and antiinflammatory actions it would be the most researched substance in history.
    \end{answer}
\end{standalonequote}

\begin{standalonequote}{Aspirin}
    \metadata{topic={Aspirin Degradation}, source={Ray Peat Forum}}

    \begin{answer}
      Aspirin breaks down into acetic acid and salicylic acid; if it's extremely sour, a little baking soda would neutralize the acid, and salicylic acid has almost the same effects as aspirin.
    \end{answer}
\end{standalonequote}

\subsection{Antibiotics}

\begin{qaexchange}{Antibiotics}
    \metadata{topic={Topical Antibiotics For Acne}, source={Email Wiki}}

    \begin{question}
        Best topical form of tetracycline for acne?
    \end{question}

    \begin{answer}
        I think the easiest thing would be an over-the-counter neomycin lotion.
    \end{answer}
\end{qaexchange}

\begin{standalonequote}{Antibiotics}
    \metadata{topic={Antibiotic Intermittent Use}, source={Email Wiki}}

    \begin{answer}
        No, I just do it occasionally.
    \end{answer}
\end{standalonequote}

\begin{standalonequote}{Antibiotics}
    \metadata{topic={Antibiotic Dosing By Symptoms}, source={Email Wiki}}

    \begin{answer}
        For myself, I judge by symptoms; if I feel an effect from a first dose, I take a smaller dose, usually 100mg, the next time, and similar amounts as long as the symptom is decreasing, and when I don't notice any symptom, I take a few smaller doses.
    \end{answer}
\end{standalonequote}

\begin{standalonequote}{Antibiotics}
    \metadata{topic={Antibiotics And Vitamin K}, source={Email Wiki}}

    \begin{answer}
        Since most people get some vitamin K from intestinal bacteria, it's important to eat liver or to take a K supplement if you use antibiotics for a long time. After a first big dose or two, you should be able to sense when you have enough in your tissues; it has a noticeable smell or sensation while exhaling. I have found that 3 doses of 100mg per day for a few days is usually enough, after one or two bigger doses.
    \end{answer}
\end{standalonequote}

\begin{standalonequote}{Antibiotics}
    \metadata{topic={Antibiotics For Intestinal Bacteria}, source={Email Wiki}}

    \begin{answer}
        Aspirin has a mild germicidal effect. Sometimes 30 to 50 milligrams of tetracycline or penicillin can help. Flowers of sulfur, a pinch a day for a few days will often establish a new flora.
    \end{answer}
\end{standalonequote}

\begin{qaexchange}{Antibiotics}
    \metadata{topic={Rifaximin Short-Term Safety}, source={Ray Peat Forum}}

    \begin{question}
        Could rifaximin be helpful for the liver?
    \end{question}

    \begin{answer}
      Rifaximin is probably safe for short term use.
    \end{answer}
\end{qaexchange}

\begin{qaexchange}{Antibiotics}
    \metadata{topic={Antibiotic Resistance Causes}, source={Ray Peat Forum}}

    \begin{question}
        I was wondering your opinion on antibiotic resistance? The mainstream ideas being that frequent, short doses of antibiotics cause antibiotic resistance, which seems common for news companies to report about every now and then. I've heard you say you use antibiotics occasionally and rather low doses. Do you not believe this use of antibiotics can cause antibiotic resistance?
    \end{question}

    \begin{answer}
      Most of the resistant bacteria are developed in hospitals and industrial animal production, where constant, uninterrupted, use of antibiotics invariably accumulates the resistant strains. Doctors and corporations, knowingly misusing them for increased profit, are the culprits.
    \end{answer}
\end{qaexchange}

\begin{standalonequote}{Antibiotics}
    \metadata{topic={Penicillin Tablet Administration}, source={Ray Peat Forum}}

    \begin{answer}
       I usually suck on it slowly, to avoid the risk of the tablet sticking to my stomach membrane. 
    \end{answer}
\end{standalonequote}

\begin{standalonequote}{Antibiotics}
    \metadata{topic={Penicillin Comprehensive Protocol}, source={Ray Peat Forum}}

    \begin{answer}
       Sometimes I notice that a few doses of 50 to 100 mg of penicillin will stop a particular symptom; a few times, for a distinct infection, I have used a standard dose of 250 mg 3 or 4 times in a day, stopping as soon as the symptom is gone, usually by the second day. The antiseptic fibrous foods (raw carrot, cooked mushrooms or bamboo shoots) eaten regularly, and avoiding the very rottable indigestible foods such as green salads, give chronic protection against bacteria. Avoiding excess phosphate lowers stress and inflammation, and getting plenty of calcium and vitamin D helps to balance the phosphate. Some aspirin at bedtime might be helpful. 
    \end{answer}
\end{standalonequote}

\begin{standalonequote}{Antibiotics}
    \metadata{topic={Erythromycin Anti-inflammatory}, source={Ray Peat Forum}}

    \begin{answer}
       The antibiotics erythromycin and tetracycline are anti-inflammatory, and might work better than penicillin. 
    \end{answer}
\end{standalonequote}

\begin{standalonequote}{Antibiotics}
    \metadata{topic={Antibiotic Safety General}, source={Ray Peat Forum}}

    \begin{answer}
       Antibiotics vary in toxicity, but in general I think it's best to use a minimally effective dose if it's to be continued very long. I think tetracycline, erythromycin, and penicillin are fairly safe. 
    \end{answer}
\end{standalonequote}

\begin{standalonequote}{Antibiotics}
    \metadata{topic={Pen-V-K Dosing Protocol}, source={Ray Peat Forum}}

    \begin{answer}
       I use 250 mg tablets of Pen Vi K, which aren't expensive, and take about a fourth of a tablet at a time, maybe twice a day until a symptom is gone, usually one or two days. 
    \end{answer}
\end{standalonequote}

\begin{standalonequote}{Antibiotics}
    \metadata{topic={Penicillin for Allergies}, source={Ray Peat Forum}}

    \begin{answer}
       It isn't habit forming. The antibiotics erythromycin and tetracycline are anti-inflammatory, and might work better than penicillin. An allergy can cause swelling of the surface of the eye, often it's from something in the air, but it's possible that something you ate was involved. Vitamin K is involved in some essential chemical processes in the brain, besides helping to produce metabolic energy; I think it might help with relaxation, too. 
    \end{answer}
\end{standalonequote}

\begin{standalonequote}{Antibiotics}
    \metadata{topic={Testing Antibiotic Effects}, source={Ray Peat Forum}}

    \begin{answer}
       I think it's good to try a small amount of an antibiotic at first, watching for a general effect such as mood; I've noticed that when I first have an odor-like sensation after taking a little penicillin (when it reaches a certain level in the tissues), it comes with a general sense of comfort. 
    \end{answer}
\end{standalonequote}

\begin{standalonequote}{Antibiotics}
    \metadata{topic={Antibiotic Preference Order}, source={Ray Peat Forum}}

    \begin{answer}
       My own preference for antibiotics would be in the order Pen-V-K (about 30 to 50 mg at a time, with some carrot), erythromycin, tetracycline, and neomycin (for example Kaomycin). All of the antibiotics are somewhat toxic to people, but at a certain level, they can suppress bacteria without noticeable toxicity for the person. The goal is to establish a better internal ecosystem. Chronic sinus infections are usually the result of chronic irritation of the intestine. 
    \end{answer}
\end{standalonequote}

\begin{standalonequote}{Antibiotics}
    \metadata{topic={Antibiotic Dosing with Vitamin K}, source={Ray Peat Forum}}

    \begin{answer}
       I usually break the tablets up, and use fourths or halves, at intervals according to need. It's important to get some vitamin K\textsubscript{1} or K\textsubscript{2} when you use an antibiotic (liver or kale, or supplements).
    \end{answer}
\end{standalonequote}

\begin{standalonequote}{Antibiotics}
    \metadata{topic={Penicillin Intermittent Dosing}, source={Ray Peat Forum}}

    \begin{answer}
       I think penicillin is most effective in such situations when it's used intermittently, 2 to 4 days at a time, at intervals of about a week. 
    \end{answer}
\end{standalonequote}

\begin{standalonequote}{Antibiotics}
    \metadata{topic={Minocycline Dosing}, source={Ray Peat Forum}}

    \begin{answer}
       People often use minocycline for a long time, but usually 50 or 100 mg in a day. It could possibly make you sensitive to sunlight if it accumulates in your body. 
    \end{answer}
\end{standalonequote}

\begin{standalonequote}{Antibiotics}
    \metadata{topic={Minocycline Safety}, source={Ray Peat Forum}}

    \begin{answer}
       I think minocycline is safer than doxycycline, and is very safe. It is antiinflammatory, and has some protective effect against cancer. 
    \end{answer}
\end{standalonequote}

\begin{standalonequote}{Antibiotics}
    \metadata{topic={Low-Dose Antibiotic Protocol}, source={Ray Peat Forum}}

    \begin{answer}
       Sometimes the antiseptic foods (raw carrots, cooked mushrooms or bamboo shoots), eaten regularly, will take care of it. I have found, for myself, that small doses of penicillin, such as 200,000 units 3 times a day, are effective in just a day or two. The tetracyclines are more often used for acne, because they have a general antiinflammatory effect, besides the germicidal action. The conventional doses are usually unnecessarily large, based on an assumption that the person has no functional immune system. Large doses of antibiotics have a slight toxic effect on human cells, so I think it's best to use them according to results, rather than formula. 
    \end{answer}
\end{standalonequote}

\begin{qaexchange}{Antibiotics}
    \metadata{topic={Amoxicillin Safety}, source={Ray Peat Forum}}

    \begin{question}
        Do you think amoxicillin is one of the safer antibiotics?
    \end{question}

    \begin{answer}
      I would call it one of the less risky.
    \end{answer}
\end{qaexchange}

\begin{standalonequote}{Antibiotics}
    \metadata{topic={Augmentin Dangers}, source={Ray Peat Forum}}

    \begin{answer}
      I think augmentin is among the more dangerous antibiotics
    \end{answer}
\end{standalonequote}

\begin{qaexchange}{Antibiotics}
    \metadata{topic={Tetracyclines Anti-inflammatory}, source={Ray Peat Forum}}

    \begin{question}
        Do you know if subantimicrobial doses of the tetracyclines are enough to lower prostaglandin synthesis and phospholipase A2? 
    \end{question}

    \begin{answer}
      The tetracyclines and macrolides such as azithromycin are antiinflammatory; aspirin has similar inhibitory effects.
    \end{answer}
\end{qaexchange}

\begin{standalonequote}{Antibiotics}
    \metadata{topic={Erythromycin Dosage}, source={Ray Peat Forum}}

    \begin{answer}
      People vary greatly in sensitivity, but I find that 250 mg once a day has a strong effect on the bowel.
    \end{answer}
\end{standalonequote}

\subsection{Anti-Serotonin Drugs}

\begin{standalonequote}{Anti-Serotonin Drugs}
    \metadata{topic={SSRI Withdrawal Support}, source={Email Wiki}}
    \begin{note}
        Decreasing SSRI Dose
    \end{note}

    \begin{answer}
        It takes time to adapt to decreasing those drugs, keeping sugar up and inflammation down, including bag breathing, should help. Starting with a little, a sixth or fourth of a tablet, of cynoplus in the evening would be the best way to try it.
    \end{answer}
\end{standalonequote}

\begin{standalonequote}{Anti-Serotonin Drugs}
    \metadata{topic={Antidepressant Withdrawal Protocol}, source={Email Wiki}}
    \begin{note}
        Weaning Off Anti-Depressants
    \end{note}

    \begin{answer}
        Keeping the metabolic rate and cholesterol up is important, so that repair and adaptation will be quick. Progesterone reduces pain and anxiety, and pregnenolone would be the most convenient supplement for men, but it's hard to find products without allergens. Combining progesterone and DHEA or testosterone can produce the stabilizing effect without suppressing the libido. Benadryl and cyproheptadine are probably both helpful. Withdrawal from morphine and SSRIs and migraine involve some similar processes.
    \end{answer}
\end{standalonequote}

\begin{standalonequote}{Anti-Serotonin Drugs}
    \metadata{topic={Pregnenolone Dosing For Withdrawal}, source={Email Wiki}}
    \begin{note}
        Weaning Off Anti-Depressants (Continued)
    \end{note}

    \begin{answer}
        It depends on how much pregnenolone you can assimilate. People would use progesterone in amounts needed to stop the withdrawal symptoms, but pregnenolone doesn't have the powerful effects of progesterone, even in multi-gram quantities, so it's just a matter of seeing what it can do. As I understand the mechanism (migraine, withdrawal, etc.), estrogen-histamine-serotonin rise on a background of hypothyroid liver malfunction, cytomel (and/or sugar, selenium, B vitamins) allows the liver and other detoxifying systems to lower them, and the lower they are, the less progesterone or pregnenolone it takes to block the symptoms.
    \end{answer}
\end{standalonequote}

\begin{standalonequote}{Anti-Serotonin Drugs}
    \metadata{topic={Cyproheptadine Safety}, source={Email Wiki}}

    \begin{answer}
        I think cyproheptadine is a safe antiserotonin drug
    \end{answer}
\end{standalonequote}

\begin{standalonequote}{Anti-Serotonin Drugs}
    \metadata{topic={Mirtazapine Antiserotonin Effects}, source={Email Wiki}}

    \begin{answer}
        I think its antiserotonin effects might be helpful, but I haven't tried it myself.
    \end{answer}
\end{standalonequote}

\begin{standalonequote}{Anti-Serotonin Drugs}
    \metadata{topic={Tianeptine Side Effects}, source={Email Wiki}}

    \begin{answer}
        I have known a few people who had very good results with tianeptine, and a couple who got side effects from it. I think any of the antiserotonin drugs will eventually cause side effects, and should only be used until a problem is corrected, for example when an enlarged pituitary is normalized. I think the same effects can be produced with nutrition and hormones, without the possible problems.
    \end{answer}
\end{standalonequote}

\begin{qaexchange}{Anti-Serotonin Drugs}
    \metadata{topic={Antiserotonin Drugs Duration}, source={Email Wiki}}

    \begin{question}
        Can anti-serotonin drugs permanently fix a problem, even if taken only for a short while?
    \end{question}

    \begin{answer}
        Yes, but it's important to keep adjusting thyroid and progesterone according to temperature, pulse, etc.
    \end{answer}
\end{qaexchange}

\begin{standalonequote}{Anti-Serotonin Drugs}
    \metadata{topic={Ritanserin}, source={Email Wiki}}

    \begin{answer}
        I haven't had any experience with ritanserin, don't recommend it, and don't recall discussing it.
    \end{answer}
\end{standalonequote}

\begin{standalonequote}{Anti-Serotonin Drugs}
    \metadata{topic={Cyproheptadine Topical Dosing}, source={Email Wiki}}

    \begin{answer}
        It would be hard to regulate the dose via the skin.
    \end{answer}
\end{standalonequote}

\begin{standalonequote}{Anti-Serotonin Drugs}
    \metadata{topic={Cyproheptadine Initial Dosing}, source={Email Wiki}}

    \begin{answer}
        It's good to start with about half a milligram, at bedtime, to judge its effects when sedation isn't risky.
    \end{answer}
\end{standalonequote}

\begin{standalonequote}{Anti-Serotonin Drugs}
    \metadata{topic={Cyproheptadine For Cancer}, source={Email Wiki}}

    \begin{answer}
        Cyproheptadine, 2 to 4 mg at bedtime, would help with his sleep as well as the cancer. It also has calcium blocking action, aldosterone antagonism, and antagonizes serotonin's antidiuretic effect.
    \end{answer}
\end{standalonequote}

\begin{standalonequote}{Serotonin}
    \metadata{topic={Serotonin Gut Production}, source={Email Wiki}}
    \begin{note}
        SSRIs Causing Gut Problems
    \end{note}

    \begin{answer}
        The gut makes 95\% of serotonin, which is the main promoter of stress hormones, inflammation, pain, and anxiety.
    \end{answer}
\end{standalonequote}

\begin{standalonequote}{Serotonin}
    \metadata{topic={Serotonin Metabolism Timing}, source={Email Wiki}}

    \begin{answer}
        Serum serotonin fluctuates according to intestinal irritation, but for the average to change very much it's necessary for the liver and brain to adapt, and that usually takes a few months. Since the lungs are the main site of serotonin metabolism, an air ionizer near your bed can help.
    \end{answer}
\end{standalonequote}

\begin{standalonequote}{Serotonin}
    \metadata{topic={B\textsubscript{6} For Serotonin Metabolism}, source={Email Wiki}}

    \begin{answer}
        B\textsubscript{6} helps for turning tryptophan into niacin rather than serotonin.
    \end{answer}
\end{standalonequote}

\begin{standalonequote}{Anti-Serotonin Drugs}
    \metadata{topic={Antiserotonin Drug Options}, source={Email Wiki}}

    \begin{answer}
        Some people who haven't had ideal results from bromocriptine have had better results from tianeptine, and/or lisuride, and/or cyproheptadine.
    \end{answer}
\end{standalonequote}

\begin{standalonequote}{Anti-Serotonin Drugs}
    \metadata{topic={Cyproheptadine For Intestinal Sensitivity}, source={Email Wiki}}

    \begin{answer}
        Cyproheptadine might be helpful for reducing sensitivity to intestinal irritants.
    \end{answer}
\end{standalonequote}

\begin{qaexchange}{Anti-Serotonin Drugs}
    \metadata{topic={Antiserotonin Drug Permanence}, source={Email Wiki}}

    \begin{question}
        Can anti-serotonin drugs permanently fix a problem, even if taken only for a short while?
    \end{question}

    \begin{answer}
        Yes, but it's important to keep adjusting thyroid and progesterone according to temperature, pulse, etc.
    \end{answer}
\end{qaexchange}

\begin{standalonequote}{Serotonin}
    \metadata{topic={High Serotonin}, source={Email Wiki}}

    \begin{answer}
        It's important to know how it was measured, and what your platelet count was. Is your intestine inflamed? Since serotonin affects bone metabolism, have your serum calcium, phosphate, parathyroid hormone, vitamin D3, prolactin, and cortisol been measured?
    \end{answer}
\end{standalonequote}

\begin{standalonequote}{Serotonin}
    \metadata{topic={Serotonin Management Protocol}, source={Email Wiki}}

    \begin{answer}
        I don't think doctors know what to do for regulating serotonin. Vitamin B\textsubscript{6} helps to direct tryptophan toward niacinamide, away from serotonin. Gelatin contains no tryptophan, so things like consomme can be helpful. Raw carrots, because of their antiseptic effect, help to lower irritation and bloating. Antibiotics can be helpful, when the small intestine is overgrown with bacteria. Thyroid supplementation will lower cholesterol. Some people get very sleepy with just two milligrams of Periactin, so I think it's good to start with one mg. the first night. Two milligrams can make a big difference, and when symptoms stop the effects can last for days without using it.
    \end{answer}
\end{standalonequote}

\begin{qaexchange}{Anti-Serotonin}
    \metadata{topic={Cyproheptadine and Amitriptyline}, source={Ray Peat Forum}}

    \begin{question}
        I currently take cyproheptadine. From my understanding, amitriptyline and cyproheptadine have opposing effects on serotonin. Is this true, and could this be problematic, if so?
    \end{question}

    \begin{answer}
        Both of those are appetite stimulants that tend to cause weight gain. Serotonin tends to cause anorexia. The drug industry generates noise in the process of selling drugs, and \enquote{serotonin} is one of their favorite noises.
    \end{answer}
\end{qaexchange}

\begin{standalonequote}{Anti-Serotonin}
    \metadata{topic={Amitriptyline Safety}, source={Ray Peat Forum}}

    \begin{answer}
        I think it's safe.
    \end{answer}
\end{standalonequote}

\begin{standalonequote}{Anti-Serotonin}
    \metadata{topic={Amitriptyline Long Term}, source={Ray Peat Forum}}

    \begin{answer}
        Amitriptyline is an important antiinflammatory agent that helps to increase your metabolic rate by lowering serotonin's effects.
    \end{answer}
\end{standalonequote}

\begin{qaexchange}{Anti-Serotonin}
    \metadata{topic={Selegiline}, source={Ray Peat Forum}}

    \begin{question}
        What are your thoughts on Selegiline?
    \end{question}

    \begin{answer}
        It's safer than many antidepressants, it depends on the nature of the problem, though.
    \end{answer}
\end{qaexchange}

\begin{emailexchange}{Anti-Serotonin}
    \metadata{topic={Serotonin and Leadership}, source={Ray Peat Forum}}

    \begin{question}
        I heard of a study where they administered serotonin to low ranking monkeys in a pack and after administration, they became leaders in the pack. Does this have to do more with becoming more aggressive, therefore becoming more dominant in your opinion? Seems like leadership is mistaken for aggressiveness in a lot of literature and teachings I've come across.
    \end{question}

    \begin{answer}
        And in the military—leadership qualities overlap with cruelty and stupidity. A classmate of mine who talked about the importance of leadership, years later described his war crimes in Vietnam, and then was quoted as saying \enquote{I'm a professional, when the president says to shoot someone, I shoot them.}
    \end{answer}
\end{emailexchange}

\begin{qaexchange}{Antiserotonin}
    \metadata{topic={Cyproheptadine for Insomnia}, source={Ray Peat Forum}}

    \begin{question}
        I saw a study that said cyproheptadine was more effective than benzodiazepine drugs. Have you heard of anyone using it for insomnia? Or is there anything good to use for insomnia, besides benadryl?
    \end{question}

    \begin{answer}
      It works extremely well, but it's important to start with a much smaller dose than is usually recommended, 1 mg, or even less, can be effective, and 4 mg sometimes makes people goofy the next day. I know people who use it for nocturnal episodes of asthma or epilepsy-like episodes.
    \end{answer}
\end{qaexchange}

\begin{qaexchange}{Antiserotonin}
    \metadata{topic={Cyproheptadine Long-Term Use}, source={Ray Peat Forum}}

    \begin{question}
        Is cyproheptadine OK to use everyday, or will it eventually cause problems?
    \end{question}

    \begin{answer}
      I think it's safe to keep using in small amounts.
    \end{answer}
\end{qaexchange}

\begin{qaexchange}{Antiserotonin}
    \metadata{topic={Cyproheptadine Effects on Serotonin}, source={Ray Peat Forum}}

    \begin{question}
        Would daily cyproheptadine eventually permanently lower serotonin levels?
    \end{question}

    \begin{answer}
      I think it could lower stress, allowing other healing processes to correct it.
    \end{answer}
\end{qaexchange}

\begin{qaexchange}{Antiserotonin}
    \metadata{topic={Mescaline Comparison}, source={Ray Peat Forum}}

    \begin{question}
        Would small doses of something like mescaline (in cactus) be healthful, like LSD?
    \end{question}

    \begin{answer}
      One of the old medical writers in Mexico reported that the local people used it for treating heart disease. I think that probably relates to an antistress effect. The good thing about cyproheptadine is that it's legal.
    \end{answer}
\end{qaexchange}

\begin{qaexchange}{Anti-Serotonin Drugs}
    \metadata{topic={Mianserin vs Mirtazapine}, source={Ray Peat Forum}}

    \begin{question}
         Do you think it's safer than mirtazapine for moderate-term use? I've read that mirtazapine's a \enquote{successor} to mianserin.
    \end{question}

    \begin{answer}
      Although the structure suggests that it might be safer, fibrosis has been associated with both of them, and I think it's best to concentrate on optimizing the metabolism, with thyroid, pregnenolone, progesterone, etc.
    \end{answer}
\end{qaexchange}

\begin{qaexchange}{Anti-Serotonin Drugs}
    \metadata{topic={Aromatase Inhibitor Toxicity}, source={Ray Peat Forum}}

    \begin{question}
         Is Arimidex or another estrogen inhibitor safe to use, or would it have other hormonal consequences?
    \end{question}

    \begin{answer}
      I think the (liver, brain) toxicity of the commercial aromatase inhibitors is too great except for treating cancer. A diet, with good thyroid function, that corrected the bloating should gradually shift the fat distribution. Hypothyroidism would greatly increase the toxicity.
    \end{answer}
\end{qaexchange}

\begin{standalonequote}{Anti-Serotonin Drugs}
    \metadata{topic={SSRI Critique}, source={Ray Peat Forum}}

    \begin{answer}
      Some of them have other actions that account for any benefit, but the advertised benefits are far from reality. Anything that injures the brain activates increased production of the neurosteroids. The power of placebos sustains medicine and psychiatry.
    \end{answer}
\end{standalonequote}

\subsection{Other Medications}

\begin{standalonequote}{Other Medications}
    \metadata{topic={Acetaminophen Toxicity}, source={Email Wiki}}
    \begin{note}
        Reaction to Excedrin
    \end{note}

    \begin{answer}
        Yes, although the aspirin and caffeine help to detoxify acetaminophen.
    \end{answer}
\end{standalonequote}

\begin{qaexchange}{Other Medications}
    \metadata{topic={Centrophenoxine For Lipofuscin}, source={Email Wiki}}

    \begin{question}
        What is your opinion on Centrophenoxine/Meclofenoxate, piracetam or DMAE, for lipofuscin removal? The former particularly seems to have interesting effects on neurotransmitters in the brain.
    \end{question}

    \begin{answer}
        I don't think there's nearly enough knowledge about its interactions with diet, stress, and hormones.
    \end{answer}
\end{qaexchange}

\begin{standalonequote}{Other Medications}
    \metadata{topic={Antimicrobial Drug Safety}, source={Email Wiki}}

    \begin{answer}
        It seems to be safe to use for a few days. [complementing] That isn't a chemical that I've had any experience with, it's just that I don't know of any reports of toxicity from it.
    \end{answer}
\end{standalonequote}

\begin{standalonequote}{Other Medications}
    \metadata{topic={Fluorouracil Absorption}, source={Email Wiki}}
    \begin{note}
        Topical Fluorouracil for Skin Cancer
    \end{note}

    \begin{answer}
        About 6\% of the fluorouracil is absorbed systemically.
    \end{answer}
\end{standalonequote}

\begin{qaexchange}{Other Medications}
    \metadata{topic={Prednisone Diabetogenic Effects}, source={Email Wiki}}

    \begin{question}
        Why do prednisone users get diabetes?
    \end{question}

    \begin{answer}
        The doses they prescribe as \enquote{replacement} are much more than the adrenals would produce, so they in themselves are diabetogenic. William Jefferies told people that, since the adrenals produce 20 mg of cortisol per day, they should take 30 or 40 mg, as a replacement dose, because only half of it is absorbed. They got fat faces quickly. Using pregnenolone, they were able to taper off the cortisol in a month or two.
    \end{answer}
\end{qaexchange}

\begin{standalonequote}{Other Medications}
    \metadata{topic={DNP Cataracts}, source={Email Wiki}}

    \begin{answer}
        The toxicity of DNP was known from the beginning of the 20th century. The cataract epidemic came on suddenly in the spring of 1935, probably because of a product with a larger dose. The cataracts caused by DNP appear within a few hours or days of taking the drug, and disappear spontaneously when the drug is stopped, more quickly with vitamin C supplement. The FDA used the cataract outbreak to get new powers. The production of permanent cataracts by estrogens and glucocorticoids hasn't led to any action at all by the FDA. 
    \end{answer}
\end{standalonequote}

\begin{standalonequote}{Other Medications}
    \metadata{topic={Lidocaine Intestinal Effects}, source={Email Wiki}}

    \begin{answer}
        Years ago I swallowed a sip of 2\% lidocaine gel, and within a few minutes felt something changing in my intestine, and from then on, without any more lidocaine, some of my bowel symptoms were gone.
    \end{answer}
\end{standalonequote}

\begin{standalonequote}{Other Medications}
    \metadata{topic={Naltrexone Mechanism}, source={Email Wiki}}

    \begin{answer}
        Bihari thinks naltrexone works by increasing endorphins, I think excess endorphins are often the problem, and the antagonist can sometimes be helpful. The endorphins differ in their effects on the two sides of the body, so when I knew two women (within the same year) who had been having mysterious one-sided symptoms for a few months before discovering that they had ovarian cancer (on the same side), I thought that the endorphins were probably involved, maybe to suppress pain on that side. Naloxone and naltrexone have some effects that aren't directly related to the endorphins, on estrogen and histamine.
    \end{answer}
\end{standalonequote}

\begin{standalonequote}{Other Medications}
    \metadata{topic={Naltrexone Dosing Protocol}, source={Email Wiki}}

    \begin{answer}
        I think it's safe to take 5 or 10 mg of naltrexone daily for a few days, but I don't think it should be used continuously; I have known people who had good results, repeating the short courses two or three times in a year.
    \end{answer}
\end{standalonequote}

\begin{standalonequote}{Other Medications}
    \metadata{topic={Piracetam Quality}, source={Email Wiki}}

    \begin{answer}
        I have heard that the quality of the product from different countries varies; I think they have preferred the one made in Belgium. It's possible that it could help with adaptation to a thyroid supplement, but it's important to use enough thyroid hormone to keep TSH low. Keeping the cholesterol in the range of 160 to 220 helps with stress, too.
    \end{answer}
\end{standalonequote}

\begin{qaexchange}{Medications}
    \metadata{topic={Hydrogen Peroxide Nebulizing}, source={Ray Peat Forum}}

    \begin{question}
        Would nebulizing a food graded hydrogen peroxide or anything else you recommended good for overall health and/or acute treatment?
    \end{question}

    \begin{answer}
        Blood has a capacity to remove a certain amount of H2O2, but that capacity is very limited, and beyond that it is very toxic.
    \end{answer}
\end{qaexchange}

\begin{standalonequote}{Medications}
    \metadata{topic={Lidocaine}, source={Ray Peat Forum}}

    \begin{note}
        Lidocaine (vs. Procaine or Benzocaine)
    \end{note}

    \begin{answer}
        I think lidocaine is probably most effective. It can be taken orally if it's pure, in doses of 50 to 100 mg. The pure powder of the HCl form is available in 100 gram bottles.
    \end{answer}
\end{standalonequote}

\begin{standalonequote}{Medications}
    \metadata{topic={Quinine HCL}, source={Ray Peat Forum}}

    \begin{answer}
        I think it should be used in extremely small amounts, as a digestive stimulant.
    \end{answer}
\end{standalonequote}

\begin{emailexchange}{Medications}
    \metadata{topic={Vaccines}, source={Ray Peat Forum}}

    \begin{question}
        I have a gut feeling that all companies will not let their employees continue working until they get mandatory vaccine against COVID-19. Of course, we know how harmful these vaccines can be. What measures can I take to protect myself and my family if we are forced to take them? Would taking a lot of aspirin and vitamin D and probably progesterone in the day of the required vaccination be helpful?
    \end{question}

    \begin{answer}
        The aluminum adjuvant's effects develop gradually over a period of days.
    \end{answer}

    \begin{question}
        So hopefully that means the harmful effects of vaccines is temporary and with good metabolism and right nutrition during this period, one can emerge from this storm as if nothing happened?
    \end{question}

    \begin{answer}
        No, a person is never again the same after reacting to an aluminum adjuvant. The official figures of the US government show clearly that the epidemic of chronic diseases began with the massive increase of vaccinations in 1989.
    \end{answer}
\end{emailexchange}

\begin{standalonequote}{Other Medications}
    \metadata{topic={Naloxone, Naltrexone Dosing}, source={Ray Peat Forum}}

    \begin{answer}
      I have seen good results from using naloxone for 3 or 4 days; naltrexone has similar effects. Doses of one milligram or less can sometimes be effective.
    \end{answer}
\end{standalonequote}

\begin{standalonequote}{Other Medications}
    \metadata{topic={Pentoxifyllin Benefits}, source={Ray Peat Forum}}

    \begin{answer}
      I think pentoxifyllin can be very useful, it's more fat soluble than caffeine.
    \end{answer}
\end{standalonequote}

\begin{standalonequote}{Other Medications}
    \metadata{topic={MDMA Therapeutic Use}, source={Ray Peat Forum}}

    \begin{answer}
      In pure form and moderate dose (e.g., 1 to 1.5 mg per kg body weight), I think it's likely to be helpful for changing the pattern of chronic stress/learned helplessness, and maybe the chronic degenerative diseases produced by inescapable stress. The production of nitric oxide is likely to be a problem with large doses or chronic use.
    \end{answer}
\end{standalonequote}

\begin{standalonequote}{Other Medications}
    \metadata{topic={Lidocaine Oral/Topical Use}, source={Ray Peat Forum}}

    \begin{answer}
       I know a few people besides myself who have had very good results from swallowing a little lidocaine; absorbing it through the skin works too, if you use enough. 
    \end{answer}
\end{standalonequote}

\begin{qaexchange}{Other Medications}
    \metadata{topic={Nitrous Oxide Inhalation}, source={Ray Peat Forum}}

    \begin{question}
        Is n2o inhalation harmful? The most I've read about is B\textsubscript{12} deficiency.
    \end{question}

    \begin{answer}
      Some of it breaks down to form NO; it isn't the most harmful anesthetic, but it can be slightly harmful.
    \end{answer}
\end{qaexchange}

\begin{qaexchange}{Other Medications}
    \metadata{topic={Serrapeptase Bleeding Risk}, source={Ray Peat Forum}}

    \begin{question}
        What are your thoughts on taking enzymes like serrapeptase?
    \end{question}

    \begin{answer}
      Could cause unwanted internal bleeding.
    \end{answer}
\end{qaexchange}

\begin{emailexchange}{Other Medications}
    \metadata{topic={Piracetam Safety}, source={Ray Peat Forum}}

    \begin{question}
        In \textit{A Biophysical Approach to Altered Consciousness}, you write positively about Piracetam. Do you have any thoughts on other drugs in this family which have been researched since you wrote this paper?
    \end{question}

    \begin{answer}
      I haven't been paying much attention to those derivatives; too much cholinergic or glutamatergic stimulation is harmful.
    \end{answer}

    \begin{question}
        Do you therefore believe habitual or occasional usage of Piracetam could be detrimental, since it can deplete choline and glutamate over time? Can this be mitigated by supplementation or is the stimulation itself problematic?
    \end{question}

    \begin{answer}
      I would be more concerned about its chronic effects on the liver.
    \end{answer}
\end{emailexchange}

\begin{standalonequote}{Other Medications}
    \metadata{topic={Ivermectin Risks}, source={Ray Peat Forum}}

    \begin{answer}
      It isn't something I would use, there are so many protective things without the risks.
    \end{answer}
\end{standalonequote}

\begin{standalonequote}{Other Medications}
    \metadata{topic={Ivermectin Brain Damage Risk}, source={Ray Peat Forum}}

    \begin{answer}
      Ivermectin has been associated with brain damage in a small percentage of users, so I don't think it's suitable for prolonged preventive use. Vitamin D, aspirin, antihistamines, aspirin, progesterone, etc. are safer.
    \end{answer}
\end{standalonequote}

\begin{emailexchange}{Other Medications}
    \metadata{topic={LSD Phospholipase A2 Activation}, source={Ray Peat Forum}}

    \begin{question}
        Do you think LSDs activation of phospholipase A2 is a problem in small doses?
    \end{question}

    \begin{answer}
      I don't think so.
    \end{answer}

    \begin{question}
        Because of the dosage or in general? 
    \end{question}

    \begin{answer}
      Because of the small effect from low doses.
    \end{answer}
\end{emailexchange}

\begin{standalonequote}{Other Medications}
    \metadata{topic={Niclosamide for Cancer}, source={Ray Peat Forum}}

    \begin{answer}
      I think it deserves more study, including for cancer treatment.
    \end{answer}
\end{standalonequote}

\begin{qaexchange}{Other Medications}
    \metadata{topic={Lactulose as Laxative}, source={Ray Peat Forum}}

    \begin{question}
        What do you think about Lactulose usage as the laxative? 
    \end{question}

    \begin{answer}
       I think it's safe unless it causes too much gas.
    \end{answer}
\end{qaexchange}

\begin{qaexchange}{Other Medications}
    \metadata{topic={Methylene Blue Dosage for Depression}, source={Ray Peat Forum}}

    \begin{question}
         What is a good daily dose of Methylene blue?
    \end{question}

    \begin{answer}
      People have told me that half a milligram stopped their depression.
    \end{answer}
\end{qaexchange}

\begin{standalonequote}{Other Medications}
    \metadata{topic={Methylene Blue Effects}, source={Ray Peat Forum}}

    \begin{answer}
      It works as a catalyst for energy production, and I think it can be very effective even in small doses analogous to the effects of thyroid hormone. I suspect that one milligram continues to have good effects for about a week.
    \end{answer}
\end{standalonequote}

\begin{standalonequote}{Other Medications}
    \metadata{topic={Tianeptine Compared to Alternatives}, source={Ray Peat Forum}}

    \begin{answer}
      The functions of tianeptine overlap with Periactin (notice the shapes of the molecules), cascara (emodin), vitamin K, and tetracycline, but the sulfur atom in tianeptine can make it allergenic for some people. I think combinations of safe things, including coffee, thyroid, pregnenolone, and aspirin, can work better than tianeptine.
    \end{answer}
\end{standalonequote}

\begin{qaexchange}{Other Medications}
    \metadata{topic={Pyrantel Pamoate as an Anthelminthic}, source={Ray Peat Forum}}

    \begin{question}
        Do you have any thoughts on the safety of pyrantel pamoate as an anthelminthic in humans?
    \end{question}

    \begin{answer}
      It seems to be safe.
    \end{answer}
\end{qaexchange}

\begin{standalonequote}{Other Medications}
    \metadata{topic={Lidocaine for Intestinal Bleeding}, source={Ray Peat Forum}}

    \begin{answer}
      Before I discovered that I couldn't tolerate starches, and was having fairly copious bowel bleeding, I experimented by taking some sips of 2\% dental lidocaine gel, and within a few minutes felt general relief, and didn't have any more bleeding for a long time (until Breyer's added gums to their ice cream).
    \end{answer}
\end{standalonequote}

\section{Natural Supplements}
\subsection{Herbs \& Botanicals}

\begin{standalonequote}{Herbs \& Botanicals}
    \metadata{topic={Aloe Vera Efficacy}, source={Email Wiki}}

    \begin{answer}
        A milliliter of real aloe juice is a strong laxative, and unless it's dried it doesn't last without preservatives. I doubt that it would be useful.
    \end{answer}
\end{standalonequote}

\begin{standalonequote}{Herbs \& Botanicals}
    \metadata{topic={Cascara Sources}, source={Email Wiki}}

    \begin{answer}
        The bulk powder from Farmalabor in Italy is the kind I like best, but US Customs can cause problems with that. Naturlich Kost Co-op, 4260 TR 628, Millersburg, OH. 44654, sells it mixed with glycerine, which is o.k.
    \end{answer}
\end{standalonequote}

\begin{standalonequote}{Herbs \& Botanicals}
    \metadata{topic={Cascara Reducing Nitric Oxide}, source={Email Wiki}}

    \begin{answer}
        I think cascara's most important effect is the reduction of the pro-inflammatory nitric oxide, which poisons mitochondrial energy production. Raw carrot or bamboo shoots can sometimes have a similar effect by reducing NO synthesis.
    \end{answer}
\end{standalonequote}

\begin{standalonequote}{Herbs \& Botanicals}
    \metadata{topic={Emodin And Cascara Sources}, source={Email Wiki}}

    \begin{answer}
        It's hard now (since the FDA's anticascara action) to find a standardized aged cascara product, but Western Botanical and (in Italy) Farmalabor are two sources that I know of. (Naturlich Kost Ko-op in Millersburg, Ohio, has cascara, but I think FDA is currently preventing them from doing business.) The Chinese rhubarb products are probably standardized, but I have never used them. An amount slightly less than a laxative dose has beneficial systemic effects.
    \end{answer}
\end{standalonequote}

\begin{standalonequote}{Herbs \& Botanicals}
    \metadata{topic={Forskolin vs Coffee}, source={Email Wiki}}
    \begin{note}
        For an Increase in cAMP Levels and Weight Loss
    \end{note}

    \begin{answer}
        I think coffee is much safer for similar purposes.
    \end{answer}
\end{standalonequote}

\begin{standalonequote}{Herbs \& Botanicals}
    \metadata{topic={Milk Thistle Liver Effects}, source={Email Wiki}}

    \begin{answer}
        The herbs can irritate the intestine, and I know someone whose liver function was worse while she was taking milk thistle--it's good to be cautious with them.
    \end{answer}
\end{standalonequote}

\begin{standalonequote}{Herbs \& Botanicals}
    \metadata{topic={Milk Of Magnesia vs Cascara}, source={Email Wiki}}

    \begin{answer}
        Milk of magnesia is very safe, but cascara has many protective biological effects.
    \end{answer}
\end{standalonequote}

\begin{qaexchange}{Herbs \& Botanicals}
    \metadata{topic={Herbs for SIBO Risks}, source={Ray Peat Forum}}

    \begin{question}
        I am thinking about using herbs to get rid of small intestinal bacterial overgrowth and leaky gut symptoms. Are these safe? What are your opinions? 
    \end{question}

    \begin{answer}
      I think the risk of allergic reaction balances any benefit from a germicidal effect. 
    \end{answer}
\end{qaexchange}

\begin{standalonequote}{Herbs \& Botanicals}
    \metadata{topic={Cascara, Carbon Dioxide Effects}, source={Ray Peat Forum}}

    \begin{answer}
      My article on cascara talked about the analogous effects of carbon dioxide, vitamin K, emodin, and the tetracyclines. Some of the experiments with Buckyballs show similar effects of a large resonant system, an electron-withdrawing effect that lowers inappropriate excitation and inflammation. A direct current flow through the body towards the oxidizing brain can be intensified or dulled by an external field.
    \end{answer}
\end{standalonequote}

\subsection{Amino Acids}

\begin{qaexchange}{Amino Acids}
    \metadata{topic={L-Tyrosine}, source={Ray Peat Forum}}

    \begin{question}
        I take 500 mg of the amino L-Tyrosine. I fell good with it, but I have increased moles from it. Is L-Tyrosine safe on the long term or it is pro-adrenaline and suppress thyroid function?
    \end{question}

    \begin{answer}
        I think it isn't safe.
    \end{answer}
\end{qaexchange}

\subsection{Other Supplements}

\begin{qaexchange}{Other Supplements}
    \metadata{topic={Internal Sulfur For Acne}, source={Email Wiki}}

    \begin{question}
        Can flowers of sulphur taken internally help with acne? Would minocycline also help?
    \end{question}

    \begin{answer}
        Either of them can help, but with prolonged use the intestine can develop sensitivity to the sulfur. causing irritation instead of stopping it.
    \end{answer}
\end{qaexchange}

\begin{qaexchange}{Other Supplements}
    \metadata{topic={Brimstone Sulfur Forms}, source={Email Wiki}}

    \begin{question}
        Follow up: Is bromstone safe?
    \end{question}

    \begin{answer}
        The people who wrote the label should be removed to fresh air immediately, their brains aren't getting enough oxygen. There are forms of sulfur that are ground, rather than precipitated, and they aren't as effective, but they aren't harmful. If it has a very strong smell, it might be contaminated; the smell should be mildly unpleasant. Ground sulfur/brimstone is commonly used in animal feed, so it isn't toxic.
    \end{answer}
\end{qaexchange}

\begin{standalonequote}{Other Supplements}
    \metadata{topic={Activated Charcoal Toxin Absorption}, source={Email Wiki}}

    \begin{answer}
        It does destroy some vitamins by oxidation, that's why he [V.V. Frolkis] used it only intermittently. Usually fibers, such as carrots or bamboo shoots, are preferable for reducing toxin absorption.
    \end{answer}
\end{standalonequote}

\begin{qaexchange}{Other Supplements}
    \metadata{topic={Ubiquinol vs Ubiquinone}, source={Email Wiki}}

    \begin{question}
        What do you think of ubiquinol supplementation? Is it dangerous?
    \end{question}

    \begin{answer}
        I would prefer to use ubiquinone; the reduced form is more likely to be interactive with iron, etc.
    \end{answer}
\end{qaexchange}

\begin{standalonequote}{Other Supplements}
    \metadata{topic={Sulfur Topical For Yeast}, source={Email Wiki}}

    \begin{answer}
        Flowers of sulfur, USP (or precipitated sulfur powder) can be mixed with a little water and applied topically to eliminate yeast\dots{}since the yeast live in water, they can interact immediately with the sulfur when it's in water.
    \end{answer}
\end{standalonequote}

\begin{standalonequote}{Other Supplements}
    \metadata{topic={Flowers Of Sulfur Dosing}, source={Email Wiki}}

    \begin{answer}
        I used a pinch, less than a sixteenth of a teaspoonful, putting it on my tongue and washing it down, just 3 or 4 days in a row. But a daily raw carrot is usually as effective, and can be used continuously.
    \end{answer}
\end{standalonequote}

\begin{standalonequote}{Other Supplements}
    \metadata{topic={Sulfur Before Antiparasitic Herbs}, source={Email Wiki}}

    \begin{answer}
        I think the first safer thing would be flowers of sulfur for about 3 days, then the herbs [wormwood, black walnut hull, and cloves] if you don't see results with the sulfur.
    \end{answer}
\end{standalonequote}

\begin{standalonequote}{Other Supplements}
    \metadata{topic={Lysine Supplement Safety}, source={Email Wiki}}

    \begin{answer}
        If its use relieves symptoms, it should be safe, but I think there's always some risk with manufactured amino acids. Taking it with a meal would reduce risk of inflammation.
    \end{answer}
\end{standalonequote}

\begin{qaexchange}{Other Supplements}
    \metadata{topic={Milk Of Magnesia Use}, source={Email Wiki}}

    \begin{question}
        Recommendations for taking milk of magnesia?
    \end{question}

    \begin{answer}
        No, just watch out for any signs of sensitivity to it. Have you seen my article on cascara? ?
    \end{answer}
\end{qaexchange}

\begin{qaexchange}{Supplements}
    \metadata{topic={Succinic Acid}, source={Ray Peat Forum}}

    \begin{question}
        I've read some unverified sources on the internet that you recommend the use of succinic acid for lead chelation. Is this true? If so, could you please elaborate on how it works and how you see it being used?
    \end{question}

    \begin{answer}
        No, I don't recommend it. It's produced endogenously.
    \end{answer}
\end{qaexchange}

\begin{qaexchange}{Supplements}
    \metadata{topic={Borax}, source={Ray Peat Forum}}

    \begin{question}
        What do you think of the therapeutic use of Borax? I noticed a burning sensation in my arthritic joints after using a few teaspoons in a bath, but then a notable improvement the next day. Is it worth continuing to use every so often?
    \end{question}

    \begin{answer}
        I think enough can be absorbed to be toxic.
    \end{answer}
\end{qaexchange}

\begin{qaexchange}{Supplements}
    \metadata{topic={Beryllium Chelation}, source={Ray Peat Forum}}

    \begin{question}
        Do you have any ideas on how someone with high levels of Beryllium in their body could get rid of it? All I can find are scientific articles discussing the use of pharmaceutical chelators and corticosteroids.
    \end{question}

    \begin{answer}
        The chelators in orange juice and grape juice seem to lack the toxic effects of some of the industrial chelators.
    \end{answer}
\end{qaexchange}

\section{Topical Applications}

\begin{emailexchange}{Topical Applications}
    \metadata{topic={Skin Exfoliation for Absorption}, source={Ray Peat Forum}}

    \begin{question}
        Would exfoliating skin before topical administration increase absorption?
    \end{question}

    \begin{answer}
      Damaged skin absorbs more than intact skin.
    \end{answer}

    \begin{question}
        Would you recommend against exfoliating the skin before applying vitamins?
    \end{question}

    \begin{answer}
      If you keep skin wet for about an hour, the dead skin cells loosen, and rub off easily with a light towel rub.
    \end{answer}
\end{emailexchange}

\subsection{Hormone Creams}

\begin{standalonequote}{Hormone Creams}
    \metadata{topic={Hormone Cream Ingredients}, source={Email Wiki}}
    \begin{note}
        Propylene Glycol, Cetearyl Alcohol and Plain Alcohol in Hormone Creams
    \end{note}

    \begin{answer}
        Plain alcohol is best, but in a cream the others are o.k.
    \end{answer}
\end{standalonequote}

\begin{qaexchange}{Hormone Creams}
    \metadata{topic={Applying Hormones to Testicles}, source={Ray Peat Forum}}

    \begin{question}
        What do you think about applying fat soluble vitamins or hormones such as pregnenolone in oil bases to the testicles?
    \end{question}

    \begin{answer}
       With pure vitamin E as a solvent it might be o.k., but I think it's risky to alter the lipids in the environment of the gonads. 
    \end{answer}
\end{qaexchange}

\begin{qaexchange}{Hormone Creams}
    \metadata{topic={Steroid Absorption on Lips}, source={Ray Peat Forum}}

    \begin{question}
        Do you have an idea of what percentage of absorption there is when applying steroids dissolved in vitamin-e on the lips?
    \end{question}

    \begin{answer}
      It's much higher than on the skin, probably around 50\%, but usually it migrates into the mouth, with close to 100\% absorption.
    \end{answer}
\end{qaexchange}

\begin{standalonequote}{Hormone Creams}
    \metadata{topic={DHT Topical Application}, source={Ray Peat Forum}}

    \begin{answer}
      It's effective topically, but it might take several milligrams on the skin to absorb one mg.
    \end{answer}
\end{standalonequote}

\subsection{Other Topical Treatments}

\begin{standalonequote}{Other Topical Treatments}
    \metadata{topic={DMSO Effects And Breakdown}, source={Email Wiki}}

    \begin{answer}
        I got interested in it in 1965, after reading about Stanley Jacob's ideas, and experimented with it occasionally over the next several years. I was interested in its effects on cell water, stabilizing it in a way that reduces some kinds of inflammation. It seems to accelerate some enzyme reactions. I later started to think about its own chemical properties, rather than thinking of it as just a solvent. It isn't stable in the presence of water, and the odor seems to indicate the degree of its decomposition. It occasionally helps slightly with joint pain, but it can cause intense skin reactions, rashes; I think some of its effects depend on the breakdown products. People often forget that it has an intrinsic oxidative effect when they are thinking of it as just a solvent, to transport drugs. I saw a product sold as eye drops, consisting of vitamin C and glutathione in DMSO. Each of those reductants, in the presence of DMSO, immediately breaks down into other substances, and the composition changes continually over a long period. There has been very little investigation of the actual composition of solutions of DMSO with other substances. At least some of the mixtures will produce sulfite and metabisulfite, which are very allergenic for some people.
    \end{answer}
\end{standalonequote}

\begin{standalonequote}{Other Topical Treatments}
    \metadata{topic={DMSO Safety Concerns}, source={Email Wiki}}

    \begin{answer}
        Small amounts are probably harmless; even large amounts seem harmless for some people. Its ability to release histamine and nitric oxide and to inhibit cholinesterase (articles below) suggest that its use shouldn't be prolonged. 
    \end{answer}
\end{standalonequote}

\begin{standalonequote}{Other Topical Treatments}
    \metadata{topic={DMSO Reactions}, source={Email Wiki}}
    \begin{note}
        Joint Pain
    \end{note}

    \begin{answer}
        Thanks for the information; about 40 years ago I was interested in DMSO, but I lost interest in it when I saw some reactions like that.
    \end{answer}
\end{standalonequote}

\begin{standalonequote}{Other Topical Treatments}
    \metadata{topic={DMSO Safety Concerns}, source={Ray Peat Forum}}

    \begin{answer}
      In low concentrations DMSO is usually safe, but since I have seen allergy-like reactions to it, and since the biological chemisty is complex (e.g., J Phys Chem A. 2006 Jun 22;110(24):7628-36. Theoretical study of the reduction mechanism of sulfoxides by thiols. Balta B, Monard G, Ruiz-López MF, Antoine M, Gand A, Boschi-Muller S, Branlant G.), I think it's good to use it cautiously and watchfully.
    \end{answer}
\end{standalonequote}

\begin{standalonequote}{Other Topical Treatments}
    \metadata{topic={DMSO Safety Concerns}, source={Ray Peat Forum}}

    \begin{answer}
      I think it's safe for occasional use in small amounts; in the early ‘70s I was interested in it, but I gradually decided that there wasn't enough known about its metabolism to be sure of its long range safety. People at the Livermore lab suggested some possibly toxic metabolites (in connection with the Gloria Ramirez case), that might depend on a person's unique redox situation.
    \end{answer}
\end{standalonequote}

\section{Lifestyle Modifications}

\subsection{Light Therapy}

\begin{standalonequote}{Light Therapy}
    \metadata{topic={Light Spectrum Importance}, source={Email Wiki}}

    \begin{answer}
        Incandescent bulbs have a continuous spectrum, luminous gases have intermittently distributed wavelengths. Orange and red are the metabolically most important wavelengths. I don't think the far infrared does anything special, besides heat. Ordinary incandescent bulbs have a slightly orange color compared to sunlight, and the bulbs I have mentioned are just slightly warmer in color, with very little blue, and more red. Ordinary incandescent bulbs are good, if there are enough of them, directed toward your skin.
    \end{answer}
\end{standalonequote}

\begin{standalonequote}{Light Therapy}
    \metadata{topic={Red Light And Melatonin}, source={Email Wiki}}

    \begin{answer}
        It does suppress melatonin. I think the problem with light research is that many of them weren't using similar levels of light energy at the different wavelengths. I have tried sleeping with red light, and I didn't like it; but it can be equally effective, for maintaining blood sugar or reducing inflammation, if it shines only on the feet or legs. I think the u.v. lamps are good for use in the winter.
    \end{answer}
\end{standalonequote}

\begin{standalonequote}{Light Therapy}
    \metadata{topic={Incandescent Light For Heating}, source={Email Wiki}}

    \begin{answer}
        I use them (incandescent light bulbs) for keeping my area warm, instead of centrally heating the house. My view of the energy saving light bulbs is that putting a cork in the plug saves more energy (and doesn't contain mercury).
    \end{answer}
\end{standalonequote}

\begin{standalonequote}{Light Therapy}
    \metadata{topic={Reducing Night-Time Stress}, source={Email Wiki}}

    \begin{answer}
        Brighter light in the early evening
    \end{answer}
\end{standalonequote}

\begin{qaexchange}{Light Therapy}
    \metadata{topic={Red Light Radiation Repair}, source={Email Wiki}}

    \begin{question}
        Does red light repair damage done just by U.V. light, or maybe also by X-rays?
    \end{question}

    \begin{answer}
        By any irradiation, but it's most effective within the first hour. The harm from the radiation can be interrupted most effectively right away, before the changes have been amplified and integrated with the system.
    \end{answer}
\end{qaexchange}

\begin{standalonequote}{Light Therapy}
    \metadata{topic={Blue vs Red Light Effects}, source={Email Wiki}}
    \begin{note}
        Blue Light Reducing Oral Cancer Growth
    \end{note}

    \begin{answer}
        Plain incandescent bulbs have enough of the red spectrum to work. Blue light is slightly toxic, so like ultraviolet is can slow cell division, but its toxicity also causes inflammation. Red light reduces inflammation, but it tends to increase proliferation.
    \end{answer}
\end{standalonequote}

\begin{standalonequote}{Light Therapy}
    \metadata{topic={Safe Light Exposure Distance}, source={Email Wiki}}

    \begin{answer}
        If it's comfortable it isn't harmful, but it's easy to get burned when they are so close. Your body temperature is likely to rise, otherwise I don't know of any problem from prolonged light (but even incandescent light does have some slightly harmful blue light, so you should watch your own reactions to it).
    \end{answer}
\end{standalonequote}

\begin{qaexchange}{Light Therapy}
    \metadata{topic={Light Placement On Body}, source={Email Wiki}}

    \begin{question}
        Does it matter where the bright light shines on your body to receive beneficial effects?
    \end{question}

    \begin{answer}
        Bare skin is best; for effects on the nervous system, shining on head, face, neck and back is good.
    \end{answer}
\end{qaexchange}

\begin{standalonequote}{Light Therapy}
    \metadata{topic={LCD Screens}, source={Email Wiki}}

    \begin{answer}
        People react differently to different screens. Keeping the room bright, and the screen not too bright, can reduce the eye strain. This person has very detailed information about them:
    \end{answer}
\end{standalonequote}

\begin{qaexchange}{Light Therapy}
    \metadata{topic={Light Therapy Wattage Limits}, source={Email Wiki}}

    \begin{question}
        As I add more and more incandescent lights to see how I feel, is there an approximate number of watts which is prudent to not go over?
    \end{question}

    \begin{answer}
        The heat is the limiting factor, not the light.
    \end{answer}
\end{qaexchange}

\begin{qaexchange}{Light Therapy}
    \metadata{topic={Red Light Repair Timing}, source={Email Wiki}}

    \begin{question}
        Does red light repair damage done by UV light, or maybe also by X-rays?
    \end{question}

    \begin{answer}
        By any irradiation, but it's most effective within the first hour.
    \end{answer}
\end{qaexchange}

\begin{qaexchange}{Light Therapy}
    \metadata{topic={Red Light}, source={Ray Peat Forum}}

    \begin{question}
        I've not seen many satisfactory explanations yet as to why red light therapy or photobiomodulation seems to only work up to a certain point, after which the effects are significantly reduced, if not deleterious. I was wondering a lot about this in light of your recent newsletter about HSP. Is the issue simply a matter of overheating - and does this perhaps explain why strobing/pulsing light seems to enjoy a higher tolerance threshold? Why do you suppose there is a point at which effects are no longer noticed, what is the mechanism?
    \end{question}

    \begin{answer}
        Red light of moderate wattage doesn't warm the tissues enough to be harmful; its good effects are from restoring oxidative metabolism, and it just takes a short exposure to do that.
    \end{answer}
\end{qaexchange}

\begin{standalonequote}{Light Therapy}
    \metadata{topic={Infrared Bulbs}, source={Ray Peat Forum}}

    \begin{answer}
        Plain incandescent bulbs are o.k., but the best kind are used by farmers for incubators, etc., and are designed as 130 volt bulbs, so when they operate on 120 volts they have a bias toward the longer wave red color, and they have an internal reflector. They are often called \enquote{infrared} or \enquote{heat lamps,} but they have a clear glass front.
    \end{answer}
\end{standalonequote}

\begin{qaexchange}{Light Therapy}
    \metadata{topic={Red Light for Infants}, source={Ray Peat Forum}}

    \begin{question}
        Is it safe or necessary to use red light on a newborn or infant?
    \end{question}

    \begin{answer}
        As far as I know, it's safe.
    \end{answer}
\end{qaexchange}

\begin{qaexchange}{Light Therapy}
    \metadata{topic={Red Light on Thyroid}, source={Ray Peat Forum}}

    \begin{question}
        What do you think of exposing the neck/thyroid area to red light to entice metabolism?
    \end{question}

    \begin{answer}
      In principle it could increase circulation to the thyroid, possibly increasing secretion, but I doubt that it has much effect.
    \end{answer}
\end{qaexchange}

\begin{standalonequote}{Light Therapy}
    \metadata{topic={Red Light Safety}, source={Ray Peat Forum}}

    \begin{answer}
      Red light of moderate wattage doesn't warm the tissues enough to be harmful; its good effects are from restoring oxidative metabolism, and it just takes a short exposure to do that.
    \end{answer}
\end{standalonequote}

\begin{emailexchange}{Light Therapy}
    \metadata{topic={Red Light on Testicles}, source={Ray Peat Forum}}

    \begin{question}
        You mentioned that shining red light on the testicles might be carcinogenic. May I ask how?
    \end{question}

    \begin{answer}
       It stimulates mitosis and angiogenesis, processes that have to be coordinated with the entire physiology.
    \end{answer}

    \begin{question}
        Do you think a pure red spectrum light shined anywhere on the body is probably not a good idea then?
    \end{question}

    \begin{answer}
      Generally, I think a broader spectrum is better.
    \end{answer}
\end{emailexchange}

\subsection{Stress Management}

\begin{emailexchange}{Stress Management}
    \metadata{topic={Sensory Deprivation}, source={Ray Peat Forum}}

    \begin{answer}
        People who are very involved with inward words are the ones who, after some time in the dark tank suddenly notice that there is something in their awareness besides words.
    \end{answer}

    \begin{question}
        The \enquote{self} reflection that occurs in such an environment can help one escape words long enough to see/experience something new? Similar to something like doing art or dancing?
    \end{question}

    \begin{answer}
        Not often, but sometimes.
    \end{answer}
\end{emailexchange}

\subsection{Exercise}

\section{Alternative Therapies}

\begin{standalonequote}{Alternative Therapies}
    \metadata{topic={Coffee Enemas}, source={Email Wiki}}

    \begin{answer}
        Coffee in such small amounts probably is more effective for protecting against bowel cancer and liver disease when it's used by enema, rather than orally, but I think the general effects might be better when it's drunk.
    \end{answer}
\end{standalonequote}

\begin{standalonequote}{Alternative}
    \metadata{topic={Unknown Therapy}, source={Ray Peat Forum}}

    \begin{answer}
      I have never tried it myself, but I know that some people have found it to be more effective than just thyroid. Some people have very intense reactions to it, flushing and low blood pressure for example, so it's good to be very cautious with it. I think it's best to work on thyroid and sugar metabolism, watching the temperature cycle and heart rate.
    \end{answer}
\end{standalonequote}

\begin{qaexchange}{Alternative Therapies}
    \metadata{topic={Electrotherapy for Healing}, source={Ray Peat Forum}}

    \begin{question}
        I would greatly appreciate your thoughts on electrotherapy to treat injuries. Is it beneficial?
    \end{question}

    \begin{answer}
      It can accelerate healing, especially bone.
    \end{answer}
\end{qaexchange}

\begin{qaexchange}{Alternative Therapies}
    \metadata{topic={Inclined Bed Therapy, Migraines}, source={Ray Peat Forum}}

    \begin{question}
        What are your thoughts about inclined bed therapy?
    \end{question}

    \begin{answer}
       It seems biologically reasonable. I think migraines involve excess cholinergic activity, related to the \enquote{learned helplessnes} physiology, and that slight tilt would tend to keep the balance of the autonomic nervous system from shifting too far in that \enquote{demobilized} direction.
    \end{answer}
\end{qaexchange}

\begin{qaexchange}{Alternative Therapies}
    \metadata{topic={Rife vs Hulda Clark}, source={Ray Peat Forum}}

    \begin{question}
        I was wondering if you had any opinions or experience on the work of Hulda Clark or Royal Rife and the frequency machines such as zappers for parasites?
    \end{question}

    \begin{answer}
      Royal Rife probably observed some real effects, and there is some reasonable basis for his ideas, but not for Hulda Clark's.
    \end{answer}
\end{qaexchange}

\begin{emailexchange}{Alternative Therapies}
    \metadata{topic={Hydrogen Gas Therapy Safety}, source={Ray Peat Forum}}

    \begin{answer}
       I would be cautious with the hydrogen generating tablets, but hydrogen gas is plausible as an antiinflammatory supplement. I think any produced by bacteria might be beneficial too. 
    \end{answer}

    \begin{question}
        Do you think the Hydrogen producing tablets risk is associated with alkalinity/Hydroxide formation or heavy metals?
    \end{question}

    \begin{answer}
      Impurities would be my concern with the tablets.
    \end{answer}
\end{emailexchange}

\begin{standalonequote}{Alternative Therapies}
    \metadata{topic={Molecular Hydrogen Therapy}, source={Ray Peat Forum}}

    \begin{answer}
       It's safe, and in some situations it seems to have good effects. 
    \end{answer}
\end{standalonequote}

\begin{qaexchange}{Alternative Therapies}
    \metadata{topic={Stem Cell Therapy Future}, source={Ray Peat Forum}}

    \begin{question}
         Do you believe their is. or will be, a time where stem cell injections can be a viable option? 
    \end{question}

    \begin{answer}
      Urine and menstrual blood are good sources of stem cells, but I think their main value is in learning the principles of regulating them. As a practical method for health maintenance, I think learning how to activate and regulate the existing system of stem cells is the correct approach.
    \end{answer}
\end{qaexchange}

\begin{qaexchange}{Alternative Therapies}
    \metadata{topic={Stem Cell Banking}, source={Ray Peat Forum}}

    \begin{question}
        Do you think that incubating one's own stem cells for possible future therapies is a worthy investment?
    \end{question}

    \begin{answer}
      No.
    \end{answer}
\end{qaexchange}

\begin{qaexchange}{Alternative Therapies}
    \metadata{topic={Ultrasound for Tumors}, source={Ray Peat Forum}}

    \begin{question}
        What do you think about ultrasound for breaking up tumor tissue? Could this be a much safer alternative for chemotherapy?
    \end{question}

    \begin{answer}
      One advantage is that any damage to good tissue is localized, while chemotherapy damages the brain and other organs.
    \end{answer}
\end{qaexchange}

\begin{qaexchange}{Alternative Therapies}
    \metadata{topic={Hyperbaric Oxygen, Ozone Therapy}, source={Ray Peat Forum}}

    \begin{question}
        What is your opinion on Hyperbaric oxygen therapy (HBOT)? I've read many people having success in treating hearing loss for example and many other ailments. And also what do you think about ozone therapy?
    \end{question}

    \begin{answer}
      For a deep infected wound, both can be appropriate, but I think for most things the damage they cause is too much.
    \end{answer}
\end{qaexchange}

\begin{standalonequote}{Alternative Therapies}
    \metadata{topic={Brain Stimulation Therapy}, source={Ray Peat Forum}}

    \begin{answer}
      The brain is always organizing itself, and although some of the DC, AC, and magnetic stimulation effects are positive, I think they are also likely to have other harmful effects. Any stimulating radiation is a potentially disorganizing influence (red light has an antiexcitatory effect). I think supporting the brain's own organizing processes is the right approach to therapy. Neurosteroids stabilize nerves, and allow repair processes to proceed. Good nutrition, light, thyroid, etc., provide the energy. Meaningful activity is the other necessity.

    \end{answer}
\end{standalonequote}

\begin{standalonequote}{Alternative Therapies}
    \metadata{topic={\ce{CO2} Immersion Therapy}, source={Ray Peat Forum}}

    \begin{answer}
      I get mine from a welder's shop, and the valve on the tank works for filling a big plastic bag, but because the tanks are heavy, it's convenient to use a hose for filling a tub. My tank's valve is like a water faucet, and a simple rubber hose screws onto it, it doesn't need a pressure gauge or anything fancy. A lighted candle is good for telling when the tank is full enough, it goes out when the \ce{CO2} reaches the flame. The flow should be fairly slow so that it doesn't create turbulence that blows the gas out of the tank.
    \end{answer}
\end{standalonequote}

\chapter{Environmental \& Lifestyle Factors}

\section{Light \& Radiation}

\begin{standalonequote}{Radiation}
    \metadata{topic={Airplanes and Radiation}, source={Email Wiki}}

    \begin{answer}
        Not likely. The biological effects of radiation decrease as altitude increase. LET and mesons explain the relationship.
    \end{answer}
\end{standalonequote}

\begin{qaexchange}{Radiation}
    \metadata{topic={Radiation Bystander Effects}, source={Email Wiki}}

    \begin{question}
        You mentioned that \enquote{A millionth of a gray is known to produce bystander effects.} Do you have any studies suggesting this?
    \end{question}

    \begin{answer}
        I don't remember the context, but it probably referred to in vitro experiments, in which alpha particles produce very large bystander effects with very little energy. A slight disturbance of cell water activates nitric oxide synthesis, and that interacts both directly and indirectly with many things, including reduced energy production and destabilization of DNA. Martin Pall and others have shown that millimeter waves, too, can interact with cell water and increase nitric oxide, suggesting that some of their effects will be the same as those of ionizing radiation. An article by Betskii and Lebedeva says millimeter waves are being used therapeutically in Russia, so the harmful effects that Pall and others describe apparently aren't immediately apparent.
    \end{answer}
\end{qaexchange}

\subsection{Sun Exposure}

\begin{qaexchange}{Sun Exposure}
    \metadata{topic={Niacinamide For Sunburn}, source={Email Wiki}}

    \begin{question}
        Considering that aspirin works in some ways like niacinamide, would niacinamide help prevent sunburn?
    \end{question}

    \begin{answer}
        I haven't tried niacinamide for sun protection; the fact that it can lighten skin pigment might mean that it's blocking some free radical processes.
    \end{answer}
\end{qaexchange}

\begin{standalonequote}{Sun Exposure}
    \metadata{topic={Topical Sun Protection}, source={Email Wiki}}
    \begin{note}
        Sun Damaged Skin
    \end{note}

    \begin{answer}
        Topical vitamin A with vitamin E would be protective. Progesterone and caffeine are other powerfully protective things. Both caffeine and progesterone are protective topically as well as orally.
    \end{answer}
\end{standalonequote}

\begin{standalonequote}{Sun Exposure}
    \metadata{topic={Aspirin For Wrinkles}, source={Email Wiki}}

    \begin{answer}
        It's [aspirin] protective against sun aging, like vitamin E. I think the most helpful thing for wrinkles is pregnenolone (internally), since it increases the tone of connective tissues, causing the fascia and similar tissues to contract, if they have been sagging from a metabolic energy problem (caused by accumulated PUFA).
    \end{answer}
\end{standalonequote}

\section{Temperature \& Altitude}

\begin{standalonequote}{Temperature \& Altitude}
    \metadata{topic={Altitude And Sun Exposure}, source={Email Wiki}}

    \begin{answer}
        High altitudes are usually sunny.
    \end{answer}
\end{standalonequote}

\begin{standalonequote}{Temperature \& Altitude}
    \metadata{topic={Altitude Threshold Effects}, source={Email Wiki}}
    \begin{note}
        5,740 feet = 1,750 m
    \end{note}

    \begin{answer}
        That's high enough to make a difference.
    \end{answer}
\end{standalonequote}

\begin{standalonequote}{Temperature \& Altitude}
    \metadata{topic={Altitude Benefits, Nearsightedness}, source={Email Wiki}}
    \begin{note}
        6,000 feet = 1,830 m
    \end{note}

    \begin{answer}
        I think those moderate elevations are very helpful. My place in Coeneo is only 6600 feet, but my nearsightedness always improves when I'm there for a few weeks. People who are very sensitive to altitude would have a headache at 14000 feet, so that was a good test.
    \end{answer}
\end{standalonequote}

\begin{standalonequote}{Temperature \& Altitude}
    \metadata{topic={Altitude Adaptation Time}, source={Email Wiki}}
    \begin{note}
        6,560 feet
    \end{note}

    \begin{answer}
        2000 meters has a noticeable effect after a couple of weeks, higher is better, but it's necessary to take some time to adapt to the higher altitudes before being very active.
    \end{answer}
\end{standalonequote}

\begin{qaexchange}{Temperature \& Altitude}
    \metadata{topic={Altitude Benefit Timeline}, source={Email Wiki}}

    \begin{question}
        How long does it take to reap benefits?
    \end{question}

    \begin{answer}
        During the first couple of weeks, the body usually has stress reactions that have to settle down, then the improvement continues for years. A couple of months at altitude will usually cause changes that last for several months even at lower altitude.
    \end{answer}
\end{qaexchange}

\begin{qaexchange}{Temperature \& Altitude}
    \metadata{topic={Minimum Beneficial Altitude}, source={Email Wiki}}

    \begin{question}
        What is the lowest elevation, that one could reap good benefits?
    \end{question}

    \begin{answer}
        Statistics for New Mexico showed improvement for every increase of altitude within the state--I think it's continuous, from below sea level up to around 12,000 feet.
    \end{answer}
\end{qaexchange}

\begin{qaexchange}{Temperature \& Altitude}
    \metadata{topic={Altitude And Respiratory Diseases}, source={Email Wiki}}

    \begin{question}
        Are there any conditions which high altitude is negative for? I think I remember saying something about asthma?
    \end{question}

    \begin{answer}
        I should have said respiratory diseases, meaning things like emphysema and pulmonary fibrosis. Asthma is usually improved at high altitude, above 6000 feet, for several reasons. Even in polluted Mexico City, at 7500 feet, there's very little asthma, but people who vacation in Acapulco often get asthma.
    \end{answer}
\end{qaexchange}

\begin{qaexchange}{Temperature \& Altitude}
    \metadata{topic={Exercise At Altitude}, source={Email Wiki}}

    \begin{question}
        After settling in for a week or two, is light exercise OK? Short runs?
    \end{question}

    \begin{answer}
        I think walking should be the main exercise for the first several weeks.
    \end{answer}
\end{qaexchange}

\begin{emailexchange}{Location}
    \metadata{topic={Living Environment}, source={Ray Peat Forum}}

    \begin{question}
        What do you think is the best pro-metabolic, pro-fruit living environment in the US?
    \end{question}

    \begin{answer}
        High, sunny places like Santa Fe, NM, are as expensive as California, and can be very cold; Santa Barbara grows some tropical fruit, otheriwise, cities with big universities usually have some for their international students. The \enquote{hill country} of NW Texas has a compromise altitude and weather and nice scenery, but not much else. The lowest latitude places in the US, i.e., Florida, are unlivably hot and humid for many people. For fruit, goat milk, altitude, weather, and costs, there are thousands of good places in Mexico.
    \end{answer}

    \begin{question}
        Can you suggest a few cities that I can check out, to start my search?
    \end{question}

    \begin{answer}
        I think it's good to spend a week or two at a moderate altitude, 4000 to 5000 feet, to adapt, before doing much at higher altitude. Towns around Guadalajara without heavy industry are probably less polluted; Zapotlanejo, Jalisco, Zamora, Michoacan, and Tequila, Jalisco might be good places to start. The Patzcuaro area is very nice, towns around the big lake; Valle de Bravo, State of Mexico, is expensive but nice, usually has good fruit.
    \end{answer}
\end{emailexchange}

\begin{standalonequote}{Temperature \& Altitude}
    \metadata{topic={Hot Tub Safety, Body Temperature}, source={Ray Peat Forum}}

    \begin{answer}
      Getting the body temperature too high can deplete glycogen and lower blood glucose, sometimes causing fainting, probably why people occasionally drown in hot tubs; high temperature can damage the testes, reducing sperm production.
    \end{answer}
\end{standalonequote}

\begin{standalonequote}{Temperature \& Altitude}
    \metadata{topic={High Altitude Adjustment}, source={Ray Peat Forum}}

    \begin{answer}
      If a person's thyroid function is borderline, adaptation to altitude is harder. It's always best to rest for the first couple of days at altitude. Diamox, Coke, sugar, aspirin, and maybe baking soda can help. It's good to make sure that it's \ce{CO2} in the tanks, other, harmful gases are sometimes sold for similar uses.
    \end{answer}
\end{standalonequote}

\begin{standalonequote}{Temperature \& Altitude}
    \metadata{topic={Oxygen Saturation, Altitude}, source={Ray Peat Forum}}

    \begin{answer}
      I have noticed that when I'm most relaxed and at high altitude my oxygen saturation is in the range of 89\% to 94\%, lower with more relaxation. When I'm walking fast uphill, it is around 99\% (I think that could mean that my fingers are consuming less). If my fingers are cold (i.e., not using much oxygen) the number is higher. If you pump blood through a cold corpse, the hemoglobin will stay 100\% saturated.
    \end{answer}
\end{standalonequote}

\begin{standalonequote}{Temperature \& Altitude}
    \metadata{topic={Living at High Altitude}, source={Ray Peat Forum}}

    \begin{answer}
      5000 feet is helpful, higher is better. The low latitudes from Bolivia to Mexico have many comfortable little cities at good altitude. I have usually preferred places between 2000 and 2600 meters. Since the days warm up quickly at high altitude, the old houses were designed to retain the heat during the cold nights, by having thick adobe walls, about a meter thick.
    \end{answer}
\end{standalonequote}

\section{Environmental Toxins}

\begin{qaexchange}{Environmental Toxins}
    \metadata{topic={Tattoo Ink Nanoparticles}, source={Ray Peat Forum}}

    \begin{question}
        What is the safety of tattoos ink and if they cause long term risks?
    \end{question}

    \begin{answer}
      Infections are always a risk, but the presence of nanoparticles has been found in the tissues of some people with tattoos. Nanoparticles of substances such as aluminum can travel to the brain, causing chronic inflammation.
    \end{answer}
\end{qaexchange}

\begin{qaexchange}{Environmental Toxins}
    \metadata{topic={Nuclear Plant Proximity}, source={Ray Peat Forum}}

    \begin{question}
        What's the minimum distance from a nuclear plant you would want to live?
    \end{question}

    \begin{answer}
       I would prefer not to be closer than at least 12,000 miles.
    \end{answer}
\end{qaexchange}

\subsection{Heavy Metals}

\begin{standalonequote}{Heavy Metals}
    \metadata{topic={Aluminum And Bentonite Clay}, source={Email Wiki}}

    \begin{answer}
        <Please read  and the following post.>
    \end{answer}
\end{standalonequote}

\begin{standalonequote}{Heavy Metals}
    \metadata{topic={Heavy Metal Removal}, source={Email Wiki}}

    \begin{answer}
        Milk, orange juice, and coffee safely accelerate the removal of heavy metals from the tissues. Everyone's body accumulates PUFA's, which progressively interfere with metabolism and raise TSH. Iron, as well as other heavy metals (except for copper) tends to accumulate. Drinking coffee also helps to shift the hormone balance in the right direction.
    \end{answer}
\end{standalonequote}

\begin{standalonequote}{Heavy Metals}
    \metadata{topic={Chelator Mobilization Risks}, source={Email Wiki}}

    \begin{answer}
        Raising the body temperature and using chelators can mobilize things, but it can increase the damage they do on the way out. The liver doesn't store toxins for more than a few hours, and coffee enemas are intended to intensely stimulate the liver. Oral coffee lets the caffeine circulate slowly, keeping everything moderately active, and with orange juice, the mobilized metals are kept from injuring things until they are excreted.
    \end{answer}
\end{standalonequote}

\begin{qaexchange}{Heavy Metals}
    \metadata{topic={DMPS/DMSA Chelator Toxicity}, source={Email Wiki}}

    \begin{question}
        Question: Does the body quickly or gradually get rid of DMPS or DMSA chelating agents? I have many people who nearly died when they took DMPS or DMSA. But, I should think that the body would eventually detox it. What do you think?
    \end{question}

    \begin{answer}
        The idea of using it to remove metals is that it leaves the body rapidly. The damage produced by moving the metals around could be fairly permanent, but the chelator leaves very quickly. Environmental pollutants, food fats, and cosmetics are the things people should worry about accumulating in their tissues.
    \end{answer}
\end{qaexchange}

\begin{qaexchange}{Heavy Metals}
    \metadata{topic={Beryllium Chelation}, source={Ray Peat Forum}}

    \begin{question}
        Do you have any ideas on how someone with high levels of Beryllium in their body could get rid of it? All I can find are scientific articles discussing the use of pharmaceutical chelators and corticosteroids.
    \end{question}

    \begin{answer}
        The chelators in orange juice and grape juice seem to lack the toxic effects of some of the industrial chelators.
    \end{answer}
\end{qaexchange}

\begin{qaexchange}{Heavy Metals}
    \metadata{topic={Cilantro for Iron Chelation}, source={Ray Peat Forum}}

    \begin{question}
        I've been reading that cilantro is good for chelating heavy metals, specifically Iron. I was wondering if you think cilantro is a safe food to use as an effective way to remove iron from the body?
    \end{question}

    \begin{answer}
       I think the amount that would make a difference with heavy metals might be allergenic. Having coffee, milk, and orange juice as regular parts of your diet help to move iron out safely.
    \end{answer}
\end{qaexchange}

\begin{standalonequote}{Heavy Metals}
    \metadata{topic={Silver Toxicity}, source={Ray Peat Forum}}

    \begin{answer}
       Silver is just slightly less toxic than mercury. Silver would be about as toxic as the organic antibiotics in the short term, and maybe worse if used chronically. Topically, it's probably not as toxic as mercurochrome; permanganate and copper sulfate might be just a little safer. How does silver kill bacteria? It works pretty much by the same mechanism that makes mercury a powerful antiseptic. Heavy metals are relatively indiscriminate oxidants (potent oxidizing agents). All heavy metals---lead, silver, mercury, nickel, cadmium---are very toxic, potent enzyme poisons and go into the brain, causing diseases like Alzheimer's, ALS, MS, etc. 
    \end{answer}
\end{standalonequote}

\begin{standalonequote}{Heavy Metals}
    \metadata{topic={Heavy Metal Testing, Chelation}, source={Ray Peat Forum}}

    \begin{answer}
       Hair picks up things from water and air, for example lead from gun smoke. If he hasn't been exposed for a few months, then the short hair would give a clue to the amount that's stored. Animal studies showed that chelation can increase metal toxicity to the brain and kidneys. Vitamin C and other acids in orange juice reduce the toxicity of mercury and allow it to be excreted safely. At least for some metals, coffee seems to have a similar effect.
    \end{answer}
\end{standalonequote}

\begin{standalonequote}{Heavy Metals}
    \metadata{topic={Stainless Steel Cookware Safety}, source={Ray Peat Forum}}

    \begin{answer}
		I normally use glass pans, Pyrex or Vision, but sometimes use the steel pans without nickel. Although they darken a little when they age, they don't rust.

		There are two main types of stainless steel, magnetic and nonmagnetic. The nonmagnetic form has a very high nickel content, and nickel is allergenic and carcinogenic. It is much more toxic than iron or aluminum. You can use a little \enquote{refrigerator magnet} to test your pans. The magnet will stick firmly to the safer type of pan.

		I think the nickel content should be less than 2\%; the magnetic pans are hard to find (used stores sometimes have old ones), because people generally prefer the slick high nickel type.
    \end{answer}
\end{standalonequote}

\begin{standalonequote}{Heavy Metals}
    \metadata{topic={Nickel in Cookware}, source={Ray Peat Forum}}

    \begin{answer}
      The shiny, high nickel kind can when it's new, but nearly all the mobile nickel goes into the first ten batches.
    \end{answer}
\end{standalonequote}

\begin{standalonequote}{Heavy Metals}
    \metadata{topic={Lead in Ceramics}, source={Ray Peat Forum}}

    \begin{answer}
      If it doesn't have a shiny glaze, it's probably not a lead problem. I use glass and a paper filter, but I've often though that a perforated ceramic or glass dripper would be more convenient. I like to start with some warm water before adding hot water, to avoid breaking down aromatic things before extracting the caffeine.
    \end{answer}
\end{standalonequote}

\begin{standalonequote}{Heavy Metals}
    \metadata{topic={Chromium Absorption from Leather}, source={Ray Peat Forum}}

    \begin{answer}
      I haven't seen the studies, but it seems likely, if there's moisture.
    \end{answer}
\end{standalonequote}

\subsection{Plastics \& Endocrine Disruptors}

\begin{qaexchange}{Plastics \& Endocrine Disruptors}
    \metadata{topic={Microplastics Exposure}, source={Ray Peat Forum}}

    \begin{question}
        With microplastics becoming so pervasive in food and water supply, do you have any specific solutions to reduce exposure? Once ingested, do micro plastics behave similarly to raw starch granules adsorption?
    \end{question}

    \begin{answer}
      Reverse osmosis is be most effective practical way to filter water. Nanoparticles enter cells more easily than starch granules.
    \end{answer}
\end{qaexchange}

\subsection{Other Toxins}

\begin{standalonequote}{Other Toxins}
    \metadata{topic={Fluoride And Bromide Toxicity}, source={Email Wiki}}

    \begin{answer}
        It's good to avoid fluoridated water as far as possible. Certain forms of bromine, including bromate and polybrominated biphenyls, are definitely toxic, but simple bromide isn't very toxic; it took large amounts of Bromo-Seltzer used for a long time to produce harmful effects, hundreds of milligrams per day. Seawater contains bromide, so all seafood contains a lot; milk and meat naturally contain it, because soil generally contains a moderate amount. A few of the promoters of large iodine supplements--Abraham, Flechas, and Brownstein--are giving a wrong impression of bromine. 
    \end{answer}
\end{standalonequote}

\begin{standalonequote}{Other Toxins}
    \metadata{topic={Organophosphate Protection}, source={Ray Peat Forum}}

    \begin{answer}
       I would have a good supply of progesterone, calcium (milk, cheese), sugar (cokes, orange juice, candy). 
    \end{answer}
\end{standalonequote}

\section{Water \& Air Quality}

\begin{standalonequote}{Water \& Air Quality}
    \metadata{topic={Distilled Water Safety}, source={Email Wiki}}

    \begin{answer}
        Distilled water is fine. The idea that distilled water is harmful probably derives from the fact that in areas where the water has a high mineral content, people have been healthier on average than in areas with naturally \enquote{soft} water, but that involves several factors, especially the fact that hard water doesn't dissolve as much lead from the plumbing (such as soldered connections of copper pipes), and also that agricultural products in those areas are likely to have a higher trace mineral content. Generally, water is softer in areas with higher rainfall, and that means that people in those regions are more likely to have less sunlight, and a vitamin D deficiency affects mineral metabolism. In general, it's best to drink water only when you're thirsty.
    \end{answer}
\end{standalonequote}

\begin{qaexchange}{Water \& Air Quality}
    \metadata{topic={Public Pool Chlorine Exposure}, source={Ray Peat Forum}}

    \begin{question}
        What is your opinion about swimming in public swimming pools? Are the chlorine and human urine a concern since our skins are live large organs that can interact with the environment?
    \end{question}

    \begin{answer}
       I don't think it's something that a person should spend a lot of time doing. 
    \end{answer}
\end{qaexchange}

\begin{standalonequote}{Water \& Air Quality}
    \metadata{topic={Water Filtration Methods}, source={Ray Peat Forum}}

    \begin{answer}
      That's the kind I use. Distillation can produce purer water, but it's probably expensive and unnecessary.
    \end{answer}
\end{standalonequote}

\begin{standalonequote}{Water \& Air Quality}
    \metadata{topic={Rainwater Collection Systems}, source={Ray Peat Forum}}

    \begin{answer}
      Big sheets of plastic to line cisterns aren't very expensive. I made a concrete tank, one meter deep, that's about half above ground, that I intended as a small swimming pool, and during the rainy season we pump some of the water into the well, helping with the ground water. Plastic-lined ponds work in some areas. Friends in Coeneo had a big house built in 1900, and they used city water, but when the city's pump failed, having thousands of gallons under the patio was convenient.
    \end{answer}
\end{standalonequote}

\section{Personal Care \& Household Products}

\begin{standalonequote}{Personal Care \& Household Products}
    \metadata{topic={Bathing Temperature Effects}, source={Email Wiki}}

    \begin{answer}
        Warm showers can lower stress, and if the bath isn't too warm, it's effective, too; if the bath raises the body temperature, that can cause the metabolism to increase, sometimes causing low blood sugar.
    \end{answer}
\end{standalonequote}

\begin{standalonequote}{Personal Care \& Household Products}
    \metadata{topic={Fluoride And Soap In Bathing}, source={Email Wiki}}

    \begin{answer}
        I don't think it's a problem. The soaps and shampoos people use are worse problems. Just washing the skin with pure soap alters the skin's endocrine function for days. and doing it every day is an \enquote{endocrine disrupter,} even if there are no toxic additives in the soap.
    \end{answer}
\end{standalonequote}

\begin{qaexchange}{Personal Care \& Household Products}
    \metadata{topic={Deodorant And Shaving Cream Safety}, source={Email Wiki}}

    \begin{question}
        Is there anything to worry about/look out for when using deodorant or shaving cream?
    \end{question}

    \begin{answer}
        It's possible to make them without toxic ingredients, but I don't know of any such products.
    \end{answer}
\end{qaexchange}

\begin{standalonequote}{Personal Care \& Household Products}
    \metadata{topic={Fluoride In Shower Water}, source={Email Wiki}}

    \begin{answer}
        I don't think it's a problem. The soaps and shampoos people use are worse problems. Just washing the skin with pure soap alters the skin's endocrine function for days. and doing it every day is an \enquote{endocrine disrupter,} even if there are no toxic additives in the soap.
    \end{answer}
\end{standalonequote}

\begin{standalonequote}{Personal Care \& Household Products}
    \metadata{topic={Plastic Container Safety}, source={Email Wiki}}

    \begin{answer}
        It depends on the type of plastic; if it's in a big plastic bucket, the plastic isn't as bad as in a 400 mililiter bottle, if it contains harmful chemicals.
    \end{answer}
\end{standalonequote}

\begin{standalonequote}{Personal Care \& Household Products}
    \metadata{topic={Carrageenan in Toothpaste Safety}, source={Email Wiki}}

    \begin{answer}
        No, it isn't likely to be a problem unless you are very sensitive to it.
    \end{answer}
\end{standalonequote}

\begin{emailexchange}{Housing}
    \metadata{topic={House Selection}, source={Ray Peat Forum}}

    \begin{question}
        What would you look for in a house or a property if you were to buy one yourself?
    \end{question}

    \begin{answer}
        I would look for a stone house, good soil and sun exposure, and a reliable water source, with good neighbors.
    \end{answer}

    \begin{question}
        Do you think bricks or blocks are ok? What do you think about modular homes?
    \end{question}

    \begin{answer}
        I think mud is best, bricks next, then cement blocks.
    \end{answer}
\end{emailexchange}

\begin{standalonequote}{Personal Care \& Household Products}
    \metadata{topic={Deodorant Alternatives}, source={Ray Peat Forum}}

    \begin{answer}
      A topical aspirin solution might slightly reduce odors. Women with low thyroid, high estrogen, are over-sensitive to odors.
    \end{answer}
\end{standalonequote}

\begin{standalonequote}{Personal Care \& Household Products}
    \metadata{topic={Hair Washing Methods}, source={Ray Peat Forum}}

    \begin{answer}
      I use either coconut soap or baking soda for washing my hair. I rarely use soap on my skin.
    \end{answer}
\end{standalonequote}

\begin{qaexchange}{Personal Care \& Household Products}
    \metadata{topic={Teeth Cleaning, Fluoride}, source={Ray Peat Forum}}

    \begin{question}
         What do you think would be best for daily teeth cleaning? Do you think fluoride toothpastes are beneficial or harmful?
    \end{question}

    \begin{answer}
       I think daily use of fluoride is likely to be harmful to the gums, but applying it occasionally to the teeth can harden the enamel, improving its resistance to cavities. I think the alkaline effect of baking soda is helpful for thorough cleaning. 
    \end{answer}
\end{qaexchange}

\begin{qaexchange}{Personal Care \& Household Products}
    \metadata{topic={Tattoo Ink Long-Term Safety}, source={Ray Peat Forum}}

    \begin{question}
      Do you think the [tattoo] ink's safe or do you think it'll cause problems over the long-term?
    \end{question}

    \begin{answer}
      I don't know of any safe ink, and people's reactions vary; the body can begin to react long after it was done.
    \end{answer}
\end{qaexchange}

\begin{qaexchange}{Personal Care \& Household Products}
    \metadata{topic={Dishwashing with Coconut Soap}, source={Ray Peat Forum}}

    \begin{question}
        I wondered how you cleaned all the cups, plates, cutlery that you use and do you consider standard washing up liquid safe?
    \end{question}

    \begin{answer}
       I use coconut oil soap or Ivory soap only, with hot water it takes only a very small amount. 
    \end{answer}
\end{qaexchange}

\begin{qaexchange}{Personal Care \& Household Products}
    \metadata{topic={Perfumes on Clothing}, source={Ray Peat Forum}}

    \begin{question}
         Perfumes still harmful if only applied to cloths and not to skin?
    \end{question}

    \begin{answer}
       I don't think they are harmful. 
    \end{answer}
\end{qaexchange}

\begin{emailexchange}{Personal Care \& Household Products}
    \metadata{topic={Baking Soda Shampoo}, source={Ray Peat Forum}}

    \begin{question}
        What is your opinion on using diluted baking soda on hair instead of commercial shampoos? Would it be good for the hair health?
    \end{question}

    \begin{answer}
      It's what I normally use.
    \end{answer}

    \begin{question}
        Do you have a specific ratio for baking soda to water in order to minimize the alkalizing effect of the solution and get the best results?
    \end{question}

    \begin{answer}
      No, it takes very little to clean the hair.
    \end{answer}
\end{emailexchange}

\begin{qaexchange}{Personal Care \& Household Products}
    \metadata{topic={Showering Frequency, Immunity}, source={Ray Peat Forum}}

    \begin{question}
        Would it be a good idea to not shower often in order for the skin absorb and build immunity to local bacteria and pathogens?
    \end{question}

    \begin{answer}
      I think general good health is supporting the same system, and that specially activating it isn't necessary---during plagues, a considerable portion of the people are unaffected, the healthy ones.
    \end{answer}
\end{qaexchange}

\begin{qaexchange}{Personal Care \& Household Products}
    \metadata{topic={Skincare Preservatives}, source={Ray Peat Forum}}

    \begin{question}
        Do you know whether Sodium Levulinate and Sodium Anisate are safe to use in skincare cosmetics as preservatives?
    \end{question}

    \begin{answer}
       I think they are safe.
    \end{answer}
\end{qaexchange}

\begin{qaexchange}{Personal Care \& Household Products}
    \metadata{topic={Bathing, Skin Barrier}, source={Ray Peat Forum}}

    \begin{question}
         Do you know if ammonia oxidizing bacteria have a negative side effect in their metabolism in regards to human physiology?
    \end{question}

    \begin{answer}
      I think bathing is the problem, but for different reasons. Soapy water removes the lipid barrier, and that disturbs the complex lipid-steroid metabolism of the skin, interfering with the natural antibiotic processes of the skin.
    \end{answer}
\end{qaexchange}

\begin{standalonequote}{Personal Care \& Household Products}
    \metadata{topic={Root Canals, Teeth as Sense Organs}, source={Ray Peat Forum}}

    \begin{answer}
      Root canals can sometimes be repaired—with more skill and care—to correct a problem. Teeth are, among other things, sense organs, affecting brain functions.
    \end{answer}
\end{standalonequote}

\begin{qaexchange}{Personal Care \& Household Products}
    \metadata{topic={Electric Shaver}, source={Ray Peat Forum}}

    \begin{question}
        What products do you shave with? 
    \end{question}

    \begin{answer}
      I have used only electric shavers since I was 16. Shaving with a blade was too irritating.
    \end{answer}
\end{qaexchange}

\begin{qaexchange}{Personal Care \& Household Products}
    \metadata{topic={Synthetic Clothing Toxicity}, source={Ray Peat Forum}}

    \begin{question}
        Do you think there are any concerns with synthetic clothing?
    \end{question}

    \begin{answer}
      Some of the new things with nano-silver for example are very toxic.
    \end{answer}
\end{qaexchange}

\begin{standalonequote}{Personal Care \& Household Products}
    \metadata{topic={Dental Filling Toxicity}, source={Ray Peat Forum}}

    \begin{answer}
      If they are very small they are probably o.k.
    \end{answer}
\end{standalonequote}

\begin{standalonequote}{Personal Care \& Household Products}
    \metadata{topic={Dental Filling Safety}, source={Ray Peat Forum}}

    \begin{note}
        Zinc Eugenol Fillings Safer Than Ceramic or Gold?
    \end{note}

    \begin{answer}
      They are all very safe (especially the real baked ceramic used for inlays and crowns).
    \end{answer}
\end{standalonequote}

\begin{standalonequote}{Personal Care \& Household Products}
    \metadata{topic={Dental Filling Materials}, source={Ray Peat Forum}}

    \begin{answer}
        I think Sorel cement is the best filling material, but few dentists know about it. Gold is safe in the presence of RF radiation.

		The zinc eugenol fillings are often used, and they can last a long time; I've known several dentists who were willing to use them, but they call them temporary.
    \end{answer}
\end{standalonequote}

\begin{qaexchange}{Personal Care \& Household Products}
    \metadata{topic={Mattress Firmness}, source={Ray Peat Forum}}

    \begin{question}
        Is a firm mattress desireable?
    \end{question}

    \begin{answer}
      Common bed spring technology in the 19th century resulted in a hammock-like sag as the springs aged. The inner-spring mattress people showed how much better it was if the bed didn't sag in the middle, and for about a generation Simmons and other companies made very soft innerspring mattresses that allowed the hips and shoulder to sink, letting the back stay straight.
	  
	  The idea of mattress firmness blended with ideas of duty, toughness, exercise, and health, until the most expensive mattresses felt like a futon on concrete.
    \end{answer}
\end{qaexchange}

\begin{standalonequote}{Personal Care \& Household Products}
    \metadata{topic={Latex Mattresses Safety}, source={Ray Peat Forum}}

    \begin{answer}
      Some people are allergic to it, but it can make a very good foam mattress. In recent years, most of the latex mattresses are designed to be stiff and firm, but that isn't intrinsic to the latex.
    \end{answer}
\end{standalonequote}

\section{Light \& Environment}

\begin{qaexchange}{Light}
    \metadata{topic={Indoor Lighting Choices}, source={Ray Peat Forum}}

    \begin{question}
        Can halogen bulbs provide helpful indoor light for northern winter months? If halogen bulbs give off more blue light, I was wondering if infrared bulbs might be better.
    \end{question}

    \begin{answer}
      Regular incandescent bulbs, including those called infrared with clear glass, are better, because of the low amount of blue.
    \end{answer}
\end{qaexchange}

\section{Radiation \& EMF}

\begin{qaexchange}{Radiation}
    \metadata{topic={High Altitude and Cosmic Rays}, source={Ray Peat Forum}}

    \begin{question}
        You once mentioned that radiation at higher altitude or in airplane is bigger or has bigger waves or something like that - therefore it is less dangerous than low level radiation because it cannot penetrate the tissues and you mention some Russian researcher if I remember correctly.

		Would you mind sharing some studies which support that?
    \end{question}

    \begin{answer}
      You can find the research under \enquote{linear energy transfer,} high energy radiation, cosmic rays, etc. It's well known basic science, but medical people rarely understand it.
    \end{answer}
\end{qaexchange}

\begin{emailexchange}{EMF}
    \metadata{topic={Laptop Radiation and Cell Towers}, source={Ray Peat Forum}}

    \begin{question}
        Do you know what would reduce radiation from using the laptop in your lap or on your knees from time to time, besides the obvious one like stop using the laptop in that manner?
    \end{question}

    \begin{answer}
      Stopping it is the only protection.
    \end{answer}
	
    \begin{question}
        Do you think buying a property in the countryside---370m away from the cell tower---is safe?
    \end{question}

    \begin{answer}
      I think that's a safe distance.
    \end{answer}
\end{emailexchange}

\begin{standalonequote}{Radiation \& EMF}
    \metadata{topic={EMF Protection Methods}, source={Ray Peat Forum}}

    \begin{answer}
      The radiation from telephones and such can be blocked by wire mesh, that's well grounded. Some people use screen to cover the walls and ceiling of their bedroom, grounding it by attaching it to the plumbing, or to the electrical ground wire. A metal roof that has a wire or metal drain-pipe connecting it to the ground reduces the radiation from radio and television. Much of the natural earth e.m. resonance will be excluded.
    \end{answer}
\end{standalonequote}

\begin{qaexchange}{Radiation \& EMF}
    \metadata{topic={Grounding Metal Bed Frame}, source={Ray Peat Forum}}

    \begin{question}
        If one had to sleep on a mostly metal bed frame with a metal spring matress, do you think it would be wise to wire them to the Earth?
    \end{question}

    \begin{answer}
      I think it might be good, for example a wire to a water pipe would be an effective ground.
    \end{answer}
\end{qaexchange}

\begin{standalonequote}{Radiation \& EMF}
    \metadata{topic={Induction Burner Safety, EMF}, source={Ray Peat Forum}}

    \begin{answer}
      That looks like the old fashioned electric range, that's very safe. We noticed that when we turned on a new induction burner, our cat looked worried, and wouldn't get near it; I think they put out a much larger field than the standard old resistance element.
    \end{answer}
\end{standalonequote}

\section{Housing \& Built Environment}

\begin{emailexchange}{Housing}
    \metadata{topic={Property Selection}, source={Ray Peat Forum}}

    \begin{question}
        What would you look for in a house or a property if you were to buy one yourself?
    \end{question}

    \begin{answer}
       I would look for a stone house, good soil and sun exposure, and a reliable water source, with good neighbors.
    \end{answer}
	
    \begin{question}
        Do you think bricks or blocks are ok? What do you think about modular homes?
    \end{question}

    \begin{answer}
      I think mud is best, bricks next, then cement blocks.
    \end{answer}
\end{emailexchange}

\begin{emailexchange}{Housing \& Built Environment}
    \metadata{topic={Best Living Locations in US/Mexico}, source={Ray Peat Forum}}

    \begin{question}
        What do you think is the best pro-metabolic, pro-fruit living environment in the US?
    \end{question}

    \begin{answer}
      High, sunny places like Santa Fe, NM, are as expensive as California, and can be very cold; Santa Barbara grows some tropical fruit, otheriwise, cities with big universities usually have some for their international students. The \enquote{hill country} of NW Texas has a compromise altitude and weather and nice scenery, but not much else. The lowest latitude places in the US, i.e., Florida, are unlivably hot and humid for many people. For fruit, goat milk, altitude, weather, and costs, there are thousands of good places in Mexico.
    \end{answer}

    \begin{question}
        Can you suggest a few cities that I can check out, to start my search?
    \end{question}

    \begin{answer}
       I think it's good to spend a week or two at a moderate altitude, 4000 to 5000 feet, to adapt, before doing much at higher altitude. Towns around Guadalajara without heavy industry are probably less polluted; Zapotlanejo, Jalisco, Zamora, Michoacan, and Tequila, Jalisco might be good places to start. The Patzcuaro area is very nice, towns around the big lake; Valle de Bravo, State of Mexico, is expensive but nice, usually has good fruit.
    \end{answer}
\end{emailexchange}

\begin{standalonequote}{Housing \& Built Environment}
    \metadata{topic={Living Locations in Mexico, Earthquakes}, source={Ray Peat Forum}}

    \begin{answer}
      The low altitude of Yucatan makes the climate oppressive; the Patzcuaro area is perfect. House construction is the main safety factor; old adobe was reinforced with horse hair and was flexible, safer than concrete. I think electrical fields of the earth/solar system are important for earthquakes. The earthquakes I've seen during the night were accompanied by lightning.
    \end{answer}
\end{standalonequote}

\begin{standalonequote}{Housing \& Built Environment}
    \metadata{topic={House Construction, Temperature Regulation}, source={Ray Peat Forum}}

    \begin{answer}
      It depends on the climate. Thick cement walls, with insulination on the outside, keep the temperature relatively steady, so that the day's heat keeps the nights warm, and cool nights keep the days from overheating. My favorite houses have been adobe, with thick mud walls and high ceilings. For example, an old house in Mexico with neither heating nor cooling stayed around 69 degrees F most of the time.
    \end{answer}
\end{standalonequote}

\begin{qaexchange}{Housing \& Built Environment}
    \metadata{topic={Air Conditioning Effects}, source={Ray Peat Forum}}

    \begin{question}
        Your thoughts on living in an A/C controlled environment; can that be harmful?
    \end{question}

    \begin{answer}
      Some of them can spread allergens and infections.
    \end{answer}
\end{qaexchange}

\chapter{Additional Topics}

\section{Philosophy \& Worldview}

\begin{qaexchange}{Philosophy}
    \metadata{topic={Legacy}, source={Ray Peat Forum}}

    \begin{question}
        What would you like your legacy to be?
    \end{question}

    \begin{answer}
        Ending oligarchy and ending the digital culture are important goals.
    \end{answer}
\end{qaexchange}

\begin{qaexchange}{Philosophy}
    \metadata{topic={Consciousness}, source={Ray Peat Forum}}

    \begin{question}
        Do you believe that our soul or consciousness lives on somewhere, or are we just pure fertilizer?
    \end{question}

    \begin{answer}
        Rupert Sheldrake, Max Freedom Long, Andrija Puharich, Michael Persinger and Albert North Whitehead have described reasons for thinking that consciousness is in some ways objective and \enquote{exterior,} not simply a bodily state. Persinger's talk \enquote{No more secrets} is an introduction to some of the implications.
    \end{answer}
\end{qaexchange}

\begin{qaexchange}{Philosophy}
    \metadata{topic={Soul and Energy}, source={Ray Peat Forum}}

    \begin{question}
        So if I am reading a lot of this right, it's possible that our energy, or soul one would, say lives on? One alive could feel and see these signs from the deceased, but what about the deceased?
    \end{question}

    \begin{answer}
        I think that question is the source for much of the sense of reverence, and responsibility, for everything in the world.
    \end{answer}
\end{qaexchange}

\begin{emailexchange}{Philosophy}
    \metadata{topic={Art and Perception}, source={Ray Peat Forum}}

    \begin{question}
        I've been revisiting tesselation patterns throughout nature and attempting to \enquote{refresh} my perception on creating, asking better questions, etc. I see and have heard in your interviews that you paint and sculp (I believe,) and wondering what kind of key insights you may have had over the years with anything regarding tesselation patterns, etc. When did you first realize the association with art and science/philosophy and so on?
    \end{question}

    \begin{answer}
        Painting led to thinking about the process of perception, and that to phenomenology, and the problems of flowing time and multiple perspectives caused me to think about the way meaning exists in a painting—generality revealed in the specific. That leads to holism in science, the realization of the impossibility of reductionism.
    \end{answer}
\end{emailexchange}

\begin{emailexchange}{Philosophy}
    \metadata{topic={Attracting People}, source={Ray Peat Forum}}

    \begin{question}
        What can one do to attract the right kind people into one's life? As I find as the culture becoming increasingly authoritarian, I'm feeling especially afraid here in NYC. There's something to this quote by Jung.

        \begin{quote}
        \enquote{If you do your work truly and conscientiously, unknown friends will come and seek you.} --- CG Jung
        \end{quote}

        Do you agree, and do you have anything to add?
    \end{question}

    \begin{answer}
        However, the wrong kind of people can also be attracted to such unusual conscientious workers.
    \end{answer}

    \begin{question}
        Yes, that has happened to me. What would be the biological imbalance underpinning such unusually conscientious work?
    \end{question}

    \begin{answer}
        I don't think it's an imbalance, more like a coherence.
    \end{answer}
\end{emailexchange}

\begin{qaexchange}{Philosophy \& Worldview}
    \metadata{topic={Monogamy and Property}, source={Ray Peat Forum}}

    \begin{question}
        I'd be interested to find out what you think about monogamy in human relationships. Do you think, for example, that to commit to a theoretically life-long contract of sexual exclusivity is to stifle our true nature in a potentially harmful way? Monogamous, long-lasting relationships are generally held up as the model in this domain, with anything else usually considered as a failure or a perversity. Is it right that the sacrifices inherent in the commitment of monogamy should be held in such esteem?
    \end{question}

    \begin{answer}
      I think it's based on the property mentality.
    \end{answer}
\end{qaexchange}

\begin{qaexchange}{Philosophy \& Worldview}
    \metadata{topic={Personal Philosophy}, source={Ray Peat Forum}}

    \begin{question}
         Why do you do it? What makes you give, like you give? 
    \end{question}

    \begin{answer}
       I don't know, but I have always thought that it's necessary to do as much as you can do. Current societies seem to be set up largely to keep people from doing what they should be doing. 
    \end{answer}
\end{qaexchange}

\begin{standalonequote}{Philosophy \& Worldview}
    \metadata{topic={Polyamory, Sexual Relationships}, source={Ray Peat Forum}}

    \begin{answer}
      Our society is committed to creating artificial scarcity to maintain authoritarian relationships everywhere, especially sexual, interpersonal, reproductive relations. The basic sense of reality tends toward coherence, so one's sense of dutiful conforming affects the sense of erotic sensuality. The organism's well being is limited by all of the anti-empathic, exploitative institutions we interact with.
    \end{answer}
\end{standalonequote}

\begin{qaexchange}{Philosophy \& Worldview}
    \metadata{topic={Eugenics and Anti-Soviet Propaganda}, source={Ray Peat Forum}}

    \begin{question}
         In what way were American eugenicists inolved in anti-soviet propoganda?
    \end{question}

    \begin{answer}
      The eugenicists, seeing themselves as a better class, wanted to eliminate poor people and immigrants, so there was a great overlap with the anti-union, anti-socialist political campaign. Working class people who were enthusiastic racists and anti-red were the instruments of the upper class who mostly stayed in the background.
    \end{answer}
\end{qaexchange}

\begin{qaexchange}{Philosophy \& Worldview}
    \metadata{topic={Staying Positive, Evil Awareness}, source={Ray Peat Forum}}

    \begin{question}
        How do you stay positive in the face of the widespread corruption, stupidity and malice that you have witnessed throughout your life?
    \end{question}

    \begin{answer}
      There's probably a biological tendency to shift attention away from evil toward pleasanter things, but I think it's important not to indulge too much in positivity. Have you read or listened to Vernon Coleman, The Old Man in a Chair?
    \end{answer}
\end{qaexchange}

\begin{qaexchange}{Philosophy \& Worldview}
    \metadata{topic={Luck, Synchronicity}, source={Ray Peat Forum}}

    \begin{question}
        I was wondering if you believed in luck? From my perspective the universe seems to collectively interact with me in a symbolically, intelligent manner. Luck almost seems like an energy or substance to could also be viewed as measuring my relationship to the universe and how well it treats me. I also seem to find that not only does having a certain attitude toward things increase my feeling of luck, but different substances and molecules interact with seem to have varying effects. Psychedelic substances such as LSD, mushromms and dimethyltryptamine seem to be the most profound luck enhancers, but other things that improve mitochondrial health such as vitamin B\textsubscript{1}, B\textsubscript{3}, pau d'archo, vitamin K\textsubscript{2}, etc., seem to correlate to me having good fortune.
    \end{question}

    \begin{answer}
      Coffee is another enhancer. F David Peat discusses similar ideas in his book Synchronicity. Andrija Puharich and John Bockris have suggested a variety of interpretations.
    \end{answer}
\end{qaexchange}

\begin{standalonequote}{Philosophy \& Worldview}
    \metadata{topic={Hegel, Critical Theory Critique}, source={Ray Peat Forum}}

    \begin{answer}
      American professors who have called themselves marxists, in recent generations, have generally been closer to Hegel in their attitude toward abstractions, protecting some authoritarian commitments, avoiding Lenin's completely open understanding of matter, which I think was close to Aristotle's view of substance. I think Hegel's idea that punishment was for the benefit of the criminal shows that abstraction had precedence over everything. Dialectic thinking makes Idealism as good as it can be, but what it lacks is everything to be discovered in the richness of reality. All the variations of neokantianism and Popperism evade a dialectical understanding, and seem to be motivated by a need to reduce possibilities of knowing.
    \end{answer}
\end{standalonequote}

\begin{standalonequote}{Philosophy \& Worldview}
    \metadata{topic={Eastern Philosophy Value}, source={Ray Peat Forum}}

    \begin{answer}
      Yes, it's worth serious study. I think it makes study of the productions of the Western cultures less harmful.
    \end{answer}
\end{standalonequote}

\begin{standalonequote}{Philosophy \& Worldview}
    \metadata{topic={Evil, Serotonin, Estrogen}, source={Ray Peat Forum}}

    \begin{answer}
      Uncontrolled serotonin and estrogen are major factors in destructive, seemingly unmotivated aggression, and can be corrected by a properly supportive environment. Less extreme inclinations of the same sort, when combined with social and economic power and what superficially seems like self-interest, become potentially world destroying. The system becomes self-propelling, mostly independent of the physiology of the actors.
    \end{answer}
\end{standalonequote}

\begin{qaexchange}{Philosophy \& Worldview}
    \metadata{topic={IQ Test Score}, source={Ray Peat Forum}}

    \begin{question}
        I know you're not a fan of IQ tests, but have you ever taken one years ago by any chance? Curious what you scored, I would imagine high.
    \end{question}

    \begin{answer}
      Yes. In graduate school (1956) the Miller Analogies Test was required. My score wasn't returned until months after the others, because it was far outside what they expected.
    \end{answer}
\end{qaexchange}

\begin{emailexchange}{Philosophy \& Worldview}
    \metadata{topic={Best/Worst US Presidents}, source={Ray Peat Forum}}

    \begin{question}
        Who do you think was the best president in the history of the U.S.?
    \end{question}

    \begin{answer}
      FDR.
    \end{answer}

    \begin{question}
        And the worst?
    \end{question}

    \begin{answer}
      Truman.
    \end{answer}
\end{emailexchange}

\begin{qaexchange}{Philosophy \& Worldview}
    \metadata{topic={Ghosts, Paranormal}, source={Ray Peat Forum}}

    \begin{question}
        I was curious to know if you had any thoughts on ghosts and ghosts stories? Do you personally believe in ghosts or have a take on it?
    \end{question}

    \begin{answer}
      A physical chemisst, John O'Meara Bockris, compiled a collection of well documented stories (\textit{The New Paradigm}); I think it's likely that interaction among the living can account for what appear to be actions of ghosts—usually.
    \end{answer}
\end{qaexchange}

\begin{qaexchange}{Philosophy \& Worldview}
    \metadata{topic={Transhumanism, WEF}, source={Ray Peat Forum}}

    \begin{question}
        Do you have any thoughts on a transhumanist future?
    \end{question}

    \begin{answer}
      I think the only practical option is to prosecute the members of WEF and their collaborators for criminal conspiracy, and to begin a radical de-digitization of society.
    \end{answer}
\end{qaexchange}

\begin{qaexchange}{Philosophy \& Worldview}
    \metadata{topic={Universal Basic Income}, source={Ray Peat Forum}}

    \begin{question}
        What do you think of the idea of a universal basic income?
    \end{question}

    \begin{answer}
      Prices of basic foods and rent would soon rise; I think the proposal is tied to the plan for digital ID, digital money, and complete political control.
    \end{answer}
\end{qaexchange}

\begin{qaexchange}{Philosophy \& Worldview}
    \metadata{topic={Marriage, Metabolism}, source={Ray Peat Forum}}

    \begin{question}
        Do you think looking for marriage (instead of doing casual sex and uncommitted relationships) and wanting to have children are both signs of high metabolism?
    \end{question}

    \begin{answer}
      I think good health goes with a desire for a permanent relationship.
    \end{answer}
\end{qaexchange}

\begin{standalonequote}{Philosophy \& Worldview}
    \metadata{topic={Medical Knowledge, Consciousness}, source={Ray Peat Forum}}

    \begin{answer}
      Knowledge isn't a commodity, especially not a fungible commodity, as the medical business sees it. Consciousness and culture are part of the life process. It is exactly the commoditization of medical knowledge that makes it dangerous, and generally stupid. Doctors buy their knowledge, and then resell it over and over; it's valuable as a commodity, so its value has to be protected by the equivalent of a copyright, the system of laws establishing the profession. Without its special status, its worthlessness would be quickly demonstrated. When A.C. Guyton wrote his textbook of medical physiology (the most widely used text in the world) in the 1950s, it was trash; as it was studied and applied by generations of physicians, it was still trash. The most compliant patients who bought their treatment from the most authoritative, Guytonesque, doctors were buying their own disability and death. 
	  
	  Each time you learn something, your consciousness becomes something different, and the questions you ask will be different; you don't know what the next appropriate question will be when you haven't assimilated the earlier answers. Until you see something as the answer to an urgent question, you can't see that it has any value. The unexpected can't be a commodity. When people buy professional knowledge they get what they pay for, a commodity in a system that sustains ignorance.

    \end{answer}
\end{standalonequote}

\begin{qaexchange}{Philosophy \& Worldview}
    \metadata{topic={Empathy, Social Dynamics}, source={Ray Peat Forum}}

    \begin{question}
        Meaning of empathetic reaction to self-destructive NPC?
    \end{question}

    \begin{answer}
      I think it's kind of a political judgment—if you want to optimize your efforts, supporting the efforts of others to do good things, and defending yourself against the energy-consuming processes of destructive people, is more effective than focussing on the suppression of evil.
    \end{answer}
\end{qaexchange}

\begin{standalonequote}{Philosophy \& Worldview}
    \metadata{topic={Bitcoin Economics}, source={Ray Peat Forum}}

    \begin{answer}
      When you mine gold, you have something that exists separately from what anyone might think about it. Bank lending counts ownership of debt as assests, if the debtors fail, the system fails, but at least some real value is there, as something mortgaged; it isn't all pure fiat money. Bitcoin mining is a way of creating belief, nothing more.
    \end{answer}
\end{standalonequote}

\begin{standalonequote}{Philosophy \& Worldview}
    \metadata{topic={Analog Machines}, source={Ray Peat Forum}}

    \begin{note}
        \enquote{Relationship} With Analog Machines
    \end{note}

    \begin{answer}
      Yes, that's the property that makes it possible for us to relate to them in a natural way. Norbert Wiener wrote about it.
    \end{answer}
\end{standalonequote}

\begin{qaexchange}{Philosophy \& Worldview}
    \metadata{topic={Protocols of Elders of Zion}, source={Ray Peat Forum}}

    \begin{question}
        Does Allen Dulles' involvement with the \textit{Protocols of the Elders of Zion} prove that it is authentic?
    \end{question}

    \begin{answer}
      Wikipedia emphasizes the plagiarism in the Protocols, from fictional works, but neither plagiarism nor forgery affects its importance. A novelist who said things like that in the 1860s was obviously onto something, since there are prominant people now in Israel who are saying similar, and worse, things. People who just speak honestly about what they observe are called prophets---it can take others a very long time to recognize that such things happen.
    \end{answer}
\end{qaexchange}

\section{Practical Considerations}

\begin{standalonequote}{Practical Considerations}
    \metadata{topic={Exercise Duration Limits}, source={Email Wiki}}

    \begin{answer}
        I think periods of intense muscular exertion should be limited to 20 or 30 seconds, followed by rest periods. Otherwise, T\textsubscript{3} falls and the stress signals rise. If mental activity has a sense of obligation, of being pushed, it can raise the same stress mediators (serotonin, TSH, prolactin, CRH, cortisol, etc.), but if the attitude is one of opening and exploring new possibilities, it activates restorative processes throughout the body.
    \end{answer}
\end{standalonequote}

\begin{standalonequote}{Practical Considerations}
    \metadata{topic={Exercise Types Comparison}, source={Email Wiki}}

    \begin{answer}
        Concentric resistance training has an anabolic effect on the whole body. Sprinting is probably o.k. Endurance exercise is the worst. I don't think martial arts are necessarily too stressful.
    \end{answer}
\end{standalonequote}

\begin{standalonequote}{Practical Considerations}
    \metadata{topic={Exercise Nutrition}, source={Email Wiki}}

    \begin{answer}
        Orange juice is very helpful, maybe some salty thing; I don't think niacinamide would be necessary, though it would be an interesting experiment.
    \end{answer}
\end{standalonequote}

\begin{qaexchange}{Practical Considerations}
    \metadata{topic={Stretching, Yoga, Weights Safety}, source={Email Wiki}}

    \begin{question}
        Do you consider stretching or yoga healthy? Or lifting weights or sprinting infrequently for that matter?
    \end{question}

    \begin{answer}
        Those can all be helpful. The two things that most often make exercise harmful are activity that keeps the lactic acid high chronically, and \enquote{eccentric} exercise, in which muscles are stretched while contracting, as in running downhill.
    \end{answer}
\end{qaexchange}

\begin{qaexchange}{Practical Considerations}
    \metadata{topic={Concentric Exercise Protocol}, source={Email Wiki}}

    \begin{question}
        So to clarify, lifting weights with only concentric exercises, while making sure to not get out of breath would be the best practice? Does high intensity/low volume produce greater lactic acid then high volume/low intensity?
    \end{question}

    \begin{answer}
        If volume refers to the mass of muscle involved, probably not, depending on the exact intensities and volumes. Ten pound dumbells, lifted quickly for 30 to 60 seconds, for example, is usually good for increasing the anabolic and protective hormones.
    \end{answer}
\end{qaexchange}

\begin{standalonequote}{Practical Considerations}
    \metadata{topic={Intermittent Training Effects}, source={Email Wiki}}

    \begin{answer}
        I think intermittent training is good if it avoids increased cortisol. Some nutrients, like vitamin K, can be stored in the fat and liver for a long time. Intense stress activates epigenetic processes that I think are hard to reverse. Temporary excess of some nutrients can probably help to restore processes to normal, or to higher functional levels. Deprivation increases the ability to tolerate deprivation. The mind is always involved, with imagination being part of the body-forming processes, and it's important to keep the whole life development in mind.
    \end{answer}
\end{standalonequote}

\begin{qaexchange}{Practical Considerations}
    \metadata{topic={PUFA And Exercise Recovery}, source={Email Wiki}}

    \begin{question}
        Exercise induced hypothyroidism: Is a person who doesn't eat PUFAs and runs long distance less likely, to become hypothyroid?
    \end{question}

    \begin{answer}
        Yes, a person relatively free of PUFA will be likely to recover very quickly from prolonged stress.
    \end{answer}
\end{qaexchange}

\begin{standalonequote}{Practical Considerations}
    \metadata{topic={Optimal Sleep Duration}, source={Email Wiki}}

    \begin{answer}
        It varies with the season, but 8 to 8 1/2 hours is usually best.
    \end{answer}
\end{standalonequote}

\begin{standalonequote}{Practical Considerations}
    \metadata{topic={Longevity Priorities}, source={Email Wiki}}

    \begin{answer}
        Minimizing intestinal inflammation/endotoxin/excess iron, methionine, tryptophan consumption.
    \end{answer}
\end{standalonequote}

\begin{standalonequote}{Practical Considerations}
    \metadata{topic={Ultradian Rhythm Awareness}, source={Email Wiki}}

    \begin{answer}
        Going to sleep and the few minutes after waking up are good times to see how things are working. Stresses and obligations shape the digestive and metabolic processes, and the rhythms of the intestine add to the shape of the day's thoughts. There are usually about 16 small cycles during a day, and watching for them can make things more spontaneous.
    \end{answer}
\end{standalonequote}

\begin{standalonequote}{Practical Considerations}
    \metadata{topic={Sensory Awareness Practices}, source={Email Wiki}}
    \begin{note}
        Question on How to Access/Reassess Emotions Independent of Language
    \end{note}

    \begin{answer}
        There are some ways of directing the attention that I have found to be useful. Thinking about the sensing surface as distinct from the thing sensed is a way to start to get away from language's control of consciousness. After-images of bright objects are a convenient place to start. Coffee and vitamin B\textsubscript{1} are helpful while doing the practices. With eyes closed, watch for spontaneous visual events after the after-image has faded. Putting attention on the solar plexus region while thinking about people you know, noticing the abdominal sensations related to different people, is another kind of sensory exercise.
    \end{answer}
\end{standalonequote}

\begin{standalonequote}{Practical Considerations}
    \metadata{topic={Pet Aspirin Dosing}, source={Email Wiki}}
    \begin{note}
        Giving Aspirin (10 mg/day) to Cat for Motility Problems
    \end{note}

    \begin{answer}
        That seems like a safe dose. I think some of the studies confuse the effects of the stress of intravenous treatment with the effects of aspirin.
    \end{answer}
\end{standalonequote}

\begin{qaexchange}{Practical Considerations}
    \metadata{topic={Sex And Stress Effects}, source={Email Wiki}}

    \begin{question}
        Does masturbation or sex have positive or negative effects on the hormones or stress?
    \end{question}

    \begin{answer}
        Generally positive, but intense arousal can have unwanted consequences, such as herpes virus outbreaks.
    \end{answer}
\end{qaexchange}

\begin{standalonequote}{Practical Considerations}
    \metadata{topic={Exercise For Belly Fat}, source={Email Wiki}}

    \begin{answer}
        Some muscle-building resistance exercise might help to increase the anabolic ratio, reducing the belly fat.
    \end{answer}
\end{standalonequote}

\begin{qaexchange}{Practical}
    \metadata{topic={University Education}, source={Ray Peat Forum}}

    \begin{question}
        Would you recommend a person to go to the university if had the opportunity, with the basic understanding of scientists and intellects you refer too in your work? For example studying biochemistry while keeping in the back of the mind, work by gilbert ling's, szent gyorgyi, Mae-Wan-Ho and others.

        Or would it be a complete waste of time due to the direction that the university takes? Could conforming to the studies of the university have a large impact on ones ability to learn and accept work individually?
    \end{question}

    \begin{answer}
        I think it's very worthwhile, if you can keep a critical mind. Funding and publication are provided only for things that serve their interests, but they can't suppress everything of value.
    \end{answer}
\end{qaexchange}

\begin{emailexchange}{Practical}
    \metadata{topic={Monogamy}, source={Ray Peat Forum}}

    \begin{question}
        I'd be interested to find out what you think about monogamy in human relationships. Do you think, for example, that to commit to a theoretically life-long contract of sexual exclusivity is to stifle our true nature in a potentially harmful way? Monogamous, long-lasting relationships are generally held up as the model in this domain, with anything else usually considered as a failure or a perversity. Is it right that the sacrifices inherent in the commitment of monogamy should be held in such esteem?
    \end{question}

    \begin{answer}
        I think it's based on the property mentality.
    \end{answer}
\end{emailexchange}

\begin{emailexchange}{Practical Considerations}
    \metadata{topic={Hospitalization Decisions}, source={Ray Peat Forum}}

    \begin{question}
         I was wondering, is there any ailment, other than broken bones or something, that you would consider going into hospital for?
    \end{question}

    \begin{answer}
       I don't think I would for a broken bone, unless it happened in Romania or some other less medicalized country, but I don't suggest that other people should be so cautious. 
    \end{answer}

    \begin{question}
        Interesting, would you do in the case of a broken bone? What symptoms do you think it would be reasonable for somebody to visit the hospital?
    \end{question}

    \begin{answer}
       1982 I slipped on ice while carrying a heavy machine, and sat down on my ankle, causing a sharp pain in the lower half of my fibula; the area was swollen, with a discolored area along the bone, and for a few weeks I couldn't put my weight on it. Nothing was displaced noticeably, so there was nothing to do but let it heal. I had known people with broken bones that had been badly set by doctors, and others whose bones had been perfectly set by farmers, so I would have had a friend help if I couldn't reset a major fracture by myself. If a person doesn't understand what's happening to them, and thinks it's something that a hospital could help, they should go, but they should also have a reasonable amount of knowledge about their body, and about the dangers of hospitalization. A couple of years ago I was involved, by telephone, with someone who suddenly developed extreme weakness and bleeding in his lungs, and whose doctor quickly did a series of appropriate tests and used appropriate treatments, and brought him back from a comatose state in a few days. It was the sort of thing that Dr. House might have done, the intelligent application of a large amount of knowledge. Such things are possible, but after having seen hundreds of people ruined by medical ignorance, I was surprised to see that it could actually happen.
    \end{answer}
\end{emailexchange}

\begin{qaexchange}{Practical Considerations}
    \metadata{topic={Walking to Burn PUFA}, source={Ray Peat Forum}}

    \begin{question}
        Do you think low intensity walking could safely help burn up PUFA stored in the body?
    \end{question}

    \begin{answer}
       Yes, it's a safe way to do it.
    \end{answer}
\end{qaexchange}

\begin{qaexchange}{Practical Considerations}
    \metadata{topic={Exercise Timing, Carbohydrates}, source={Ray Peat Forum}}

    \begin{question}
        When exercising for fat loss, so long as it isn't breathless, is an empty stomach better for losing fat?
    \end{question}

    \begin{answer}
      I think it's best to have had some carbohydrate within the hour, to prevent the toxic effects of a large increase in free fatty acids in the blood.
    \end{answer}
\end{qaexchange}

\begin{qaexchange}{Practical Considerations}
    \metadata{topic={Physical Activity, Free Movement}, source={Ray Peat Forum}}

    \begin{question}
        If you don't mind me asking, how much physical activity is incorporated in your life that doesn't count as exercise?
    \end{question}

    \begin{answer}
      Painting and sculpting involve continuous activity, buying and cooking food, occasionally playing cello, etc., continually interrupt sitting to answer emails or to read. Free movement is essential. It's stereotyped movement that I think harmfully affects people. For rodents living in little boxes, running in a wheel is a little closer to free living.
    \end{answer}
\end{qaexchange}

\begin{qaexchange}{Practical Considerations}
    \metadata{topic={Yoga and Meditation}, source={Ray Peat Forum}}

    \begin{question}
        What are your thoughts about yoga and meditation?
    \end{question}

    \begin{answer}
      Best when incorporated into regular activities.
    \end{answer}
\end{qaexchange}

\begin{standalonequote}{Practical Considerations}
    \metadata{topic={Emergency Preparedness}, source={Ray Peat Forum}}

    \begin{answer}
      I keep enough cash on hand to get to Mexico in situations of frozen accounts or imminent supply chain breakdown; in rural Mexico, the essential supply chains are so short that they are very stable. In Coeneo, we used to use fallen branches as fuel in the water heater, but now there's a plastic solar water heater on the roof, that didn't cost much. Solar electricity is a very good idea—electricity and city water are subject to scarcity and rationing. Having a good supply of durable staples is always convenient, even without major calamities.
    \end{answer}
\end{standalonequote}

\begin{qaexchange}{Practical Considerations}
    \metadata{topic={Exercise Philosophy}, source={Ray Peat Forum}}

    \begin{question}
        Do you exercise at all?
    \end{question}

    \begin{answer}
      I occasionally do things like cutting wood, doing repairs, etc.; physical exertion is good, especially when it feels good.
    \end{answer}
\end{qaexchange}

\section{Special Populations}

\begin{standalonequote}{Special Populations}
    \metadata{topic={Dog Diet Recommendations}, source={Ray Peat Forum}}

    \begin{answer}
      I think meat, fish, cottage cheese, eggs, well cooked potatoes, squash, and fruit are safe, same as for people.
    \end{answer}
\end{standalonequote}

\begin{standalonequote}{Special Populations}
    \metadata{topic={Cat Anxiety Treatment}, source={Ray Peat Forum}}

    \begin{answer}
      Milk and vitamin D, and maybe a little progesterone or pregnenolone, can reduce fearfulness. If the kittens are more open to being petted, the mother could learn by watching. An extremely small amount of cyproheptadine helps with anxiety.
    \end{answer}
\end{standalonequote}

\begin{standalonequote}{Special Populations}
    \metadata{topic={Cats and Milk Tolerance}, source={Ray Peat Forum}}

    \begin{answer}
      Everyone seems to have their idea of who is lactose intolerant. Cats obviously adapt very well to milk, the way people do. 
    \end{answer}
\end{standalonequote}

\begin{standalonequote}{Special Populations}
    \metadata{topic={Cat Diet, Carbohydrates}, source={Ray Peat Forum}}

    \begin{answer}
      They can handle some carbohydrate, but most of the diet should be meat, fish, eggs, liver, skin, cartilage, and ground bone or cheese or milk for extra calcium. There are some very interesting videos on youtube of cats enjoying candies, ice cream, melons, etc., that I think disprove the idea that cats can't taste sugar.
    \end{answer}
\end{standalonequote}

\subsection{Pregnancy \& Lactation}

\begin{qaexchange}{Pregnancy \& Lactation}
    \metadata{topic={Pregnancy And Acne Treatment}, source={Email Wiki}}

    \begin{answer}
        Antibiotics generally shouldn't be used during pregnancy. Sulfur (precipitated or sublimed) can be mixed into a slurry with water and patted onto the area. Dissolved aspirin used with it increases the antiseptic effect and reduces inflammation.
    \end{answer}
\end{qaexchange}

\begin{standalonequote}{Pregnancy \& Lactation}
    \metadata{topic={Pregnancy Nausea}, source={Email Wiki}}

    \begin{answer}
        Salt is often the most important thing for pregnancy nausea. Two quarts of milk daily, cheese, eggs, and orange juice, but with anything salty, even sips of salty water first thing in the morning, should stop it. Low thyroid function, with a low ratio of progesterone to estrogen, causes the kidneys to be unable to retain salt efficiently.
    \end{answer}
\end{standalonequote}

\begin{standalonequote}{Pregnancy \& Lactation}
    \metadata{topic={Prenatal Hypothyroidism Effects}, source={Email Wiki}}
    \begin{note}
        Hypothyroidism During Pregnancy, Effect on Offspring, Treatment in the 50s
    \end{note}

    \begin{answer}
        I think a background of hypothyroidism, even when it's compensated by high production of the stress hormones so that the classical symptoms aren't present, is a major factor in reproductive problems, and in increasing susceptibility to injury by toxins, including DES and anesthetics. Progesterone production depends on good thyroid function, and as it declines cortisol and other stress hormones increase.
    \end{answer}
\end{standalonequote}

\begin{standalonequote}{Pregnancy \& Lactation}
    \metadata{topic={Historical Pregnancy Treatments}, source={Email Wiki}}

    \begin{answer}
        A few doctors at that time [1956] used real progesterone supplements during pregnancy, but generally they were much more likely to use a synthetic progestin, or DES or estrogen.
    \end{answer}
\end{standalonequote}

\begin{standalonequote}{Pregnancy \& Lactation}
    \metadata{topic={Prenatal Hormone Imprinting}, source={Email Wiki}}

    \begin{answer}
        The prenatal environment can imprint a pattern of hormone balance, especially hypothyroidism, that tends to persist until new patterns can be formed, and that usually requires prolonged supplementation and a very good diet. With a good balance of nutrients and thyroid function, the protective progesterone, pregnenolone and DHEA are produced sufficiently to reduce the burden on the adrenal glands.
    \end{answer}
\end{standalonequote}

\begin{standalonequote}{Pregnancy \& Lactation}
    \metadata{topic={Hypothyroidism Family History}, source={Email Wiki}}

    \begin{answer}
        Gallbladder problems are extremely common in hypothyroidism [family history], and the compensating stress hormones produce problems with blood sugar regulation.
    \end{answer}
\end{standalonequote}

\begin{standalonequote}{Pregnancy \& Lactation}
    \metadata{topic={C-Section}, source={Email Wiki}}

    \begin{answer}
        C-sections, anesthesia, DES, and mechanical attitudes toward pregnancy and nutrition all have their place in the authoritarian medical culture, part of a generally stressful culture. That culture is coherent and self-validating, and escape from it has to be equally systematic to be able to persist.
    \end{answer}
\end{standalonequote}

\begin{standalonequote}{Pregnancy \& Lactation}
    \metadata{topic={Aspirin Safety In Pregnancy}, source={Email Wiki}}

    \begin{answer}
        Aspirin has a good record of safety in pregnancy, paracetamol doesn't. It became the most commonly used analgesic, displacing aspirin, because of advertising, not science.
    \end{answer}
\end{standalonequote}

\begin{standalonequote}{Pregnancy \& Lactation}
    \metadata{topic={Aspirin Safety in Pregnancy}, source={Ray Peat Forum}}

    \begin{answer}
      Aspirin has a good record of safety in pregnancy, paracetamol doesn't. It became the most commonly used analgesic, displacing aspirin, because of advertising, not science.

      Aspirin has been studied mainly in women with a tendency to miscarry or to develop preeclampsia, and it improved the health of their babies, but I don't know of any good animal studies that would involve a realistic intermittent use of larger doses.
    \end{answer}
\end{standalonequote}

\begin{standalonequote}{Pregnancy \& Lactation}
    \metadata{topic={Vaccines Affecting Breastmilk}, source={Ray Peat Forum}}

    \begin{answer}
      Substances in a vaccine are able to trigger a reaction in a nursing baby who is allergic to that substance. Any immune reaction affects the mother's physiology, and vaccination can affect the quality and quantity of the milk. Commercial dairies are interested in this because it can affect their profits, but the medical business prefers not to investigate the subject, because it could affect their profits.
    \end{answer}
\end{standalonequote}

\begin{emailexchange}{Pregnancy \& Lactation}
    \metadata{topic={Progesterone During Pregnancy}, source={Ray Peat Forum}}

    \begin{question}
        What would be your approach with using supplemental progesterone if you were a newly pregnant female?
    \end{question}

    \begin{answer}
      It's necessary to know the general situation, including the balance of nutrients in the diet, the metabolic rate, temperature, pulse rate, body weight, age. and menstrual cycle history.
    \end{answer}
	
    \begin{question}
        Would it be a matter of using more the worse the individual situation is? Would I be right in thinking that the dose should gradually increase during the pregnancy? I.e; starting with one drop per day, and adding a drop every month?
    \end{question}

    \begin{answer}
      Katharina Dalton sometimes gave the same amount all the way through pregnancy, but sometimes it's no longer needed when the placenta makes enough. The nutritional discoveries of Tom Brewer, Jay Hodin and others support progesterone production. Thyroid hormone makes the use of progesterone easier.
    \end{answer}
\end{emailexchange}

\subsection{Children \& Adolescents}

\begin{standalonequote}{Children \& Adolescents}
    \metadata{topic={Teen Hormone Fluctuations}, source={Email Wiki}}

    \begin{answer}
        Around puberty the changing hormones, especially momentarily high estrogen in boys, can cause some obsessive episodes, and I suppose stress could be involved with early high testosterone production. A few small supplementations with thyroid or pregnenolone might reduce the stress and extend his growing years, but he could judge by whether it made him feel better. Lots of milk and fruit are appropriate in the teens, with eggs, seafoods, and meats according to appetite.
    \end{answer}
\end{standalonequote}

\begin{standalonequote}{Children \& Adolescents}
    \metadata{topic={Teen Thyroid and Vitamin A Dosing}, source={Email Wiki}}

    \begin{answer}
        An eighth or tenth of a cynoplus tablet is a good amount for a trial, it's a little less than the body would normally produce in an hour, so it's enough to detect as a change of state, slight change of heart rate, warming of hands, for example. Vitamin A is such a basic metabolic factor, in the brain and endocrine glands, liver and kidneys, etc., I think it's dangerous to experiment with drugs that interfere with it.
    \end{answer}
\end{standalonequote}

\begin{standalonequote}{Children \& Adolescents}
    \metadata{topic={Primary Teeth Treatment}, source={Email Wiki}}
    \begin{note}
        Milk Teeth
    \end{note}

    \begin{answer}
        I don't think anesthesia should be combined with cyproheptadine, but I think dentists are too aggressive in treating deciduous teeth; the important thing is to improve her digestion and hormones as her permanent teeth are developing. The so-called temporary fillings, made of zinc oxide and eugenol, are very easy to put in, and are antiseptic and mildly anesthetic topically. (I had them in wisdom teeth, and they lasted for years.) X-rays, anesthesia, and drilling into invisible cavities have their place in especially problematic adult teeth, but seem inappropriate for teeth that will soon be gone.
    \end{answer}
\end{standalonequote}

\begin{standalonequote}{Children \& Adolescents}
    \metadata{topic={Children Protein Requirements}, source={Ray Peat Forum}}

    \begin{answer}
      Growing children need quite a bit of protein containing tryptophan.
    \end{answer}
\end{standalonequote}

\subsection{Athletes}

\begin{qaexchange}{Athletes}
    \metadata{topic={Athletes And Thyroid Supplementation}, source={Email Wiki}}

    \begin{question}
        Do you think all athletes should supplement thyroid?
    \end{question}

    \begin{answer}
        No, if the diet and level of activity are right, it shouldn't be necessary.
    \end{answer}
\end{qaexchange}

\section{Scientific Topics}

\begin{standalonequote}{Scientific Topics}
    \metadata{topic={Genetic Enzyme Variants}, source={Email Wiki}}
    \begin{note}
        On SNPs That Have Been Linked to Autism, Chronic Fatigue, etc. (Yasko)
    \end{note}

    \begin{answer}
        Yes, a few slow enzyme systems could make the system easier to disrupt.
    \end{answer}
\end{standalonequote}

\begin{qaexchange}{Scientific Topics}
    \metadata{topic={Genetic Stress Hormone Patterns}, source={Email Wiki}}

    \begin{question}
        Can some people be genetically \enquote{programmed} to have high stress hormones to compensate for hypothyroidism?
    \end{question}

    \begin{answer}
        The population has a range of those effects, it just takes longer to change them when they are strongly imprinted.
    \end{answer}
\end{qaexchange}

\begin{standalonequote}{Scientific Topics}
    \metadata{topic={Cellular Inhibition Mechanisms}, source={Email Wiki}}

    \begin{answer}
        Inhibition occurs when the proteins are in a relatively acidic (electron-withdrawn, electronegative) state, because of the conditioning factors, including high ATP, \ce{CO2}, and ketone and quinone groups associated with them, and the relative exclusion of disordering groups and ions, such as hydroxyls, phosphate, and sodium. 
    \end{answer}
\end{standalonequote}

\begin{qaexchange}{Scientific Topics}
    \metadata{topic={Lidocaine Epigenetic Changes}, source={Email Wiki}}

    \begin{question}
        Epigenetic?
    \end{question}

    \begin{answer}
        I think that must be it, for the improvement to be so stable.
    \end{answer}
\end{qaexchange}

\begin{standalonequote}{Scientific Topics}
    \metadata{topic={Microdosing LSD Effects}, source={Email Wiki}}

    \begin{answer}
        The small doses, like coffee, help to optimize normal processes. Giant doses of either will deplete energy stores. Sugar, salt, milk, gelatin, juice, etc., help to restore the reserves.
    \end{answer}
\end{standalonequote}

\begin{qaexchange}{Scientific Topics}
    \metadata{topic={LSD Electronic Resonance}, source={Email Wiki}}

    \begin{question}
        follow-up: Is this effect on electronic resonance just related to its anti serotonin effects or is it somthing more? Would cyproheptadine or lisuride have this effect?
    \end{question}

    \begin{answer}
        Things that protect against \enquote{reductive stress} (an excess of metabolic electrons) protect the sensitivities of cells that make coherent integrated function possible. Szent-Gyorgyi talked about an intracellular integration made possible by maintaining a partially oxidized state of proteins, and I'm thinking about intercellular communication of this electronic state. Stress shifts metabolism away from this condition. Once the state exists, it tends to be stable by itself, without drugs, in the absence of stress.
    \end{answer}
\end{qaexchange}

\begin{standalonequote}{Scientific Topics}
    \metadata{topic={LSD Therapeutic Use}, source={Email Wiki}}

    \begin{answer}
        If I were in a place where it's not illegal, I think I would want to occasionally use 10 mcg quantities. I think it's one of the things that can help to maintain the proper electronic resonance of the organism. (Have you heard any of Luca Turin's talks on resonance?)
    \end{answer}
\end{standalonequote}

\begin{standalonequote}{Scientific Topics}
    \metadata{topic={LSA}, source={Email Wiki}}

    \begin{answer}
        I'm not familiar with it, but I assume it would be a serotonin antagonist.
    \end{answer}
\end{standalonequote}

\begin{standalonequote}{Scientific Topics}
    \metadata{topic={LSD Depleting Glycogen}, source={Email Wiki}}
    \begin{note}
        Long-term Effects
    \end{note}

    \begin{answer}
        It acts as a learning ce, and can affect your general attitudes; if the amount is excessive, causing depletion of brain glycogen, I think it can lead to prolonged defensive attitudes, probably with a rebound of serotonin.
    \end{answer}
\end{standalonequote}

\begin{standalonequote}{Scientific Topics}
    \metadata{topic={Reductive Stress Protection}, source={Email Wiki}}
    \begin{note}
        Mechanism of LSD
    \end{note}

    \begin{answer}
        Things that protect against \enquote{reductive stress} (an excess of metabolic electrons) protect the sensitivities of cells that make coherent integrated function possible. Szent-Gyorgyi talked about an intracellular integration made possible by maintaining a partially oxidized state of proteins, and I'm thinking about intercellular communication of this electronic state. Stress shifts metabolism away from this condition. Once the state exists, it tends to be stable by itself, without drugs, in the absence of stress.
    \end{answer}
\end{standalonequote}

\begin{standalonequote}{Scientific Topics}
    \metadata{topic={Memory Electron Theory}, source={Email Wiki}}
    \begin{note}
        Physiology
    \end{note}

    \begin{answer}
        I've been gradually clarifying my ideas about memory, and currently I'm putting more emphasis on the role of electrons in maintaining biological coherence, including the construction of a variably coherent model of the world, as we develop in it. I see it as being a very large generalization-like construction with spatial quality, with memories being constructed as needed for whatever situation we find ourselves in, somewhat like deductions from our present state. Helmut Schwartz showed that electrons (in a beam) have \enquote{memory,} picking up a modulation and expressing it after a delay. I think that very flexible ability of electrons to be \enquote{modulated} is a central part of the memory process.
    \end{answer}
\end{standalonequote}

\begin{standalonequote}{Scientific Topics}
    \metadata{topic={Mind-Body Philosophy}, source={Email Wiki}}

    \begin{answer}
        While I was (\dots{}) a psychology major, I did some surveys (1957) relating to creativity and types of thought and dreaming, following up some ideas I found in Brewster Ghiselin's book The Creative Process. I felt that the current US view of the brain as a computing device with nerves serving as wires and switches was completely inappropriate, even for understanding things such as the perception of odors and musical pitch, and around that time a practical study of creativity was published, in a book called Synectics, and I saw that Pavlov's colleague P.K. Anokhin had been developing a much better understanding of brain function. The fact that sensations and perception of space in dreams can be so convincing led me to feel that biological/metabolic processes in the brain reproduce in fairly direct or literal ways things in the external world, i.e., that our experience of internal colors and smells and sounds are probably a sort of electrochemical resonance within nerves---with a nerve and its surroundings, spatial parts of the brain, taking on energetic states with the frequencies that are closely analogous to the frequencies produced by the external objects, colors, chemical odors, sound vibrations, as well as other kinds of patterned relationships. If \enquote{photons} or electromagnetic interactions within the organism are the substance of consciousness, then the electronic properties of nutrients, hormones, and drugs are important, rather than their geometric form, as interpreted by the \enquote{lock-and-key} \enquote{receptor and ligand} doctrine. I think the active chemical in St. John's wort is hypericin, an anthroquine (very similar to emodin, in cascara, and to vitamin K and tetracycline), which is a large system of conjugated electrons, that interacts powerfully with our cellular regulatory systems. (\dots{}) I suspect that growing up with creativity involves opportunities that cause the brain to develop various sensitivities and resonances, and that the brain functioning in these ways calls up the energetic and hormonal resources that it needs, and ideally that includes an array of chemicals that enrich and intensify consciousness, allowing very complex internal experiences to be generated.
    \end{answer}
\end{standalonequote}

\begin{standalonequote}{Scientific Topics}
    \metadata{topic={Consciousness And Body Awareness}, source={Email Wiki}}

    \begin{answer}
        At any moment, one's position in the world is part of one's image of the world, and body awareness is part of our consciousness of our position. Being is the basic thing, and there is really no understanding separate from that, although there are symbolic patterns that can be manipulated as if they were separate from the substance, but that's just a matter of habitual attention. The \enquote{faint glass} people are identifying with the constructed story about life, rather than seeing it as an aspect of a single substance-awareness. Toes (and internal organs) are part of everything we do, making up part of the substance and meaning of things, except when indoctrination directs attention away from them.
    \end{answer}
\end{standalonequote}

\begin{standalonequote}{Scientific Topics}
    \metadata{topic={Ultradian Cycle Research}, source={Email Wiki}}

    \begin{answer}
        Besides articles in psychology and medical journals, the ultradian cycles have been described from a variety of perspectives. R.O Becker discussed weak natural electromagnetic rhythms, Frank Brown did many experiments showing the effects of surrounding fields on biorhythms, Solco Tromp's publications on biometeorology and Michel Gauquelin's statistical studies showed other effects. There have been quite a few Hindu publications on body cycles. When I taught school and had to get up at the same time every day, I developed a strong metabolic rhythm that made me go to sleep immediately at 10:30, and if I had to stay awake, I had a sudden loss of energy exactly at 10:30. A daytime nap that's timed according to the small cycles can be very effective.     
    \end{answer}
\end{standalonequote}

\begin{qaexchange}{Scientific Topics}
    \metadata{topic={Attention To Sensation}, source={Email Wiki}}

    \begin{question}
        Follow-up: Sensing surface as distinct from thing sensed?
    \end{question}

    \begin{answer}
        Just for an example, if you touch a marble with the tips of crossed fingers, the first reaction is that there are two marbles, because of the normal projection of awareness of objects in the world. When you look at an empty sky, you can usually notice different kinds of \enquote{objects} or textures, that aren't the sky (some people insist that those are something in the sky; others that they are nothing but debris in the eye); when you direct your attention to the sensory surface, rather than to the object, you can notice that the process of noticing affects what you notice. Attention to this process makes it possible to feel the process of thought interacting with sensations. Inattention to those processes leaves a person trapped in the system of verbal concepts and rules.
    \end{answer}
\end{qaexchange}

\begin{qaexchange}{Scientific Topics}
    \metadata{topic={Proprioceptive Awareness}, source={Email Wiki}}

    \begin{question}
        Follow-up: what does \enquote{projection of awareness of objects in the world} mean?
    \end{question}

    \begin{answer}
        Every tissue contains nerves, and for some of these, the proprioceptive nerves, their object is what we feel as our body. Other nerves sense things that we understand as objects in some sense---sound is felt to come from somewhere in the space around us, smells usually the same, and tastes represent the objects that we are eating. But in each case, it's possible to experience our sensing without imagining objects as the cause of the sensation. The value of that is that it gives you an absolute, uninterpreted, experience, which makes it possible to put the verbal life history that we normally inhabit, into a new context. In the case of seeing with your eyes closed (with light on the closed eyelids, the situation is similar to looking at an empty sky), a finely granular texture is the retina itself, and/or the optic nerve. A textureless, dreamlike substance of space filling images, a relatively free activity of the visual cortex and more complex brain systems, will gradually be noticed, when there's the right combination of nervous arousal and relaxation. The behavior of after-images will change according to the state of the whole organism, for example the length of time that it stays positive, and the length of time that a following negative image lasts. It's the same with after-images of motion; their differences between people are very interesting.
    \end{answer}
\end{qaexchange}

\begin{standalonequote}{Scientific Topics}
    \metadata{topic={Quantum Theory Critique}, source={Email Wiki}}
    \begin{note}
        Consciousness, Electric Universe
    \end{note}

    \begin{answer}
        With an orientation of radical empiricism and process philosophy, I have some sympathy for Einstein's project and his reluctance to accept quantum theory. I think the quantum theory was created by philosophically inadequate people.
    \end{answer}
\end{standalonequote}

\begin{standalonequote}{Scientific Topics}
    \metadata{topic={Electric Consciousness Theory}, source={Email Wiki}}

    \begin{answer}
        Conventional views of electrons were built on just a few kinds of experiment, and I think new approaches to understanding matter will be found. While I think consciousness is electronic, I don't think it's appropriate to think of it as being just inside cells (much less simply a matter of synaptic interactions). Electricity's space-filling property is relevant. The process (or background, that we call body or self) that gives continuity and meaning to our perceptions and actions is something that's always happening, and people usually turn their attention away from it when they aren't in some practical or objective activity. The organism has many potential intentions, and if we let our attention respond, they can appear as hypnagogic images or dreams. Ordinary metabolism, and its variations, are always producing these parallel spaces, and their quality varies under the influence of various metabolites and \enquote{dopants.} I think the electric universe is analogous to the electric organism.
    \end{answer}
\end{standalonequote}

\begin{qaexchange}{Scientific Topics}
    \metadata{topic={ATP Synthesis During Muscle Stretch}, source={Email Wiki}}

    \begin{question}
        Why are migratory birds able to do such long distance flights?
    \end{question}

    \begin{answer}
        Studies in insect flight muscle found that ATP is synthesized when the muscle is (passively) stretched; i suppose it happens in birds too. Studies of nerves show that after heat is emitted during activity, heat is absorbed during recovery, i.e. The nerve slightly refrigerates its surroundings. The absorption of heat (besides preventing over-heating) as ATP is resynthesized would make the usual Expectations about chemical energy and work less applicable.
    \end{answer}
\end{qaexchange}

\begin{standalonequote}{Scientific Topics}
    \metadata{topic={Political History Importance}, source={Email Wiki}}
    \begin{note}
        Polititcal Power and Deception
    \end{note}

    \begin{answer}
        I think understanding political history is essential for understanding culture and science, because the manipulations and deceptions of power are aimed toward total control. I've been aware of U.S. fascism for most of my life, having read Smedley Butler's account of the 1934 coup attempt against FDR, involving DuPont and J.P. Morgan officials and Henry Luce's publications. Their plan was to say that they were taking over because of the president's poor health. Because Butler went to congress to reveal the plot, it was delayed for 11 years. The Dulles brothers managed the US government from 1945 to 1963, and Gladio guided Europe. In 1965 at Blake College, we had a course on the role of the Masons in the politics of Latin America, taught by Iso Brante, who had done considerable research on the subject. At that time, the Sinarquistas were still very active in Mexico. For a long time I have thought of Russia and the west side of Latin America as the places where civilization would have its best chance to escape.
    \end{answer}
\end{standalonequote}

\begin{standalonequote}{Scientific Topics}
    \metadata{topic={Receptor Theory Critique}, source={Email Wiki}}
    \begin{note}
        Alternative to \enquote{Receptors}
    \end{note}

    \begin{answer}
        There are proteins that bind chemicals while those chemicals are producing effects on cells, but the schemes that use them to explain cell physiology are very ideological, usually arbitrarily excluding many alternative explanations. In some cases, they are fraudulent, in others, just stupid, but usually important for the drug industry. I think it's valuable to investigate the development of the estrogen receptor idea by Elwood Jensen.
    \end{answer}
\end{standalonequote}

\begin{standalonequote}{Scientific Topics}
    \metadata{topic={Nuclear Decay Environmental Effects}, source={Email Wiki}}

    \begin{answer}
        J.L. Anderson, \enquote{Non-Poisson Distributions Observed During Counting of Certain Carbon-14 Labeled (Sub) Monolayers,} Journal of Physical Chemistry, Vol. 76, No. 4 (1972). (Nuclear decay isn't random and depends on its environment)
    \end{answer}
\end{standalonequote}

\begin{standalonequote}{Scientific Topics}
    \metadata{topic={Ether Drift Experiments}, source={Email Wiki}}

    \begin{answer}
        Dayton Miller's Ether-Drift Experiments
    \end{answer}
\end{standalonequote}

\begin{standalonequote}{Scientific Topics}
    \metadata{topic={Resonance In Biology}, source={Email Wiki}}
    \begin{note}
        Luca Turins' TED Talk on Resonance
    \end{note}

    \begin{answer}
        He talks about the need for resonant electronic interaction in drug actions and smell; similar arguments have been made for vision, hearing, and other senses. In his model of the \enquote{receptor} that responds to a drug he didn't talk about the things surrounding the receptor, which transmit the effect into the cell; the standard idea is that \enquote{molecular cascades} of interaction diffuse the signal through the cytoplasm, but an alternative view is that the microtrabecular system is a communication system. It would be within this molecular network that the electronic resonance coherently transmits the excitations that make up consciousness. Resonant theories of sense and awareness go back at least 70 years. In the 1960s to 1980s, when all the textbooks described the cytoplasm as a liquid in which reactions were governed by free diffusion, with Michaelis-Menten kinetics, Sidney Bernard showed, stoichiometrically, that there is no free diffusion involved in the reactions of glycolysis, and glycolysis was the very basis for the belief that biochemistry could be studied in test-tubes, in watery solution. This requires a different view of cell organization. The living cell can be seen as an excitable medium, supporting oscillating reactions, with an inherent directionality. A.G.Gurwich, P.K. Anokhin, A. Szent-Gyorgyi, Mae-Wan Ho, and many others have contributed to developing this view
    \end{answer}
\end{standalonequote}

\begin{standalonequote}{Scientific Topics}
    \metadata{topic={Biology Book Recommendations}, source={Email Wiki}}
    \begin{note}
        Good Biology Books
    \end{note}

    \begin{answer}
        Harold Hillman's books are good for surveying the nonfactual aspects of \enquote{biology.} Sometimes I put \enquote{membranes} in quotation marks to indicate that people mean very different things by the word. (For example, the centrifugation \enquote{pellet} is often meant when they say \enquote{membranes.}) F.S. Sjostrand and other electron microscopists working in the 1950s to 1970s are worth looking at. The irrelevancy of the \enquote{membrane} is explained in Gilbert Ling's work.
    \end{answer}
\end{standalonequote}

\begin{standalonequote}{Scientific Topics}
    \metadata{topic={Alternative Biology Researchers}, source={Email Wiki}}

    \begin{answer}
        I don't know of any single book that assembles the important things, it's probably still necessary to read the original work and some of the things in each field that have built on those. G. N. Lewis, Peter A. Stewart (the \enquote{acid base tutorial} on the internet summarizes his approach), Bungenberg de Jong (coacervates), Sidney Fox, Walter Drost-Hansen, A.S. Troshin, Gerald Pollack, Szent-Gyorgyi, Carlos Sonnenschein, James A. Shapiro are people who have tried to avoid the mainstream mistakes, and have suggested new possibilities by the facts they chose to study.
    \end{answer}
\end{standalonequote}

\begin{standalonequote}{Scientific Topics}
    \metadata{topic={Virus Alternative Theories}, source={Email Wiki}}

    \begin{answer}
        I think the best approach to understanding viruses is to investigate recent research on microvesicles, \enquote{retrotransposons,} and incorporation of foreign (food) DNA into our cells, and to look at Bonghan Kim's ideas. C.C. Lindegren's book, Cold War in Biology, discusses some of the older ideas about horizontal transmission of DNA between very different types of organism. Bacteria can \enquote{engineer} their own genes (James A. Shapiro), and useful packets of new genetic material can be shared by unrelated types of bacteria. She's right about viruses being produced as a result of stresses, including toxic chemicals.
    \end{answer}
\end{standalonequote}

\begin{emailexchange}{Science}
    \metadata{topic={Bose and Sap Flow}, source={Ray Peat Forum}}

    \begin{question}
        Just in case you're familiar with those writings of Bose too\dots{} (The Physiology of the Ascent of Sap (1923), etc.) I think I've seen around 8 theories for sap flow\dots{} There's one in Pollack's The Fourth Phase of Water, etc. My friend Atom Bergstrom [aware of IBT] has read his share of Bose and feels that Bose probably proved the true workings of sap flow with over 200 experiments\dots{} Is that something you would agree with\dots{}? (solved long ago, but forgotten?) Or was some percentage still lacking\dots{}?
    \end{question}

    \begin{answer}
        I think Bose was right. The reason people scoffed at his idea was similar to the reason people scoffed at the idea that nerves have a contractile movement during the conduction of an impulse, or that heat generation is continuous, not saltatory, between the nodes of myelinated nerves. The people who explain the movement of sap by transpiration and \enquote{capillary attraction} don't like to think about guttation, which is most obvious at 100\% relative humidity when there's no transpiration. It's just one place where the ignorance of official biology is very visible, and the funny thing is that biologists aren't embarrassed by it.
    \end{answer}
\end{emailexchange}

\begin{qaexchange}{Scientific Topics}
    \metadata{topic={Ocean Water, Cell Hydration}, source={Ray Peat Forum}}

    \begin{question}
        Why do we not become dehydrated when we go into the ocean? Do we not absorb much sodium transdermally?
    \end{question}

    \begin{answer}
      Contemporary medicine and \enquote{official physiology} believe that cells are osmometers; they aren't. When I was a kid I read about someone who spent many weeks on the ocean without fresh water, and he said he started taking sips of sea water before their supplies of water ran out, and while the others on the boat were dying of dehydration, he was able to subsist on the seawater. Another later book described survival in a similar situation, after a nurse administered seawater enemas, knowing that the colon is able to extract water from a \enquote{hyperosmotic} solution. When a person is swimming for many hours, even in seawater, some water is absorbed through the skin. The way to understand it is to think of cell water as a solute, rather than a solvent—sodium is less soluble in cells than water is. Carbon dioxide, cholesterol, magnesium, progesterone, and urea are things that, by regulating the structure of the cytoplasm, affect its solubility properties.
    \end{answer}
\end{qaexchange}

\begin{standalonequote}{Scientific Topics}
    \metadata{topic={Cell Electronic Properties}, source={Ray Peat Forum}}

    \begin{answer}
      I connect Schwarz's electron behavior to my view of cell function. Polanyi's energy behavior in metals, Luca Turin's resonance responses, PK Anokhin's axonal conduction of complex signals (not all-or-none), and Alexander Rothen's complex long-range action, Deryagin's long-range influences, are all compatible with, supportive of, the idea that the functional integration of macromolecules, cells, and organisms, is an \enquote{electronic} state analogous to the delocalized electronic state of metals, a tunable gel in which there are no identifiable separate electrons—but able to interact with the discrete valence electrons of molecules and ions, and with appropriate fields, possibly including the fields of nuclear spin and unpaired electron spin.
    \end{answer}
\end{standalonequote}

\begin{qaexchange}{Scientific Topics}
    \metadata{topic={Body Magnetism}, source={Ray Peat Forum}}

    \begin{question}
       If someone has not been vaccinated with the covid vaccine but finds that their tissues (back and upper body) are magnetic how can this be explained? What remedies can reduce the magnetism?
    \end{question}

    \begin{answer}
       I don't think body magnetism is a problem unless you cause a compass needle to deviate from its north-south alignment.
    \end{answer}
\end{qaexchange}

\begin{standalonequote}{Scientific Topics}
    \metadata{topic={Kozyrev Time Theory}, source={Ray Peat Forum}}

    \begin{note}
        How to Visualize Kozyrev's Idea of Time's Physical Effect on Objects
    \end{note}

    \begin{answer}
      I think it has to be seen in terms of \enquote{fields} or \enquote{ether.} The same perspective helps to interpret Rupert Sheldrake's and Frank Brown's and Halton Arp's ideas. I think Horace Dudley's \enquote{neutrino sea} is a reasonable approach. Changes, energy, absorbed by the \enquote{sea} affect other processes in the sea; the idea of a gravitational lens describes similar facts, but with different assumptions.
    \end{answer}
\end{standalonequote}

\section{Art \& Culture}

\begin{qaexchange}{Art}
    \metadata{topic={Color Perception in Paintings}, source={Ray Peat Forum}}

    \begin{question}
        I was curious about the colors in the paintings on your website, especially the blue that sick women can have. It seems to me like the hormonal state of the women you're painting seeps into their surroundings. I've started to notice that the way I feel can dramatically change the way I perceive the world around me---is that something you're intending in your paintings?
    \end{question}

    \begin{answer}
      Yes, tangentially thinking about the effects of colors.
    \end{answer}
\end{qaexchange}

\begin{standalonequote}{Art \& Culture}
    \metadata{topic={Oil Paint Toxicity, Turpentine}, source={Ray Peat Forum}}

    \begin{answer}
      Starting in the 1950s I would buy three of the smallest tubes of color, and a bigger tube of zinc white, and a bottle of turpentine, and paint with a water-color-like technique, mostly just tinting the paper (no blending on the paper) so that the little tubes would last for months. In the 1980s my girl friend complained about the turpentine smell, so when I couldn't paint outside (Oregon winter) I started using mostly latex house paints, and sometimes oil pastels. The disadvantage of a latex or acrylic paint is that it dries so fast that it doesn't let you think while mixing colors on a palette. Since I didn't have skin contact with the pigments I wasn't worried about their toxicity, but I used some paint sticks 20 years ago, and noticed that an umber color was extremely allergenic or toxic to touch. A few years ago I decided to go back to oil with turpentine, and learned that the pine turpentine industry has disappeared—the familiar green and white cans of \enquote{pure gum spirits of turpentine} now contained a gasoline-like petroleum distillate. If pure turpentine was available, that, with oils, would be my preferred medium.
    \end{answer}
\end{standalonequote}

\begin{standalonequote}{Art \& Culture}
    \metadata{topic={Playing Musical Instruments, Brain Health}, source={Ray Peat Forum}}

    \begin{answer}
      Soprano recorders are inexpensive (and fit the hands better than the more expensive, mellower altos), and are convenient for sporadic playing. Playing tunes stimulates the brain in some of the same ways that speaking does, but without the pressure; for example, people who stutter when they speak usually don't when they sing. The good thing about recorders is that they are convenient, so you can play a little whenever you feel like it, while doing other things. When I was a kid I played violin for a while, but gradually realized that my neck was much too long, and my little finger too weak and slow, for that instrument. In high school I played trumpet, mostly because it was the cheapest instrument, but eventually I bought an old french horn for \$25, and an oboe, and in Paracho, Michoacan, ten years ago I finally got a cello--that had always been my favorite instrument. Every time you make sounds on a musical instrument, you are stimulating organized processes in your body--it's a kind of nourishment.
    \end{answer}
\end{standalonequote}

\begin{standalonequote}{Art \& Culture}
    \metadata{topic={Turpentine Scarcity}, source={Ray Peat Forum}}

    \begin{answer}
      Lately I have been using mainly water based paints, as a result of the scarcity of turpentine, which I had previously used with oil paints. About ten years ago I got a can with the familiar \enquote{pure gum spirits of turpentine} label, that contained mineral spirits. I tried three other brands that turned out to be variations on gasoline. The main turpentine stills in the US closed several years ago. Your mention of someone in Mexico reminded me that I had read that it's still being produced there--where does that person live? If it's possible to locate a store in Mexico that sells it, they might be willing to send a bottle of it to my place in Mexico.
    \end{answer}
\end{standalonequote}

\section{Literature \& Education}

\begin{emailexchange}{Literature}
    \metadata{topic={Writers and Languages}, source={Ray Peat Forum}}

    \begin{question}
        What sort of writers or poets do you like, aside from Blake? Any of the beat writers?
    \end{question}

    \begin{answer}
      Yes, I liked the style of several of the beat writers, Kerouac's fluidity, and Ginsberg's more intense confrontation of the culture. Aldous Huxley was another person strongly influenced by Blake.
    \end{answer}
	
    \begin{question}
        Do you read in a lot of languages?
    \end{question}

    \begin{answer}
      I felt that my school English classes had given the language a sense of artificiality for me. When I first saw verses by Blake I realized that was how English should work; later, I saw that the Beats were reclaiming English. When I was in Mexico, an American student commented that when I spoke English I sounded depressed, but that I sounded happy when I was speaking Spanish. I think that was because I learned it just by listening and talking to people in Mexico City, and so it contained my feelings about those experiences. My projects were very similar to Illich's, freeing ourselves from the culture's stereotypes. (I was often in Mexico between 1955 and 1965, overlapping Illich's time there. I had Blake College there in the early '60s.) The only other languages I read often are French and Italian; oddly, reading French feels more efficient than reading English, probably because I've never spoken it.
    \end{answer}
	
    \begin{question}
        Blake College sounds like the type of school I'd like to have gone to. I read a little about it in the book \textit{Ungodly} about Madelyn Murray O'Hair. Is there anywhere else to read about it? Or do you happen to know if there are any similar schools today?
    \end{question}

    \begin{answer}
      There were some newspaper and magazine articles about it in the 'sixties, but not much. In 1965 I knew someone with a friend in the Mexico City branch of the Ford Foundation, and we submitted a proposal to them for funding. Right after the Madalyn events, I learned that they were funding an experimental, student centered college at Fordham University, and I think other places; the foundation and the government worked closely together, and I think they saw the importance of co-opting the movement, to keep it from getting too independent, i.e., anti-imperialist. I think our idea of giving a diploma whenever the person scored very high on the Graduate Record Exam, which would let them get into good graduate study programs, was the main point they wanted to suppress, since it eliminated all of the judgmental points of control in the process.
    \end{answer}
\end{emailexchange}

\begin{qaexchange}{Literature \& Education}
    \metadata{topic={Mathematics vs Visual Activities}, source={Ray Peat Forum}}

    \begin{question}
        I was listening to one of your interviews where you talk about excessive reading as harmful, and that's why you paint. Do you think mathematics (specifically calculus or geometry) exercises the visual/spatial mode of learning similar to painting, sculpting, etc., as an adjunct to reading?
    \end{question}

    \begin{answer}
      Some kinds of thinking about geometry use some of the brain's spatial functions, but the absence of integral sensory stimulation can still leave the organism under-used.
    \end{answer}
\end{qaexchange}

\begin{qaexchange}{Literature \& Education}
    \metadata{topic={Medical School Curriculum Critique}, source={Ray Peat Forum}}

    \begin{question}
        Do you think there's a particular medical school that preserves the analogical process of intelligence in its students, or what avenue would you recommend for an aspiring physician?
    \end{question}

    \begin{answer}
       I suppose there is some variation in the personalities at the different institutions in the US, but I think an essential part of medical authoritarianism is the curriculum. If a person can stay aware, year after year, that the curriculum is an ideology with many facts embedded in it, then it should be possible to get through it without seriously impaired intelligence. 
    \end{answer}
\end{qaexchange}

\begin{qaexchange}{Literature \& Education}
    \metadata{topic={Writing and Learning Process}, source={Ray Peat Forum}}

    \begin{question}
         How do you strike the balance between writing and reading, as in would it be reasonable to dedicate years to consuming the written works of others followed by a \enquote{capstone} of your own work, or do you think it's beneficial to do things piecemeal and more spontaneously, reading, writing, and revising as time goes on, as if this helps the learning process?
    \end{question}

    \begin{answer}
      Learning is always a revising process, and writing helps to understand what you know at a particular time.
    \end{answer}
\end{qaexchange}

\begin{qaexchange}{Literature \& Education}
    \metadata{topic={Teaching Literature}, source={Ray Peat Forum}}

    \begin{question}
        If you were to teach literature to teenagers, how would you facilitate their reading and learning? What might a lesson look like?
    \end{question}

    \begin{answer}
      Choosing things that interest them. Going over short excerpts in class, showing that different ways of using language are important for everything. Finding things in \enquote{classic} authors that can be criticized and lampooned, taking literature out of its museum atmosphere.
    \end{answer}
\end{qaexchange}

\begin{qaexchange}{Literature \& Education}
    \metadata{topic={Teaching Philosophy, Syllabi}, source={Ray Peat Forum}}

    \begin{question}
        Are there any textbooks/journals you wish you could have used as a teacher; or is there a syllabus you would give to new students now, assuming you weren't restricted to a set list?
    \end{question}

    \begin{answer}
      Sources that I've mentioned in newsletters over the years are the sort of thing that I have used as background reading for courses. Many years ago, testbooks written by one person were useful when the author happened to be an insightful and imaginative person, but increasingly, textbooks are created by publishers for sales, including everything stylish enough to increase profits. Any syllabus should function as just a direction in which to exercise critical thinking; everything worth knowing becomes more when it's taken up and used.
    \end{answer}
\end{qaexchange}

\begin{qaexchange}{Literature \& Education}
    \metadata{topic={Recommended Philosophers}, source={Ray Peat Forum}}

    \begin{question}
        Which thinkers, philosophers, or writers would you recommend to someone trying to gain an appropriate grasp of ethics, morality, and worldview?
    \end{question}

    \begin{answer}
       Bergson, Whitehead, and Albert Schweitzer.
    \end{answer}
\end{qaexchange}

\section{Philosophy \& Science}

\begin{qaexchange}{Philosophy}
    \metadata{topic={Electric Universe Theory}, source={Ray Peat Forum}}

    \begin{question}
        Do you think the electric universe theory and \extlink{https://thunderbolts.info}{thunderbolts.info} is on the right track? Does it mean anything for us on the personal biological level?
    \end{question}

    \begin{answer}
      I think it's right. The same simple application of realistic physics that led to that view of cosmology has biological implications, e.g., R.O. Becker, Yuri Holodov, Solco Tromp, Madeleine Barnothy, Michele Gauquelin, et al.
    \end{answer}
\end{qaexchange}

\begin{qaexchange}{Philosophy}
    \metadata{topic={Reich and Puharich}, source={Ray Peat Forum}}

    \begin{question}
        Where do you think Reich went wrong in his thinking? Do you think his Contact with Space was full of errors where he mistook cause for effect? What about Puharich, where did he wrong? Was it related to his work with Uri Geller and Arigo the surgeon of the rusty knife?
    \end{question}

    \begin{answer}
       Even people with good insights can get confused in. the complexity, and in trying to promote their special views.
    \end{answer}
\end{qaexchange}

\begin{qaexchange}{Philosophy \& Science}
    \metadata{topic={Reincarnation, Formative Fields}, source={Ray Peat Forum}}

    \begin{question}
        Do you have any thoughts on Ian Stevenson's work? Specifically in relation to reincarnation?
    \end{question}

    \begin{answer}
      I don't know much about his work, but about 60 years ago I read about people who had detailed knowledge of preceding lives. Since I am always starting from the radical empirical awareness of complexity constantly changing in meaningful ways, I am also always considering ways to understand the meanings of the regularities. I think a quality of coherence in things, covering situations that used to be explained by a luminiferous ether, can be thought of as a \enquote{formative ether,} with resonant processes that span spaces and times. We can resonate in the same time with organisms in different places, affecting our complex developmental processes. Substances are always participating in particular situations or fields, and are never merely random. The \enquote{orthogenetic} theory of evolution described a developmental inertia, in which the existence of a structure leads to more of the same structure. A structure in one organism affects its interactions, so that a functional (or eco-) system tends to develop its own inertia. Within a certain society, these functional systems could span generations, eliciting \enquote{phenocopies} transgenerationally. Lancelot Whyte's \enquote{formative principle} and Rupert Sheldrake's \enquote{formative causation} just need a more concrete physical substrate, that I think exists in fragments, from Leibniz to Bose to Polanyi, to Horace Dudley and J.L. Anderson, etc. The idea of resonance needs a better understanding of substance as/incorporating its fields—and nothing has private independent fields.
    \end{answer}
\end{qaexchange}

\begin{standalonequote}{Philosophy \& Science}
    \metadata{topic={Flat Earth, Cosmology Theories}, source={Ray Peat Forum}}

    \begin{answer}
       I've never known a literal flat earther, but I've known several physicists and mathematicians who passionately believe that number is the basis of existence. Their timeless and colorless world is worse than a simply flat world. Our culture is swamped with highly authoritative disinformation; maybe the flat-earthers can be persuaded to be skeptical about other elements of the predominant world view. In Generative Energy I wrote about the expanding earth theory, and I've commented favorably on Halton Arp's work on galaxies, and Kozyrev's theory of stellar energy, Hannes Alfvén's plasma cosmology—nothing that's easy to interpret as flatness. 
    \end{answer}
\end{standalonequote}

\begin{qaexchange}{Philosophy \& Science}
    \metadata{topic={Ocean Acidification, Climate Change}, source={Ray Peat Forum}}

    \begin{question}
        Lately I've been hearing a lot about how awful ocean acidification is for marine life. What are your thoughts about this - could \ce{CO2} be detrimental in this context?
    \end{question}

    \begin{answer}
      I think the nuclear industry is promoting most of the propaganda on climate change, and deflecting attention away from solar cycles, and the roles of deforestation, desertification, \ce{CO2} emission from melting tundra, increased phytoplankton growth, and the decreased solubility of \ce{CO2} in warming oceans. It was just after the nuclear accident at Three Mile Island when people were turning against nuclear power, that the public's attention was diverted from cooling of the planet and a predicted new ice age, that had been discussed for decades, to the new warming supposedly caused by use of fossil fuels.
    \end{answer}
\end{qaexchange}

\begin{qaexchange}{Philosophy \& Science}
    \metadata{topic={Gene Therapy Critique}, source={Ray Peat Forum}}

    \begin{question}
        I understand and share you distain for modern medicine's obsession with the 'faulty gene' model of disease; however, for the few disorders that are truly of monogenic origin, do you see any therapeutic potential in the upcoming gene therapies---the adenovirus associated vectors, for example?
    \end{question}

    \begin{answer}
      I think the supposedly monogenic disorders involve deeper problems that would show up in different ways if a single gene were replaced.
    \end{answer}
\end{qaexchange}

\begin{qaexchange}{Philosophy \& Science}
    \metadata{topic={Gold Jewelry Effects}, source={Ray Peat Forum}}

    \begin{question}
        Do you personally think wearing gold or keeping it close to your body has any effect on a person, whether its metabolically or spiritually?
    \end{question}

    \begin{answer}
      I think the experiences of training dogs to find hidden or buried gold---indicating that they could \enquote{smell} it even when no atoms could be reaching them, show that it has biologically detectable fields.
    \end{answer}
\end{qaexchange}

\begin{qaexchange}{Philosophy \& Science}
    \metadata{topic={Denis Noble, Biological Causation}, source={Ray Peat Forum}}

    \begin{question}
        Could you share your insights about Denis Noble? Do you think it is good material? 
    \end{question}

    \begin{answer}
      His ideas on biological causation are very important. Too much phosphorus in fresh water is a problem, but eventually brown kelp and plankton will be practical sources of phosphorus for fertilizer.
    \end{answer}
\end{qaexchange}

\clearpage

\backmatter

\end{document}

